% !TEX root = ../thesis_main.tex


% Abstract
\clearpage
\begin{abstract}
%The nuclear weak force is known to be a predominantly left-handed vector and axial-vector (V-A) interaction.  An experiment is proposed to further test that observation, constraining the strength of right-handed (V+A) currents by exploiting the principle of conservation of angular momentum within a spin-polarized beta decay process.  Here, we focus on the decay \mbox{$^{37}\textrm{K} \rightarrow \,^{37}\textrm{\!Ar} + \beta^{+} + \nu_e$}.  The angular correlations between the emerging daughter particles provide a rich source of information about the type of interaction that produced the decay.
The nuclear weak interaction is known to feature both vector and axial-vector couplings in a dominant role, however the presence of scalar and tensor couplings cannot be ruled out entirely.  In beta decay physics, the Fierz interference, $\bFierz$, is an observable comprised of a linear combination of scalar and tensor couplings, and can be measured as an adjustment to the shape of the resultant beta energy spectrum.  A precision measurement experiment is conducted to observe the $\beta^+$ decay of spin-polarized $^{37}$K from an atom cloud intermittently confined by a magneto-optical trap, and  the beta energy spectra are observed in two detectors on opposing sides of the cloud, along the axis of polarization.  This geometry, combined with a knowledge of the polarization, allows the superratio asymmetry to be constructed, providing an observable which is particularly sensitive to the value of $\bFierz$, while simultaneously eliminating contributions from a variety of systematic effects.  Geant4 simulations are used to model scattering effects that could mimic the signal being searched for.  The resulting measurement gives $\bFierz = $ ?? $\pm$ ??(stat) $\pm$ ??(sys), consistent with the Standard Model.
%A precision search for a linear combination of these exotic couplings is performed 
%The nuclear weak force is understood to 
%Although the nuclear weak force primarily involves
%comprised of vector and axial-vector couplings, however the possibility of 
\end{abstract}

