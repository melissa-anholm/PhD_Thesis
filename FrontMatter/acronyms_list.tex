% !TEX root = ../thesis_main.tex
% 
% 
% 
%\listofacronyms  % "List of Acronyms"

\DeclareAcronym{MOT}
{
	short = MOT,
	long = magneto-optical trap,
%	list = the regular-est trap
}
\DeclareAcronym{AC-MOT}
{
	short = AC-MOT,
	long = alternating current magneto-optical trap
}
\DeclareAcronym{DC-MOT}
{
	short = DC-MOT,
	long = direct current magneto-optical trap
}
\DeclareAcronym{OP}
{
	short = OP,
	long = optical pumping
}
\DeclareAcronym{SOE}
{
	short = SOE,
	long = shake-off electron
}
\DeclareAcronym{SM}
{
	short = SM,
	long = standard model of particle physics
}
\DeclareAcronym{BSM}
{
	short = BSM,
	long = beyond the standard model
}
\DeclareAcronym{PDF}
{
	short = PDF,
	long = probability density function
}
\DeclareAcronym{JTW}
{
	short = JTW,
	long = Jackson--Treiman--Wyld,
	list = Jackson--Treiman--Wyld \cite{jtw}\cite{jtw_coulomb}
}
\DeclareAcronym{QED}
{
	short = QED,
	long = quantum electrodynamics
}
\DeclareAcronym{ROC}
{
	short = ROC,
	long = recoil-order corrections
}
\DeclareAcronym{TOF}
{
	short = TOF,
	long = time of flight
}
\DeclareAcronym{MCP}
{
	short = MCP,
	long = microchannel plate
}
\DeclareAcronym{G4}
{
	short = G4,
	long = Geant4
}
\DeclareAcronym{SMC}
{
	short = SMC,
	long = simple monte carlo
}
\DeclareAcronym{MC}
{
	short = MC,
	long = monte carlo
}
\DeclareAcronym{TDC}
{
	short = TDC,
	long = time-to-digital converter
}
\DeclareAcronym{DSSD}
{
	short = DSSD,
	long = double-sided silicon strip detector
}
\DeclareAcronym{LE}
{
	short = LE,
	long = leading edge
}
\DeclareAcronym{TE}
{
	short = TE,
	long = trailing edge
}
\DeclareAcronym{RF}
{
	short = RF,
	long = response function
}
\DeclareAcronym{PMT}
{
	short = PMT,
	long = photomultiplier tube
}
\DeclareAcronym{CFD}
{
	short = CFD,
	long = constant fraction discriminator
}
\DeclareAcronym{CVC}
{
	short = CVC,
	long = conserved vector current
}
\DeclareAcronym{CKM}
{
	short = CKM,
	long = Cabibbo--Kobayashi--Maskawa
}
\DeclareAcronym{TRINAT}
{
	short = TRINAT,
	long = TRIUMF Neutral Atom Trap
}
\DeclareAcronym{UHV}
{
	short = UHV,
	long = ultra-high vacuum
}













