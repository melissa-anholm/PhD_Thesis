% !TEX root = ../thesis_main.tex



%%%% --- * --- %%%%	
\clearpage
\chapter*{Long Comments}
\addcontentsline{toc}{chapter}{Long Comments}

\note[gg, nolist]{From Gerald:
\\...\\
Hi Melissa,
\\...\\
Attached is the report by Georg. I contains many valid points, and even for those where I'm not fully on board, it's clear that this has to be addressed. To be clear: This is not really that big of a deal. The actual work is not questioned in any way, and that's of course the really relevant part in the end. But it does have to be framed within a context, especially for the non-experts. I should have probably called off things like the handwritten appendix earlier, but thought that the committee members would ask for that in passing (on the basis that everything in the thesis must stand on its own, which those personal notes arguably don't; if this is relevant, it should be typed up).
\\...\\
I will write to the other committee members and call off their effort on this version, because Georg already won't sign off on it, so there is not point for them to go through it, as they most likely will agree with his points.
\\...\\
The course of action will now be for you to address these issues and then we'll resubmit.
\\...\\
I'm not entirely sold myself on a full "Standard Model intro" in the sense that the beta decay work stands on its own, and can be told as its own story. This is reflected by the fact that JTW predates the SM is is still fully relevant. So I'd go a somewhat light SM exposure. The most important goal is to make the non-experts comfortable in introducing what this is all about.
\\...\\
This will also bolster the citation count, which indeed looks suspiciously low (even though i'm not keen on such bean counting arguments).
\\...\\
Gerald
}


\note[gs, nolist]{From Georg Schreckenbach:
\\
Re: Melissa Anholm, Ph.D. thesis, version April 18, 2022
\\
In response to your request for pre-evaluation of Melissa Anholm’s Ph.D. thesis, dated April 18, 2022, I believe that the thesis in its current form is \emph{not} ready for distribution.
}
\note[gs, nolist]
{
In my view, the following are critical problems:
\begin{enumerate}
    \item 1. The thesis in its current form is completely missing any embedding into a “bigger picture”.
    This would be the task of the Introduction chapter primarily, but should also significantly permeate other parts such as, notably, the conclusions. One could phrase this in the form of questions, including – but certainly not limited to – the following:
    \begin{itemize}
        \item - What is the Standard Model?
        \item - Why are we even searching for “physics beyond the standard model”? Why is there a need or desire to do so?
        \item - What other efforts have been or are being made to search for physics beyond the standard model? Have any of them been successful? If so – or if not, what does this mean?
        \item - Closely related to the previous question, how does the current work fit into this overall effort? What is the unique contribution here?
    \end{itemize}
    Etc.
    \item 2. These aspects should be picked up in the Conclusions. What do the results mean for the “big
    picture”? Again, how do the fit in with the other efforts, prior, concurrent, or planned? Etc.
    \item 3. As an outsider, it is my understanding that the author worked as part of a larger collaboration.
    It would be beneficial to clearly lay out what her unique contributions are. This could be
    done, for instance, in the Introduction, and/or the Conclusions.
    \item 4. Formulas and symbols contained in these need to be defined, as do symbols that appear in the
    text. Currently, this is only done partly. It starts right at the beginning with Eq. 1.1 and continues throughout. Sure, one can of course guess the meaning of the symbols in Eq. 1.1 – but one should not have to guess! – and this is certainly not true for later equations. (For instance, I have no idea about the meaning of most of the symbols in Eq. 1.9, to give but one example.)
    \item 5. Similarly, several figures need better explanations, starting with Fig. 1.1. Which, btw., also needs a source attribution (unless it was created by the author) and, presumably, a copyright note. Similar comments apply to various other figures as well.
\end{enumerate}
...\\
}
\note[gs, nolist]{
...\\
\begin{enumerate}    
	\setcounter{enumi}{5}
    \item 6. At first glance, I find a bibliography that contains only 28 entries to be problematic for a Ph.D. thesis. This goes, to some degree, back to my earlier points about embedding the work in the “bigger picture” – doing so would necessarily lead to several additional references – but it might be more than that.
    \item 7. Appendix C, adding scanned pictures of handwritten notes as figures that, moreover, seem not to be referenced at all, is – well – problematic. I don’t think that it conforms to the instructions as per Supplemental Regulations either (though I did not explicitly check).
\end{enumerate}
}

\note[gs, nolist]{
Given these various issues, I have not fully read the thesis in detail; I have instead focused on the question at hand: what is needed for the thesis to be in a form that can be distributed? I will provide a more formal report on the version of the thesis that I will receive from FGS.
}
\note[gs, nolist]{
Besides, there are some additional points that I would suggest for modification but that do not, in my mind, prevent distribution. These include:

\begin{enumerate}    
	\setcounter{enumi}{7}
    \item 8. The abstract in its current form comprises 186 words or so. The FGS regulations allow for
    350 words maximum; I would encourage the author to make better use of that space – not the
    least in light of my earlier comments regarding the “big picture” and such.
    \item 9. The thesis would benefit from a list of abbreviations.
    \item 10. P. 93 contains a reference to “Figure ??”
    \item 11. Likewise, p. 98, a reference to “Ch. ??”, as well as a reference to “our recent PRL article” –
    the latter needs to be referenced properly!
    \item 12. I have no idea what the title of Appendix B means. It is for sure original and even funny
    though ... Also, the actual appendix title in the thesis is not the same as that listed in the TOC.
\end{enumerate}
}




%
%%%% --- * --- %%%%	
%
\clearpage

\note[white, nolist]{Old Notes!} \noindent

\note[jb1, nolist]{JB on intuitive concepts that are missing (are they *still* missing?!?):  
\\
The SM couples to left-handed neutrinos and right-handed antineutrinos. Since the neutrinos only have weak interactions, there are no right-handed nu's nor left-handed antinu's in nature. The neutrino asymmetry $B_\nu$ is a number with no energy dependence. 
\\
Similarly, the SM weak interaction only couples to right-handed positrons and left-handed electrons. Since these are massive particles, the average helicity of positrons is not 1, but instead v/c. One can always boost to a frame where the positron keeps its circulation but is moving in the opposite direction. This is why the beta asymmetry is A v/c, not just A.
\\
The Fierz term's additional energy dependence of m/E also comes from helicity arguments, stemming from the fact that it still is coupling to SM nu's and antinu's only, so the beta's are generated with wrong handedness.  
\\
The details are built at 4th-year undergrad level in Garcia's paper with his student and postdoc~\cite{hong_sternberg_garcia}.
}
%\note[jb1]{JB on intuitive concepts that are missing:  
%\\
%The beta asymmetry dependence on the Fierz term only comes through the normalization of $W(\theta) = 1 + \bFierz m/E + \Abeta \cos(\theta)$.
%\\ i.e.:
%\\$W'(\theta) = 1 + \Abeta/(1+ \bFierz m/E) \cos(\theta)$.  (the angular distribution must be unity where cos(theta) vanishes, by definition).
%}

.
\note[org, nolist]{I have legit just deleted a section in Ch.6:  "Relation to Other Measurements and New Overall Limits".  It's gone now.  Because it was stupid.  Only John's comments about what it might do in the future remain.}
\note[jb1, nolist]{JB on Ch.~\ref{sec:relation_to_other_measurements}:  "Relation to Other Measurements and New Overall Limits"
\\
6.3 might be better left to a collaboration memo. You have to decide what to do here,
and decide it quickly.
\\...\\
I already gave you the latest Perkeo Fierz paper,
and the info that we are relatively less sensitive to tensor by
by the ratio of Mgt$^2$ = 3/0.6 = 5.
\\...\\
so if you make a linearized exclusion plot of tensor vs. scalar as straight
lines with the uncertainty of bFierz, using Perkeo and your result, you
will see even with 0.09 uncertainty you compete with them on the scalar limit.
\\...\\
(However, see attached figure from my article with Alexandre on 38mK's
beta-nu scalar limit. I think the constraints on Lorentz scalars are hard to
compete with.)
}
\note[jb1, nolist]{JB says:   To put your work in context, please add at the end of that minimal S,T section, or at the end of "Our Decay" section
\\ ... \\
The best existing measurement of $\bFierz$ is in the decay of the neutron~\cite{Saul2020},
$\bFierz$ = 0.017 $\pm$ 0.021, consistent with the Standard Model prediction of zero.
Our measurement is strongly related, yet complementary.
In terms on non-Standard Model Lorentz current structures, to lowest order in the non-SM  currents the same equation applies:
\\
$\bFierz$ = $\pm$ $(C_S +C_S' + (C_T - C_T') \lambda^2)/( 1 + \lambda^2)$
\\
(the plus is for $\beta^-$ decay and the - for $\beta^+$ decay)~\cite{jtw}.  [to be continued...] }
\note[jb1, nolist]{[...continued from prev.]
\\
In our $^{37}$K case, $\lambda^2$ = $|M_{\rm GT}|^2$/$|M_{\rm F}|^2$ is close to 3/5 (the expected value j/(j+1) for a single j=3/2 d3/2 nucleon)~\cite{deShalit1963},
while for the neutron $\lambda^2$ is close to 3 (the expected value for an (j+1)/j j=1/2 s1/2 nucleon).
$|M_F|$, the Fermi matrix element, is nearly the same for both of these isospin = 1/2 decays (the largest correction is the larger isospin mixing of $\sim$0.01 in $^{37}$K).
So our observable is relatively less sensitive to Lorentz tensor currents, and will predominantly constrain or discover Lorentz scalar currents.
\\...\\ 
Full considerations would require a weighted fit of $\bFierz$ experiments and similar observables~\cite{Falkowski2021}, and are beyond the scope of this thesis.
The info from this thesis, values of $\Abeta$ and $\bFierz$ with their uncertainties, can together with the known $fT$ value (lifetime and
branching ratio) allow the community and/or the collaboration to include the results in a future constraint or discovery of scalar and tensor Lorentz currents
contributing to $\beta$ decay.}
\note[org, nolist]{End of John's comments on the section I deleted.}


%
%%%% --- * --- %%%%	
%
\clearpage

