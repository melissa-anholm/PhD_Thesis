% !TEX root = ../thesis_main.tex

%
\chapter{Introduction and Motivation}
\label{IntroductionChapter}
\section{Motivation}
\label{motivation}
Since the magneto-optical trap (MOT) was first described in 1987 by Raab et. al.~\cite{raab}, it has become a standard technique for confining cold samples of neutral atoms.  These cold trapped atoms may subsequently be used in the measurement of a variety of physical quantities.  

\section{Overview}
\label{overview}

%The entirety of Chapter~\ref{ch:MOT} is devoted to presenting a description of the mechanisms involved in a functional (AC- or DC-) MOT, and Chapter~\ref{ch:acmot} describes the additional requirements for an AC-MOT.
%
%We describe our own offline AC-MOT and characterize some of its properties in Chapter~\ref{ch:characteristics}, including a qualitative explanation for the previously noted~\cite{manchester} AC-MOT instability at low frequencies.
%
%In Chapter~\ref{ch:off} we discuss and demonstrate optimal strategies for turning off trapping forces in our AC-MOT.  
%
%Chapter~\ref{ch:online} deals only with optimizing parameters in the online MOT setup.
%
%In Chapter~\ref{ch:obe}, attention is given to quantifying the polarization problems caused by residual magnetic fields.  Beginning with a derivation of the well-known Optical Bloch Equations, we introduce terms into the Hamiltonian to model the effect of a non-zero magnetic field on the polarization of a sample of atoms.  The equations are evolved numerically and the results are discussed.



%Until recently, one limitation of such samples was the necessity for the presence of a relatively large magnetic field, which is expected to partially destroy atomic polarization, limiting the precision of many types of measurements.  Here we discuss the construction of a newer type of MOT, the AC-MOT, which minimizes residual magnetic fields.  
%
%The guys in~\cite{manchester} came up with the idea of the AC-MOT.  They made it work and did some stuff with it.  Good for them.
%
