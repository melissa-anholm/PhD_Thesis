% !TEX root = ../thesis_main.tex


%%%% --- * --- %%%%	
\clearpage
\chapter{The Experimental Setup}
\label{setup_chapter}

%\color{oldcolor}
\section{Overview of the Double MOT System and Duty Cycle}
\label{section:overview}

\note{
We obtain a sample of neutral, cold, nuclear spin-polarized $^{37}\textrm{K}$ atoms with a known spatial position, via the TRIUMF accelerator facility, by intermittently running a magneto-optical trap (MOT) to confine and cool the atoms, then cycling the trap off to polarize the atoms.  With $\beta$ detectors placed opposite each other along the axis of polarization, we are able to directly observe the momenta of $\beta^+$ particles emitted into 1.4\% of the total solid angle nearest this axis.  We also are able to extract a great deal of information about the momentum of the recoiling $^{37\!}$Ar daughters by measuring their times of flight and hit positions on a microchannel plate detector with a delay line.  Because the nuclear polarization is known to within $<0.1\%$~\cite{ben_OP}, and we are able to account for many systematic effects by periodically reversing the polarization and by collecting unpolarized decay data while the atoms are trapped within the MOT, we expect to be well equipped to implement a test of `handedness' within the nuclear weak force.
}


The experimental subject matter of this thesis was conducted at TRIUMF using the apparatus of the TRIUMF Neutral Atom Trap (TRINAT) collaboration.  The TRINAT laboratory offers an experimental set-up which is uniquely suited to precision tests of Standard Model beta decay physics, by virtue of its ability to produce highly localized samples of cold, isotopically pure atoms within an open detector geometry.  \aside[color=org]{Surely most of this paragraph goes in an intro chapter somewhere.}  Although the discussion in this chapter will focus on the methodologies used to collect one particular dataset, taken over approximately 7 days of beamtime in June 2014, the full apparatus and the techniques used are fairly versatile, and can be (and have been) applied to
several related experiments using other isotopes.
\note{Cite a bunch of papers here.}
%\note{mumble mumble 7ish days of beamtime, mumble mumble 2014.}

\begin{figure}[t!h]
	\centering
	\includegraphics[width=.999\linewidth]
	{Figures/doublemot4.pdf}
	\caption{The TRINAT experimental set-up, viewed from above.  The two MOT system reduces background in the detection chamber.  Funnel beams along the atom transfer path keep the atoms focused.}	
	\label{fig:doublemot}
\note{Figure was originally created by Alexandre, modified by ... someone else?  Or Alexandre?  And I got it from ... probably an experimental proposal?  I should figure out how to cite a proposal...}
\end{figure}

The TRINAT lab accepts radioactive ions delivered by the ISAC beamline at TRIUMF.  These ions are collected on the surface of a hot zirconium foil where they are electrically neutralized, and subsequently escape from the foil into the first of two experimental chambers (the ``collection chamber").  Further details on the neutralization process are presented in a previous publication~\cite{gorelov2000}.  Within the collection chamber, atoms of one specific isotope -- for the purposes of this thesis,  \isotope[37]{K} -- are continuously collected into a magneto-optical trap (MOT)\aside[color=org]{defined `MOT' in Ch.3.} from the tail end of the thermal distribution.  Although this procedure preferrentially traps only the slowest atoms, once trapped, atoms will be cooled further as a side-effect of the MOT's trapping mechanism.  The result is a small ($\sim\!1\,$mm diameter), cold ($\sim\!1\,$mK) cloud of atoms of a particular isotope.  
\note{Mumble mumble UHV.  Mumble mumble tail end of the Boltzmann distribution.}

These properties of the atomic cloud allow for a relatively clean transfer of linear momentum from an appropriately tuned laser beam to the atoms within the cloud, and we use this mechanism to ``push'' the atoms out of the collection MOT and into the ``detection chamber'', where they are loaded into a second MOT (see Fig.~\ref{fig:doublemot}).  During regular operation, atoms are transferred approximately once per second.  

There is no need to release previously trapped atoms in the second MOT when a new group of atoms is loaded.  Although the trap loses atoms over time as a result of a variety of physical processes,\aside{discussed ... idk, somewhere else.} during typical operation the majority of atoms loaded in a given transfer will still be trapped at the time the next set of atoms is loaded, and after several transfer cycles, something like a steady state is obtained.

%It should be noted that this loading process does not require atoms already trapped within the second MOT to be released when the next set of atoms is loaded.  Although there are several loss mechanisms 

Because the transfer and trapping mechanisms rely on tuning laser frequencies to specific atomic resonances, these mechanisms act on only a single isotope, and all others remain unaffected.  The result is a significant reduction of background contaminants within the detection chamber relative to initial beamline output.  The transfer methodology is discussed in some detail within another publication~\cite{swanson}.

%this setup allows for the selection of only a single isotope within the detection MOT, and a significantly reduced background relative to the initial beamline output. 

%\note{Probably describe the laser transfer method slightly.}  

%The TRIUMF Neutral Atom Trap (TRINAT) offers an experimental set-up which is uniquely suited to precision tests of Standard Model beta decay physics.  Radioactive ions are delivered from the ISAC beamline and neutralized before being trapped in the first of two magneto-optical traps (MOTs).  Approximately once per second, atoms from the first MOT are transferred to the second, where their decay products can be observed with significantly less background than would have been possible in the first trap (see Figure~\ref{fig:doublemot}).  The transfer methodology is discussed in some detail in a paper by Swanson et al~\cite{swanson}. \aside{The point is that this eliminates background from the decays of other stuff.  Or the same stuff.  Stuff that's not centered at the trap.}

We now turn our attention to what happens to the atom cloud in the detection chamber between loading phases (see Fig.~\ref{fig:dutycycle}).  One of the goals for the 2014 $^{37}\textrm{K}$ beamtime required that the atom cloud must be spin-polarized, as well as being cold and spatially confined.  Although the MOT makes it straightforward to produce a cold and well confined cloud of atoms, it is fundamentally incompatible with techniques to polarize these atoms. The physical reasons behind this are discussed in Section~\ref{section:acmot_and_polarization}.  \aside[color=org]{I *do* discuss this, right?  Right??}

% One goal for the 2014 $^{37}\textrm{K}$ beamtime was to perform a precision measurement of the beta asymmetry, 

\begin{figure}[h!!]
	\centering
	\includegraphics[width=.999\linewidth]
	{Figures/DutyCycle_2014.pdf}
	\caption{The duty cycle used for transferring, cooling, trapping, and optically pumping $\isotope[37]{K}$ during the June 2014 experiment.  Not drawn to scale.  Question marks indicate timings that varied either as a result of electronic jitter or as a result of variable times to execute the control code.  Atoms are transferred during operation of the DC-MOT.  Though the push beam laser itself is only on for $30\,$ms, the bulk of the DC-MOT's operation time afterwards is needed to collect and cool the transferred atoms.  After 100 on/off cycles of optical pumping and the AC-MOT, the DC-MOT resumes and the next group of atoms is transferred in.  After 16 atom transfers, the polarization of the optical pumping laser is flipped to spin-polarize the atoms in the opposite direction, in order to minimize systematic errors.}	
	\label{fig:dutycycle}
\end{figure}

Once the newly transferred set of $^{37}\textrm{K}$ atoms has been collected into the cloud, the entire MOT apparatus cycles 100 times between a state where it is `on' and actively confining atoms, and a state where it is `off' and instead the atoms are spin-polarized by optical pumping while the atom cloud expands ballistically before being re-trapped.  These 100 on/off cycles take a combined total of $488\,$ms.  The laser components of the trap are straightforward to cycle on and off on these timescales, but the magnetic field is much more challenging to cycle in this manner.  

\note[color=org]{How to segue here?  Do I...  Cite myself?  Cite Harvey+Murray?  Reference the next section?  Reference the previous chapter?  This is clunky and stupid.}
%Once the newly transferred atoms have arrived at the second trap, the MOT cycles 100 times between a state where it is `on' and actively confining atoms to a region of approximately 2\,mm$^3$, to a state where it is `off' and instead the atoms are spin-polarized by optical pumping while the atom cloud expands ballistically before being re-trapped.  
Immediately following each set of 100 optical pumping cycles, another set of atoms is transferred in from the collection chamber to the detection chamber, joining the atoms that remain in the trap (see Fig.~\ref{fig:dutycycle}).  The details of the trapping and optical pumping cycles are described further in Section~\ref{section:acmot_and_polarization}, and the optical pumping technique and its results for this beamtime are the subject of a recent publication~\cite{ben_OP}.



%	
%	
%%%% --- * --- %%%%	
\section{The AC-MOT and Polarization Setup}
\label{section:acmot_and_polarization}
%\\*

\note{Probably document things about the waveform and frequency used for the beamtime, since I don't think it's in my MSc.}


As alluded to in the previous section~(\ref{section:overview}), the measurement in question required a spin-polarized sample of atoms, and a precise knowledge of what that polarization was.  This was primarily needed in order to facilitate a measurement of $A_{\mathrm{\beta}}$ \aside[color=org]{have I defined the beta asymmetry yet in some previous chapter?  otherwise I have to do it here..} that was performed on the same data that is the subject of discussion here.~\cite{ben_Abeta}   
%In order to facilitate a measurement of $A_{\mathrm{\beta}}$, great efforts were taken to polarize the atom cloud, and to quantify that polarization.  
% %This resulted in a duty cycle in which the atoms were intermittently trapped in the AC-MOT, then optically pumped to polarize them.  
While this is arguably less critical to a measurement of $b_{\mathrm{Fierz}}$,\aside[color=org]{have I defined $b_{\mathrm{Fierz}}$ yet?} it can still be an asset for eliminating systematic effects.  \aside{Plus, it barely makes sense to talk about measuring $\bFierz$ if you don't know $\Abeta$.}  
We use only the polarized portion of the duty cycle in order to minimize other systematic errors, such as the scintillator energy calibration and overall trap position.  It also makes for a more straightforward interpretation of the relationship of the measured values of $\Abeta$ and $\bFierz$ when the systematic effects are the same for both measurements. Finally, using only polarized data allows us to make use of the `superratio' construction in data analysis, a powerful tool for reducing (many) systematic errors at the expense of statistical precision (see Chapter~\ref{signature_chapter}).
\note[color=org]{I could move this (above) whole paragraph over to Ch.5, where I surely discuss this in more detail.  It provides nice context here though.}

%\note{In order to eliminate systematic effects, the polarization direction is flipped every 16 seconds.}
The Magneto-Optical Trap is a well-known technique from atomic physics, used to confine and cool neutral atoms~\cite{raabprentiss}, and it is also discussed in more detail in Chapter~\ref{atomicphysics_chapter}.\aside[color=org]{Should I move a bunch of this section's content over there, to live in Ch.3??  Should I move a bunch of content from Ch.3 to live here?  Current thinking:  It seems best to split the content up into a general picture of the MOT (Ch.3), and stuff that's unique to this experiment (here). }

%The technique is used predominantly with alkalis due to their simple orbital electron structure, so is appropriate for use with $^{37}\textrm{K}$.  Once set up, it is quite robust, and the trapping force is specific to the isotope for which the trap has been tuned. This feature makes it ideal for use in radioactive decay experiments, since the daughters are unaffected by the trapping forces keeping the parent confined.
%\note[color=org]{I think most of the above paragraph is also written/paraphrased elsewhere.  eg, the chapter on MOTs in general, and the previous section describing our 2-MOT system.  Do I need to move things around?}

%There are two primary components necessary for any MOT:  a laser, and a magnetic field.  The laser, which must be circularly polarized in the appropriate directions and tuned slightly to the red of an atomic resonance, is split into three perpendicular retroreflected beams, doppler cooling the atoms and (with the appropriate magnetic field) confining them in all three dimensions (see Figure~\ref{fig:mot}).  
\note{Removed:  stuff about how a MOT works.  It's in Ch.3.  It lives there now.}
The TRINAT science chamber includes 6 `viewports' specifically designed to be used for the trapping laser (see Fig.~\ref{fig:thechamber}.).

\missingfigure{This is going to need another edge-on G4 picture of the chamber to label all the atomic components.  }

\[\]

\note{Is my photoionization description adequate?  ... in light of John's feedback:  no.}
\note[color=jb]{JB says:  ``Since you worked hard on the logic triggers, a photoion spectrum with duty cycle would be appropriate if you want."}

\note{Need to describe how polarization works.}
\note[color=jb]{JB says:  all polarization details could be deferred to ~\cite{ben_OP}.  (be sure to list all authors including [me]).  )}

A MOT also requires a quadrupolar magnetic field, which we generate with two current-carrying anti-Helmholtz coils located within the vacuum chamber itself.  The coils themselves are hollow, and are cooled continuously by pumping temperature-controlled water through them.   

One feature which makes our MOT unusual has been developed as a result of our need to rapidly cycle the MOT on and off -- that is, it is an ``AC-MOT''.  Rather than running the trap with one particular magnetic field and one set of laser polarizations to match, we run a sinusoidal AC current in the magnetic field coils, and so the sign and magnitude of the magnetic field alternate smoothly between two extrema, and the trapping laser polarizations are rapidly swapped to remain in sync with the field~\cite{harveymurray}\cite{thesis}.  See Figure~\ref{fig:acmot}.  

\note{Note that because the atoms within a MOT can be treated as following a thermal distribution, some fraction of the fastest atoms continuously escape from the trap's potential well.  Even with the most carefully-tuned apparatus, the AC-MOT cannot quite match a similar standard MOT in terms of retaining atoms.  The TRINAT AC-MOT has a `trapping half-life' of around 6 seconds, and although that may not be particularly impressive by the standards of other MOTs, it is more than adequate for our purposes.  $^{37}\textrm{K}$ itself has a radioactive half-life of only 1.6 seconds 
(cite someone), so our dominant loss mechanism is radioactive decay rather than thermal escape. }

\note[color=lgrey]{Anyway, here's some figures.  Or possibly one figure.  Whatever.  Also, here's a reference to a figure.  See Fig.~\ref*{fig:themot} (works -- currently ``3.4"), or also its subfigures, eg Fig.~\ref{fig:acmot} (works -- currently ``3.4b") and Fig.~\ref{fig:mot} (works -- currently ``3.4a").  Maybe I have to subref them?  Like, eg, Subfig.~\subref{fig:acmot} (works -- currently ``3.4b") and Subfig.~\subref*{fig:mot} (works -- currently ``3.4a").  What if we try to subref everything?  Consider, eg, Fig.~\subref{fig:themot} (doesn't work).  Yeah, ok, so fortunately the note cites like the text.  This gives an example of shit to do and not to do.  Also, can't do a linebreak within a note.}  

%% !TEX root = ../thesis_main.tex

% fig:themot
% 	fig:mot
% 	fig:acmot

\begin{figure}[ht]
	\centering
	\begin{subfigure}[t]{0.242\textwidth}
		\centering
		\includegraphics[width=\textwidth]{mot.png}
		\caption{Components of a magneto-optical trap, including current-carrying magnetic field coils and counterpropagating circularly polarized laser beams.}
		\label{fig:mot}
	\end{subfigure}
	\hfill
	\begin{subfigure}[t]{0.728\textwidth}
		\centering
		\includegraphics[width=\textwidth]{acmot.png}
		\caption{One cycle of trapping with the AC-MOT, followed by optical pumping to spin-polarize the atoms.  After atoms are transferred into the science chamber, this cycle is repeated 500 times before the next transfer.  The magnetic dipole field is created by running parallel (rather than anti-parallel as is needed for the MOT) currents through the two coils.}
		\label{fig:acmot}
	\end{subfigure}
	\caption{An alternating-current magneto-optical trap with a duty cycle optimized for producing polarized atoms}	
	\label{fig:themot}
\end{figure}

\begin{figure}[ht]
	\centering
		\includegraphics[width=.999\linewidth]{acmot.png}
		\caption{One cycle of trapping with the AC-MOT, followed by optical pumping to spin-polarize the atoms.  After atoms are transferred into the science chamber, this cycle is repeated 100 times before the next transfer.  The magnetic dipole field is created by running parallel (rather than anti-parallel as is needed for the MOT) currents through the two coils.}
		\label{fig:acmot}
\end{figure}


We spin-polarize $^{37}\textrm{K}$ atoms within the trapping region by optical pumping~\cite{ben_OP}.  A circularly polarized laser is tuned to match the relevant atomic resonances, and is directed through the trapping region along the vertical axis in both directions.  When a photon is absorbed by an atom, the atom transitions to an excited state and its total angular momentum (electron spin + orbital + nuclear spin) along the vertical axis is incremented by one unit.  When the atom is de-excited a photon is emitted isotropically, 
%\comment{(is it still isotropic when it's polarized?  I bet it's not.)}
so it follows that if there are available states of higher and lower angular momentum, the \emph{average} change in the angular momentum projection is zero.  If the atom is not yet spin-polarized, it can absorb and re-emit another photon, following a biased random walk towards complete polarization.  

%\missingfigure{Need a picture of the whole duty cycle.  Possibly combine with ~\ref{fig:acmot}.}


In order to optimally polarize a sample of atoms by this method, it is necessary to have precise control over the magnetic field.  This is because absent other forces, a spin will undergo Larmor precession about the magnetic field lines.  In particular, the magnetic field must be aligned along the polarization axis (otherwise the tendency will be to actually depolarize the atoms), and it must be uniform in magnitude over the region of interest (otherwise its divergencelessness will result in the field also having a non-uniform direction, which results in a spatially-dependent depolarization mechanism).  Note that this type of magnetic field is not compatible with the MOT, which requires a linear magnetic field gradient in all directions (characteristic of a quadrupolar field shape), and has necessitated our use of the AC-MOT as described in Section~\ref{section:acmot_and_polarization}.\aside{that's this section.  I should really describe the AC-MOT.}



\section{Microchannel Plates and Electric Field}
\label{section:mcps}
MCPs.  Hoops.  Only one thing works at a time!  Blarg.  Upon decay, atoms literally aren't trapped anymore by the trap.  No trapping forces, no slowing forces, because it's all isotope-specific.
\missingfigure{Back-to-back MCPs in an electric field to tag events from the trap, and to measure the trap position and polarization.  Hoops to produce the electric field.}

%\color{black}
%	\subsection{\textbf{Nuclear Setup}}
\section{Measurement Geometry and Detectors}
\label{section:betadetectors}
\note{This section is disorganized and repeats itself.}
\comment{
%\note{Do I want to make an entirely new section for the MCPs and electric field stuff?  It doesn't seem to quite fit in either of the two sections here...}
	Needs several diagrams.  %Back-to-back beta detectors along the polarization axis.  Back-to-back MCPs in an electric field to tag events from the trap, and to measure the trap position and polarization.  Hoops to produce the electric field.  
	Many laser ports to make the MOT functional, and for optical pumping.  Fancy mirror geometry to combine optical pumping and trapping light along the vertical axis.  Water-cooled (anti-)Helmholz coils within the chamber for the AC-MOT, fast switching to produce an optical pumping field.  
%	\subsection{\textbf{All the Detectors}}
}

% !TEX root = ../thesis_main.tex


% "fig:thechamber"

\begin{figure}[h!!!tb]
	\centering
%	\hspace*{\fill}%
	\subfloat[A decay event within the TRINAT science chamber.  After a decay, the daughter will be unaffected by forces from the MOT.  Positively charged recoils and negatively charged shake-off electrons are pulled towards detectors in opposite directions.  Although the $\beta^+$ is charged, it is also highly relativistic and escapes the electric field with minimal perturbation.
	%\comment{The pic is still kind-of fuzzy.}
	]
	{\includegraphics[width=.530\linewidth]{Figures/chamber_decayevent3.png}\label{chamber_decayevent} }
	\hspace*{\fill}
%	\hfill
	\hspace*{\fill}
	\subfloat[Inside the TRINAT science chamber.  This photo is taken from the vantage point of one of the microchannel plates, looking into the chamber towards the second microchannel plate.  The current-carrying copper Helmholtz coils and two beta telescopes are visible at the top and bottom.  The metallic piece near the center is one of the electrostatic `hoops' used to generate an electric field within the chamber.  The hoop's central circular hole allows access to the microchannel plate, and the two elongated holes on the sides allow the MOT's trapping lasers to pass unimpeded at an angle of 45 degress `out of the page'.]	
	{\includegraphics[width=.444\linewidth]{Figures/chamber_photo_2.png}}
%	\hspace*{\fill}%
	\caption{The TRINAT detection chamber.}	
	\label{fig:thechamber}
\note{Mirrors are 275~$\mu$m thick, not the 254~$\mu$m shown in picture.}
\end{figure}


% "fig:thechamber"

%\note{Below is pretty vague.  I could do better, even for just an overview-summary thing.  Obvs I have to describe it in detail later on *somewhere*, though maybe not in the overview-summary...} 
Detectors are positioned about the second MOT for data collection.  The detection chamber 
%(shown in Figure \ref{fig:thechamber}) 
operates at ultra-high vacuum (UHV) and provides not only the apparatus necessary to intermittently confine and then spin-polarize atoms, but also the variety of detectors and implements required to quantify their position, temperature, and polarization.  The detection chamber further boasts an array of electrostatic hoops to collect both positively and negatively charged low energy particles into two microchannel plates (MCPs),  and a further set of two beta detectors positioned along the polarization axis, each of which consists of a 40x40 pixel double-sided silicon strip detector (DSSD) and a scintillator and photomultiplier tube (PMT).  %The details of the detection chamber setup are described in detail in Section~\ref{section:betadetectors}. 
%\note{ ...where I basically repeat this same content.  Blarg.}




\note{...(shown in Figure \ref{fig:thechamber}) ...}
The beta detectors, located above and below the atom cloud along the axis of polarization (see Figure~\ref{chamber_decayevent}), are each the combination of a plastic scintillator and a set of silicon strip detectors.  Using all of the available information, these detectors are able to reconstruct the energy of an incident beta, as well as its hit position, and provide a timestamp for the hit's arrival.  Together the upper and lower beta detectors subtend approximately 1.4\% of the total solid angle as measured with respect to the cloud position. 

%\section{Beta Detectors}
	The two sets of beta detectors were positioned directly along the axis of polarization.  Each beta detector consists of a plastic scintillator and photo-multiplier tube (PMT) \aside{There's gotta be a better way to describe it} placed directly behind a 40$\times$40-pixel double-sided silicon strip detector (DSSD).  \aside{what's the open area of the detector?  how big is each pixel?}  The scintillator is used to measure the overall energy of the incoming particles, as well as to assign a timestamp to these events, while the DSSD is used both to localize the hit position to one (or in some cases, two) individual pixel(s), and also to discriminate between different types of incoming particles.  In particular, though the scintillator will measure the energy of an incoming beta or an incoming gamma with similar efficiency, the beta will lose a portion of its kinetic energy as it passes through the DSSD into the scintillator.  By contrast, an incident gamma will deposit only a very small amount of energy in the DSSD layer, making it possible to reject events with insufficient energy deposited in the DSSD as likely gamma ray events.  Given that the decay of interest to us emits positrons, we expect a persistent background 511 keV gamma rays that are not of interest to us, so it is extremely important that we are able to clean these background events from our spectrum. 


It must be noted that the path between the cloud of trapped atoms and either beta detector is blocked by two objects:  a 275$\,\mu$m silicon carbide mirror (necessary for both trapping and optical pumping), and a 229$\,\mu$m beryllium foil (separating the UHV vacuum within the chamber from the outside world).  In order to minimize beta scattering and energy attenuation, these objects have had their materials selected to use the lightest nuclei with the desired material properties, and have been manufactured to be as thin as possible without compromising the experiment.  As the $^{37}\textrm{K} \rightarrow \,^{37}\textrm{\!Ar} + \beta^{+} + \nu_e$ decay proceess releases $Q=5.125$\,MeV of kinetic energy~\cite{Q_value}, the great majority of betas are energetic enough to punch through both obstacles without significant energy loss before being collected by the beta detectors.  

On opposing sides of the chamber, and perpendicular to the axis of polarization, two stacks of $\sim$ 80\,mm diameter microchannel plates (MCPs) have been placed (see Figure~\ref{fig:thechamber}) as detectors, providing a time stamp when a particle is incident on their surfaces.  Behind each stack of MCPs there is a set of delay lines, which provide  position sensitivity for these detectors.   

In order to make best use of these MCPs, we create an electric field in order to draw positively charged particles into one MCP, while drawing negatively charged electrons into the other MCP.  Seven electrostatic hoops have been placed within the chamber (see Figure~\ref{fig:thechamber}), and are connected to a series of high voltage power supplies.  See Sections~\ref{photoions} and~\ref{pos_recoils} for a discussion of what sort of charged particles we expect to observe in these detectors and how they are created.  
  
Scientific data has been collected at field strengths of 395 V/cm, 415 V/cm, and 535 V/cm.  It should be noted that these field strengths are too low to significantly perturb any but the least energetic of the (positively charged) betas from the decay process, and these low energy betas would already have been unable to reach the upper and lower beta detectors due to interactions with materials in the SiC mirror and Be foil vacuum seal.  

