% !TEX root = ../thesis_main.tex


%%%% --- * --- %%%%
\clearpage	

\chapter{The Experimental Signature}
\label{analysis_chapter}

\section{The Superratio and Asymmetry}
%\\*
The data can be combined into a superratio asymmetry.  This has the benefit of causing many systematics to cancel themselves out at leading order.  It also will increase the fractional size of the effects we're looking for.  This can be shown by using math.  

\section{Signature of a Fierz Term in This Experiment}
%\\*
Not all systematics effects are eliminated.  We'll want to be careful to propagate through any effects that are relevant.  Using the superratio asymmetry as our physical observable makes this process a bit messier for the things that don't cancel out, but it's all just math.  

\section{Comparative Merits of the Superratio and Supersum for Measurement}
%\\*
Some other groups have performed similar measurements using the supersum as the physical observable.  There are pros and cons to both methods.  I can show, using a back-of-the-envelope calculation, that for this particular dataset, the superratio asymmetry method produces a better result.  

