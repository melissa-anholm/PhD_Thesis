% !TEX root = ../thesis_main.tex



%%%% --- * --- %%%%	
\clearpage
\chapter*{Long Comments}
\section{Round 0 and Earlier}
\addcontentsline{toc}{chapter}{Long Comments}

%%%%%%%%%%%%%%%%%%%%%%%%%%%


%%%% --- * --- %%%%	
%%%%%%%%%%%%%%%%%%%
\FloatBarrier
%\clearpage

\note[white, nolist]{Old Notes!} \noindent

\note[jb1, nolist]{JB on intuitive concepts that are missing (are they *still* missing?!?):  }
\note[jb1, nolist]{
The SM couples to left-handed neutrinos and right-handed antineutrinos. Since the neutrinos only have weak interactions, there are no right-handed nu's nor left-handed antinu's in nature. The neutrino asymmetry $B_\nu$ is a number with no energy dependence. 
\\...\\
Similarly, the SM weak interaction only couples to right-handed positrons and left-handed electrons. Since these are massive particles, the average helicity of positrons is not 1, but instead v/c. One can always boost to a frame where the positron keeps its circulation but is moving in the opposite direction. This is why the beta asymmetry is A v/c, not just A.
}
\note[jb1, nolist]{
The Fierz term's additional energy dependence of m/E also comes from helicity arguments, stemming from the fact that it still is coupling to SM nu's and antinu's only, so the beta's are generated with wrong handedness.  
\\...\\
The details are built at 4th-year undergrad level in Garcia's paper with his student and postdoc~\cite{hong_sternberg_garcia}.
}
\note[jb1]{JB on intuitive concepts that are missing:  
\\
The beta asymmetry dependence on the Fierz term only comes through the normalization of $W(\theta) = 1 + \bFierz m/E + \Abeta \cos(\theta)$.
\\ i.e.:
\\$W'(\theta) = 1 + \Abeta/(1+ \bFierz m/E) \cos(\theta)$.  (the angular distribution must be unity where cos(theta) vanishes, by definition).
}

%
%%%% --- * --- %%%%	
\clearpage
\section{Stuff that Remains after Round 0}

\subsection{Ch. 1}


\note[orange, nolist]{
\begin{itemize}
	\item (we're not proposing a new feynman diagram.  Just, a new vertex factor thingy, I guess.)
	\item Weak force lagrangian!  I think possibly it's only for the *charged* weak force??
	\item ...then we demand that the Lagrangian must be Lorentz invariant, otherwise we break physics.
	\item within this thesis, we will focus primarily on the \emph{charged} weak interactions (using $W^{\pm}$ as mediators rather than $Z^0$), with first-generation normal matter quarks (\emph{i.e.} up ($u$) and down ($d$) quarks, but not charm ($c$), strange ($s$), top ($t$), or bottom ($b$) quarks; no anti-quarks ($\bar{q}$ for $q=\{u,d,c,s,t,b \}$ )) and first-generation leptons (\emph{i.e.} electrons, positrons, electron neutrinos and electron anti-neutrinos, but not positive or negative muons or taus, nor muon or tau neutrinos or anti-neutrinos).
	\item Really, we'll only be looking at weak force interactions involving the two most common nucleons -- protons and neutrons.  ...each of which is a (meta-)stable bound state involving three quarks.  
	\item masses of W and Z bosons?  
\end{itemize}
}


%\subsection{bullet points.}
%%Beta Decay:
%%\begin{itemize}
%%\item 1931 - Pauli "discovers" neutrinos and is very confused.
%%\item 1934 - Fermi's Golden Rule to describe the rate of beta decay in terms of transition matrix elements.
%%\item ...
%%\item 1956 - Lee and Yang wonder if parity is even conserved in beta decay?  
%%\item 1957 - C.S. Wu shows that parity isn't conserved.  At all.
%%\item ???? - turns out it's V-A.
%%\end{itemize}
%Standard Model:
%\begin{itemize}
%\item ???? - Schwinger 
%\item 1959 - Glashow
%\item 1964 - Salam and Ward
%\item 1967 - Weinberg
%\item 1971 - 't Hooft
%\item ...
%\item 1954 - Yang + Mills  .... huh?
%\end{itemize}
%%Radioactivity:
%%\begin{itemize}
%%\item 1896 - Henry Bequerel in Uranium.  Presumably that's alpha decay.
%%\item 1898 - Thorium is radioactive -- Marie Curie + Gerhard Carl Schmidt, independently.
%%\item 	 - Curie:  Thorium, polonium, radium -- all radioactive!
%%\item 1899 - Rutherford notices that there's two types of emissions:  alpha and beta (beta minus).
%%\item 1900 - Paul Villard discovers gamma rays, doesn't realize they're different?  Or something?
%%\item 1903 - Rutherford realizes that gamma rays are a fundamentally new thing.
%%\item ...
%%\item 1900 - Becquerel realizes beta particles have the same charge-to-mass ratio as electrons, postulates that they're the same thing.
%%\item 1901 - Rutherford + Soddy show that alpha and beta decay turn elements into different elements.  
%%\item ...
%%\item 1911 - Lise Meitner + Otto Hahn:  emitted beta particles come out at many different energies.  Unlike alpha and gamma decays.
%%\item 1914 - James Chadwick shows that it's a continuous distribution.
%%...
%%\item 1930 - Pauli proposes neutrinos (but he called them "neutrons") as a solution to the missing energy problem.
%%\item 1931 - Fermi calls them neutrinos.  
%%\item 1933 - Fermi's landmark theory?  Surely it's 1934.
%%...
%%\item 1934 - Irène and Frédéric Joliot-Curie discover (induced) beta+ decay.  it got 'em a nobel prize.
%%\item 1934 - Wick proposes electron capture as a thing.  Yukawa and others maybe have thoughts about that too.
%%\item 1937 - Luis Alvarez observes electron capture.
%%\item ???? - Parity?  Coupling constants??
%%\end{itemize}


\FloatBarrier
%%%% --- * --- %%%%	
\subsubsection{Beta Decay within the Standard Model -- a section that no longer exists.}

~\aside[jbn]{Eqs. ~\ref{eq:betaplus_decay} and ~\ref{eq:electroncapture} are energetically disallowed for an unbound proton, but allowed energetically when bound in nuclei as in 37K decay.}
\note[jbn]{Electron capture decay of Eq. 1.3 is calculated to be an 0.080\% branch compared to positron decay in 37K decay (Table VII of N. Severijns, M. Tandecki, T. Phalet, and I.S. Towner)~\cite{SeverijnsTandecki2008}, an important correction when interpreting the total decay rate of 37K to determine the theoretical prediction for 37K Abeta  (P.D. Shidling, R.S. Behling, B. Fenker, J.C. Hardy, V.E. Iacob, M. Mehlman, H.I. Park, B.T. Roeder, D. Melconian Phys Rev C 98 015502 (2018)~\cite{shidling2018}; A. Ozmetin, D.G. Melconian, V.E. Iacob, P. Shidling, V.S. Kolhinen, D.J. McClain, M. Nasser, B. Schroeder, B. Roeder, H. Park, M. Anholm, A.J. Saastamoinen, APS DNP Abstract KF.00005 2020 ``Improving the ft value of 37K via a precision measurement of the branching ratio'')~\cite{ozmetin2020}.
}
Limiting the focus of this discussion to Eq.~\ref{eq:betaplus_decay}, we note that this expression provides no information at all about the momenta or spin of the outgoing daughter particles.  This behaviour is governed by the form of the Weak coupling that mediates the decay.  

Within the field of nuclear physics, it is common to classify beta decay processes as being either ``Allowed'' or ``Forbidden''~\aside[jbn]{don't capatalize "Allowed" and "Forbidden."
The quotes are justified.} (sometimes with an associated number to describe the extent to which it is Forbidden), where Forbidden processes are generally suppressed but not truly forbidden.  In an Allowed transition, the positron and anti-neutrino are treated as being created at the nuclear centre, and as a result they may not carry away any \emph{orbital} angular momentum.  However, since the outgoing leptons both have spin $S=1/2$, it is still possible for the total nuclear angular momentum, $J$, to be changed in an Allowed decay.  This implies that an Allowed transition must \emph{always} change the total nuclear angular momentum by either $0$ or $\pm1$.  
~\note[jbn]{First forbidden decay emits leptons with total orbital angular momentum 1, changing the nuclear parity-- suppressed in a long-wavelength approximation ~ ((beta momentum)/(hbar c))$^2$ or about two orders of magnitude. This is one reason decays to negative-parity excited states of 37K are so small (Fig.~\ref{fig:nuclearleveldiagram}).
}

The Allowed decays traditionally are further separated into a Fermi singlet~\aside[jbn]{singlet in lepton spin} in which there is no change to nuclear angular momentum ($\Delta J = 0$) and therefore the two leptons are required to have anti-parallel spins, and a Gamow-Teller triplet~\cite{severijns_beck_cuncic_2006}, where the projection of the nuclear angular momentum is changed by $\pm1$.%, and so the two lepton spins must be aligned in parallel to one another under the Allowed approximation.
\note[done]{My only comment is to ask where the Fermi "singlet" and "triplet" came from, referring to coupling of the lepton spins.
\\
I have not seen this-- it could be fine, yes, I don't know.}

This implies that the total nuclear angular momentum is changed by $\Delta J = \{0, \pm1\}$ during a Gamow-Teller transition.  A mixed transition is also possible, however we note that the $J_i = J_f = 0$ decays must always be pure Fermi transitions, because there is no way to produce this result from two outgoing leptons with parallel spins.~\cite{krane}~\cite{wong1990}~\cite{severijns_beck_cuncic_2006}.

Given the differing behaviour within the angular momenta of the daughters in Fermi and Gamow-Teller transitions, it is perhaps not suprising that that the \emph{linear} momenta of the outgoing particles should also follow a different set of distributions in these two cases.  At the level of the Weak~\aside[done,nolist]{Don't capitalize Weak anywhere, either. You can keep quotes around the first "weak", I guess.} coupling, Fermi- and Gamow-Teller~\aside[done,nolist]{you should indeed capitalize Fermi and Gamow-Teller because these are people.} transitions are governed by different operators, with the Fermi interaction mediated by a so-called ``vector'' ($V$) coupling, and the Gamow-Teller interaction mediated by an ``axial-vector'' ($A$) coupling.
~\note[jbn]{The vector and axial-vector couplings refer to Lorentz transformation of the Lagrangian terms involved, which we come to next.}



%%%% --- * --- %%%%	

%%%%\note{btw:  as of 2018, we have $m_W = 80.379(12)$\,GeV/$c^2$ ~\cite{pdg2018}.}

%%%% --- * --- %%%%	

%%%


\subsubsection{Draft of Intro Section -- with paragraphs!}


\note[jb1]{JB:  The Gamow-Teller operator sigma dot tau refers to nucleon spins, not lepton spins, i.e. the nucleon spin can flip, but that doesn't tell me about this lepton rule either way.
You correctly state the lepton and antilepton chirality farther down.
}

%%%%\note[done,nolist]{John's suggestions for fixing the stuff that Georg wanted fixed -- that giant email.
%%%%i) Among the necessary technical corrections, there's a solid request for more background info. I think you know what to do.
%%%%\\...\\
%%%%A large thing is some kind of extra qualitative description of what the SM is. You could do that qualitatively without any trouble.
%%%%}
%%%%\note[done, nolist]{
%%%%You might point out some part of this:
%%%%\begin{itemize}
%%%%	\item the charged weak interaction you're writing down predates the Weinberg-Salam model by more than a decade. The version you've written down assumes protons and neutrons are fundamental particles.
%%%%	\item The exchange boson is much heavier in mass than the energy and momentum in the decay, so it can be approximated by an interaction with zero range. 
%%%%	\begin{itemize}
%%%%		\item Fermi did that very early on, 
%%%%		\item and Gamow and Teller added a process that change the nucleon spin. \cite{GamowTeller}
%%%%		\item Lee and Yang, which you cite, added the possible currents with different Lorentz transformations (do you mention that since any further combination of Dirac matrices can be reduced to these, so they span the space), and the possibility of parity violation by writing out helicity projections for the leptons. I.e. Lee and Yang assumed some fields (the nucleons and leptons) and a general interaction preserving the symmetries of the theory, which by definition is then an effective field theory. \cite{LeeYang}
%%%%	\end{itemize}
%%%%\end{itemize}
%%%%}
%%%%\note[done, nolist]{(...I might point out some part of this:)
%%%%\begin{itemize}
%%%%	\item Feynman and Gell-Mann's 1958 paper is the one that postulated V-A, again a decade before Weinberg-Salam, and they did this by making analogies between the boson exchanged and the photon, i.e. an analogy between the charged weak interaction and the electromagnetic current, so you could cite them instead of saying "SM predicts everything" \cite{FeynmanGellMann1958} 
%%%%\end{itemize}
%%%%People had assumed there was a massive boson exchanged for a very long time.
%%%%\\
%%%%What Weinberg (and, independently, Salam) did was come up with a consistent mathematical theory of massive bosons, incorporating Yang Mills gauge theory (looks like E\&M but with nonabelian operators) to do that, and the result was the weak neutral interaction prediction.
%%%%}

\note[oldnote,nolist]{A paragraph that used to live in my "draft of the intro" section:
\\...\\
There exists an extensive body of experimental evidence to demonstrate that the above model is overall a very good description of the beta decay process~\cite{wu}.  Despite the success of the $(V-A)$ model, there are still certain lingering questions that must be addressed by precision measurements.  Any deviation from maximal parity violation (i.e., a ``$(V+A)$'' contribution to the Weak force) would be of great interest to the community, as would the presence of certain other exotic couplings, such as the so-called Scalar ($S$) and Tensor ($T$) interactions.  Any such behaviour beyond the Standard Model (BSM) would represent a non-dominant contribution to the interaction, however the possibility cannot be entirely ruled out.  
}

\note[jbn]{(Any quark-lepton pseudoscalar couplings have usually been ignored in beta decay, because they are suppressed by (beta momentum)/(nucleon mass).  Note that more recently it's been pointed out that C\_P is naturally quite large in the nucleon (M. Gonzalez-Alonso and J. Martin Camalich Phys Rev Lett 112 042501 (2014)) and allows for significant constraints from allowed beta decay.)}



%%%% --- * --- %%%%	
%
%
%
%
%
%
%
%%%% --- * --- %%%%	



%%%% --- * --- %%%%	



\subsubsection{....whatever comes after the draft-only subsections}

%%%%%%% --- * --- %%%%	

%%	\note[orange, nolist]{Authors: John Cameron, Jun Chen and Balraj Singh, Ninel Nica.
%%	\\
%%	Citation:Nuclear Data Sheets 113, 365 (2012) \cite{nucleardata2012}.
%%	\\...\\
%%	Also, it's generated by the page from:  National Nuclear Data Center (NNDC) at Brookhaven National Laboratory.  Possibly from the NuDat3 database.
%%%	https://www.nndc.bnl.gov/nudat3/decaysearchdirect.jsp?nuc=37K&unc=NDS
%%	}
%	\caption[New 37K decay scheme]{A newer, better, less generated-by-Dan level diagram for the decay of $\isotope[37]{K}$.  \cite{nucleardata2012} \cite{ChinPhysC2012}. (probably only use the first citation?) }

%\\
%Also:  describe what "I(\%)" is, or maybe just re-do the label.



\subsection{Ch. 2}

%The TRINAT lab has adopted this technique from atomic physics to perform nuclear physics experiments.  


\note{
When this is combined with a circularly polarized laser beam, the effect is to move the atomic resonance closer to- or farther from- the frequency of the laser.  The circular polarization, combined with some selection rules, means a circularly polarized laser will only couple to one particular transition, w.r.t. angular momentum.  ie, for a $\sigma_+$ polarized laser, the atom's overall angular momentum projection (along some axis) will be incremented by $+1$.  The Zeeman shift means that in a magnetic field, this transition (M+=1) not be the same as the M-=1 transition.  So, if you have a magnetic field that changes linearly across space, you can make it so that in $+ B_z$ regions, the laser beam with one certain polarization is closer to resonance and therefore more likely to be absorbed -- and similarly, in $- B_z$ regions, a different laser with the opposite polarization will be more likely to be absorbed.  Again, if the B-field is linear in space, you can do it so that as the atoms get further and further from the `centre' region, the effect gets progressively stronger.  So, if you've done this right, you can make it so that the atoms get a stronger ``push'' back towards the center the farther away they've drifted.
\\
They still get the optical molasses cooling effect for free.}

%%%% % % % %%%

\note[done, nolist]{  JB:  \\
*It would be simplest to write down what you mean by a quadrupole field. You could say: "approximately in our geometry near the trap center
\\
$\vec{B} = 2 B_o z \hat{z} - (B_0 x \hat{x} + B_0 y \hat{y})$
}

\note{Upon decay, atoms literally aren't trapped anymore by the trap.  No trapping forces, no slowing forces, because it's all isotope-specific.  This is super useful for us.}



\note[done, nolist]{Direct quote from John below:  
\\...\\
The optical pumping process is described in detail within our collaboration's Ref.~\cite{ben_OP}. The primary detail described here is that the optical pumping is disturbed by any component of magnetic field not along the quantization axis. (Ours is the vertical axis, defined by the direction of the optical pumping light, and along which the detectors are placed.) This required sophistication with an AC-MOT described in Section~\ref{sec:acmot} below.
}

\note[jb1]{JB says:  ``I would say you don't need an atomic level diagram.  You could just describe in words the semiclassical picture of atoms absorbing photons until they are nearly fully polarized, then they stop absorbing.  The optical pumping + photoionization is then an in situ probe of the polarization. ... You would need to add in words that quantum mechanical corrections to this picture are in the optical Bloch equation approach in B. Fenker et al.  The depolarized states still have high nuclear polarization (1/2 for $F=2, M_F=1$, 5/6 for $F=1, M_F=1$) and determining the ratio of those two populations provides most of the info we need -- we model with the O.B.E, measure the optical pumping light polarization, and float an average transverse magnetic field.  This is adequate to determine the depolarized fraction to 10\% accuracy, which is all that is needed.'' }


\note[oldnote, nolist]{
We obtain a sample of neutral, cold, nuclear spin-polarized $^{37}\textrm{K}$ atoms with a known spatial position, via the TRIUMF accelerator facility, by intermittently running a magneto-optical trap (MOT) to confine and cool the atoms, then cycling the trap off to polarize the atoms.  With $\beta$ detectors placed opposite each other along the axis of polarization, we are able to directly observe the momenta of $\beta^+$ particles emitted into 1.4\% of the total solid angle nearest this axis.  We also are able to extract a great deal of information about the momentum of the recoiling $^{37\!}$Ar daughters by measuring their times of flight and hit positions on a microchannel plate detector with a delay line.  Because the nuclear polarization is known to within $<0.1\%$~\cite{ben_OP}, and we are able to account for many systematic effects by periodically reversing the polarization and by collecting unpolarized decay data while the atoms are trapped within the MOT, we expect to be well equipped to implement a test of `handedness' within the nuclear weak force.
}

%	The two MOT system reduces background in the detection chamber.  Funnel beams along the atom transfer path keep the atoms focused.}	

%\note{Mumble mumble UHV.  Mumble mumble tail end of the Boltzmann distribution.}

%\note[done, nolist]{First, in Section 2.4 the sentence after Harvey and Murray are referenced could be "There are some details of the present implementation of the AC MOT in Ref.~\cite{thesis}, done with a separate MOT geometry from
%this beta decay work."
%\\
%(Presently your thesis is Ref. 14, which is only cited so far for Fig. 2.1)
%}
\note[done, nolist]{Chapter 2 corrections from John:  "all of these are trivial...."}
\note[jb1]{
*section 2.1.3 "using a Helmoltz coil" add "but with currents in opposite direction." (saying it correctly here then implicitly defines anti-Helmholtz which you use correctly later for the AC MOT)
}

%\note[done, nolist]{*2.5.1 "roughly" -> "approximately" The Efield is not rough at all, it's pretty good really. }
%\note[done, nolist]{Is the rMCP really just a 2-plate chevron in 2014? I don't remember. Why can't I find Ben's thesis?}
%
%\note[done, nolist]{a glaring omission:
%\\
%\textbf{You don't state the polarization achieved 99.1$\pm$0.1 percent anywhere that I can see., citing the publication.} You might also state that's important for Abeta but has much more precision than needed for the near-zero bFierz term.
%\\...\\
%You mention the polarization is different by approximately 0.3\% in Section 6.1 from the cut, but never say what P is.
%\\...\\
%Appendix A has ?? for a discussion of the changes of the polarization cut, so you need to clean up that \ref{}. It is presently in Section 6.1.
%}


% 
%\note[done, nolist]{*you need a subsection in Section 2.4 "The AC-MOT and Polarization" on your field trimming.}
%\note[done, nolist]{quote from JB:
%\\...\\
%Time-constant ambient fields were trimmed in all dimensions using two horizontal pairs of Helmholtz coils and the AC MOT coils for the vertical direction.
%These ambient B fields were first trimmed to be near zero using a giant magnetoresistance 3-axis probe at the trap center, with the MCP assembly removed and the vacuum chamber up to air.
%\\...\\
%Final trimming was done by optimizing polarization of stable $^{41}$K atoms with the apparatus assembled (cite[Fenker PRL Suppl Mat] has some details).
%\\...\\
%The AC MOT and polarization B fields were generated by SRS DS345 arbitrary waveform generators, amplified by
%Matsusada bipolar 20 KHz bandwidth 80A 20V amplifiers. The waveforms were carefully trimmed in amplitude and time to minimize B fields from eddy currents during the optical pumping cycle, again with the MCP assembly removed and the vacuum chamber up to air.
%\\...\\
%The bipolar amplifier bandwidth was inadequate when using current control mode, so voltage control had to be used, a much more time-exacting process requiring empirical iteration. These waveforms are not the same for the top and bottom coils, as during the optical pumping cycle one coil had to be flipped with respect to the MOT cycle to create the uniform vertical field for optical pumping by a Helmholtz rather than antiHelmholtz configuration.
%\\...\\
%The beta detector full assemblies were in place during the field trimming.
%\\...\\
%All materials near the trap cloud are chosen to minimize both magnetic permeability to suppress time-constant B field gradients and conductivity to suppress eddy currents. E.g., the E field electrodes are made from either glassy carbon semiconductor or titanium alloy. A copper ring (not pictured in Fig. 2.5) with a slit mounted on each beta detector stainless steel reentrant flange suppresses the worst eddy currents fighting the B field along the z-axis. These designs were all confirmed by finite element calculations of another collaboration member.
%}


%\note[done, nolist]{Chapter 2 corrections from John: (trivial...):
%\\...\\
%*2.5.1 state clearly in an extra paragraph at the end that the polarization-determining data was taken with the rMCP, while the Abeta and bFierz data were taken with the eMCP. The polarization was assumed to be the same during the Abeta+bFierz data. This is backed up by constant 41K polarization data for weeks
%of optimization, with all optical pumping and B field switching parameters kept constant. 
%\\...\\
%Ambient magnetic field changes of 50 milliGauss could cause some polarization perturbations at the precision achieved, yet the stray fields are under control at that level. The TRINAT lab is in a basement well-shielded from the experimental hall by concrete with rebar, and though 50 mG fields are seen in that Hall from an open Helmholtz ion trap, they and the 5-ton crane produce negligible fields when measured at the atom trap. The cyclotron field makes 0.5 Gauss, predominantly vertical, and a smaller horizontal component, but trim Helmholtz coils at TRINAT are adjusted during calibrations with cyclotron on vs. off, and of course the cyclotron is on with constant field during the 37K delivery.
%}


\subsection{Ch. 3}

\subsection{Ch. 4}

\subsection{Ch. 5}

%I really need an excuse to include more pictures of data.  Also, more pictures of simulations.
%\missingfigure{Show simulated spectra separated by scattering category.}

%\note[done,nolist]{Things that the G4 simulation did that I kept include:  an accurate representation of the complex details of our experimental geometry.  Also, the noise spectra on the DSSDs.}

\note{OK, somewhere I really have to say specifically the parameter that I varied and chi-squared-ed...  it's some sort of pseudo-Holstein-eque $g_S$, which gives "left handed" scalar couplings.  Also, no tensors.  Pretty sure it comes out equivalent to looking for tensors within bFierz, but gives a slightly different value of Abeta. But maybe this goes in the Analysis chapter?}

%\note[done,nolist]{I changed $E_{0}$ to $E_{\textrm{in}}$ in all the equations.  I'm leaving $p_{\textrm{dE0}}$ alone though.  I think.  Probably need to check that all the pictures still make sense.
%\\...\\
%Also, changing $x$ to $E_{\textrm{out}}$.  }
%
%\note[done,nolist]{In these response functions, $x=$?}

%\note[jb]{JB says: I would say you have a well-determined TOF cut to minimize this error-- a cut that could not have been done blind without an unreasonably perfect simulation.  Thus the exact spot of the cut should not be considered to introduce a systematic. }

%\note{The SOE events were generated only up to 100 eV.  That smaller number of eV (20??) wasn't enough, even though the tail gets tiny.  Also, for higher energy SOEs, we observe a smaller fraction of the ones that are generated anyway.}


%\note[bluetodo]{In the end, we used $(0.09)*(\textrm{0eV}) + (0.91)*(0.85*(\textrm{4S}) + 0.15*(\textrm{3P}))$.   That's $9+77+14$.  I will (eventually) say that *here*, not in the intro section.  Also, John used Eq.20 for the 4S, and Eq.24 for the 3P.}


%%%%%%%%

%%%%%%\note[jb]{JB on 2.4 SOE:
%%%%%%\\
%%%%%%Say what you did, with as little info as possible now. Georg is a theoretical
%%%%%%chemist, so he may be very curious about this.
%%%%%%\\ ... \\
%%%%%%"Atomic electrons with kinetic energies 0 to 100 eV are produced as part
%%%%%%of the beta decay process.
%%%%%%If their energy is below a certain value, our detection process is not perturbed,
%%%%%%so we provide physics about the energy spectrum here.
%%%%%%\\ ... \\ 
%%%%%%Levinger~\cite{Levinger}, assuming the sudden approximation,
%%%%%%calculated the overlap between an electron in the initial  atom with
%%%%%%an outgoing electron or an electron in  the final atom. Levinger approximated
%%%%%%everything with hydrogenic wavefunctions, so his calculations become
%%%%%%analytic. The collaboration has found that Levinger's formulae
%%%%%%fit our measurements of the position and TOF  info of our atomic electrons,
%%%%%%so they are used in our simulations 
%%%%%%\\
%%%%%%OR
%%%%%%\\
%%%%%%but the precise energy spectrum was found to be unimportant, so we used X in our
%%%%%%simulations.
%%%%%%\\
%%%%%%whichever is true.
%%%%%%\\...\\
%%%%%%This is then fine. You have most of the basics down.
%%%%%%}



\subsection{Ch. 6}
%\note[done]{Remember to cite these dudes for (I think) the low-energy tail in the spectrum:  ~\cite{stragglingLonergan1970} \cite{stragglingRester1971}.}

\note{Go through and change all instances of ``Beta - SOE''/``Beta---SOE''/etc to ``SOE -  Beta''/etc., *within the whole thesis*.}

\subsection{Ch. 7}
\note[done, nolist]{JB on Ch. ~\ref{sec:measured_limits}:
\\
%\\...\\
%you just have to summarize the figures. 
Point out the natural dilution
of Abeta result if energy-dependent new physics is allowed to float.
}
%Results go here, with measured limits described and quantified in all formats anyone could ever care about.
\note[jb1]{John says to just skip doing the $C_S$ and $C_T$ stuff, for now.  No time.  ... Really, $C_S$ is already basically done, but then that'll lead to awkward questions about $C_T$.}


\note{Some citations of things I'll probably want to use:
\\...\\
Falkowski (v1, v2):  \cite{Falkowski2021}\cite{Falkowski2022}.  
\\
Gonzales-Alonso, Naviliat-Cuncic, Severijns (2019) review on searches:  \cite{GonzalesalonsoNaviliatcuncicSeverijns2019}.
\\
Hardy and Towner review limits from superallowed beta decay, 2020:  \cite{HardyTownerSuperallowed2020}.
\\
UCNA does $\bFierz$, 2020 version.  They got $\bFierz = 0.066 \pm 0.041$(stat) $\pm 0.024$(sys) for the neutron\cite{UCNAfierz2020}.
\\
Saul and some other collaboration does $\bFierz$, 2020 version.  They got $\bFierz = 0.017 \pm 0.021$ \cite{Saul2020}.  (they also make the claim that I make about how the SR trades stats for reduced systematics.)
}

\note[done, nolist]{
Full considerations would require a weighted fit of $\bFierz$ experiments and similar observables~\cite{Falkowski2021}, and are beyond the scope of this thesis.
The info from this thesis, values of $\Abeta$ and $\bFierz$ with their uncertainties, can together with the known $fT$ value (lifetime and
branching ratio) allow the community and/or the collaboration to include the results in a future constraint or discovery of scalar and tensor Lorentz currents
contributing to $\beta$ decay.}



%%%%%%% %%%%%%%%% %%%%%%%%
\note[done,nolist]{JB on Ch. \ref{sec:discussion_corrections_uncertainties} -- `discussion of corrections and uncertainties':
\\...\\
could be as simple as
"The bFierz result in Table 5.1 is dominated by statistical uncertainty.
Largely because of this result,
the collaboration is working to reduce the largest systematics,
using lower-Z materials to reduce backscattering, and changing the silicon
delta-E to a multi-wire proportional chamber with very thin windows.
The collaboration has already implemented very thin pellicle mirrors.
The projected systematic uncertainty could approach 0.01 in a future
experment, which would then likely continue to be limited by statistics."
\\...\\
or all of that could go at the end of Chapter 5.
}

\note[jbn, nolist]{Another Round of Thoughts from John!
\\...\\
Fitting Ben's data (using for the bFierz function my convolution of 1/E with a Clifford tail) as I've shown before, compare what happens if I let Abeta float, or if I then fix Abeta arbitrarily to that floated value.
\\
That's not a well-motivated thing to do on its own, but it's not that much different than using Abeta[Cs$^2$,Ct$^2$] constrained by Ft[Cs$^2$,Ct$^2$] since that's only dependent on squares of small quantities.
\\...\\
So I would expect the ability to extract Cs and Ct from the 37K data to have more
sensitivity if Abeta[Cs$^2$,Ct$^2$] and Ft[Cs$^2$,Ct$^2$] and bFierz[Cs,Ct] are floated together in a 2d fit rather then floating Abeta and bFierz without any theoretical constraints and then considering getting linear combinations of Cs and Ct from the bFierz value.
\\...\\
Note that my simply fixing Abeta lowers bFierz uncertainty by a factor of 3. I expect similar improvement in sensitivity to Cs and Ct.
\\...\\
(Here I'm using Cs as a stand-in for (Cs+Cs')/2 for simplicity)
}

\note[jb1]{JB on simple things still missing:
\\
I would have expected a separate uncertainty for b and Abeta for each data set,
either on each 2D figure in ch. 6, are collected in a table in Ch. 6.
}

\note[jbn, nolist]{Immediate follow-up email from John:  
\\...\\
another basic big point. What would it mean if there were nonzero Cs, Ct? 
\\
In the context of the more elegant SM, a theory of quarks and leptons and mathematically consistent interactions, that would imply the existence of at least one extra unknown exchange boson. The mass would not be measured, just its coupling
strenghts to the particles participating in the beta decay.
\\...\\
Maybe background to understand that includes:
\\
If one were to write the SM weak interaction, $C_A$ and $C_V$ and $C_A$ and $C_V g_A$ would all be constants with abs value 1.
The (1+- gamma\_5)'s are projection operators-- in the SM, the W boson only couples to left-handed nu's and right-handed antinu's
\\
('more will be said in the forward-looking Ch. 6.2 about tests of that part.)
\\
So all the known information about Ca and Cv is accounted for.
\\
($g_A = 1.26$ for the neutron... it's not equal to 1 because of strong interactions between the quarks in the neutron.)
}
















































%
%%%% --- * --- %%%%	
\clearpage

