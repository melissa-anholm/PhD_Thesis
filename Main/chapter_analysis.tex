% !TEX root = ../thesis_main.tex
%
%
%
%%%% --- * --- %%%%	
\clearpage
%\chapter{Calibrations and Analysis}
\chapter{Calibrations}

\note[color=org]{I have some tables summarizing which types of data were measured.  Presently, that's in Ch-4, but probably I should move it here.}



%%%% --- * --- %%%%	
%\section{Calibrations}
\label{calibrations_chapter}

\section{Measurements of the Atom Cloud}
\label{sec:cloud_calibration}

\note[color=org]{Content that previously was in this subsection is now in Section~\ref{cloud}.  But also, see Section~\ref{sec:photoprobe}.  Probably in this section here, I need to just jump right in assuming everyone understands the physical principle of how the data works.  Then, I can talk about what the data *is*, and how it gets interpreted.}

%\label{cloud}
%\label{photoions}
%\note[color=org]{Some of this stuff is in Ch.3 and/or Ch.4.  Some of it *should* be.}
%In order to measure properties of the trapped $^{37}\textrm{K}$ cloud, a 10\,kHz pulsed laser at 355\,nm is directed towards the cloud.  These photons have sufficient energy to photoionize neutral $^{37}\textrm{K}$ from its excited atomic state, which is populated by the trapping laser when the MOT is active, releasing 0.77\,eV of kinetic energy, but do not interact with ground state $^{37}\textrm{K}$ atoms.  The laser is of sufficiently low intensity that the great majority\aside[color=jb]{JB:  ``On order 1\% are photoionized."} of excited state atoms are \emph{not} photoionized, so the technique is only very minimally destructive.  
%\note{Probably worth mentioning that we test this stuff offline on stable \isotope[41]{K}. }
%
%
%Because an electric field has been applied within this region (see Section~\ref{field})\aside[color=jb]{JB:  ``you could reference the letter for the value of the field 150V/cm.''} the $^{37}\textrm{K}^+$ ions are immediately pulled into the detector on one side of the chamber, while the freed $e^-$ is pulled towards the detector on the opposite side of the chamber.  Because  $^{37}\textrm{K}^+$ is quite heavy relative to its initial energy, it can be treated as moving in a straight line directly to the detector, where its hit position on the microchannel plate is taken as a 2D projection of its position within the cloud.  Similarly, given a sufficient understanding of the electric field, the time difference between the laser pulse and the microchannel plate hit allows for a calculation of the ion's initial position along the third axis.  
%
%\note{As a check:  the camera measurements for photons from de-excitation.  It's aimed 35 degrees from vertical, with its horizontal axis the same as ..... one of the other axes.  I think it's the TOF axis.  I can check this when my computer comes back.   Also, there's an unknown additional delay between some of our DAQ channels that can't be explained by accounting for cable lengths, so we really like having the check there.}
%\note[color=jb]{JB says:  ``yes, camera x-axis is tof axis.''}
%
%
%With this procedure, it is possible to produce a precise map of the cloud's position and size, both of which are necessary for the precision measurements of angular correlation parameters that are of interest to us here.  However, it also allows us to extract a third measurement:  the cloud's polarization.
%
%The key to the polarization measurement is that only atoms in the excited atomic state can be photoionized via the 355 nm laser.  While the MOT runs, atoms are constantly being pushed around and excited by the trapping lasers, so this period of time provides a lot of information for characterizing the trap size and position.  When the MOT is shut off, the atoms quickly return to their ground states and are no longer photoionized until the optical pumping laser is turned on.  As described in Section~\ref{op}, and in greater detail in~\cite{ben_OP}, the optical pumping process involves repeatedly exciting atoms from their ground states until the atoms finally cannot absorb any further angular momentum and remain in their fully-polarized (ground) state until they are perturbed.  Therefore, there is a sharp spike in excited-state atoms (and therefore photoions) when the optical pumping begins, and none once\aside[color=jb]{JB points out that this should be ``if", not ``once".} the cloud has been completely polarized.  The number of photoion events that occur once the sample has been maximally polarized, in comparison with the size and shape of the initial spike of photoions, provides a very precise characterization of the cloud's final polarization~\cite{ben_OP}.



\note{Trap position -- Measured using the same dataset that was used to quantify the polarization.  The trap drifts slightly over the course of our data collection.  Describe the rMCP calibration needed to extract this info.}
\note{Polarization measurement was conducted on a different set of data, collected in between the measurements used for $A_{\mathrm{\beta}}$ and $b_{\mathrm{Fierz}}$, and at a higher electric field, because we were unable to run both our MCP detectors simultaneously.  }

\missingfigure{Need pictures of the cloud.  Possibly need projections of the cloud as a function of time, for AC/OP cycles.}
%\note{Need to describe how polarization works.  Needs a level diagram.  Needs another(?) level diagram for the photoionization, and maybe a third for the MOT.  Can I combine them all?  idk.}

Anyway, here is a nice table describing the atom cloud, for each of 3 runsets, and I'll immediately reference it right now, as Table~\ref{table:cloudpositions}:

% !TEX root = ../thesis_main.tex
%
%
%
\begin{table}[h!!!!t]
	\begin{center}
	\begin{tabular}{ c  | r || lcl | lcl || lcl | lcl |}
			\multicolumn{1}{c}{ Runsets  }{ } & \multicolumn{1}{  c  }{ } & 
				\multicolumn{3}{  c  }{ \!\!Initial Position\!\! } &  \multicolumn{3}{   c  }{ Final Position } &  \multicolumn{3}{   c  }{ Initial Size } &  \multicolumn{3}{   c  }{ Final Size } \\
			\cline{2-14}
			\multirow{3}{*}{EB $\leftarrow$ RB} %&\multirow{3}{*}{RB}  
								& $x$ & \,\,1.77 & \!\!$\!\! \pm  \!\!$\!\! & 0.03   & \,\,2.06   & \!\!$\!\! \pm  \!\!$\!\! & 0.08    & 0.601 & \!\!$\!\! \pm  \!\!$\!\! & 0.013 & 1.504 & \!\!$\!\! \pm  \!\!$\!\! & 0.047 \\
								& $y$ & -3.51    & \!\!$\!\! \pm  \!\!$\!\! & 0.04   & -3.33     & \!\!$\!\! \pm  \!\!$\!\! & 0.05    & 1.009 & \!\!$\!\! \pm  \!\!$\!\! & 0.013 & 1.551 & \!\!$\!\! \pm  \!\!$\!\! & 0.018 \\
								& $z$ & -0.661  & \!\!$\!\! \pm  \!\!$\!\! & 0.005 & -0.551   & \!\!$\!\! \pm  \!\!$\!\! & 0.021  & 0.891 & \!\!$\!\! \pm  \!\!$\!\! & 0.004 & 1.707 & \!\!$\!\! \pm  \!\!$\!\! & 0.015 \\
			\cline{2-14}
			\multirow{3}{*}{EC $\leftarrow$ RD} %&\multirow{3}{*}{RD}  
								& $x$ & \,\,2.22  & \!\!$\!\! \pm  \!\!$\!\! & 0.05  & \,\,2.33   & \!\!$\!\! \pm  \!\!$\!\! & 0.11    & 1.18   & \!\!$\!\! \pm  \!\!$\!\! & 0.04   & 1.538 & \!\!$\!\! \pm  \!\!$\!\! & 0.087 \\
								& $y$ & -3.68     & \!\!$\!\! \pm  \!\!$\!\! & 0.04  & -3.31      & \!\!$\!\! \pm  \!\!$\!\! & 0.06   & 0.965 & \!\!$\!\! \pm  \!\!$\!\! & 0.012 & 1.460 & \!\!$\!\! \pm  \!\!$\!\! & 0.030 \\
								& $z$ & -0.437   & \!\!$\!\! \pm  \!\!$\!\! & 0.09  & -0.346    & \!\!$\!\! \pm  \!\!$\!\! & 0.037 & 0.927 & \!\!$\!\! \pm  \!\!$\!\! & 0.007 & 1.797 & \!\!$\!\! \pm  \!\!$\!\! & 0.026 \\
			\cline{2-14}
			\multirow{3}{*}{ED $\leftarrow$ RE} %&\multirow{3}{*}{RE}  
								& $x$ & \,\,2.274 & \!\!$\!\! \pm  \!\!$\!\! & 0.012 & \,\,2.46 & \!\!$\!\! \pm  \!\!$\!\! & 0.06   & 0.386 & \!\!$\!\! \pm  \!\!$\!\! & 0.016 & 1.382 & \!\!$\!\! \pm  \!\!$\!\! & 0.046 \\
								& $y$ & -4.54      & \!\!$\!\! \pm  \!\!$\!\! & 0.04   & -4.28    & \!\!$\!\! \pm  \!\!$\!\! & 0.04   & 0.986 & \!\!$\!\! \pm  \!\!$\!\! & 0.08   & 1.502 & \!\!$\!\! \pm  \!\!$\!\! & 0.013 \\
								& $z$ & -0.587    & \!\!$\!\! \pm  \!\!$\!\! & 0.04   & -0.481  & \!\!$\!\! \pm  \!\!$\!\! & 0.018 & 0.969 & \!\!$\!\! \pm  \!\!$\!\! & 0.003 & 1.861 & \!\!$\!\! \pm  \!\!$\!\! & 0.013 \\
			\cline{2-14}
	\end{tabular}
	\end{center}
%	\note{These positions are for the *electron* runsets of those names.  Might want a chart of which rMCP runs are used to measure position for which eMCP runs.  Possibly in that other section.}
	\note{Sig figs here need work.}
	\note{Parameters measured with the recoil runs, and applied on the appropriate electron runs.}
	\caption[Cloud Position and Size]{Cloud Positions and Sizes -- Measured immediately before and immediately following the optical pumping phase of the trapping cycle.  Measurements are evaluated using rMCP runs, and are taken to describe the cloud during associated eMCP runs as well.  All entries are expressed in units of millimetres, and the size parameters describe the gaussian width.}
	\label{table:cloudpositions}
\end{table}




\note{Also, we noticed the trap drifting after one of the runs, because one of the batteries on one of the thingies adjusting the laser frequency (I think) was failing. }
\note[color=jb]{JB:  ``If we rejected the data with the MOT moving (indeed a battery determining the voltage controlled oscillator frequency offset between absorption in stable \isotope[41]{K} cell and the \isotope[37]{K} resonance) then that's all you need to say.''}
\note{describe how you'd turn this into a physical description of the cloud, with like a temperature and a sail velocity and shit.  with equations.}



\begin{figure}[h!!t]
	\centering
	\includegraphics[width=.999\linewidth]
	{Figures/y_from_camera_electron_recoil.pdf}
	\caption[Trap Position along TOF Axis]{Trap Position along the ``Time-of-Flight'' Axis.  Electron runs and recoil runs plotted by run number. \comment{(I should probably re-plot this.  Maybe combine info with Fig.~(\ref{fig:cameraposition_by_time}).) } }	
	\label{fig:camera_electron_recoil}
\end{figure}

\begin{figure}[h!!t]
	\centering
	\includegraphics[width=.999\linewidth]
	{Figures/TrapPosition_FromCamera.pdf}
	\caption[Trap Position from Camera]{Trap Position along the ``Time-of-Flight'' Axis and the ``Vertical+30" Axis.  All runs plotted by time of run. \comment{(Need to re-plot this.)}}	
	\label{fig:cameraposition_by_time}
\end{figure}




%\section{Beta Detector Cuts}
%\label{section:calibrations_betadetectors}
%	Setup is as described in Section~\ref{section:betadetectors}.	 
%	\note{This is a stupid section name.  Also, I really do need to describe how the cuts were made here somewhere, because it's non-trivial in many cases, and possibly different than what Ben did in some cases.  But it won't make sense to describe what I did different if I don't describe the thing as a whole, at least a bit.  The point is, this is a set of cuts/systematics that isn't really that straightforward to understand.  }
%\section{Plastic Scintillators}
%	\note[color=org]{Maybe this goes in Chapter~\ref{analysis_chapter}?}
%	\note{Energy calibration for the scintillator+PMT setup changed dramatically at one point.    Describe how calibration was done.  Like, one sentence or something.  Something about the endpoint energy, and something about the compton edge for 511s, IIRC.}
%\section{Strip Detectors}
%	\note{How the fuck WERE these things done?!}



\section{The eMCP}
	I can describe the eMCP calibration here, even though it mostly wasn't implemented by me.  It is tangentially relevant to data selection and background estimation by providing an experimental energy spectrum for shake-off electrons.  It's actually a pretty neat algorithm that I basically wasn't involved with.
	\note[color=jb]{JB:  eMCP.  You need to describe the timing information obtained.  You also need a statement of whether or not you used the position information in your cuts.}

\missingfigure{Needs an SOE timing spectrum.  At least one of them.  Experimental and simulated.  Also, I have to describe how I did the simulating, and how I check that it's OK despite the fact that the simulated spectrum looks nothing like the experimental spectrum.}


\section{The rMCP}
	I did this, and they're absolutely needed to make any sense of the trap position data.  
%\missingfigure{Needs two pictures from the rMCP with the grid lines -- before and after corrections are performed.  Just for fun, I could throw in one with the stupid stripes.}
\begin{figure}[h!!t]
	\centering
	\includegraphics[width=.999\linewidth]
	{Figures/rMCP_Calibration}
	\caption[rMCP Calibration]{rMCP Calibration.  Definitely comment about how this calibration went here, in the figure caption. \comment{(Do I definitely want a picture with the stupid line-y runs?  Maybe it's better to just *not*...)}}	
	\label{fig:rmcp_calibration}
\end{figure}
	These calibrations are done during AC-MOT time, and we're actually interested in the rMCP data taken during OP time.  Can I find pictures to estimate the size of the change resulting from the magnetic field?  In any case, the change is pretty small.  

\subsection{rMCP Bullet Points!}
%	\begin{itemize}
%	\item 
	Using the ``other'' data set with the rMCP:  Measure the trap position/size/velocity/expansion with the rMCP and with the camera.  Necessitates calibrating the rMCP, which is its own whole thing.  Also measure polarization.
		\begin{itemize}
		\item rMCP calibration probably gets its own section, if not its own chapter..
		\item We took the mask off before the 2014 run, to give us more detector area.  Use previous reference calibration \emph{with the mask} during the test run in Nov 2013.  The delay line's non-linearities should be the same, assuming we can get the centering the same.  Cables have changed and stuff, so we have to re-center the pre-calibration image to where the old pre-calibration image was.  ...  So, center the new runs w.r.t the old run.
		\item We'll want to make some sum cuts for these things.  We might like them to be identical, or at least identical-ish, but the peaks don't really look the same.  So we'll settle for ``decent sum cuts for all!".  ...  So, apply sum cuts to the new runs and old run.
		\item Calibrate the old run, with the mask.  In fact, I don't remember which order I did things in.  But I have a record of it here, somewhere...
		\end{itemize}
%	\end{itemize}

%\missingfigure{Position as a function of run number.  I have this somewhere.}


%%%% --- * --- %%%%	
\clearpage
\chapter{Analysis}
%\chapter{Data Selection}
\label{analysis_chapter}
\note[color=jb]{JB says:  ``ch 7 looks fine to me.''  ...even in spite of the giant paragraph suggestions below, I guess.}
\note[color=org]{Do I actually want to combine this with Chapter~\ref{calibrations_chapter}?}
\note[color=org]{from JB:  you ask at the start if you should combine it (Ch.7, `Analysis') with Ch.6 (`Calibrations'. I see no special reason either way.  I would say the topics seem distinct.}

\note[color=jb]{JB:  Here is what I suggest for the intro the Analysis section:  (n.b.:  at the time of this comment, that was Ch.7)
\\...\\
7.0
Analysis is critical to a precision measurement, as most of the research is in
determining systematic uncertainties by self-consistent analysis and simulations.
Each detector in this experiment is critical and has independent calibrations and cuts.
Full explanation is here (and in two following chapters) justifying choice of the deterministic cuts,
because in the analysis in this thesis the data was not blinded.
The main goal of blinding was nevertheless achieved-- to make sure all analysis is done
completely with full redundancy of checks wherever possible-- so the discipline entailed must
be described in full detail.
Here there are details of detectors:
eMCP
rMCP
beta DSSD
scintillator.
\\...\\
The collaboration has done an independent analysis fixing bFierz to zero.
Differences with that analysis are interleaved in this section.
Critical physics improvements concern
an emcp-beta timing walk correction which enabled an improved cut against background, also incorporating
a more complete modelling of decay backgrounds from untrapped atoms.
Technical corrections include a correct treatment of the polarization cycle.
An arbitrary change in the deltaE radius cut is kept self-consistent.
\\...\\
Then Section 7.1...
}

%\section{Data Selection and Pre-Processing}
%\section{Run Parameters and Preliminary Cuts}
\section{Data Selection and Preliminary Cuts}
Although the detection chamber was designed to feature two MCP detectors on opposing sides of an applied electric field intended for simultaneous use (see Section~\ref{section:mcps}), % to collect positively charged recoiling ions on the rMCP and negatively charged shake-off electrons on the eMCP 
in practice the two detectors produced quite a bit of feedback when operated at the same time.  In order to salvage usable data from the beamtime, we ended up running only one detector at a time, but switched which detector was in use every few hours, collecting approximately the same amount of data with each detector.  Thus, the runs are sorted into `electron' and `recoil' runs, depending on what the detector in use was intended to detect.  

While the beta asymmetry and Fierz interference are best evaluated using the electron runs, the recoil runs are best for evaluating the polarization (a dominant uncertainty in the beta asymmetry measurement) and the cloud position.  The polarization measurement is the subject of a recent publication (see~\cite{ben_OP}), and the cloud position evaluation is discussed in more detail in Chapter~\ref{calibrations_chapter}.

The data is further split up into four runsets:  A, B, C, and D based on when certain detection settings were adjusted, and each of these runsets contains both electron and recoil runs.  These four runsets were then treated separately for nearly all parts of the analysis.  In particular, Data Set A was neglected completely during analysis after it was determined that one scintillator had an improperly set hardware threshold such that lower energy betas weren't being detected at all.  Additionally, there was a QDC module failure between Runsets B and C, resulting in an abrupt change in calibration for the two scintillators.  
\note{Also, the background was too noisy in Set A I think.  Also, pretty sure there isn't really that much data in Set A at all.  Also-also, Set A has E=66.7, so can't use it for more stats on the other backgrounds. } 

%\note{What did I change between Runsets B and C?  Between Runsets C and D?  ...OP time, yes, but also and one point a PMT(?) (eta:  it was a QDC) blew out and we replaced it, so then the scintillator calibration was a bit different afterward.  Anyway, I should maybe just make a goddamn table.  Table goes here.}

% !TEX root = ../thesis_main.tex


\begin{table}[h!!!!t]
	\begin{center}
	\begin{tabular}{ c || p{6cm} | p{6cm} | }
			\multicolumn{1}{  c  }{ } & 
				\multicolumn{1}{  c  }{ \!\!Electron Runs\!\! } &  \multicolumn{1}{   c  }{ Recoil Runs }  \\
			\cline{2-3}
			\multirow{1}{*}{Runset B}	& \,\,1.77, something, something, something, somethignsomething something, something & 3  blah blah blah blah, blahblah, blah  \\
			\cline{2-3}
			\multirow{1}{*}{Runset C}	& \,\,2.22  & 3    \\
			\cline{2-3}
			\multirow{1}{*}{Runset D}	& \,\,2.274 &  3    \\
			\cline{2-3}
	\end{tabular}
	\end{center}
	\caption[List of Runs]{Runs categorized by whatever. \comment{Yeah, I have to fix this so it has like any content at all.}}
	\label{table:runlist}
\end{table}



Before proceeding further, several basic cuts are performed on the data.  For the Electron Runs which are to be processed directly into a physical measurement, we consider only events in which there was a recorded hit \emph{both} on the eMCP \emph{and} on (at least) one of the scintillators.  The required scintillator hit, of course, is potentially a beta, and so it is obvious why this must be present.  The eMCP hit requirement -- particularly with the timing of the eMCP hit occuring within a certain time range relative to the scintillator hit -- is used to tag beta decay events originating from the cloud, as opposed to those originating from some other surface within the chamber.  The precise time interval to be used, and how its results should be evaluated,\aside{also discussed: how we decide what counts as an eMCP hit at all} is discussed in Section~\ref{section:emcp_cuts}, but for the first pass through the data it is good enough to simply require that an eMCP hit occurred.\aside{Awkward stupid phrasing.}  Although not every beta decay event from the cloud will produce a hit on the eMCP, this requirement eliminates a great deal of background that would otherwise be challenging to evaluate.  \aside{Also, we claim that it doesn't bias the data.  Much.  Didn't I try to evaluate how much it biased the data at one point?}  
 
%The scintillator hit is considered to be potentially a beta from a decay
Because the eMCP hit is required as a `tag' of good events, it is also necessary to remove from direct consideration any event which is coincident with a pulse of the photoionization laser.  When photoionization occurs within the atom cloud, an orbital electron is removed from the atom and will be accelerated by the electric field into the eMCP, just as a shake-off electron from a decay might be.  If, by chance, this photoelectron arrives in coincidence with a scintillator hit, it would be interpreted as a decay event from the trap -- unless we preemtively discard it.  


Over the course of the runtime, there were several instances where we noted an apparent electrical discharge within the experimental chamber, producing enormous backgrounds for a short time.  The detectors typically recovered quickly afterward, so it was neither necessary nor useful to stop an entire run to wait for the system to recover.  Instead, the time when the discharge occurred was recorded, and events within approximately one minute of the spark time were discarded.  

%Because it is of the utmost importance to understand the nuclear spin-polarization immediately prior to a decay, 
We use only the ``fully polarized'' events for which we have a detailed understanding of the nuclear polarization (described in more detail in ~\cite{ben_OP}).  This means we must use \emph{only} events from the ``optical pumping'' portion of the duty cycle (see Fig.~\ref{fig:dutycycle}), and discard events when the DC- or AC-MOT is active.  After the AC-MOT is shut off, there is a short delay before optical pumping begins (see Table~\ref{table:runlist}) to allow the magnetic field to decay, and it is only after $100\,\mu s$ of optical pumping that we consider the atoms to be fully polarized.  Furthermore, because the magnetic field from the DC-MOT is slow to decay (relative to the field from the AC-MOT), all events from the first five AC/OP cycles after every atom transfer are discarded.  A secondary benefit of our insistence on considering only polarized data is that the scintillators' gains are more stable in the presence of only the (small, stable) magnetic field used for optical pumping than they are in the presence of a larger oscillating magnetic field used for trapping.
\note{change by 0.2\% of its value vs change by 0.5\% of its value, according to Ben's thesis pg 143.}

%Another class of event that must be removed from direct analysis is 
Finally, because this analysis depends heavily on energy measurements from the two scintillators as a proxy for beta energy, it is necessary to remove events in which the pulser LED fired.  Although the pulser LED is useful for evaluating the stability of the scintillators, in the case where an LED pulse occurs together with a true beta hit in the scintillator, it may change the measured energy.  Therefore, we discard all events that include an LED pulse.   
	

\section{Further Cuts Using the DSSD}
%\note[color=org]{Have I even defined `DSSD' yet?}
\note[color=org]{Pretty sure I'm repeating myself from Chapter~\ref{???}}
\note[color=jb]{JB says: To repeat a comment, since the Appendix on analysis changes is being dispersed throughout, when you state somewhere (I think it's in Ch. 6) make sure you state clearly that the only change for BB1 cuts
is the radius cut, e.g. that you took the same T and E from the waveforms (I'm not even sure whether the
waveforms are recorded anyway). You  ask 'how can I state this' but there's no reason to be subtle.
Just say upfront that Ben and Spencer's theses did all the groundwork on the BB1, and here you include selected details needed to understand the present analysis. If you need to include some redundant material, don't worry too much about that.}

Although it was not possible to use the DSSD in real-time analysis or event triggering, the DSSDs may be used, after the data has been collected, to distinguish between different types of particles incident on the detector, as more energy will be deposited by heavier particles.  When a scintillator hit is triggered by a particle originating within the experimental chamber, that particle will typically have passed through the DSSD before arriving at the scintillator.

In the present experiment, the two primary particles that will concern us are $\beta^+$ particles originating from the decay of $\isotope[37]{K}$, and $\gamma$ rays, which may be produced through a variety of processes, e.g. directly from the $2\%$ decay branch, through annihilation of $\beta^+$ particles upon their interaction with regular-matter electrons, or bremsstrahlung radiation from emitted $\beta$s.  

We would like to look specifically at events involving $\beta^+$ particles arriving direct from a decay within the atom cloud, and the DSSD may be used to eliminate events in which the scintillator is triggered by a $\gamma$.  An incident $\beta$ will typically deposit some portion of its energy in the DSSD as it passes through, however an incident $\gamma$ will deposit significantly less energy; for this setup the energy deposited by a $\gamma$ is generally indistinguishable from background on the DSSDs.  Therefore, we require that a `good' event must include a `good' hit to the DSSD as well as a hit to the associated scintillator.   

In order to proceed at this point, and because the DSSD readout records so much information, it is necessary to develop some criteria to determine whether or not we will accept any given DSSD readout as a $\beta$ hit.

\note{How do I *say* that Ben was the one who did most of the DSSD calibration stuff?  I maybe don't need to describe all of it here, but I *could*, and maybe it's needed in order to understand like 4 rows in my error budget.}  

\note[color=jb]{JB:  ``You can describe anything you did differently or improved, but you can and should otherwise defer all details of the scintillator calibration and DSSD calibration to Ben's paper and his thesis and Spencer's.  E.g. Section~\ref{section:bb1_systematics} ``statistical agreement between BB1 X and Y detectors' energies only makes a small effect on results" does not need the technical details beyond that statement."
\label{thesisconventionjb} }
\note[color=jb]{JB:  ``If you have some way of documenting the coding you used, that would be great."  ... yeah, it would, wouldn't it?}


We read out the full waveform for every strip at each event with a scintillator hit, but in post-processing take \emph{only} the `time' and `energy' from the peak waveform height and the time in the waveform at which that occurs. Each strip will have its own noise spectrum and energy calibration.  To classify an event as a good DSSD hit, we require at least one `x' strip and one `y' strip record an energy above the noise threshold.  We require that the x strip and the y strip agree (to within some number of standard deviations) in amount of energy deposited, and in the time at which that hit occurred.  In order to avoid problems resulting from the strips' non-uniform noise thresholds, we further require that the energy deposited be greater than some lower-end cutoff which is selected so as to be higher than every individual strip's noise threshold.  In this case, the DSSD's lower energy uniform threshold was set at 50 keV.  

\note{I think Ben might have selected 60 keV?  That's maybe something for the appendix.}

We also elect to use only events where a beta hit the DSSD within a $15.5\,$mm radius of the center of the detector, so as to avoid scattering effects from the collimator walls. \aside{Did I even mention the collimator?  Like, in the previous chapter or something..?}

%Look at all the events to determine how noisy each individual strip is.  Do that for all strips.  Is the energy higher than some noise threshold?  by like a few sigma?


\section{Further Cuts Using the eMCP}
\label{section:emcp_cuts}
\note[color=org]{I described the HEX-75 somewhere in a previous chapter, right??}

The eMCP features a set of three delay lines, intended to be used to record the position of a hit, as in Fig.~\ref{fig:emcp_position}.  \aside{Do I need to describe MCPs and delay lines somewhere?  Maybe not...}  Though only two delay lines is sufficient to determine the position within the plane of the MCP if they are both hit, the presence of a third delay line allows for some redundancy.  In practice, however, a large fraction of otherwise `good' events include a hit on the eMCP, but have insufficient information recorded on the delay line channels to reconstruct a position.  

\begin{figure}[h!!!!t!]
	\centering
	\includegraphics[width=.999\linewidth]
	{Figures/eMCP_position.pdf}
	\caption{Position as measured on the eMCP, after some data cleaning.}	
	\label{fig:emcp_position}
\end{figure}

Because a SOE from the trap is most likely to land in the centre of the plate, while the background from other sources is roughly constant across the plate, it might make sense to accept only events where the eMCP hit is within some radius of the central peak.  This methodology was seriously considered because the remaining data has a much lower fraction of background events polluting it -- however this results in a loss of around half of the events even for the most generous eMCP radius cuts (see Fig.~\ref{fig:soe_tof_positioncompare}).  Therefore, it was decided that no position cuts on the eMCP would be made in the final analysis.

\begin{figure}[h!!!!t!]
	\centering
	\includegraphics[width=.999\linewidth]
	{Figures/SOE_TOF_positioncompare.pdf}
	\note{Um.  Did I for sure get the labels correct on this???  It seems really wrong.}
	\caption{Beta-electron TOF, for events with and without eMCP hit position information.  A cut will eventually be taken to accept only events sufficiently near the largest peak -- in this case the number of events is `only' decreased by a factor of 2.}	
	\label{fig:soe_tof_positioncompare}
\end{figure}

Several years after the data was initially collected, a problem was discovered with our low-level analyzer software, which we had been using to convert large and unwieldy MIDAS data sets into somewhat smaller and more manageable ROOT data sets.  In particular, for every timestamp recorded, our raw MIDAS data actually included both a timestamp for the leading edge (LE) of the pulse, and a timestamp for the trailing edge (TE).  The analyzer had--for years--been reporting the timestamp associated with the trailing edge of the pulse.  Initially it was unclear if there might have been a reason behind this choice, but a closer examination of the data showed less timing noise and a sharper distribution of timing pulses across the board (see Fig.~\ref{fig:LE_TE}), with some channels showing a larger effect than others.  This was corrected, and the entirety of this analysis has been performed now using the cleaner LE spectra.  

\begin{figure}[h!!t!]
	\centering
	\includegraphics[width=.999\linewidth]
	{Figures/LE_TE_peaks.pdf}
	\caption{SOE TOF peaks (eMPC - Scintillator), using the leading edge (LE) and using the trailing edge (TE).  Data is sorted according to runset.  For each individual runset, the TE peak is broader than the LE peak.  The centroid of each runset is also more variable in the TE plots.}	
	\label{fig:LE_TE}
\end{figure}

The place where this change between the TE and LE timestamps had the biggest impact on the analysis is in the shake-off electron time-of-flight spectra, on which a cut must eventually be taken. Although this problem was not discovered in time to be used in the previous measurement of $\Abeta$ using this same data~\cite{ben_Abeta}, it likely would have had a negligible effect on the final result, because the SOE TOF cut that was used there was comparatively loose, and the evaluation of the background that remained was not a dominant systematic effect. \aside[color=org]{This goes in that one appendix, if I haven't already put it there.}

With the data reprocessed using the leading edge for timestamps, I wanted to eliminate as much background as possible from the SOE TOF spectrum.  With this goal in mind, the next step was to correct the scintillator timing for its low energy `walk' (see Fig.~\ref{fig:WalkAdjust}).  A quartic polynomial was fit to each of the 2D timing vs energy spectra (the top and bottom detectors were treated separately), and the result was used to produce a `straightened' SOE TOF spectrum with respect to measured scintillator energy, and as expected, the resulting SOE TOF spectrum was a bit more sharply peaked.  


\begin{figure}[h!!!!t!]
	\centering
	\includegraphics[width=.999\linewidth]
	{Figures/WalkAdjust.png}
	\caption{SOE TOF walk, before (left) and after (right) applying a quartic adjustment to straighten out the effective TOF.}	
	\label{fig:WalkAdjust}
\end{figure}

At this point it becomes necessary to attempt to model the SOE TOF spectrum, as in Fig.~\ref{fig:soetof}.  To check the agreement of the model with reality, we compare the averaged superratio asymmetries from both, as in Fig.~\ref{fig:asymmetry_by_tof}.
\note{Okay.  This needs more description.}

\begin{figure}[h!!!!t!]
	\centering
	\includegraphics[width=.999\linewidth]
	{Figures/SOE_TOF_Spectra.pdf}
	\caption{SOE TOF, model and data.  In the end, I cut the data to use only events with a TOF between -0.928 ns and +1.416 ns.  Max. possible background is like a factor of two too big.  Similar quality results no matter how you distribute the Levinger spectra between 4S and 3P, however adding the 0 eV electrons makes a big improvement to the agreement. }	
	\label{fig:soetof}
\end{figure}


\begin{figure}[h!!!!t!]
	\centering
	\includegraphics[width=.999\linewidth]
	{Figures/asymmetry_by_tof.pdf}
	\caption{The superratio asymmetry, averaged over all scintillator energies between 400-4800 keV, is used to compare the experimental data and simulated TOF model.  All other cuts have been applied.}	
	\label{fig:asymmetry_by_tof}
\end{figure}


%\section{Bullet Points!}
%Right, so.  Here's (more of) how I processed the data into an answer.  In bullet point form, so I don't forget stuff while I'm obsessively trying to phrase everything well.  
%\newline
%
%With the Data:
%\begin{itemize}
%%	\item \greycomment{Lower-level data cleaning.  Discard events during parts of the duty cycle when atoms weren't polarized.  Discard events near a recorded spark time. Discard events when the photoionization laser fires.  Discard events when the LED pulser used to calibrate the scintillators fires.}
%%	\item \greycomment{Split up runs into sets, to account for changing experimental conditions.  Possibly I should list what the differences between runs were somewhere.  But not in this section.  OTOH, ...maybe?}
%	%
%%	\item Make some more careful cuts to clean the data.  
%	%	\begin{itemize}
%%		\item \greycomment{Discard events without a ``good'' DSSD hit.  Eliminates vast majority of background 511s.  Necessitates having a definition of what a ``good'' DSSD hit is.  It's subtle enough that we'll want to leave some part of this definition of ``good'' to be varied as a systematic effect.  Notably, we consider energy agreement for each hit pixel, individual strip SNR, and overall DSSD energy threshold.  Also, hit radius w.r.t. center of detector.  This is a lot of stuff, all implemented by Ben -- and it needs to be done fairly early on in data processing in order to keep processing times for everything else manageable.  }
%		\item Discard events where SOE-Beta TOF falls outside a certain range.  Necessitates picking a ``good'' range.  The precise definition of ``good'' is varied as a systematic.  
%		\item \greycomment{There was a scintillator timing walk correction before taking the SOE-Beta TOF cut.}
%		\item \greycomment{Before the scintillator walk correction, there was the thing where the data was re-processed using the leading edge pulse rather than the trailing edge pulse.  Makes it cleaner.  Ben just got stuck with the trailing edge.}
%		\item \greycomment{Could have implemented an eMCP hit *position* requirement, but that kills too many stats.  It's like a factor of two.}
%	%	\end{itemize}
%\end{itemize}





%\section{Simulations}
\chapter{Simulations}
\note[color=jb]{JB:  ``I doubt I will have further useful comments on the Ch. 5, 8, 9 as they are now.'' }
\note[color=org]{Do I want this in a separate chapter?  Or maybe I want it to be combined with the stuff on data selection and analysis.  Surely that chapter is getting too big even without this stuff.}
%With the Simulations:
\begin{itemize}
	\item Multithreading.  There was multithreading, and it took a stupidly long time to get working.
	\item Update G4 event generator to be able to model non-zero scalar and tensor coupling.  These things show up in $\Abeta$ too, not just in $\bFierz$.  Though, the effects on $\Abeta$ are much smaller.
	\item Run 3 sets of G4 simulations with a bunch of statistics (N events, for data with like N/10 events).  Each one has the same nominal value of $\Abeta$, but with 3 different values of the scalar coupling $C_S$:  zero, and +/-(whatever).  Keep $C_T=0$.  Because reasons, we're not really able to distinguish between $C_S$ and $C_T$ in this experiment anyway, so might as well keep the analysis simple.
	\item Just run one set of 0.02*N events for the two percent branch.  We can't neglect it, but it isn't going to change (much?) when we adjust BSM couplings.
	\item Match cuts in simulated data up to the cuts on experimental data.  Obviously.  DSSD cut, DSSD energy, one hit DSSD, one hit scint.  TOF cut, which requires a whole extra model of background in the TOF spectrum..
		\begin{itemize}
		\item Suppose background in the TOF spectrum is coming from decays of atoms that have gotten themselves stuck to surfaces within the chamber...
		\item Run G4 to get a beta TOF spectrum (w.r.t. the decay)
		\item Run COMSOL (credit to Alexandre) to track low-energy SOEs through the electric field from wherever they started, into the detectors.  Energy spectra from Levinger.
		\item Combine G4 and COMSOL spectra, event-by-event, while requiring that both the beta detector and the eMCP are hit according to the set of random numbers generated by each monte carlo separately.  Then, the beta and SOE will each have a TOF from decay to detector, and subtracting one from the other gives a timing spectrum that can be observed experimentally.  See Fig.~\ref{fig:soetof}.
		\item Upper limit for the fraction of events generated this way can be estimated by assuming that all losses from the trap not due to radioactive decay emerge isotropically from the trap and then stick to whatever chamber wall is in its path.  This upper limit is too big by a factor of 2.
		\end{itemize}
	\item For each of those 3 simulations, sort the ``good'' data according to emission angle relative to the detector.  Do each detector individually.  For both polarizations.
	\item Assemble the (simulated) superratio asymmetry.  We'll compare it to data, and the $\chi^2$ from that comparison will be our figure of merit.  
	\item We can make a whole 2D parameter space for different values of $\Abeta$ and $\bFierz$, and compare them all (via their superratio asymmetries) to the experimental data.  We get the ``best'' values of $\Abeta$ and $\bFierz$, where $\chi^2$ is minimized.
	\item We can do this whole thing again for simulated data sets with different values of parameters that we vary as systematics.  Note how the best values of $\Abeta$ and $\bFierz$ change when each of the systematics are varied.
	\item Then there's the lineshape thing.  My god, this wasn't nearly as useful as I thought it was going to be.  Certainly not worth the whole goddamn year that I spent on it.
\end{itemize}


