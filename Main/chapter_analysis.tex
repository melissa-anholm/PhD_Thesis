% !TEX root = ../thesis_main.tex



%%%% --- * --- %%%%	
\clearpage
\chapter{Analysis}
\label{analysis_chapter}

Right, so.  Here's how I processed the data into an answer.

With the Data:
\begin{itemize}
	\item Higher-level data cleaning.  Discard events during parts of the duty cycle when atoms weren't polarized.  Discard events near a recorded spark time.  Discard events when the photoionization laser fires.  Discard events when the LED pulser used to calibrate the scintillators fires.  
	\item Split up runs into sets, to account for changing experimental conditions.  Possibly I should list what the differences between runs were somewhere.  But not in this section.
	\item Using the ``other'' data set with the rMCP:  Measure the trap position/size/velocity/expansion with the rMCP and with the camera.  Necessitates calibrating the rMCP.  Also measure polarization.
	\item Make some more careful cuts to clean the data.  
		\begin{itemize}
		\item Discard events without a ``good'' DSSD hit.  Eliminates vast majority of background 511s.  Necessitates having a definition of what a ``good'' DSSD hit is.  It's subtle enough that we'll want to leave some part of this definition of ``good'' to be varied as a systematic effect.
		\item Discard events where SOE-Beta TOF falls outside a certain range.  Necessitates picking a ``good'' range.  The precise definition of ``good'' is varied as a systematic.
		\end{itemize}
\end{itemize}

With the Simulations:
\begin{itemize}
	\item Update G4 event generator to be able to model non-zero scalar and tensor coupling.   
	\item Run 3 sets of G4 simulations with a bunch of statistics (N events, for data with like N/10 events).  Each one has the same nominal value of $\Abeta$, but with 3 different values of the scalar coupling $C_S$:  zero, and +/-(whatever).  Keep $C_T=0$.  Because reasons, we're not really able to distinguish between $C_S$ and $C_T$ in this experiment anyway, so might as well keep the analysis simple.
	
	
\end{itemize}

