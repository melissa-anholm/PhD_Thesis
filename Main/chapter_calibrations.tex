% !TEX root = ../thesis_main.tex



%%%% --- * --- %%%%	
\clearpage	
\chapter{Calibrations}
\label{calibrations_chapter}
	
%\section{Polarization}
%	%\\*
%	Polarization measurement was conducted on a different set of data, collected in between the measurements used for $A_{\mathrm{\beta}}$ and $b_{\mathrm{Fierz}}$, and at a higher electric field, because we were unable to run both our MCP detectors simultaneously.  
%	
%\section{Trap Position}
%	%\\*
%	Measured using the same dataset that was used to quantify the polarization.  The trap drifts slightly over the course of our data collection.  Describe the rMCP calibration needed to extract this info.  
%

\section{Cloud Measurements via Photoionization}
\label{cloud}
\label{photoions}
In order to measure properties of the trapped $^{37}\textrm{K}$ cloud, a 10\,kHz pulsed laser at 355\,nm is directed towards the cloud.  These photons have sufficient energy to photoionize neutral $^{37}\textrm{K}$ from its excited atomic state, which is populated by the trapping laser when the MOT is active, releasing 0.77\,eV of kinetic energy, but do not interact with ground state $^{37}\textrm{K}$ atoms.  The laser is of sufficiently low intensity that the great majority of excited state atoms are \emph{not} photoionized, so the technique is only very minimally destructive.  

Because an electric field has been applied within this region (see Section~\ref{field}) the $^{37}\textrm{K}^+$ ions are immediately pulled into the detector on one side of the chamber, while the freed $e^-$ is pulled towards the detector on the opposite side of the chamber.  Because  $^{37}\textrm{K}^+$ is quite heavy relative to its initial energy, it can be treated as moving in a straight line directly to the detector, where its hit position on the microchannel plate is taken as a 2D projection of its position within the cloud.  Similarly, given a sufficient understanding of the electric field, the time difference between the laser pulse and the microchannel plate hit allows for a calculation of the ion's initial position along the third axis.  

\note{As a check:  the camera measurements for photons from de-excitation.  It's like ~30 degrees from .... one of the axes.  Also, there's an unknown additional delay between some of our DAQ channels that can't be explained by accounting for cable lengths, so we really like having the check there.}

With this procedure, it is possible to produce a precise map of the cloud's position and size, both of which are necessary for the precision measurements of angular correlation parameters that are of interest to us here.  However, it also allows us to extract a third measurement:  the cloud's polarization.

The key to the polarization measurement is that only atoms in the excited atomic state can be photoionized via the 355 nm laser.  While the MOT runs, atoms are constantly being pushed around and excited by the trapping lasers, so this period of time provides a lot of information for characterizing the trap size and position.  When the MOT is shut off, the atoms quickly return to their ground states and are no longer photoionized until the optical pumping laser is turned on.  As described in Section~\ref{op}, and in greater detail in~\cite{ben_OP}, the optical pumping process involves repeatedly exciting atoms from their ground states until the atoms finally cannot absorb any further angular momentum and remain in their fully-polarized (ground) state until they are perturbed.  Therefore, there is a sharp spike in excited-state atoms (and therefore photoions) when the optical pumping begins, and none once the cloud has been completely polarized.  The number of photoion events that occur once the sample has been maximally polarized, in comparison with the size and shape of the initial spike of photoions, provides a very precise characterization of the cloud's final polarization~\cite{ben_OP}.


\note{Trap position -- Measured using the same dataset that was used to quantify the polarization.  The trap drifts slightly over the course of our data collection.  Describe the rMCP calibration needed to extract this info.}
\note{Polarization measurement was conducted on a different set of data, collected in between the measurements used for $A_{\mathrm{\beta}}$ and $b_{\mathrm{Fierz}}$, and at a higher electric field, because we were unable to run both our MCP detectors simultaneously.  }
\note{Need to describe how polarization works.  Needs a level diagram.  Needs another(?) level diagram for the photoionization, and maybe a third for the MOT.  Can I combine them all?  idk.}

Anyway, here are some nice tables describing the atom cloud, for each of 3 runsets:



\begin{table}[h!!!!t]
	\begin{center}
	\begin{tabular}{ c | r || lcl | lcl || lcl | lcl |}
	%	\hline
	%		\multicolumn{5}{ | c | }{Cloud Measurements -- $x$-Projection} \\
	%	\hline\hline
	%	\cline{3-6}
			\multicolumn{2}{  c  }{ } & 
				\multicolumn{3}{  c  }{ \!\!Initial Position\!\! } &  \multicolumn{3}{   c  }{ Final Position } &  \multicolumn{3}{   c  }{ Initial Size } &  \multicolumn{3}{   c  }{ Final Size } \\
		%	\cline{2-6}
		%	\hline
		%	\cline{2-6}
			\cline{2-14}
			\multirow{3}{*}{Runset B}  & $x$ & \,\,1.77 & \!\!$\!\! \pm  \!\!$\!\! & 0.03   & \,\,2.06   & \!\!$\!\! \pm  \!\!$\!\! & 0.08    & 0.601 & \!\!$\!\! \pm  \!\!$\!\! & 0.013 & 1.504 & \!\!$\!\! \pm  \!\!$\!\! & 0.047 \\
								& $y$ & -3.51    & \!\!$\!\! \pm  \!\!$\!\! & 0.04   & -3.33     & \!\!$\!\! \pm  \!\!$\!\! & 0.05    & 1.009 & \!\!$\!\! \pm  \!\!$\!\! & 0.013 & 1.551 & \!\!$\!\! \pm  \!\!$\!\! & 0.018 \\
								& $z$ & -0.661  & \!\!$\!\! \pm  \!\!$\!\! & 0.005 & -0.551   & \!\!$\!\! \pm  \!\!$\!\! & 0.021  & 0.891 & \!\!$\!\! \pm  \!\!$\!\! & 0.004 & 1.707 & \!\!$\!\! \pm  \!\!$\!\! & 0.015 \\
			\cline{2-14}
			\multirow{3}{*}{Runset C}  & $x$ & \,\,2.22  & \!\!$\!\! \pm  \!\!$\!\! & 0.05  & \,\,2.33   & \!\!$\!\! \pm  \!\!$\!\! & 0.11    & 1.18   & \!\!$\!\! \pm  \!\!$\!\! & 0.04   & 1.538 & \!\!$\!\! \pm  \!\!$\!\! & 0.087 \\
								& $y$ & -3.68     & \!\!$\!\! \pm  \!\!$\!\! & 0.04  & -3.31      & \!\!$\!\! \pm  \!\!$\!\! & 0.06   & 0.965 & \!\!$\!\! \pm  \!\!$\!\! & 0.012 & 1.460 & \!\!$\!\! \pm  \!\!$\!\! & 0.030 \\
								& $z$ & -0.437   & \!\!$\!\! \pm  \!\!$\!\! & 0.09  & -0.346    & \!\!$\!\! \pm  \!\!$\!\! & 0.037 & 0.927 & \!\!$\!\! \pm  \!\!$\!\! & 0.007 & 1.797 & \!\!$\!\! \pm  \!\!$\!\! & 0.026 \\
			\cline{2-14}
			\multirow{3}{*}{Runset D}  & $x$ & \,\,2.274 & \!\!$\!\! \pm  \!\!$\!\! & 0.012 & \,\,2.46 & \!\!$\!\! \pm  \!\!$\!\! & 0.06   & 0.386 & \!\!$\!\! \pm  \!\!$\!\! & 0.016 & 1.382 & \!\!$\!\! \pm  \!\!$\!\! & 0.046 \\
								& $y$ & -4.54      & \!\!$\!\! \pm  \!\!$\!\! & 0.04   & -4.28    & \!\!$\!\! \pm  \!\!$\!\! & 0.04   & 0.986 & \!\!$\!\! \pm  \!\!$\!\! & 0.08   & 1.502 & \!\!$\!\! \pm  \!\!$\!\! & 0.013 \\
								& $z$ & -0.587    & \!\!$\!\! \pm  \!\!$\!\! & 0.04   & -0.481  & \!\!$\!\! \pm  \!\!$\!\! & 0.018 & 0.969 & \!\!$\!\! \pm  \!\!$\!\! & 0.003 & 1.861 & \!\!$\!\! \pm  \!\!$\!\! & 0.013 \\
			\cline{2-14}
	\end{tabular}
	\label{table:cloudpositions}
	\end{center}
	\caption[Cloud Position and Size]{Cloud Positions and Sizes -- Measured immediately before and immediately following the optical pumping phase of the trapping cycle.  All entries are expressed in units of mm, and the ``size" parameters describe the gaussian width.}
\end{table}

\note{Also, we noticed the trap drifting after one of the runs, because one of the batteries on one of the thingies adjusting the laser frequency (I think) was failing. }
\note{describe how you'd turn this into a physical description of the cloud, with like a temperature and a sail velocity and shit.  with equations.}

%\begin{table}[h!!!!t]
%	\begin{center}
%	\begin{tabular}{ | c | r || l | l || l | l |}
%	%	\hline
%	%		\multicolumn{5}{ | c | }{Cloud Measurements -- $x$-Projection} \\
%	%	\hline\hline
%		\cline{3-6}
%			\multicolumn{2}{  c || }{ } & 
%				\multicolumn{1}{  l  }{ Initial Position } &  \multicolumn{1}{   l  }{ Final Position } &  \multicolumn{1}{   l  }{ Initial Size } &  \multicolumn{1}{   l |  }{ Final Size } \\
%		%	\cline{2-6}
%			\hline
%			\multirow{3}{*}{Runset B}  & $x$ & thing & thing & thing & thing \\
%								& $y$ & thing & thing & thing & thing \\
%								& $z$ & thing & thing & thing & thing \\
%			\hline
%			\multirow{3}{*}{Runset C}  & $x$ & thing & thing & thing & thing \\
%								& $y$ & thing & thing & thing & thing \\
%								& $z$ & thing & thing & thing & thing \\
%			\hline
%			\multirow{3}{*}{Runset D}  & $x$ & thing & thing & thing & thing \\
%								& $y$ & thing & thing & thing & thing \\
%								& $z$ & thing & thing & thing & thing \\
%		\hline
%	\end{tabular}
%	\label{table:nonlin}
%	\end{center}
%	\caption{Nonlinear Model Results}
%\end{table}
%
%
%\begin{table}[h!!!!t]
%	\begin{center}
%	\begin{tabular}{ | l || l | l | l | l |}
%		\hline
%			\multicolumn{1}{  | c  }{ } & 
%				\multicolumn{1}{   c  }{ $x_i$} &  \multicolumn{1}{   c  }{ $x_f$} &  \multicolumn{1}{   c  }{ $\sigma_{xi}$ } &  \multicolumn{1}{   c |  }{ $\sigma_{xf}$ } \\
%			\hline
%			Runset B & thing & thing & thing & thing \\
%			\hline
%			Runset C & thing & thing & thing & thing \\
%			\hline
%			Runset D & thing & thing & thing & thing \\
%		\hline
%	\end{tabular}
%	\label{table:nonlin}
%	\end{center}
%	\caption{Nonlinear Model Results}
%\end{table}


\section{Beta Detectors}
	%\\*
	Energy calibration for the scintillator+PMT setup changed dramatically at one point.    Describe how calibration was done.  Also describe how the DSSD calibration was done, even though it wasn't implemented by me.    
	\note{How the fuck WERE these things done?!}
\section{The eMCP}
	%\\*
	I can describe the eMCP calibration here, even though it mostly wasn't implemented by me.  It is tangentially relevant to data selection and background estimation by providing an experimental energy spectrum for shake-off electrons.  

\section{The rMCP}
	I did this, and they're absolutely needed to make any sense of the trap position data.  












