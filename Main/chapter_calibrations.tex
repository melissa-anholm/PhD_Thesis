% !TEX root = ../thesis_main.tex



%%%% --- * --- %%%%	
\clearpage	
\chapter{Calibrations and Data Selection}
\label{calibrations_chapter}
	
%\section{Polarization}
%	%\\*
%	Polarization measurement was conducted on a different set of data, collected in between the measurements used for $A_{\mathrm{\beta}}$ and $b_{\mathrm{Fierz}}$, and at a higher electric field, because we were unable to run both our MCP detectors simultaneously.  
%	
%\section{Trap Position}
%	%\\*
%	Measured using the same dataset that was used to quantify the polarization.  The trap drifts slightly over the course of our data collection.  Describe the rMCP calibration needed to extract this info.  
%

\section{Cloud Measurements via Photoionization}
\label{cloud}
\label{photoions}
In order to measure properties of the trapped $^{37}\textrm{K}$ cloud, a 10\,kHz pulsed laser at 355\,nm is directed towards the cloud.  These photons have sufficient energy to photoionize neutral $^{37}\textrm{K}$ from its excited atomic state, which is populated by the trapping laser when the MOT is active, releasing 0.77\,eV of kinetic energy, but do not interact with ground state $^{37}\textrm{K}$ atoms.  The laser is of sufficiently low intensity that the great majority of excited state atoms are \emph{not} photoionized, so the technique is only very minimally destructive.  
\note{Probably worth mentioning that we test this stuff offline on stable \isotope[41]{K}. }

Because an electric field has been applied within this region (see Section~\ref{field}) the $^{37}\textrm{K}^+$ ions are immediately pulled into the detector on one side of the chamber, while the freed $e^-$ is pulled towards the detector on the opposite side of the chamber.  Because  $^{37}\textrm{K}^+$ is quite heavy relative to its initial energy, it can be treated as moving in a straight line directly to the detector, where its hit position on the microchannel plate is taken as a 2D projection of its position within the cloud.  Similarly, given a sufficient understanding of the electric field, the time difference between the laser pulse and the microchannel plate hit allows for a calculation of the ion's initial position along the third axis.  

\note{As a check:  the camera measurements for photons from de-excitation.  It's aimed 35 degrees from vertical, with its horizontal axis the same as ..... one of the other axes.  I think it's the TOF axis.  I can check this when my computer comes back.   Also, there's an unknown additional delay between some of our DAQ channels that can't be explained by accounting for cable lengths, so we really like having the check there.}

With this procedure, it is possible to produce a precise map of the cloud's position and size, both of which are necessary for the precision measurements of angular correlation parameters that are of interest to us here.  However, it also allows us to extract a third measurement:  the cloud's polarization.

The key to the polarization measurement is that only atoms in the excited atomic state can be photoionized via the 355 nm laser.  While the MOT runs, atoms are constantly being pushed around and excited by the trapping lasers, so this period of time provides a lot of information for characterizing the trap size and position.  When the MOT is shut off, the atoms quickly return to their ground states and are no longer photoionized until the optical pumping laser is turned on.  As described in Section~\ref{op}, and in greater detail in~\cite{ben_OP}, the optical pumping process involves repeatedly exciting atoms from their ground states until the atoms finally cannot absorb any further angular momentum and remain in their fully-polarized (ground) state until they are perturbed.  Therefore, there is a sharp spike in excited-state atoms (and therefore photoions) when the optical pumping begins, and none once the cloud has been completely polarized.  The number of photoion events that occur once the sample has been maximally polarized, in comparison with the size and shape of the initial spike of photoions, provides a very precise characterization of the cloud's final polarization~\cite{ben_OP}.



\note{Trap position -- Measured using the same dataset that was used to quantify the polarization.  The trap drifts slightly over the course of our data collection.  Describe the rMCP calibration needed to extract this info.}
\note{Polarization measurement was conducted on a different set of data, collected in between the measurements used for $A_{\mathrm{\beta}}$ and $b_{\mathrm{Fierz}}$, and at a higher electric field, because we were unable to run both our MCP detectors simultaneously.  }
%\note{Need to describe how polarization works.  Needs a level diagram.  Needs another(?) level diagram for the photoionization, and maybe a third for the MOT.  Can I combine them all?  idk.}

Anyway, here is a nice table describing the atom cloud, for each of 3 runsets, and I'll immediately reference it right now, as Table~\ref{table:cloudpositions}:


% !TEX root = ../thesis_main.tex
%
%
%
\begin{table}[h!!!!t]
	\begin{center}
	\begin{tabular}{ c  | r || lcl | lcl || lcl | lcl |}
			\multicolumn{1}{c}{ Runsets  }{ } & \multicolumn{1}{  c  }{ } & 
				\multicolumn{3}{  c  }{ \!\!Initial Position\!\! } &  \multicolumn{3}{   c  }{ Final Position } &  \multicolumn{3}{   c  }{ Initial Size } &  \multicolumn{3}{   c  }{ Final Size } \\
			\cline{2-14}
			\multirow{3}{*}{EB $\leftarrow$ RB} %&\multirow{3}{*}{RB}  
								& $x$ & \,\,1.77 & \!\!$\!\! \pm  \!\!$\!\! & 0.03   & \,\,2.06   & \!\!$\!\! \pm  \!\!$\!\! & 0.08    & 0.601 & \!\!$\!\! \pm  \!\!$\!\! & 0.013 & 1.504 & \!\!$\!\! \pm  \!\!$\!\! & 0.047 \\
								& $y$ & -3.51    & \!\!$\!\! \pm  \!\!$\!\! & 0.04   & -3.33     & \!\!$\!\! \pm  \!\!$\!\! & 0.05    & 1.009 & \!\!$\!\! \pm  \!\!$\!\! & 0.013 & 1.551 & \!\!$\!\! \pm  \!\!$\!\! & 0.018 \\
								& $z$ & -0.661  & \!\!$\!\! \pm  \!\!$\!\! & 0.005 & -0.551   & \!\!$\!\! \pm  \!\!$\!\! & 0.021  & 0.891 & \!\!$\!\! \pm  \!\!$\!\! & 0.004 & 1.707 & \!\!$\!\! \pm  \!\!$\!\! & 0.015 \\
			\cline{2-14}
			\multirow{3}{*}{EC $\leftarrow$ RD} %&\multirow{3}{*}{RD}  
								& $x$ & \,\,2.22  & \!\!$\!\! \pm  \!\!$\!\! & 0.05  & \,\,2.33   & \!\!$\!\! \pm  \!\!$\!\! & 0.11    & 1.18   & \!\!$\!\! \pm  \!\!$\!\! & 0.04   & 1.538 & \!\!$\!\! \pm  \!\!$\!\! & 0.087 \\
								& $y$ & -3.68     & \!\!$\!\! \pm  \!\!$\!\! & 0.04  & -3.31      & \!\!$\!\! \pm  \!\!$\!\! & 0.06   & 0.965 & \!\!$\!\! \pm  \!\!$\!\! & 0.012 & 1.460 & \!\!$\!\! \pm  \!\!$\!\! & 0.030 \\
								& $z$ & -0.437   & \!\!$\!\! \pm  \!\!$\!\! & 0.09  & -0.346    & \!\!$\!\! \pm  \!\!$\!\! & 0.037 & 0.927 & \!\!$\!\! \pm  \!\!$\!\! & 0.007 & 1.797 & \!\!$\!\! \pm  \!\!$\!\! & 0.026 \\
			\cline{2-14}
			\multirow{3}{*}{ED $\leftarrow$ RE} %&\multirow{3}{*}{RE}  
								& $x$ & \,\,2.274 & \!\!$\!\! \pm  \!\!$\!\! & 0.012 & \,\,2.46 & \!\!$\!\! \pm  \!\!$\!\! & 0.06   & 0.386 & \!\!$\!\! \pm  \!\!$\!\! & 0.016 & 1.382 & \!\!$\!\! \pm  \!\!$\!\! & 0.046 \\
								& $y$ & -4.54      & \!\!$\!\! \pm  \!\!$\!\! & 0.04   & -4.28    & \!\!$\!\! \pm  \!\!$\!\! & 0.04   & 0.986 & \!\!$\!\! \pm  \!\!$\!\! & 0.08   & 1.502 & \!\!$\!\! \pm  \!\!$\!\! & 0.013 \\
								& $z$ & -0.587    & \!\!$\!\! \pm  \!\!$\!\! & 0.04   & -0.481  & \!\!$\!\! \pm  \!\!$\!\! & 0.018 & 0.969 & \!\!$\!\! \pm  \!\!$\!\! & 0.003 & 1.861 & \!\!$\!\! \pm  \!\!$\!\! & 0.013 \\
			\cline{2-14}
	\end{tabular}
	\end{center}
%	\note{These positions are for the *electron* runsets of those names.  Might want a chart of which rMCP runs are used to measure position for which eMCP runs.  Possibly in that other section.}
	\note{Sig figs here need work.}
	\note{Parameters measured with the recoil runs, and applied on the appropriate electron runs.}
	\caption[Cloud Position and Size]{Cloud Positions and Sizes -- Measured immediately before and immediately following the optical pumping phase of the trapping cycle.  Measurements are evaluated using rMCP runs, and are taken to describe the cloud during associated eMCP runs as well.  All entries are expressed in units of millimetres, and the size parameters describe the gaussian width.}
	\label{table:cloudpositions}
\end{table}




\note{Also, we noticed the trap drifting after one of the runs, because one of the batteries on one of the thingies adjusting the laser frequency (I think) was failing. }
\note[color=jb]{JB:  ``If we rejected the data with the MOT moving (indeed a battery determining the voltage controlled oscillator frequency offset between absorption in stable \isotope[41]{K} cell and the \isotope[37]{K} resonance) then that's all you need to say.''}
\note{describe how you'd turn this into a physical description of the cloud, with like a temperature and a sail velocity and shit.  with equations.}


\section{Plastic Scintillators}
	Energy calibration for the scintillator+PMT setup changed dramatically at one point.    Describe how calibration was done.  
	\note{How the fuck WERE these things done?!}
	\note[color=jb]{JB:  ``You can describe anything you did differently or improved, but you can and should otherwise defer all details of the scintillator calibration and DSSD calibration to Ben's paper and his thesis and Spencer's.  E.g. Section~\ref{section:bb1_systematics} ``statistical agreement between BB1 X and Y detectors' energies only makes a small effect on results" does not need the technical details beyond that statement."
	\label{thesisconventionjb} }
	\note[color=jb]{JB:  ``If you have some way of documenting the coding you used, that would be great."  ... yeah, it would, wouldn't it?}
	
\section{Strip Detectors}
	Also describe how the DSSD calibration was done, even though it wasn't implemented by me. 
	\note{How the fuck WERE these things done?!}

\section{The eMCP}
	I can describe the eMCP calibration here, even though it mostly wasn't implemented by me.  It is tangentially relevant to data selection and background estimation by providing an experimental energy spectrum for shake-off electrons.  It's actually a pretty neat algorithm that I basically wasn't involved with.
	\note[color=jb]{JB:  eMCP.  You need to describe the timing information obtained.  You also need a statement of whether or not you used the position information in your cuts.}

\missingfigure{Needs an SOE timing spectrum.  At least one of them.  Experimental and simulated.  Also, I have to describe how I did the simulating, and how I check that it's OK despite the fact that the simulated spectrum looks nothing like the experimental spectrum.}

\section{The rMCP}
	I did this, and they're absolutely needed to make any sense of the trap position data.  
\missingfigure{Needs two pictures from the rMCP with the grid lines -- before and after corrections are performed.  Just for fun, I could throw in one with the stupid stripes.}
	These calibrations are done during AC-MOT time, and we're actually interested in the rMCP data taken during OP time.  Can I find pictures to estimate the size of the change resulting from the magnetic field?  In any case, the change is pretty small.  

\missingfigure{Position as a function of run number.  I have this somewhere.}











