% !TEX root = ../thesis_main.tex



%%%% --- * --- %%%%	
\clearpage
\chapter{Introduction}
\label{intro_chapter}
\note[color=jb]{JB on the contents of Chapter 1:
\\
Move what you have in 1.1 and 1.3 to the first section of
Chapter 2, and otherwise omit Chapter 1.
}

\note[color=jb]{more from JB on the contents of Chapter 1:
\\
With three changes:
\\ ... \\
1)"and we shall be interested especially in scalar (S) and tensor (T) couplings."
-> "our observable is mostly sensitive to scalar (S) and tensor (T) couplings."
\\ ...\\
2)
"These couplings all refer to parameters in a Lagrangian that takes the
relativistic inner product of a current for the lepton with a current for the
proton or neutron.
The resulting Lagrangian must be a scalar under Lorentz transformations, so
these currents must have transformations like these V,A,S, and T and be
combined into a scalar."
\\...\\
3) Add one reference to the latest review:
\\
Adam Falkowski, Martín González-Alonso, Oscar Naviliat-Cuncic. Comprehensive analysis of beta
decays within and beyond the Standard Model. Journal of High Energy Physics, Springer, 2021, 04,
pp.126. 10.1007/JHEP04(2021)126.
\\ ... \\
You have time for nothing else.
}


\section{Background and Motivation}
The nuclear weak force is one of four fundamental forces described within physics.  It mediates the process of beta decay, which is of particular interest to us here.  Although beta decay is generally well understood, it presents a unique opportunity to search for physics beyond the Standard Model within the behaviour of the weak coupling.  By observing the kinematics and angular correlations involved in the decay process, one gains access to a wealth of information about the precise form of the operators mediating the decay.  

According to the predictions of the Standard Model, beta decay involves only so-called vector (V) and axial-vector (A) couplings, with a relative phase angle producing the left-handed ``(V--A)'' form of the interaction.  \aside{Did I even get this right?  Is the phase angle really what makes it left-handed? \\ JB says:  \\ ... \\ Relative sign.  look at the quark-lepton Lagrangian, which has $(1 \pm \gamma_5)$ } There exists an extensive body of experimental evidence to demonstrate that this is overall a very good description of the beta decay process.  \aside{Cite a bunch of people here.  (Who?)  Might be nice to have a picture also.}  Despite the success of the (V--A) model, the additional presence of certain other ``exotic'' couplings cannot be entirely ruled out, and we shall be interested especially in scalar (S) and tensor (T) couplings.  \note{According to present limits, these couplings would have to be pretty small relative to the (V) and (A) couplings.}

\bluetodo{Need to figure out how the exotic couplings actually work, mathematically.  What the fuck does ``$(V-A)$'' even *mean*?  IIRC John wants a brief mention of $\gamma_5$'s and $\gamma_\mu$'s, and probably a brief mention of whatever mumble-mumble group is mumble-mumble represented or something.
\\
...
\\
JB says:
\\
the current transforms like a Lorentz scalar or tensor -- this does not refer to the angular momentum.
\\
If you write down the Lagrangian for beta decay, that's eough. All these things refer to the structure of the Lagrangian. The theory considers all possible Lorentz transformations of the currents. 
\\
Please don't talk about SU(2)xU(1) for electroweak unification. It's textbook material that's beyond the scope.
}

\note[color=jb]{JB on intuitive concepts that are missing:  
\\
The SM couples to left-handed neutrinos and right-handed antineutrinos. Since the neutrinos only have weak interactions, there are no right-handed nu's nor left-handed antinu's in nature. The neutrino asymmetry $B_\nu$ is a number with no energy dependence. 
\\
Similarly, the SM weak interaction only couples to right-handed positrons and left-handed electrons. Since these are massive particles, the average helicity of positrons is not 1, but instead v/c. One can always boost to a frame where the positron keeps its circulation but is moving in the opposite direction. This is why the beta asymmetry is A v/c, not just A.
\\
The Fierz term's additional energy dependence of m/E also comes from helicity arguments, stemming from the fact that it still is coupling to SM nu's and antinu's only, so the beta's are generated with wrong handedness.  
\\
The details are built at 4th-year undergrad level in Garcia's paper with his student and postdoc~\cite{hong_sternberg_garcia}.
\\
The beta asymmetry dependence on the Fierz term only comes through the normalization of $W(\theta) = 1 + \bFierz m/E + \Abeta \cos(\theta)$.
\\ i.e.:
\\$W'(\theta) = 1 + \Abeta/(1+ \bFierz m/E) \cos(\theta)$.  (the angular distribution must be unity where cos(theta) vanishes, by definition).
}


\section{Exotic Couplings}
%	In particular, we're interested in so-called scalar and tensor couplings within the nuclear weak force. Standard model beta decay involves only vector and axial-vector couplings, combined with a ``$(V-A)$'' handedness (left-handed).  

\section{Fierz Interference -- The Physical Signature}
	The physical effects resulting from the presence of scalar or tensor couplings include a small perturbation to the energy spectrum of betas produced by radioactive decay.  

\missingfigure{I need that simulated picture of the different beta energy spectra, with different values of $\bFierz$.}
\note[color=jb]{JB on that missing figure:    ``A dependence of Abeta on beta energy is also introduced.
\\
UCNA fits energy spectrum and Abeta[Ebeta] simultaneously now."}

\section{Present Limits}
	A bit about other people's physics.

\section{A Toy Experiment}
	A quick overview of how an experiment like this one would be set up to extract the physics of interest, to keep the reader from getting too lost in the rest of the thesis.
\note{Do I really even *want* to include a toy experiment?  And would I want to do it here??  What even is the point?  I think in the past I decided it was easier to build up a description of .... something .... starting this way.  But why??  Possibly as I continue to add content, it will become obvious again why I originally wanted to do this.}
\note[color=jb]{JB says:  ``Your experiment is simple enough to describe without a toy.''  (He's said this twice now, so I should probably just kill this section.)}
