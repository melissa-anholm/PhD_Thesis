% !TEX root = ../thesis_main.tex



%%%% --- * --- %%%%	
\clearpage
\chapter{Background and Motivation}
\label{background_chapter}

\section{Exotic Couplings}
	In particular, we're interested in so-called scalar and tensor couplings within the nuclear weak force. Standard model beta decay involves only vector and axial-vector couplings, combined with a ``$(V-A)$'' handedness (left-handed).  
%\missingfigure{Make a sketch of the structure of a trebuchet.}

\section{Fierz Interference -- The Physical Signature}
	The physical effects resulting from the presence of scalar or tensor couplings include a small perturbation to the energy spectrum of betas produced by radioactive decay.  

\missingfigure{I need that simulated picture of the different beta energy spectra, with different values of $\bFierz$.}

\section{Present Limits}
	A bit about other people's physics.

\section{A Toy Experiment}
	A quick overview of how an experiment like this one would be set up to extract the physics of interest, to keep the reader from getting too lost in the rest of the thesis.
\note{Do I really even *want* to include a toy experiment?  And would I want to do it here??  What even is the point?  I think in the past I decided it was easier to build up a description of .... something .... starting this way.  But why??  Possibly as I continue to add content, it will become obvious again why I originally wanted to do this.}


%%%% --- * --- %%%%	
% !TEX root = ../thesis_main.tex



\clearpage
\chapter{Theoretical Overview}
\label{theory_chapter}

\section{The Basics of Beta Decay}
	%\\*
	Standard Model beta decay is well understood.  The Fermi model of beta decay is in all the textbooks, but you have to dig slightly harder to understand Gamow-Teller or mixed decays, all of which are relevant here.  
	
	via Krane~\cite{krane}
	Under the Allowed Approximation, we require that a beta decay may not carry away any orbital angular momentum, because we treat the nucleus as pointlike \aside{Is this even true?  The pointlike thing?} and work in the CM frame.  An Allowed decay can, however, change the total nuclear angular momentum, because the outgoing leptons have spin$=1/2$ and therefore carry angular momentum.  Therefore, in an allowed decay, the total nuclear angular momentum must always change by either $0$ or $1$.  
	
	From a 2006 paper by Severijns et al ~\cite{severijns_beck_cuncic_2006}, the selection rules for an allowed transition are:
	
\bea
\Delta I = I_f - I_i = \{0, \pm 1\} \\ 
\hat{\Pi}_i \, \hat{\Pi}_f = +1
\eea

	Then, you can separate the allowed transitions into singlet (anti-parallel lepton spins, $S=0$ -- a Fermi transition) and triplet states (parallel lepton spins, $S=1$ -- a Gamow-Teller transition).
	
	
	Fermi decays are so-called ``vector'' interactions, and happen when the spin of the two leptons involved are antiparallel, so there can be no change in angular momentum (at least in the case of the Allowed approximation).  
	
	Gamow-Teller decays involve two leptons with parallel spins, so the decay must change the projection of the nuclear angular momentum, $M_I$, by exactly one unit (in the case of the Allowed approximation).  They transition may or may not simultaneously change the total nuclear spin, $I$, by one unit.  These are ``axial-vector'' interactions.  (Note that $I=0 \rightarrow I=0$ interactions are never Gamow-Teller decays.  
	
	Probably everything in this section is yoinked from ~\cite{wong1990}, pg 212.  
	
	
%\section{JTW Formalism}	
%	%\\*
%	Describes how to search for a variety of BSM terms within beta decay.  Does not account for certain well-understood effects of similar (or greater) magnitude.
%	
%	% !TEX root = ../thesis_main.tex
%
%
%
% "A PDF for the People"
\bea
\textrm{d}^5\Gamma_{\textrm{JTW}} \!\!\!\! \!\! && \equiv \,\,
%\omega(\cdots) \mathrm{d} \E \, \dOmegae \, \dOmeganu \,\, = 
\,\,  \frac{1}{(2\pi)^5} \, \FFpm \pe \Ee (E_0 - \Ee)^2 \dEe \, \dOmegae \, \dOmeganu \, \nonumber\\ 
&&	\!\!\!\!  \!\!\!\!  \!\!\!\!  
	\times \,\, \xi \left[
	1 + \a \frac{\vecpe\cdot\vecpnu}{\Ee\Enu} + \bFierz \frac{\m c^2}{\Ee} 
    + \,\,  \calign \,\, \Talign(\vecJ) 
	\left(
		\frac{\vecpe \cdot \vecpnu}{3\Ee\Enu}
		- \frac{ (\vecpe\cdot \hatj) (\vecpnu\cdot\hatj) }{\Ee\Enu}
	\right)
	\!
\right. \nonumber\\ 
&&	\!\!\!\!  \!\!\!\!  \!\!\!\!  
	\left. + 
	 \frac{\vecJ}{J} \cdot
	\left(
		\A \frac{\vecpe}{\Ee} 
		+ \B \frac{\vecpnu}{\Enu} 
		+ \D \frac{\vecpe \times \vecpnu}{\Ee\Enu} 
	\right)
\right],
\label{equation:jtw_master}
\eea
%\textrm{d}^5\Gamma_{\textrm{JTW}} \,\, \equiv \,\, \omega(\cdots) \!\!\!\! \!\!\!\! \!\!\!\! \!\!\!\! && \,\,\,\, \,\,\,\, \mathrm{d} \E \, \dOmegae \, \dOmeganu 
%	% equation:jtw_master
%	
%\note{Probably I should now give values for things, or expressions for letters, or something.  }
%We haven't integrated out the neutrino momentum.  Neutrino energy itself is a redundant parameter, I think, because we are already using an endpoint energy and a beta energy, and we are not taking recoil-order effects into account.
%
%For ``convenience'', let's define a nuclear alignment term, $\Talign$, so that:
%\bea
%\Talign(\vecJ) &=& \TalignExpand
%\eea
%
%
%
%\section{Holstein Formalism}
%	An in-depth mathematical description of beta decay, including many smaller effects.  It does not include a description of the BSM physics of greatest interest to us.   Here, we've already integrated over neutrino momentum at least.  That's something.  Here's Holstein's Eq.~(52):
%% !TEX root = ../thesis_main.tex
%
%
%
% "A PDF for the People"
\bea
\mathrm{d}^3 \Gamma_{\textrm{Holstein}} &=& 2  G_v^2 \cos^2\theta_c \frac{\FF}{(2\pi)^4} \, \pe \Ee (E_0 - \Ee)^2 \dEe \, \dOmegae 
\nonumber\\
&& \times
\left\{
	F_0(\E) 
	+ \Lambda_1 F_1(\E) \hatn \cdot \frac{\vecpe}{\Ee}
	+ \Lambda_2 F_2(\E) \left[ \left( \nhat \cdot \frac{\vecpe}{\Ee} \right)^2 - \frac{1}{3}\frac{\pe^2}{\Ee^2} \right]
	\right. \nonumber\\ && \left.
	+ \Lambda_3 F_3(\E) 
		\left[ 
			\left( \hatn \cdot \frac{\vecpe}{\Ee} \right)^3
			- \frac{3}{5}\frac{\pe^2}{\Ee^2}\hatn \cdot \frac{\vecpe}{\Ee}
		\right]
\right\}
\label{equation:holstein52}
\eea
%	
%% equation:holstein52
%
%\section{Relation between JTW and Holstein Formalisms}
%	%\\*
%	To conduct a precision search for scalar and tensor couplings, it is necessary to combine the Holstein and JTW models into a single cohesive probability distribution.  
\section{Mathematical Formalism}
	In order to proceed with a measurement, we must find a master equation to describe the probability of beta decay events with any given distribution of energy and momenta among the daughter particles, as a function of the strength of the specific couplings of interest to us.  To do this, two sets of formalisms are combined -- the older formalism from Jackson, Treiman, and Wylde (JTW)~\cite{jtw},~\cite{jtw_coulomb}, which describes the effects of all types of Standard Model and exotic couplings of interest to us here, but which truncates its expression at first order in the (small) parameter of recoil energy, and a newer formalism from Holstein ~\cite{holstein}, which includes terms up to several orders higher in recoil energy, but which does not include any description of the exotic couplings of particular interest to us.  We note that because any exotic couplings present in nature have already been determined to be either small or nonexistant, it is sufficient to describe these parameters with expressions truncated at first order, despite the fact that it is still necessary to describe the larger Standard Model couplings with higher-order terms. 
	
	The procedure for combining the two formalisms is described in detail in Appendix~\ref{appendix_forthepeople}, so we will simply provide the combined master equation here:
	
\bea
\textrm{put  a master equation here.}
\eea
\aside{Do it!  Do the master equation!}





%%%% --- * --- %%%%	
\clearpage
\chapter{Atomic Physics Overview}
\label{atomicphysics_chapter}
\section{Magneto-Optical Traps}
	\subsection{Doppler Cooling}
	\subsection{Zeeman Splitting}
	Needs a level diagram.
	\subsection{Atom Trapping with a MOT}

\section{Optical Pumping}
Needs a level diagram.


\section{Shake-off Electron Spectrum}
Shake-off electrons:  where do they come from, and where do they go?  ~\cite{Levinger}.

John made some nice plots of these from the eMCP data.  I did *not* use it to make a cut on eMCP hit position in the end, despite the fact that it makes the spectrum more clean, because a lot of good events don't have full hit position information, and you lose an awful lot of statistics by making the cut.  I used this for modeling the background spectrum, but in the end it wasn't as elegant a result as I might have hoped.  Also, it's still an open question exactly which fraction of SOEs come from which atomic shell, but it doesn't change the resulting spectrum very much.
\missingfigure{SOE Spectrum goes here.}


%%%% --- * --- %%%%	
% !TEX root = ../thesis_main.tex


%%%% --- * --- %%%%	
\clearpage
\chapter{The Experimental Setup}
\label{setup_chapter}

%\color{oldcolor}
\section{Overview}

The experimental subject matter of this thesis was conducted at TRIUMF using the apparatus of the TRIUMF Neutral Atom Trap (TRINAT) collaboration.  The TRINAT laboratory offers an experimental set-up which is uniquely suited to precision tests of Standard Model beta decay physics, by virtue of its ability to produce highly localized samples of isotopically pure cold atoms within an open detector geometry.  

The TRINAT lab accepts radioactive ions delivered by the ISAC beamline at TRIUMF.  These ions are collected on the surface of a hot zirconium \aside{Is it definitely zirconium?  I don't remember.} foil where they are electrically neutralized, and subsequently escape from the foil into the first of two experimental chambers (the ``collection chamber").  Within the collection chamber, atoms of one specific isotope -- for the purposes of this thesis,  \isotope[37]{K} -- are continuously collected into a magneto-optical trap (MOT).  Approximately once per second, the atoms in the collection MOT are transferred to a second experimental chamber (the ``detection chamber'') and loaded into a second MOT (see Fig.~\ref{fig:doublemot}).  Because the transfer and trapping mechanisms rely on tuning to specific atomic resonances, this setup allows for the selection of only a single isotope within the detection MOT, and a significantly reduced background relative to the initial beamline output.  The transfer methodology is discussed in some detail within another publication~\cite{swanson}.

\note{Probably describe the laser transfer method slightly.}  

%The TRIUMF Neutral Atom Trap (TRINAT) offers an experimental set-up which is uniquely suited to precision tests of Standard Model beta decay physics.  Radioactive ions are delivered from the ISAC beamline and neutralized before being trapped in the first of two magneto-optical traps (MOTs).  Approximately once per second, atoms from the first MOT are transferred to the second, where their decay products can be observed with significantly less background than would have been possible in the first trap (see Figure~\ref{fig:doublemot}).  The transfer methodology is discussed in some detail in a paper by Swanson et al~\cite{swanson}. \aside{The point is that this eliminates background from the decays of other stuff.  Or the same stuff.  Stuff that's not centered at the trap.}

\begin{figure}[t!h]
	\centering
	\includegraphics[width=.999\linewidth]
	{Figures/doublemot4.pdf}
	\caption{The TRINAT experimental set-up utilizes a two MOT system in order to reduce background in the detection chamber.}	
	\label{fig:doublemot}
\end{figure}

Once the newly transferred atoms have arrived at the second trap, the MOT cycles 500 times between a state where it is `on' and actively confining atoms to a region of approximately 2\,mm$^3$, to a state where it is `off' and instead the atoms are spin-polarized by optical pumping while the atom cloud expands ballistically before being re-trapped.  In order to eliminate systematic effects, the polarization direction is flipped every 16 seconds.  This optical pumping technique and its results are the subject of a recent publication~\cite{ben_OP}.

\note{Below is pretty vague.  I could do better, even for just an overview-summary thing.  Obvs I have to describe it in detail later on *somewhere*, though maybe not in the overview-summary...} 
 
Detectors are positioned about the second MOT for data collection.  The science chamber (shown in Figure \ref{fig:thechamber}) operates at ultra-high vacuum (UHV) and provides the apparatus necessary to intermittently confine atoms within a MOT and then spin-polarize them, and quantify their position, temperature, and initial polarization, and electrostatic hoops to allow for collection and observation of charged recoiling daughter nuclei, as well as further detectors to observe the outgoing betas and reconstruct angular correlations.  
%\comment{(should I elaborate here?) -- "no".} 

\begin{figure}[h!!!tb]
	\centering
%	\hspace*{\fill}%
	\subfloat[A decay event within the TRINAT science chamber.  After a decay, the daughter will be unaffected by forces from the MOT.  Positively charged recoils and negatively charged shake-off electrons are pulled towards detectors in opposite directions.  Although the $\beta^+$ is charged, it is also highly relativistic and escapes the electric field with minimal perturbation.
	%\comment{The pic is still kind-of fuzzy.}
	]
	{\includegraphics[width=.530\linewidth]{Figures/chamber_decayevent3.png}\label{chamber_decayevent} }
	\hspace*{\fill}
%	\hfill
	\hspace*{\fill}
	\subfloat[Inside the TRINAT science chamber.  This photo is taken from the vantage point of one of the microchannel plates, looking into the chamber towards the second microchannel plate.  The current-carrying copper Helmholtz coils and two beta telescopes are visible at the top and bottom.  The metallic piece near the center is one of the electrostatic `hoops' used to generate an electric field within the chamber.  The hoop's central circular hole allows access to the microchannel plate, and the two elongated holes on the sides allow the MOT's trapping lasers to pass unimpeded at an angle of 45 degress `out of the page'.]	
	{\includegraphics[width=.444\linewidth]{Figures/chamber_photo_2.png}}
%	\hspace*{\fill}%
	\caption{The TRINAT detection chamber}	
	\label{fig:thechamber}
\end{figure}
%\clearpage



%\color{black}
%\section{Double MOT System to Supply Atoms}
%%\\*
%Mostly, 
%this requires a diagram.  We take ions supplied by the ISAC beamline, neutralize them and trap them in the first MOT, then periodically transfer them to a second MOT.  Detectors are 
%positioned about the second MOT for data 
%collection.  This double MOT system eliminates a great deal of background.  
%Also, here's some random equation that doesn't really go with the topic of this section:
%\bea
\frac{\partial \rho}{\partial t} &\!\!\!\!\!\! \bigg|_{relax} \!\!\! &= \:\:\: -\frac{1}{2} \left( \hat{\Gamma} \rho + \rho \,\hat{\Gamma} \right) 
\label{eq:relax} \\
\frac{\partial \rho}{\partial t} &\!\!\!\!\!\!  \bigg|_{repop} \!\!\! &= \:\:\: \hat{\Lambda},
\label{eq:repop} 
\eea

%Furthermore, here's a picture that doesn't really go with the topic of this section.
%collection.  This double MOT system 
%\margintodo[color=green]{A thing that's worth noting is that (I think!) recoil-order corrections have been implicitly excluded at some point here.  ...Is this even true??}
%eliminates a great deal of background.  


\section{AC-MOT and Polarization Setup}

%\\*
In order to facilitate a measurement of $A_{\mathrm{\beta}}$, we went to great efforts to polarize the atom cloud, and quantify that polarization.  This resulted in a duty cycle in which the atoms were intermittently trapped in the AC-MOT, then optically pumped to polarize them.  While knowledge of the polarization is less critical in a measurement of $b_{\mathrm{Fierz}}$, we still use only the polarized portion of the duty cycle in order to minimize other systematic errors, such as the scintillator energy calibration and overall trap position.
\note[color=green]{Is that $\uparrow$ even true??  Because I'm really not sure it is.  Via Kofoedhansen, $(E_0 - E_e) = E_\nu$.  So there.}  
Anyway, here's some figures.  Or possibly one figure.  Whatever.  Also, here's a reference to a figure.  See Fig.~\ref*{fig:themot}, or also its subfigures, eg Fig.~\ref{fig:acmot} and Fig.~\ref{fig:mot}.  Maybe I have to subref them?  Like, eg, Subfig.~\subref{fig:acmot} and Subfig.~\subref*{fig:mot}.  What if we try to subref everything?  Consider, eg, Fig.~\subref{fig:themot}.
\margintodo{Does this work?  It really should.}


The Magneto-Optical Trap is a well-known technique from atomic physics, used to confine and cool neutral atoms~\cite{raabprentiss}.  The technique is used predominantly with alkalis due to their simple orbital electron structure, and is quite robust, so is appropriate for use with $^{37}\textrm{K}$.  Once set up, the trapping force is specific to the isotope for which the trap has been tuned, which makes it ideal for use in radioactive decay experiments, since the daughters are unaffected by the trapping forces keeping the parent confined.

There are two primary components necessary for any MOT:  a laser, and a magnetic field.  The laser, which must be circularly polarized in the appropriate directions and tuned slightly to the red of an atomic resonance, is split into three perpendicular retroreflected beams, doppler cooling the atoms and (with the appropriate magnetic field) confining them in all three dimensions (see Figure~\ref{fig:mot}).  The TRINAT science chamber includes 6 `viewports' specifically designed to be used for the trapping laser.

A MOT also requires a quadrupolar magnetic field, which we generate with two current-carrying anti-Helmholtz coils located within the vacuum chamber itself.  The coils themselves are hollow, and are cooled continuously by pumping temperature-controlled water through them.   

One feature which makes our MOT unusual has been developed as a result of our need to rapidly cycle the MOT on and off -- that is, it is an ``AC-MOT''.  Rather than running the trap with one particular magnetic field and one set of laser polarizations to match, we run a sinusoidal AC current in the magnetic field coils, and so the sign and magnitude of the magnetic field alternate smoothly between two extrema, and the trapping laser polarizations are rapidly swapped to remain in sync with the field~\cite{harveymurray}\cite{thesis}.  See Figure~\ref{fig:acmot}.  

\note{Note that because the atoms within a MOT can be treated as following a thermal distribution, some fraction of the fastest atoms continuously escape from the trap's potential well.  Even with the most carefully-tuned apparatus, the AC-MOT cannot quite match a similar standard MOT in terms of retaining atoms.  The TRINAT AC-MOT has a `trapping half-life' of around 6 seconds, and although that may not be particularly impressive by the standards of other MOTs, it is more than adequate for our purposes.  $^{37}\textrm{K}$ itself has a radioactive half-life of only 1.6 seconds 
(cite someone), so our dominant loss mechanism is radioactive decay rather than thermal escape. }



\begin{figure}[ht]
\centering
	\begin{subfigure}[t]{0.242\textwidth}
		\centering
		\includegraphics[width=\textwidth]{mot.png}
	%	\label{fig:mot}
		\caption{Components of a magneto-optical trap, including current-carrying magnetic field coils and counterpropagating circularly polarized laser beams.}
		\label{fig:mot}
	\end{subfigure}
	\hfill
	\begin{subfigure}[t]{0.728\textwidth}
		\centering
		\includegraphics[width=\textwidth]{acmot.png}
		\caption{One cycle of trapping with the AC-MOT, followed by optical pumping to spin-polarize the atoms.  After atoms are transferred into the science chamber, this cycle is repeated 500 times before the next transfer.  The magnetic dipole field is created by running parallel (rather than anti-parallel as is needed for the MOT) currents through the two coils.}
		\label{fig:acmot}
	\end{subfigure}
	\caption{An alternating-current magneto-optical trap with a duty cycle optimized for producing polarized atoms}	
	\label{fig:themot}
\color{black}
\end{figure}


We spin-polarize $^{37}\textrm{K}$ atoms within the trapping region by optical pumping~\cite{ben_OP}.  A circularly polarized laser is tuned to match the relevant atomic resonances, and is directed through the trapping region along the vertical axis in both directions.  When a photon is absorbed by an atom, the atom transitions to an excited state and its total angular momentum (electron spin + orbital + nuclear spin) along the vertical axis is incremented by one unit.  When the atom is de-excited a photon is emitted isotropically, 
%\comment{(is it still isotropic when it's polarized?  I bet it's not.)}
so it follows that if there are available states of higher and lower angular momentum, the \emph{average} change in the angular momentum projection is zero.  If the atom is not yet spin-polarized, it can absorb and re-emit another photon, following a biased random walk towards complete polarization.  

In order to optimally polarize a sample of atoms by this method, it is necessary to have precise control over the magnetic field.  This is because absent other forces, a spin will undergo Larmor precession about the magnetic field lines.  In particular, the magnetic field must be aligned along the polarization axis (otherwise the tendency will be to actually depolarize the atoms), and it must be uniform in magnitude over the region of interest (otherwise its divergencelessness will result in the field also having a non-uniform direction, which results in a spatially-dependent depolarization mechanism).  Note that this type of magnetic field is not compatible with the MOT, which requires a quadrupolar magnetic field \emph{gradient}, and has necessitated our use of the AC-MOT as described in Subsection~\ref{trap}.


%\color{black}
%	\subsection{\textbf{Nuclear Setup}}
\section{Measurement Geometry and Detectors}
	%\\*
	Needs several diagrams.  Back-to-back beta detectors along the polarization axis.  Back-to-back MCPs in an electric field to tag events from the trap, and to measure the trap position and polarization.  Hoops to produce the electric field.  Many laser ports to make the MOT functional, and for optical pumping.  Fancy mirror geometry to combine optical pumping and trapping light along the vertical axis.  Water-cooled (anti-)Helmholz coils within the chamber for the AC-MOT, fast switching to produce an optical pumping field.  
%	\subsection{\textbf{All the Detectors}}

The beta detectors, located above and below the atom cloud along the axis of polarization (see Figure~\ref{chamber_decayevent}), are each the combination of a plastic scintillator and a set of silicon strip detectors.  Using all of the available information, these detectors are able to reconstruct the energy of an incident beta, as well as its hit position, and provide a timestamp for the hit's arrival.  Together the upper and lower beta detectors subtend approximately 1.4\% of the total solid angle as measured with respect to the cloud position. 

It must be noted that the path between the cloud of trapped atoms and either beta detector is blocked by two objects:  a 254$\,\mu$m silicon carbide mirror (necessary for both trapping and optical pumping), and a 229$\,\mu$m beryllium foil (separating the UHV vacuum within the chamber from the outside world).  In order to minimize beta scattering and energy attenuation, these objects have had their materials selected to use the lightest nuclei with the desired material properties, and have been manufactured to be as thin as possible without compromising the experiment.  As the $^{37}\textrm{K} \rightarrow \,^{37}\textrm{\!Ar} + \beta^{+} + \nu_e$ decay proceess releases $Q=5.125$\,MeV of kinetic energy~\cite{Q_value}, the great majority of betas are energetic enough to punch through both obstacles without significant energy loss before being collected by the beta detectors.  

On opposing sides of the chamber, and perpendicular to the axis of polarization, two stacks of $\sim$ 80\,mm diameter microchannel plates (MCPs) have been placed (see Figure~\ref{fig:thechamber}) as detectors, providing a time stamp when a particle is incident on their surfaces.  Behind each stack of MCPs there is a set of delay lines, which provide  position sensitivity for these detectors.   

In order to make best use of these MCPs, we create an electric field in order to draw positively charged particles into one MCP, while drawing negatively charged electrons into the other MCP.  Seven electrostatic hoops have been placed within the chamber (see Figure~\ref{fig:thechamber}), and are connected to a series of high voltage power supplies.  See Sections~\ref{photoions} and~\ref{pos_recoils} for a discussion of what sort of charged particles we expect to observe in these detectors and how they are created.  
  
Scientific data has been collected at field strengths of 395 V/cm, 415 V/cm, and 535 V/cm.  It should be noted that these field strengths are too low to significantly perturb any but the least energetic of the (positively charged) betas from the decay process, and these low energy betas would already have been unable to reach the upper and lower beta detectors due to interactions with materials in the SiC mirror and Be foil vacuum seal.  




%%%% --- * --- %%%%	
% !TEX root = ../thesis_main.tex



%%%% --- * --- %%%%	
\clearpage	
\chapter{Calibrations and Data Selection}
\label{calibrations_chapter}
	
%\section{Polarization}
%	%\\*
%	Polarization measurement was conducted on a different set of data, collected in between the measurements used for $A_{\mathrm{\beta}}$ and $b_{\mathrm{Fierz}}$, and at a higher electric field, because we were unable to run both our MCP detectors simultaneously.  
%	
%\section{Trap Position}
%	%\\*
%	Measured using the same dataset that was used to quantify the polarization.  The trap drifts slightly over the course of our data collection.  Describe the rMCP calibration needed to extract this info.  
%

\section{Cloud Measurements via Photoionization}
\label{cloud}
\label{photoions}
In order to measure properties of the trapped $^{37}\textrm{K}$ cloud, a 10\,kHz pulsed laser at 355\,nm is directed towards the cloud.  These photons have sufficient energy to photoionize neutral $^{37}\textrm{K}$ from its excited atomic state, which is populated by the trapping laser when the MOT is active, releasing 0.77\,eV of kinetic energy, but do not interact with ground state $^{37}\textrm{K}$ atoms.  The laser is of sufficiently low intensity that the great majority of excited state atoms are \emph{not} photoionized, so the technique is only very minimally destructive.  
\note{Probably worth mentioning that we test this stuff offline on stable \isotope[41]{K}. }

Because an electric field has been applied within this region (see Section~\ref{field}) the $^{37}\textrm{K}^+$ ions are immediately pulled into the detector on one side of the chamber, while the freed $e^-$ is pulled towards the detector on the opposite side of the chamber.  Because  $^{37}\textrm{K}^+$ is quite heavy relative to its initial energy, it can be treated as moving in a straight line directly to the detector, where its hit position on the microchannel plate is taken as a 2D projection of its position within the cloud.  Similarly, given a sufficient understanding of the electric field, the time difference between the laser pulse and the microchannel plate hit allows for a calculation of the ion's initial position along the third axis.  

\note{As a check:  the camera measurements for photons from de-excitation.  It's aimed 35 degrees from vertical, with its horizontal axis the same as ..... one of the other axes.  I think it's the TOF axis.  I can check this when my computer comes back.   Also, there's an unknown additional delay between some of our DAQ channels that can't be explained by accounting for cable lengths, so we really like having the check there.}

With this procedure, it is possible to produce a precise map of the cloud's position and size, both of which are necessary for the precision measurements of angular correlation parameters that are of interest to us here.  However, it also allows us to extract a third measurement:  the cloud's polarization.

The key to the polarization measurement is that only atoms in the excited atomic state can be photoionized via the 355 nm laser.  While the MOT runs, atoms are constantly being pushed around and excited by the trapping lasers, so this period of time provides a lot of information for characterizing the trap size and position.  When the MOT is shut off, the atoms quickly return to their ground states and are no longer photoionized until the optical pumping laser is turned on.  As described in Section~\ref{op}, and in greater detail in~\cite{ben_OP}, the optical pumping process involves repeatedly exciting atoms from their ground states until the atoms finally cannot absorb any further angular momentum and remain in their fully-polarized (ground) state until they are perturbed.  Therefore, there is a sharp spike in excited-state atoms (and therefore photoions) when the optical pumping begins, and none once the cloud has been completely polarized.  The number of photoion events that occur once the sample has been maximally polarized, in comparison with the size and shape of the initial spike of photoions, provides a very precise characterization of the cloud's final polarization~\cite{ben_OP}.



\note{Trap position -- Measured using the same dataset that was used to quantify the polarization.  The trap drifts slightly over the course of our data collection.  Describe the rMCP calibration needed to extract this info.}
\note{Polarization measurement was conducted on a different set of data, collected in between the measurements used for $A_{\mathrm{\beta}}$ and $b_{\mathrm{Fierz}}$, and at a higher electric field, because we were unable to run both our MCP detectors simultaneously.  }
%\note{Need to describe how polarization works.  Needs a level diagram.  Needs another(?) level diagram for the photoionization, and maybe a third for the MOT.  Can I combine them all?  idk.}

Anyway, here is a nice table describing the atom cloud, for each of 3 runsets, and I'll immediately reference it right now, as Table~\ref{table:cloudpositions}:


% !TEX root = ../thesis_main.tex
%
%
%
\begin{table}[h!!!!t]
	\begin{center}
	\begin{tabular}{ c  | r || lcl | lcl || lcl | lcl |}
			\multicolumn{1}{c}{ Runsets  }{ } & \multicolumn{1}{  c  }{ } & 
				\multicolumn{3}{  c  }{ \!\!Initial Position\!\! } &  \multicolumn{3}{   c  }{ Final Position } &  \multicolumn{3}{   c  }{ Initial Size } &  \multicolumn{3}{   c  }{ Final Size } \\
			\cline{2-14}
			\multirow{3}{*}{EB $\leftarrow$ RB} %&\multirow{3}{*}{RB}  
								& $x$ & \,\,1.77 & \!\!$\!\! \pm  \!\!$\!\! & 0.03   & \,\,2.06   & \!\!$\!\! \pm  \!\!$\!\! & 0.08    & 0.601 & \!\!$\!\! \pm  \!\!$\!\! & 0.013 & 1.504 & \!\!$\!\! \pm  \!\!$\!\! & 0.047 \\
								& $y$ & -3.51    & \!\!$\!\! \pm  \!\!$\!\! & 0.04   & -3.33     & \!\!$\!\! \pm  \!\!$\!\! & 0.05    & 1.009 & \!\!$\!\! \pm  \!\!$\!\! & 0.013 & 1.551 & \!\!$\!\! \pm  \!\!$\!\! & 0.018 \\
								& $z$ & -0.661  & \!\!$\!\! \pm  \!\!$\!\! & 0.005 & -0.551   & \!\!$\!\! \pm  \!\!$\!\! & 0.021  & 0.891 & \!\!$\!\! \pm  \!\!$\!\! & 0.004 & 1.707 & \!\!$\!\! \pm  \!\!$\!\! & 0.015 \\
			\cline{2-14}
			\multirow{3}{*}{EC $\leftarrow$ RD} %&\multirow{3}{*}{RD}  
								& $x$ & \,\,2.22  & \!\!$\!\! \pm  \!\!$\!\! & 0.05  & \,\,2.33   & \!\!$\!\! \pm  \!\!$\!\! & 0.11    & 1.18   & \!\!$\!\! \pm  \!\!$\!\! & 0.04   & 1.538 & \!\!$\!\! \pm  \!\!$\!\! & 0.087 \\
								& $y$ & -3.68     & \!\!$\!\! \pm  \!\!$\!\! & 0.04  & -3.31      & \!\!$\!\! \pm  \!\!$\!\! & 0.06   & 0.965 & \!\!$\!\! \pm  \!\!$\!\! & 0.012 & 1.460 & \!\!$\!\! \pm  \!\!$\!\! & 0.030 \\
								& $z$ & -0.437   & \!\!$\!\! \pm  \!\!$\!\! & 0.09  & -0.346    & \!\!$\!\! \pm  \!\!$\!\! & 0.037 & 0.927 & \!\!$\!\! \pm  \!\!$\!\! & 0.007 & 1.797 & \!\!$\!\! \pm  \!\!$\!\! & 0.026 \\
			\cline{2-14}
			\multirow{3}{*}{ED $\leftarrow$ RE} %&\multirow{3}{*}{RE}  
								& $x$ & \,\,2.274 & \!\!$\!\! \pm  \!\!$\!\! & 0.012 & \,\,2.46 & \!\!$\!\! \pm  \!\!$\!\! & 0.06   & 0.386 & \!\!$\!\! \pm  \!\!$\!\! & 0.016 & 1.382 & \!\!$\!\! \pm  \!\!$\!\! & 0.046 \\
								& $y$ & -4.54      & \!\!$\!\! \pm  \!\!$\!\! & 0.04   & -4.28    & \!\!$\!\! \pm  \!\!$\!\! & 0.04   & 0.986 & \!\!$\!\! \pm  \!\!$\!\! & 0.08   & 1.502 & \!\!$\!\! \pm  \!\!$\!\! & 0.013 \\
								& $z$ & -0.587    & \!\!$\!\! \pm  \!\!$\!\! & 0.04   & -0.481  & \!\!$\!\! \pm  \!\!$\!\! & 0.018 & 0.969 & \!\!$\!\! \pm  \!\!$\!\! & 0.003 & 1.861 & \!\!$\!\! \pm  \!\!$\!\! & 0.013 \\
			\cline{2-14}
	\end{tabular}
	\end{center}
%	\note{These positions are for the *electron* runsets of those names.  Might want a chart of which rMCP runs are used to measure position for which eMCP runs.  Possibly in that other section.}
	\note{Sig figs here need work.}
	\note{Parameters measured with the recoil runs, and applied on the appropriate electron runs.}
	\caption[Cloud Position and Size]{Cloud Positions and Sizes -- Measured immediately before and immediately following the optical pumping phase of the trapping cycle.  Measurements are evaluated using rMCP runs, and are taken to describe the cloud during associated eMCP runs as well.  All entries are expressed in units of millimetres, and the size parameters describe the gaussian width.}
	\label{table:cloudpositions}
\end{table}




\note{Also, we noticed the trap drifting after one of the runs, because one of the batteries on one of the thingies adjusting the laser frequency (I think) was failing. }
\note[color=jb]{JB:  ``If we rejected the data with the MOT moving (indeed a battery determining the voltage controlled oscillator frequency offset between absorption in stable \isotope[41]{K} cell and the \isotope[37]{K} resonance) then that's all you need to say.''}
\note{describe how you'd turn this into a physical description of the cloud, with like a temperature and a sail velocity and shit.  with equations.}


\section{Plastic Scintillators}
	Energy calibration for the scintillator+PMT setup changed dramatically at one point.    Describe how calibration was done.  
	\note{How the fuck WERE these things done?!}
	\note[color=jb]{JB:  ``You can describe anything you did differently or improved, but you can and should otherwise defer all details of the scintillator calibration and DSSD calibration to Ben's paper and his thesis and Spencer's.  E.g. Section~\ref{section:bb1_systematics} ``statistical agreement between BB1 X and Y detectors' energies only makes a small effect on results" does not need the technical details beyond that statement."
	\label{thesisconventionjb} }
	\note[color=jb]{JB:  ``If you have some way of documenting the coding you used, that would be great."  ... yeah, it would, wouldn't it?}
	
\section{Strip Detectors}
	Also describe how the DSSD calibration was done, even though it wasn't implemented by me. 
	\note{How the fuck WERE these things done?!}

\section{The eMCP}
	I can describe the eMCP calibration here, even though it mostly wasn't implemented by me.  It is tangentially relevant to data selection and background estimation by providing an experimental energy spectrum for shake-off electrons.  It's actually a pretty neat algorithm that I basically wasn't involved with.
	\note[color=jb]{JB:  eMCP.  You need to describe the timing information obtained.  You also need a statement of whether or not you used the position information in your cuts.}

\missingfigure{Needs an SOE timing spectrum.  At least one of them.  Experimental and simulated.  Also, I have to describe how I did the simulating, and how I check that it's OK despite the fact that the simulated spectrum looks nothing like the experimental spectrum.}

\section{The rMCP}
	I did this, and they're absolutely needed to make any sense of the trap position data.  
\missingfigure{Needs two pictures from the rMCP with the grid lines -- before and after corrections are performed.  Just for fun, I could throw in one with the stupid stripes.}
	These calibrations are done during AC-MOT time, and we're actually interested in the rMCP data taken during OP time.  Can I find pictures to estimate the size of the change resulting from the magnetic field?  In any case, the change is pretty small.  

\missingfigure{Position as a function of run number.  I have this somewhere.}















%%%% --- * --- %%%%
% !TEX root = ../thesis_main.tex


%%%% --- * --- %%%%
\clearpage	

\chapter{The Experimental Signature}
\label{analysis_chapter}

\section{TBD}

I really need an excuse to include more pictures of data.  Also, more pictures of simulations

\missingfigure{Show individual beta energy spectra.  ...with a variety of different cuts, perhaps?}

\missingfigure{Show simulated spectra separated by scattering category.}

\missingfigure{Show SimpleMC spectra, show the supersum, show the superratio, show the superratio asymmetry.  Maybe do some simple fits to show how much better the superratio asymmetry is than \emph{not} the superratio asymmetry.  }



\section{The Superratio and Asymmetry}
%\\*
The data can be combined into a superratio asymmetry.  This has the benefit of causing many systematics to cancel themselves out at leading order.  It also will increase the fractional size of the effects we're looking for.  This can be shown by using math.  

\section{Signature of a Fierz Term in This Experiment}
%\\*
Not all systematics effects are eliminated.  We'll want to be careful to propagate through any effects that are relevant.  Using the superratio asymmetry as our physical observable makes this process a bit messier for the things that don't cancel out, but it's all just math.  

\section{Comparative Merits of the Superratio and Supersum for Measurement}
%\\*
Some other groups have performed similar measurements using the supersum as the physical observable.  There are pros and cons to both methods.  I can show, using a back-of-the-envelope calculation, that for this particular dataset, the superratio asymmetry method produces a better result.  





%%%% --- * --- %%%%
% !TEX root = ../thesis_main.tex



%%%% --- * --- %%%%	
\clearpage
\chapter{Analysis}
\label{analysis_chapter}

Right, so.  Here's how I processed the data into an answer.  In bullet point form, so I don't forget stuff while I'm obsessively trying to phrase everything well.  
\newline

With the Data:
\begin{itemize}
	\item Higher-level data cleaning.  Discard events during parts of the duty cycle when atoms weren't polarized.  Discard events near a recorded spark time.  Discard events when the photoionization laser fires.  Discard events when the LED pulser used to calibrate the scintillators fires.  
	\item Split up runs into sets, to account for changing experimental conditions.  Possibly I should list what the differences between runs were somewhere.  But not in this section.
	\item Using the ``other'' data set with the rMCP:  Measure the trap position/size/velocity/expansion with the rMCP and with the camera.  Necessitates calibrating the rMCP, which is its own whole thing.  Also measure polarization.
		\begin{itemize}
		\item rMCP calibration probably goes in another section.  wev.
		\item We took the mask off before the 2014 run, to give us more detector area.  Use previous reference calibration \emph{with the mask} during the test run in Nov 2013.  The delay line's non-linearities should be the same, assuming we can get the centering the same.  Cables have changed and stuff, so we have to re-center the pre-calibration image to where the old pre-calibration image was.  ...  So, center the new runs w.r.t the old run.
		\item We'll want to make some sum cuts for these things.  We might like them to be identical, or at least identical-ish, but the peaks don't really look the same.  So we'll settle for ``decent sum cuts for all!".  ...  So, apply sum cuts to the new runs and old run.
		\item Calibrate the old run, with the mask.  In fact, I don't remember which order I did things in.  But I have a record of it here, somewhere...
		%	\begin{enumerate}
		%		\item 
		%	\end{enumerate}
		\end{itemize}
	\item Make some more careful cuts to clean the data.  
		\begin{itemize}
		\item Discard events without a ``good'' DSSD hit.  Eliminates vast majority of background 511s.  Necessitates having a definition of what a ``good'' DSSD hit is.  It's subtle enough that we'll want to leave some part of this definition of ``good'' to be varied as a systematic effect.  Notably, we consider energy agreement for each hit pixel, individual strip SNR, and overall DSSD energy threshold.  Also, hit radius w.r.t. center of detector.  This is a lot of stuff, all implemented by Ben -- and it needs to be done fairly early on in data processing in order to keep processing times for everything else manageable.  
		\item Discard events where SOE-Beta TOF falls outside a certain range.  Necessitates picking a ``good'' range.  The precise definition of ``good'' is varied as a systematic.
		\end{itemize}
\end{itemize}

\vspace{24pt}
\vspace{12pt}
With the Simulations:
\begin{itemize}
	\item Update G4 event generator to be able to model non-zero scalar and tensor coupling.  These things show up in $\Abeta$ too, not just in $\bFierz$.  Though, the effects on $\Abeta$ are much smaller.
	\item Run 3 sets of G4 simulations with a bunch of statistics (N events, for data with like N/10 events).  Each one has the same nominal value of $\Abeta$, but with 3 different values of the scalar coupling $C_S$:  zero, and +/-(whatever).  Keep $C_T=0$.  Because reasons, we're not really able to distinguish between $C_S$ and $C_T$ in this experiment anyway, so might as well keep the analysis simple.
	\item Just run one set of 0.02*N events for the two percent branch.  We can't neglect it, but it isn't going to change (much?) when we adjust BSM couplings.
	\item Match cuts in simulated data up to the cuts on experimental data.  Obviously.  DSSD cut, DSSD energy, one hit DSSD, one hit scint.  TOF cut, which requires a whole extra model of background in the TOF spectrum..
		\begin{itemize}
		\item Suppose background in the TOF spectrum is coming from decays of atoms that have gotten themselves stuck to surfaces within the chamber...
		\item Run G4 to get a beta TOF spectrum (w.r.t. the decay)
		\item Run COMSOL (credit to Alexandre) to track low-energy SOEs through the electric field from wherever they started, into the detectors.  Energy spectra from Levinger.
		\item Combine G4 and COMSOL spectra, event-by-event, while requiring that both the beta detector and the eMCP are hit according to the set of random numbers generated by each monte carlo separately.  Then, the beta and SOE will each have a TOF from decay to detector, and subtracting one from the other gives a timing spectrum that can be observed experimentally.  See Fig.~\ref{fig:soetof}.
		\item Upper limit for the fraction of events generated this way can be estimated by assuming that all losses from the trap not due to radioactive decay emerge isotropically from the trap and then stick to whatever chamber wall is in its path.  This upper limit is too big by a factor of 2.
		\end{itemize}
	\item For each of those 3 simulations, sort the ``good'' data according to emission angle relative to the detector.  Do each detector individually.  For both polarizations.
\end{itemize}

\begin{figure}[t!h]
	\centering
	\includegraphics[width=.999\linewidth]
	{Figures/SOETOF_withmodel.pdf}
	\caption{SOE TOF, model and data.  In the end, I cut the data to use only events with a TOF between \comment{A and B.\;\;}  Max. possible background is like a factor of two too big.}	
	\label{fig:soetof}
\end{figure}







%%%% --- * --- %%%%
% !TEX root = ../thesis_main.tex



%%%% --- * --- %%%%
\clearpage	
\chapter{Estimating Systematic Effects}
\label{systematics_chapter}
	\note{How do I even \emph{do} these estimations?}
	\section{Low-energy Scintillator Threshold}
	%\\*
	Choice of low-energy scintillator threshold has a large systematic effect...  \aside{It's actually not nearly as big as I'd originally expected.  It's huge in the lineshape thing, but pretty tiny in everything else.}
	
	\section{BB1 Radius, Energy Threshold, Agreement}
	\label{section:bb1_systematics}
	%\\*
	BB1 radius cut can help to eliminate scattered events.  Energy threshold selection and statistical agreement between BB1 detectors' energies only makes a small effect on results.  BB1 radius itself has a pretty big effect on the result, but we can at least just G4 it away.  The remaining systematic effect is pretty small.
	\note[color=jb]{As per JB's comment in section~\ref{thesisconventionjb}:   ``statistical agreement between BB1 X and Y detectors' energies only makes a small effect on results" does not need the technical details beyond that statement."}
	
\missingfigure{Surely this requires at *least* one image of the pixelated BB1 data.  Maybe some of a few waveforms and energy distributions too.  ....Feels like cheating to include some of that stuff, since Ben was the one who actually used it mostly.}

\note{Remember:  There's noise applied to simulated BB1s, matching some spectrum.}
In the end, we get our results from the scintillator energy only, without summing the BB1 energy back in.  Energy absorbed in DSSDs is only used as (a) a tag for events, and (b) contributing to the total beta energy loss before the beta arrives at the scintillator.


\section{Background Modeling -- Decay from Surfaces within the Chamber}
	So many surfaces, all of which can get stray 37K atoms stuck to them.  Then they decay from a place that isn't the actual trap center, and it contaminates our stuff.
	
	\missingfigure{Show modelled TOF spectra in comparison with real TOF spectra.  Show the cut we made on that. }
	\missingfigure{Show the "average asymmetry" (all energies) as a function of TOF, with real data, best model normalization, and extrema of model normalizations.  Show our cut.  Turns out, it's a lot of work for a really tiny correction.  Oh well.}
	
	So we model the beta TOF from the surfaces in G4, event by event.  This is necessary because scattered events will have their TOF changed to account for a longer beta pathlength, and we're preferrentially cutting away the events that don't have a TOF in the appropriate range.  ....And then have COMSOL generate electron TOFs for SOEs starting from the start points picked by G4.  Ran COMSOL for 0 eV SOEs, and again for Levinger spectrum SOEs.  Used ~9\% 0eV SOEs in the end.  I forget which Levinger distribution I used in the end.  The point is, for each event, you've simulated a beta TOF that may or may not be scattered off of something before it hits a detector, and you have a SOE TOF for an event originating at that same point, so you subtract them to model the TOF you'd measure in an experiment.  Also, because you've done the scattering with G4, you get the beta energy corrected for any scattering that happened.  This way, you know precisely how much "bad stuff" you're getting rid of with the TOF cut.
	
%	\subsection{Decay from Chamber Surfaces}

\section{Quantifying the Effects Backscatter with Geant4}
	Beta decay (back-)scatter from surfaces within the experimental chamber is a significant systematic, and it must be evaluated, quantified, and corrected for.  This is done via a series of GEANT4 simulations.  While only a small fraction of events are affected, the process results in an energy loss in the beta that can, if not understood, be misinterpreted as the exact signal we're searching for.  It is therefore imperative that this be well understood. 
	\aside{Oh god.  Have I even \emph{tried} to quantify the combined systematic that comes out of the TOF cut?  Do I need to, or is it double-counting?  Ugh, it would be such a headache to do this.  Maybe I can at least do it at the end -- because I might never get my code back to the way it was.}


%	\missingfigure{Show simulated $\cos\theta$ vs TOF.  You can just make a lot of that go away with a properly selected cut.}
\begin{figure}[h!!!t]
	\centering
	\includegraphics[width=.999\linewidth]
	{Figures/toa_vs_costheta.pdf}
	\caption[Simulated Beta TOA vs emission angle w.r.t. detector orientation]{Simulated Beta TOA vs emission angle w.r.t. detector orientation}	
	\label{fig:toa_vs_costheta}
\end{figure}


\section{Lineshape Reconstruction}
	\note[color=jb]{This section should reference Clifford.~\cite{clifford}.}
	\subsection{Motivation}
	This process is used because the (back-)scatter, which it itself an important systematic, is largely independent of a wide variety of other experimental effects.  These other effects must all be evaluated, but it is computationally prohibitive to re-evaluate the scattering with every other effect under consideration.
	
	\subsection{What is it and how does it work?}
	Mono-energetic beta decay events are generated in GEANT4, which outputs an energy spectrum for unscattered and forward-scattered beta events in the detector.  These spectra are fit to a function to model the scintillator resolution, as well as energy loss in materials that the beta passed through before arriving at the scintillator.  These spectrum fits are performed for a set of beta energies, and parameters are extrapolated to be applied to betas emitted at intermediate energies.  Thus, the whole spectrum can be modeled.  Pictures will make this clearer. 
	
	\subsection{The Math-Specifics}
	I'll write down the specific functions I'm using, and the parameter values I'm using.  (Maybe this should go in an appendix instead?)  I'll describe the adjustments I make to the spectrum so that it can work even for the dataset where the scintillators' resolutions have changed.
	
	\subsection{The Results -- Things That Got Evaluated This Way}
	As it turns out, only cloud parameters were evaluated this way.  Trap position, size, sail velocity, temperature.  But then we varied the lineshape anyhow, to account for G4 doing a bad job of modelling the bremsstrahlung (sp?).
	%Thicknesses of the SiC mirror, the Be foil, and the DSSD.  Scintillator calibration.  
	
	\subsection{The low-energy tail uncertainty, and what it does}
	Bremsstrahlung.  It does Bremsstrahlung.





%%%% --- * --- %%%%	
% !TEX root = ../thesis_main.tex



\clearpage	
\chapter{Results and Conclusions}
\label{results_chapter}
\note[color=jb]{JB:  Dan and I independently discussed (((Ch.~\ref{results_chapter}))) yesterday, and he has suggestions to
help. So I will also schedule a meeting with Dan and you to discuss
(((Ch.~\ref{results_chapter}))) Results and whether the S,T part must be deleted and left to a paper.
You don't have enough time, and although this should be quite straightforward,
it is not your critical result and it's the only thing that can go.}



%%% %%%%%%% %%%
%\section{Measured Limits on $b_{Fierz}$, $C_S$, $C_T$}
\section{Measured Limits on $b_{Fierz}$}
	%\\*
	Results go here, with measured limits described and quantified in all formats anyone could ever care about.
	
\note[color=jb]{John says to just skip doing the $C_S$ and $C_T$ stuff, for now.  No time.  Really, $C_S$ is already basically done, but then that'll lead to awkward questions about $C_T$.}
	
\begin{figure}[h!!!t]
	\centering
	\includegraphics[width=.999\linewidth]
	{Figures/AsymmetryAndResiduals.pdf}
	\caption{A superratio asymmetry from the data, and the best fit from simulations.}	
	\label{fig:asymmetry}
\end{figure}
	
	
\begin{figure}[h!!!t]
	\centering
	\includegraphics[width=.999\linewidth]
	{Figures/Abeta_bFierz_2D_prelim.png}
	\caption{Some results.  I'll want to show at least one of these things.  Probably show a separate one for each runset, actually.}	
	\label{fig:2d_results_bcd}
\end{figure}


%%% %%%%%%% %%%
\section{Discussion of Corrections and Uncertainties}
	\note{Just write a blurb to qualitatively summarize a bunch of the stuff in Ch.~\ref{systematics_chapter}.  Do I want to put my error budget table here?  If not, here it is! (~\ref{table:budget}). }
	\note{Other things to discuss here:  which things are dominant error sources, and how viable it would be to improve those for future experiments.}
	
\section{Relation to Other Measurements and New Overall Limits}
	%\\*
	In which I'll show exclusion plots and write down new limits, combining my result with results from the literature.
	Or, y'know, maybe I'll just talk about doing that.
	
\note[color=jb]{JB says:   To put your work in context, please add at the end of that minimal S,T section, or at the end of "Our Decay" section
\\ ... \\
The best existing measurement of $\bFierz$ is in the decay of the neutron~\cite{Saul2020},
$\bFierz$ = 0.017 $\pm$ 0.021, consistent with the Standard Model prediction of zero.
Our measurement is strongly related, yet complementary.
In terms on non-Standard Model Lorentz current structures, to lowest order in the non-SM  currents the same equation applies:
\\
$\bFierz$ = $\pm$ $(C_S +C_S' + (C_T - C_T') \lambda^2)/( 1 + \lambda^2)$
\\
(the plus is for $\beta^-$ decay and the - for $\beta^+$ decay)~\cite{jtw}.  [to be continued...] }

\note[color=jb]{[...continued from prev.]
\\
In our $^{37}$K case, $\lambda^2$ = $|M_{\rm GT}|^2$/$|M_{\rm F}|^2$ is close to 3/5 (the expected value j/(j+1) for a single j=3/2 d3/2 nucleon)~\cite{deShalit1963},
while for the neutron $\lambda^2$ is close to 3 (the expected value for an (j+1)/j j=1/2 s1/2 nucleon).
$|M_F|$, the Fermi matrix element, is nearly the same for both of these isospin = 1/2 decays (the largest correction is the larger isospin mixing of $\sim$0.01 in $^{37}$K).
So our observable is relatively less sensitive to Lorentz tensor currents, and will predominantly constrain or discover Lorentz scalar currents.
\\...\\ 
Full considerations would require a weighted fit of $\bFierz$ experiments and similar observables~\cite{Falkowski2021}, and are beyond the scope of this thesis.
The info from this thesis, values of $\Abeta$ and $\bFierz$ with their uncertainties, can together with the known $fT$ value (lifetime and
branching ratio) allow the community and/or the collaboration to include the results in a future constraint or discovery of scalar and tensor Lorentz currents
contributing to $\beta$ decay.}




%
%%%%%%% %%%%%%%%% %%%%%%%%
\clearpage
\section{Other Possible Future Work for the Collaboration:  $R_{\textrm{slow}}$}
\label{section_rslow}
\note{Probably shouldn't get a clearpage in the end.  It's for my sanity during writing.}
\note[color=jb]{John says the whole $R_{\textrm{slow}}$ thing should go in here somewhere.}
\note[color=jb]{Appendix I  keep, it's excellent. It should be moved as is to Conclusions under "Future Experiment for the collaboration"! so people know you worked so hard on it!!}
%\section{An Old Rslow Abstract}
The nuclear weak force is known to be a predominantly left-handed vector and axial-vector (V-A) interaction.  An experiment is proposed to further test that observation, constraining the strength of right-handed (V+A) currents by exploiting the principle of conservation of angular momentum within a spin-polarized beta decay process.  \comment{Here, we focus on the decay \mbox{$^{37}\textrm{K} \rightarrow \,^{37}\textrm{\!Ar} + \beta^{+} + \nu_e$}.  The angular correlations between the emerging daughter particles provide a rich source of information about the type of interaction that produced the decay.}

%, meaning that \comment{the fuck does it even mean?}, but limits are \comment{something something dark side}.  However, we are able to describe the probability distribution for the kinematics of the daughter particles in a beta decay reaction in terms of both right- and left-handed interactions, and scalar, tensor, vector, and axial couplings.  
%It is therefore possible to perform tests of the Standard Model by precisely measuring the kinematics within a beta decay process.  
%This document will focus predominantly on a search for right-handed interactions within the decay process, $^{37}\textrm{K} \rightarrow \,^{37}\textrm{\!Ar}^{+n} + \beta^{+} + \nu_e + (n+1)e^-$.

%{\color{cyan} (...) }

%\comment{ We obtain a sample of neutral, cold, nuclear spin-polarized $^{37}\textrm{K}$ atoms with a known spatial position, via the TRIUMF accelerator facility, by intermittently running a magneto-optical trap (MOT) to confine and cool the atoms, then cycling the trap off to polarize the atoms.  With $\beta$ detectors placed opposite each other along the axis of polarization, we are able to directly observe the momenta of $\beta^+$ particles emitted into 1.4\% of the total solid angle nearest this axis.  We also are able to extract a great deal of information about the momentum of the recoiling $^{37\!}$Ar daughters by measuring their times of flight and hit positions on a microchannel plate detector with a delay line.  Because the nuclear polarization is known to within $<0.1\%$~\cite{ben_OP}, and we are able to account for many systematic effects by periodically reversing the polarization and by collecting unpolarized decay data while the atoms are trapped within the MOT, we expect to be well equipped to implement a test of `handedness' within the nuclear weak force. }

%We have performed precision measurements of the kinematics of the daughter particles in the decay $^{37}\textrm{K} \rightarrow \,^{37}\textrm{\!Ar}^{+n} + \beta^{+} + \nu_e + (n+1)e^-$.
%\color{blue}
% of $^{37}$K.  This isotope decays by $\beta^+$ emission in a mixed Fermi/Gamow-Teller transition to its isobaric analog, $^{37\!}$Ar.  
%Because the higher-order standard model corrections to this decay process are well understood, it is an ideal candidate for for improving constraints on interactions beyond the standard model.  

%Our setup utilizes a magneto-optical trap to confine and cool samples of $^{37}$K, which are then released and spin-polarized by optical pumping.  
%This allows us to perform measurements on both polarized and unpolarized nuclei, which is valuable for a complete understanding of systematic effects.  
%Precision measurements of this decay are expected to be sensitive to the presence of right-handed vector currents, as well as a linear combination of scalar and tensor currents.  Progress towards a final result is presented here.
%\color{black}

\subsection{Motivation}
\label{rslow_motivation}

The nuclear weak force has long been known to be a predominantly left-handed chiral interaction, meaning that immediately following an interaction (such as a beta decay) with a weak force carrying boson ($W^+,\: W^-,\: Z$), 
%leptons emerge with left-handed chirality.  
normal-matter leptons (such as the electron and electron neutrino) emerge with left-handed chirality
% as seen from the decay frame, 
%\comment{(chirality?)}, chirality, 
while the anti-leptons (e.g. the positron and electron anti-neutrino) emerge with right-handed chirality.  
%\comment{(chirality?)}.  chirality.  
In the limit of massless particles, the particle's chirality is the same as its helicity. Thus, in a left-handed model, the direction of an (ultrarelativistic) normal lepton's spin is antiparallel direction of its motion, and the direction of spin for an anti-lepton is parallel to its direction of motion.  For a non-relativistic particle the property of chirality is fairly abstract, and describes the appropriate group representation and projection operators to be used in calculations.  It should be noted that a fully chiral model is also one which is maximally parity violating.
% \comment{(cite someone)}.

This odd quirk of the nuclear weak force is not only \emph{predominantly} true, but it is, to the best of our current scientific knowledge, \emph{always} true -- that is, attempts to measure any right-handed chiral components of the weak force have produced results consistent with zero~\cite{severijns_beck_cuncic_2006}\cite{severijns_cuncic_2011}.  This project proposes a further measurement to constrain the strength of the right-handed component of the weak interaction.  

%
%\section{The Principle of the Measurement}
%\label{principle}
%\comment{ Of particular interest is the decay process: $^{37}\textrm{K} \rightarrow \,^{37}\textrm{\!Ar} + \beta^{+} + \nu_e$.  Among other useful properties, this is is a `mirror' decay, 
%meaning that the nuclear wavefunctions of the parent and daughter are identical up to their isospin quantum number.  
%the number of protons in the parent nucleus (19) is equal to the number of neutrons in the daughter, and the number of neutrons in the parent (18) is equal to the number of protons in the daughter.  
%This property allows us to place strong constraints on the size of the theoretical uncertainties for this decay process within the Standard Model.   We further exploit this property by noting that both the $^{37}\textrm{K}$ parent and the $^{37}\textrm{\!Ar}$ daughter have nuclear spin $I=3/2$, a fact which is key to this experiment. }

%We propose to exploit the principle of conservation of angular momentum to search for right-handed currents.  Given a sample of cold, spatially confined, spin-polarized atoms which decay according to a known scheme, we propose to observe the kinematics of the daughter particles and thereby reconstruct beta decay event information.  
%
%In particular, given a fully spin-polarized ($m_I=\pm I$) 
%$\asdf37K$
%$^{37}\textrm{K}$ nucleus, the $^{37}\textrm{\!Ar}$ daughter may not take on a higher spin quantum number.  This puts a limit on the kinematics of the decay process.  If we assume, as per the Standard Model, that the spin-$1/2$ $\beta^+$ must emerge in a spin eigenstate parallel to its direction of motion (up to relativistic corrections), and that furthermore, the normal-matter $\nu_e$ must emerge with spin anti-parallel to its direction of motion.  Then, in the limiting case where the $\beta^+$ and the $\nu_e$ emerge exactly back-to-back, the Standard Model predicts that we will have changed the nuclear spin quantum number $m_I$ (as quantized along the `back-to-back' axis) by one unit.  However if, as per our assumption, the parent nucleus was already fully `spun up' in state $m_I = +3/2$, then $m_I$ may only decrease.
%
%In terms of beta decay kinematics, this means that given a $^{37}\textrm{K}$ nucleus which is spin-polarized to point upwards, if we observe a `back-to-back' decay along the axis of polarization, the Standard Model tells us that we can expect that the $\beta^+$ has gone upwards and the $\nu_e$ has gone downwards.  If we observe an event where the opposite is true, then we will have observed `new' physics beyond the predictions of the Standard Model (See Figure~\ref{fig:rhc}).  
%
%\begin{figure}[h!t]
%	\centering
%	\includegraphics[width=.999\linewidth]
%	{Figures/RHC_verbose2.png}
%	\caption{ A list of the `back-to-back' polarized $\beta^+$ decays in a simplified one-dimensional geometry.  The decays are labelled according to the type(s) of interaction that could produce them (vector, axial vector, scalar, and tensor).  By fully spin-polarizing the parent nucleus in a mirror decay, we are able to exclude diagrams (a) and (b).  Diagram (c), which represents the `left-handed' linear combination of vector and axial vector interactions, is the only one of these processes allowed within the Standard Model. Subfigure (d) represents the `right-handed' combination of vector and axial vector interactions.  Subfigures (e), (f), (g), and (h) could be produced by linear combinations of scalar and tensor interactions, neither of which have been observed within the Standard Model.  
%Within our experimental set-up we do not measure the spin of daughter particles directly, and therefore the decays in the top row (a, c, e, g) would be indistinguishable from one another, and decays in the bottom row (b, d, f, h) would also be indistinguishable from one another.  However since none of the latter event types would be possible under the Standard Model, an observation of this event type would be indicative of the presence of scalar, tensor, or right-handed interactions within the nuclear weak force. 
%\comment{(top and bottom rows are what's distinguishable in our experiment.)}
%}	
%	\label{fig:rhc}
%\end{figure}
%
%The problem, then, becomes one of designing an experiment which will allow us to isolate beta decay events with the desired signature from the more mundane beta decay events predicted by the Standard Model.  



%\pagebreak

%
%\section{The Experimental Setup}
%\label{setup}
%
%\color{usedcolor}
%\subsection{Overview}
%\label{overview}
%
%\begin{figure}[t!h]
%	\centering
%	\includegraphics[width=.999\linewidth]
%	{Figures/doublemot4.pdf}
%	\caption{The TRINAT experimental set-up utilizes a two MOT system in order to reduce background in the detection chamber.}	
%	\label{fig:doublemot}
%\end{figure}
%
%The TRIUMF Neutral Atom Trap (TRINAT) offers an experimental set-up which is uniquely suited to precision tests of Standard Model beta decay physics.  Radioactive ions are delivered from the ISAC beamline and neutralized before being trapped in the first of two magneto-optical traps (MOTs).  Approximately once per second, atoms from the first MOT are transferred to the second, where their decay products can be observed with significantly less background than would have been possible in the first trap (see Figure~\ref{fig:doublemot}).  The transfer methodology is discussed in some detail in a paper by Swanson et al~\cite{swanson}.
%
%Once the newly transferred atoms -- in this case, neutral $^{37}\textrm{K}$ -- have arrived at the second trap, the MOT cycles 500 times between a state where it is `on' and actively confining atoms to a region of approximately 2\,mm$^3$, to a state where it is `off' and instead the atoms are spin-polarized by optical pumping while the atom cloud expands ballistically before being re-trapped.  In order to eliminate systematic effects, the polarization direction is flipped every 16 seconds.  This optical pumping technique and its results are the subject of a recent publication~\cite{ben_OP}.
%
%The science chamber (shown in Figure \ref{fig:thechamber}) operates at ultra-high vacuum (UHV) and provides the apparatus necessary to intermittently confine atoms within a MOT and then spin-polarize them, and quantify their position, temperature, and initial polarization, and electrostatic hoops to allow for collection and observation of charged recoiling daughter nuclei, as well as further detectors to observe the outgoing betas and reconstruct angular correlations.  
%\comment{(should I elaborate here?) -- "no".} 
%
%\begin{figure}[h!!!tb]
%	\centering
%	\hspace*{\fill}%
%	\subfloat[A decay event within the TRINAT science chamber.  After a decay, the daughter will be unaffected by forces from the MOT.  Positively charged recoils and negatively charged shake-off electrons are pulled towards detectors in opposite directions.  Although the $\beta^+$ is charged, it is also highly relativistic and escapes the electric field with minimal perturbation.
%	%\comment{The pic is still kind-of fuzzy.}
%	]
%	{\includegraphics[width=.530\linewidth]{Figures/chamber_decayevent3.png}\label{chamber_decayevent} }
%	\hspace*{\fill}
%	\hfill
%	\hspace*{\fill}
%	\subfloat[Inside the TRINAT science chamber.  This photo is taken from the vantage point of one of the microchannel plates, looking into the chamber towards the second microchannel plate.  The current-carrying copper Helmholtz coils and two beta telescopes are visible at the top and bottom.  The metallic piece near the center is one of the electrostatic `hoops' used to generate an electric field within the chamber.  The hoop's central circular hole allows access to the microchannel plate, and the two elongated holes on the sides allow the MOT's trapping lasers to pass unimpeded at an angle of 45 degress `out of the page'.]	
%	{\includegraphics[width=.444\linewidth]{Figures/chamber_photo_2.png}}
%	\hspace*{\fill}%
%	\caption{The TRINAT detection chamber}	
%	\label{fig:thechamber}
%\end{figure}
%\clearpage
%\color{black}
%
%\subsection{Trapping}
%\label{trap}
%The Magneto-Optical Trap is a well-known technique from atomic physics, used to confine and cool neutral atoms~\cite{raabprentiss}.  The technique is used predominantly with alkalis due to their simple orbital electron structure, and is quite robust, so is appropriate for use with $^{37}\textrm{K}$.  Once set up, the trapping force is specific to the isotope for which the trap has been tuned, which makes it ideal for use in radioactive decay experiments, since the daughters are unaffected by the trapping forces keeping the parent confined.
%
%There are two primary components necessary for any MOT:  a laser, and a magnetic field.  The laser, which must be circularly polarized in the appropriate directions and tuned slightly to the red of an atomic resonance, is split into three perpendicular retroreflected beams, doppler cooling the atoms and (with the appropriate magnetic field) confining them in all three dimensions (see Figure~\ref{fig:mot}).  The TRINAT science chamber includes 6 `viewports' specifically designed to be used for the trapping laser.
%
%A MOT also requires a quadrupolar magnetic field, which we generate with two current-carrying anti-Helmholtz coils located within the vacuum chamber itself.  The coils themselves are hollow, and are cooled continuously by pumping temperature-controlled water through them.   
%
%One feature which makes our MOT unusual has been developed as a result of our need to rapidly cycle the MOT on and off -- that is, it is an ``AC-MOT''.  Rather than running the trap with one particular magnetic field and one set of laser polarizations to match, we run a sinusoidal AC current in the magnetic field coils, and so the sign and magnitude of the magnetic field alternate smoothly between two extrema, and the trapping laser polarizations are rapidly swapped to remain in sync with the field~\cite{harveymurray}\cite{thesis}.  See Figure~\ref{fig:acmot}.  
%
%\begin{figure}[ht]
%	\centering
%	\subfloat[Components of a magneto-optical trap, including current-carrying magnetic field coils and counterpropagating circularly polarized laser beams.]
%	{\includegraphics[width=.237\linewidth]{Figures/mot.png}\label{fig:mot} }
%	\hspace*{\fill}
%%	\hfill
%	\hspace*{\fill}
%	\subfloat[One cycle of trapping with the AC-MOT, followed by optical pumping to spin-polarize the atoms.  After atoms are transferred into the science chamber, this cycle is repeated 500 times before the next transfer.  The magnetic dipole field is created by running parallel (rather than anti-parallel as is needed for the MOT) currents through the two coils.]	
%	{\includegraphics[width=.726\linewidth]{Figures/acmot.png}\label{fig:acmot} }
%	\caption{An alternating-current magneto-optical trap with a duty cycle optimized for producing polarized atoms}	
%	\label{fig:themot}
%\end{figure}
%
%The need to use an AC-MOT rather than a typical MOT has arisen as a direct result of our desire to optimally polarize our sample of atoms.  Polarization is incompatible with the non-uniform magnetic field used by a MOT, and rapidly shutting off the current used to produce the MOT's magnetic field produces eddy currents in the surrounding materials, which in turn produce their own non-uniform magnetic field in the region of interest.  The AC-MOT is developed as a way to ensure that the unavoidable eddy currents are behaving as we expect.  With an appropriate choice of shut-off phase, the current in the coils can be shut off when the overall magnetic field is already zero, so that no further eddy currents are induced.   

%{\color{cyan} (...) (something about why we need to run an AC-MOT.) }

%%%  Cut here?  %%%
%Note that because the atoms within a MOT can be treated as following a thermal distribution, some fraction of the fastest atoms continuously escape from the trap's potential well.  Even with the most carefully-tuned apparatus, the AC-MOT cannot quite match a similar standard MOT in terms of retaining atoms.  The TRINAT AC-MOT has a `trapping half-life' of around 6 seconds, and although that may not be particularly impressive by the standards of other MOTs, it is more than adequate for our purposes.  $^{37}\textrm{K}$ itself has a radioactive half-life of only 1.6 seconds \comment{(cite someone)}, so our dominant loss mechanism is radioactive decay rather than thermal escape. \comment{(Cut this whole paragraph?)}
%%%     %%%     %%%
%\clearpage

%


%\subsection{Optical Pumping}
%\label{op}
%We spin-polarize $^{37}\textrm{K}$ atoms within the trapping region by optical pumping~\cite{ben_OP}.  A circularly polarized laser is tuned to match the relevant atomic resonances, and is directed through the trapping region along the vertical axis in both directions.  When a photon is absorbed by an atom, the atom transitions to an excited state and its total angular momentum (electron spin + orbital + nuclear spin) along the vertical axis is incremented by one unit.  When the atom is de-excited a photon is emitted isotropically, 
%%\comment{(is it still isotropic when it's polarized?  I bet it's not.)}
%so it follows that if there are available states of higher and lower angular momentum, the \emph{average} change in the angular momentum projection is zero.  If the atom is not yet spin-polarized, it can absorb and re-emit another photon, following a biased random walk towards complete polarization.  
%
%In order to optimally polarize a sample of atoms by this method, it is necessary to have precise control over the magnetic field.  This is because absent other forces, a spin will undergo Larmor precession about the magnetic field lines.  In particular, the magnetic field must be aligned along the polarization axis (otherwise the tendency will be to actually depolarize the atoms), and it must be uniform in magnitude over the region of interest (otherwise its divergencelessness will result in the field also having a non-uniform direction, which results in a spatially-dependent depolarization mechanism).  Note that this type of magnetic field is not compatible with the MOT, which requires a quadrupolar magnetic field \emph{gradient}, and has necessitated our use of the AC-MOT as described in Subsection~\ref{trap}.


%\subsection{The Detectors}
%\label{detectors}
%\label{field}
%The beta detectors, located above and below the atom cloud along the axis of polarization (see Figure~\ref{chamber_decayevent}), are each the combination of a plastic scintillator and a set of silicon strip detectors.  Using all of the available information, these detectors are able to reconstruct the energy of an incident beta, as well as its hit position, and provide a timestamp for the hit's arrival.  Together the upper and lower beta detectors subtend approximately 1.4\% of the total solid angle as measured with respect to the cloud position. 
%
%It must be noted that the path between the cloud of trapped atoms and either beta detector is blocked by two objects:  a 254$\,\mu$m silicon carbide mirror (necessary for both trapping and optical pumping), and a 229$\,\mu$m beryllium foil (separating the UHV vacuum within the chamber from the outside world).  In order to minimize beta scattering and energy attenuation, these objects have had their materials selected to use the lightest nuclei with the desired material properties, and have been manufactured to be as thin as possible without compromising the experiment.  As the $^{37}\textrm{K} \rightarrow \,^{37}\textrm{\!Ar} + \beta^{+} + \nu_e$ decay proceess releases $Q=5.125$\,MeV of kinetic energy~\cite{Q_value}, the great majority of betas are energetic enough to punch through both obstacles without significant energy loss before being collected by the beta detectors.  
%
%On opposing sides of the chamber, and perpendicular to the axis of polarization, two stacks of $\sim$ 80\,mm diameter microchannel plates (MCPs) have been placed (see Figure~\ref{fig:thechamber}) as detectors, providing a time stamp when a particle is incident on their surfaces.  Behind each stack of MCPs there is a set of delay lines, which provide  position sensitivity for these detectors.   
%
%In order to make best use of these MCPs, we create an electric field in order to draw positively charged particles into one MCP, while drawing negatively charged electrons into the other MCP.  Seven electrostatic hoops have been placed within the chamber (see Figure~\ref{fig:thechamber}), and are connected to a series of high voltage power supplies.  See Sections~\ref{photoions} and~\ref{pos_recoils} for a discussion of what sort of charged particles we expect to observe in these detectors and how they are created.  
%  
%%\comment{(Say something about field uniformity?)}
%
%Scientific data has been collected at field strengths of 395 V/cm, 415 V/cm, and 535 V/cm.  It should be noted that these field strengths are too low to significantly perturb any but the least energetic of the (positively charged) betas from the decay process, and these low energy betas would already have been unable to reach the upper and lower beta detectors due to interactions with materials in the SiC mirror and Be foil vacuum seal.  
%


%\subsection{Cloud Measurements via Photoionization}
%\label{cloud}
%\label{photoions}
%In order to measure properties of the trapped $^{37}\textrm{K}$ cloud, a 10\,kHz pulsed laser at 355\,nm is directed towards the cloud.  These photons have sufficient energy to photoionize neutral $^{37}\textrm{K}$ from its excited atomic state, releasing 0.77\,eV of kinetic energy, but do not interact with ground state $^{37}\textrm{K}$ atoms.  The laser is of sufficiently low intensity that the great majority of excited state atoms are \emph{not} photoionized, so the technique is only very minimally destructive.  
%
%Because an electric field has been applied within this region (see Section~\ref{field}) the $^{37}\textrm{K}^+$ ions are immediately pulled into the detector on one side of the chamber, while the freed $e^-$ is pulled towards the detector on the opposite side of the chamber.  Because  $^{37}\textrm{K}^+$ is quite heavy relative to its initial energy, it can be treated as moving in a straight line directly to the detector, where its hit position on the microchannel plate is taken as a 2D projection of its position within the cloud.  Similarly, given a sufficient understanding of the electric field, the time difference between the laser pulse and the microchannel plate hit allows for a calculation of the ion's initial position along the third axis.  
%
%With this procedure, it is possible to produce a precise map of the cloud's position and size, both of which are necessary for the precision measurements of angular correlation parameters that are of interest to us here.  However, it also allows us to extract a third, slightly more subtle and significantly more important measurement:  the cloud's \emph{polarization}.
%
%The key to the polarization measurement is that only atoms in the excited atomic state can be photoionized.  While the MOT runs, atoms are constantly being pushed around and excited by the trapping lasers, so this period of time provides a lot of information for characterizing the trap size and position.  When the MOT is shut off, the atoms quickly return to their ground states and are no longer photoionized until the optical pumping beam is turned on.  As described in Section~\ref{op}, and in greater detail in~\cite{ben_OP}, the optical pumping process involves repeatedly exciting atoms from their ground states until the atoms finally cannot absorb any further angular momentum and remain in their fully-polarized (ground) state until they are perturbed.  Therefore, there is a sharp spike in excited-state atoms (and therefore photoions) when the optical pumping begins, and none once the cloud has been completely polarized.  The number of photoion events that occur once the sample has been maximally polarized, in comparison with the size and shape of the initial spike of photoions, provides a very precise characterization of the cloud's final polarization~\cite{ben_OP}.



%\comment{Photoions, the MCP, polarization stats, camera.}
%{\color{cyan} (...) }

%
\subsection{The Decay Process}
\label{rslow_decayprocess}
\label{pos_recoils}
% % %
The kinematics of nuclear $\beta^+$ decay are described by the following probability density function:
\bea
\label{jtw_pdf}
W(\langle I \rangle | E_\beta \hat{\Omega}_\beta \hat{\Omega}_\nu) 
&=& \left(\frac{1}{2\pi}\right)^{\!5} \!\! F\!\left( - Z, E_\beta \right)\, 
p_\beta E_\beta (E_0 - E_\beta)^2 \textrm{d}E_\beta \textrm{d}\hat{\Omega}_\beta \textrm{d}\hat{\Omega}_\nu \, \xi 
%
\!\!\!\! \nonumber \\ && \times
\left[ 1 + a_{\beta\nu} \frac{\vec{p}_\beta \cdot \vec{p}_\nu}{E_\beta E_\nu}
+ b_{\textrm{Fierz}} \frac{m_e}{E_\beta} 
\phantom{\frac{\left(\vec{p}_\beta\right)^2}{\vec{p}_\beta}} 
%
\!\!\!\! \right. \nonumber \\ &&\left. 
+ \, c_\textrm{align} \left( \frac{\frac{1}{3}\vec{p}_\beta \cdot \vec{p}_\nu - (\vec{p}_\beta \cdot \hat{j}) ( \vec{p}_\nu\cdot \hat{j} ) }{E_\beta E_\nu} \right) \!
 \left( \frac{I(I+1) - 3\langle (\vec{I}\cdot\hat{i})^2 \rangle}{I(2I-1)} \right) 
%
\right. \nonumber \\ && \left. 
+ \frac{\langle \vec{I} \rangle}{I} \left( A_\beta\frac{\vec{p}_\beta}{E_\beta} + B_\nu \frac{\vec{p}_\nu}{E_\nu} + D_{\textrm{TR}} \frac{\vec{p}_\beta \times \vec{p}_\nu}{E_\beta E_\nu} \right)
\right],
\eea
where $\vec{I}$ is the nuclear spin-polarization, $F\!\left( - Z, E_\beta \right)$ is the Fermi function, 
and parameters $\xi$, $a_{\beta\nu}$, $ b_{\textrm{Fierz}}$, $c_\textrm{align}$, $A_\beta$, $B_\nu$, and $D_{\textrm{TR}}$ are functions that vary with the strengths of the vector, axial, scalar, and tensor couplings (constant throughout nature), as well as the Fermi and Gamow-Teller nuclear matrix elements (specific to the individual decay)~\cite{jtw}\cite{jtw_coulomb}.

%{\color{cyan} (...) }

The decay may be treated as a three-body problem in which the available kinetic energy is divided up between the beta, the neutrino, and the recoiling $^{37}\textrm{\!Ar}$ nucleus, and (of course) the total linear and angular momentum are conserved.  While the neutrino cannot be detected directly, its kinematics may be reconstructed from observations of the beta and the recoiling daughter nucleus.  By placing detectors above and below the decaying atom along the axis of its polarization, we are able to obtain information about the outgoing beta's energy and momentum, in the cases of interest to us, where it is emitted along (or close to) the axis of polarization.  

The recoiling $^{37}\textrm{\!Ar}$ nucleus is a bit trickier to work with, but the task is not impossible.  One useful feature of the $^{37}\textrm{K} \rightarrow \,^{37}\textrm{\!Ar}$ transition is that, in addition to the $\beta^+$ emitted in the decay itself, one or more \emph{orbital} electrons from the parent atom are typically lost.  In the majority of decay events
%\comment{(80\% of them?  Give a number.  Cite someone.  Dan's thesis?~\cite{dan_thesis}  Or Alexandre's talk from like 1999.)}, 
only one orbital electron is `shaken off' and so the daughter $^{37}\textrm{\!Ar}$ atom is electrically neutral~\cite{gorelov2000}\cite{dan_thesis}.  In the remaining cases, two or more orbital electrons are lost this way, and the daughter atom is positively charged.  If we apply an electric field perpendicular to the direction of polarization, these positively charged $^{37}\textrm{\!Ar}^{(+n)}$ ions may be collected into a detector, from which hit position and time of flight information may be extracted.  These shake-off electrons are emitted with an average energy of only $\sim$ 2\,eV
%very little kinetic energy ($\sim$ 2\,eV)
%\comment{(give a number.  2 eV?  I should check.  Also, do I cite one of John's talks here?)}, 
so to a very good approximation the other decay products are not perturbed by the presence of shake-off electrons.  

It should be noted that for the class of decays of greatest interest, where the beta and the neutrino emerge back-to-back along the polarization axis, the recoiling daughter nucleus will have zero momentum along the directions perpendicular to this axis, and on average less total energy than if the beta and neutrino were emitted in a parallel direction.  Henceforth, daughter nuclei from a back-to-back decay as shown in Figure~\ref{fig:rhc} will be described as `slow' recoils.  In terms of observables, this means that if the electric field is configured to point along one of the axes perpendicular to the polarization direction, then when the recoiling ion is swept away into a detector, the slow recoil's hit position should be exactly along the projection of the polarization axis.  Furthermore, the slow recoil's time of flight should be in the middle of the time of flight spectrum, since other recoils will be emitted with momentum towards or away from the detector.  

%Of course, in any real experiment, the number of decays emitted in exactly one set of directions will be a set of measure zero.  \comment{(say something about the limits taken as beta decay events approach the optimal set of angles.  Or just delete this paragraph because it's stupid.)}
% % %

%\section{Proposed Project}
%\label{project}
%Using $^{37}\textrm{K}$ beta decay data collected in June 2014, I intend to reconstruct the recoil momenta both along the polarization axis and perpendicular to it, such that when combined with energy and hit position from the beta detectors, each event's full kinematics may be reconstructed.  The spectra created by these events will be compared against a series of Geant4-based Monte Carlo simulations.  
%
%Matched template fitting will be used to compare the experimental data to the simulation, meaning that the implicit vector, axial, scalar, and tensor couplings within Eq.~\ref{jtw_pdf} will be allowed to float separately within the simulation, and a series of simulation ``templates'' will be produced, and each separately fit to the data.  The quality of each fit will be used to determine a ``best'' experimental value for both parameters, as well as the error inherent in the measurement.  
%
%%{\color{cyan} (...) }
%%\comment{See Figure~\ref{fig:Rslow_tof} ?}
%\begin{figure}[h!!t]
%	\centering
%	\includegraphics[width=.999\linewidth]
%	{Figures/Rslow_tof_squished.png}
%	\caption{Time-of-flight spectra for charged $^{37}\textrm{\!Ar}$ recoils at 535 V/cm, sorted by whether the $\beta$ was detected after emerging \emph{with} or \emph{against} the nuclear polarization direction, compared against a simulation using a uniform electric field.  	Recoils with zero initial momentum along the flight axis arrive in the center of the distribution for their charge state.  
%	}	
%	\label{fig:Rslow_tof}
%\end{figure}

%
\subsection{Status of the $R_{\textrm{slow}}$ Measurement}
\label{rslow_status}
In June 2014, after several years of preparatory work beforehand (the author has been continuously involved with this project since 2010), approximately 7 days of beam time at TRIUMF was dedicated to the TRINAT $^{37}\textrm{K}$ beta decay experiment.  Approximately half of this data is suitable for use in this project.  During this period, approximately 10,000 atoms were held within the trap at any given time.  The cleaned spectra show around 50,000 polarized beta-recoil coincidence events in total, divided among measurements at three different electric field strengths (535 V/cm, 415 V/cm, 395 V/cm). 

%At the present time, analysis is underway.  The recoil MCP hit position data has been calibrated, and systematic effects in the trap position and size measurements are being considered.  The largest two remaining hurdles for the analysis both lie in improvements to the Monte Carlo.  

%The first challenge is to implement particle tracking within a \emph{non-uniform} electric field.  Using the true non-uniform electric field will shift the time-of-flight spectra overall by around 1\%, and is expected to change the \emph{shape} of spectra as well, since the deviation from uniformity changes significantly as a function of flight tragectory, and is greater the farther a particle ventures from the central axis.  This would affect the measured hit positions as well.  Taken together, the shape of the time-of-flight spectra (as in Figure~\ref{fig:Rslow_tof}) and the recoil hit position are critical to our reconstruction of the decay process, it is critical that we model them correctly.  

%The second challenge will be to allow our simulation to vary vector, axial, scalar, and tensor coupling constants separately while holding other physical parameters (such as the half-life and Fermi/Gamow-Teller ratio) constant.  This, too, will be absolutely critical to the analysis, and its implementation is likely to be non-trivial.

A fit to simulation has shown that the data that has already been collected has sufficient statistical power to measure the \emph{fractional} contribution of any polarized `new physics' beta decay parameter (ie right-handed, scalar, and tensor currents within the weak interaction) to a sensitivity of $\sim 2\%$ of its true value.  Systematic limitations are still being assessed.  
%Because previous measurements have already shown that any $(V-A)$, $S$, and $T$ terms within the Standard Model must be quite small~\cite{severijns_beck_cuncic_2006}\cite{severijns_cuncic_2011}, the present measurement is not likely to be able to add any new constraints to our understanding of Standard Model physics, and must instead be understood as complementary to previous measurements.
%Because previous measurements have 
%Quantifying this observable's sensitivity to physics beyond the Standard Model in comparison to previously measured constraints~\cite{severijns_beck_cuncic_2006}\cite{severijns_cuncic_2011} is work in progress.  



%Approximately of beamtime and several years of preparatory work beforehand, and its analysis is currently underway.  The statistical strength of our data is sufficient for a \comment{5\%?  (John said 5\%, but I don't really know how he calculated that...)} measurement to constrain right-handed weak interactions, however the systematic effects within the data have not yet been fully evaluated.  This constraint is far too weak to allo w for the discovery of a right-handed component of the nuclear weak force, and is likely also too weak to place any new constraints on its strength.  It could, at best, lend statistical strength to the constraints on right-handed currents that have already been observed.
%
%Unfortunately, due to beam scheduling and target creation procedures at TRIUMF, and because several components of the TRINAT science chamber have subsequently been removed and disassembled to prepare for future upgrades to the apparatus, it would be extremely difficult to collect any further data in a timely fashion.  

%\pagebreak





%%%%%% %%%%%%% %%%%%%%
\section{Conclusions}
Conclusions go here.













