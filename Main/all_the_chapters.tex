% !TEX root = ../thesis_main.tex



%%%% --- * --- %%%%	
\clearpage
%\part{Background}
%\chapter{Motivation}

\chapter{Background and Motivation}
\label{background_chapter}

\section{Exotic Couplings}
	In particular, we're interested in so-called scalar and tensor couplings within the nuclear weak force. Standard model beta decay involves only vector and axial-vector couplings, combined with a ``$(V-A)$'' handedness (left-handed).  
%\missingfigure{Make a sketch of the structure of a trebuchet.}

\section{Fierz Interference -- The Physical Signature}
	The physical effects resulting from the presence of scalar or tensor couplings include a small perturbation to the energy spectrum of betas produced by radioactive decay.  

\missingfigure{I need that simulated picture of the different beta energy spectra, with different values of $\bFierz$.}

\section{Present Limits}
	A bit about other people's physics.

\section{A Toy Experiment}
	A quick overview of how an experiment like this one would be set up to extract the physics of interest, to keep the reader from getting too lost in the rest of the thesis.
\note{Do I really even *want* to include a toy experiment?  And would I want to do it here??  What even is the point?  I think in the past I decided it was easier to build up a description of .... something .... starting this way.  But why??  Possibly as I continue to add content, it will become obvious again why I originally wanted to do this.}


%%%% --- * --- %%%%	
% !TEX root = ../thesis_main.tex



\clearpage
\chapter{Theoretical Overview}
\label{theory_chapter}

\section{The Basics of Beta Decay}
	%\\*
	Standard Model beta decay is well understood.  The Fermi model of beta decay is in all the textbooks, but you have to dig slightly harder to understand Gamow-Teller or mixed decays, all of which are relevant here.  
	
	via Krane~\cite{krane}
	Under the Allowed Approximation, we require that a beta decay may not carry away any orbital angular momentum, because we treat the nucleus as pointlike \aside{Is this even true?  The pointlike thing?} and work in the CM frame.  An Allowed decay can, however, change the total nuclear angular momentum, because the outgoing leptons have spin$=1/2$ and therefore carry angular momentum.  Therefore, in an allowed decay, the total nuclear angular momentum must always change by either $0$ or $1$.  
	
	From a 2006 paper by Severijns et al ~\cite{severijns_beck_cuncic_2006}, the selection rules for an allowed transition are:
	
\bea
\Delta I = I_f - I_i = \{0, \pm 1\} \\ 
\hat{\Pi}_i \, \hat{\Pi}_f = +1
\eea

	Then, you can separate the allowed transitions into singlet (anti-parallel lepton spins, $S=0$ -- a Fermi transition) and triplet states (parallel lepton spins, $S=1$ -- a Gamow-Teller transition).
	
	
	Fermi decays are so-called ``vector'' interactions, and happen when the spin of the two leptons involved are antiparallel, so there can be no change in angular momentum (at least in the case of the Allowed approximation).  
	
	Gamow-Teller decays involve two leptons with parallel spins, so the decay must change the projection of the nuclear angular momentum, $M_I$, by exactly one unit (in the case of the Allowed approximation).  They transition may or may not simultaneously change the total nuclear spin, $I$, by one unit.  These are ``axial-vector'' interactions.  (Note that $I=0 \rightarrow I=0$ interactions are never Gamow-Teller decays.  
	
	Probably everything in this section is yoinked from ~\cite{wong1990}, pg 212.  
	
	
%\section{JTW Formalism}	
%	%\\*
%	Describes how to search for a variety of BSM terms within beta decay.  Does not account for certain well-understood effects of similar (or greater) magnitude.
%	
%	% !TEX root = ../thesis_main.tex
%
%
%
% "A PDF for the People"
\bea
\textrm{d}^5\Gamma_{\textrm{JTW}} \!\!\!\! \!\! && \equiv \,\,
%\omega(\cdots) \mathrm{d} \E \, \dOmegae \, \dOmeganu \,\, = 
\,\,  \frac{1}{(2\pi)^5} \, \FFpm \pe \Ee (E_0 - \Ee)^2 \dEe \, \dOmegae \, \dOmeganu \, \nonumber\\ 
&&	\!\!\!\!  \!\!\!\!  \!\!\!\!  
	\times \,\, \xi \left[
	1 + \a \frac{\vecpe\cdot\vecpnu}{\Ee\Enu} + \bFierz \frac{\m c^2}{\Ee} 
    + \,\,  \calign \,\, \Talign(\vecJ) 
	\left(
		\frac{\vecpe \cdot \vecpnu}{3\Ee\Enu}
		- \frac{ (\vecpe\cdot \hatj) (\vecpnu\cdot\hatj) }{\Ee\Enu}
	\right)
	\!
\right. \nonumber\\ 
&&	\!\!\!\!  \!\!\!\!  \!\!\!\!  
	\left. + 
	 \frac{\vecJ}{J} \cdot
	\left(
		\A \frac{\vecpe}{\Ee} 
		+ \B \frac{\vecpnu}{\Enu} 
		+ \D \frac{\vecpe \times \vecpnu}{\Ee\Enu} 
	\right)
\right],
\label{equation:jtw_master}
\eea
%\textrm{d}^5\Gamma_{\textrm{JTW}} \,\, \equiv \,\, \omega(\cdots) \!\!\!\! \!\!\!\! \!\!\!\! \!\!\!\! && \,\,\,\, \,\,\,\, \mathrm{d} \E \, \dOmegae \, \dOmeganu 
%	% equation:jtw_master
%	
%\note{Probably I should now give values for things, or expressions for letters, or something.  }
%We haven't integrated out the neutrino momentum.  Neutrino energy itself is a redundant parameter, I think, because we are already using an endpoint energy and a beta energy, and we are not taking recoil-order effects into account.
%
%For ``convenience'', let's define a nuclear alignment term, $\Talign$, so that:
%\bea
%\Talign(\vecJ) &=& \TalignExpand
%\eea
%
%
%
%\section{Holstein Formalism}
%	An in-depth mathematical description of beta decay, including many smaller effects.  It does not include a description of the BSM physics of greatest interest to us.   Here, we've already integrated over neutrino momentum at least.  That's something.  Here's Holstein's Eq.~(52):
%% !TEX root = ../thesis_main.tex
%
%
%
% "A PDF for the People"
\bea
\mathrm{d}^3 \Gamma_{\textrm{Holstein}} &=& 2  G_v^2 \cos^2\theta_c \frac{\FF}{(2\pi)^4} \, \pe \Ee (E_0 - \Ee)^2 \dEe \, \dOmegae 
\nonumber\\
&& \times
\left\{
	F_0(\E) 
	+ \Lambda_1 F_1(\E) \hatn \cdot \frac{\vecpe}{\Ee}
	+ \Lambda_2 F_2(\E) \left[ \left( \nhat \cdot \frac{\vecpe}{\Ee} \right)^2 - \frac{1}{3}\frac{\pe^2}{\Ee^2} \right]
	\right. \nonumber\\ && \left.
	+ \Lambda_3 F_3(\E) 
		\left[ 
			\left( \hatn \cdot \frac{\vecpe}{\Ee} \right)^3
			- \frac{3}{5}\frac{\pe^2}{\Ee^2}\hatn \cdot \frac{\vecpe}{\Ee}
		\right]
\right\}
\label{equation:holstein52}
\eea
%	
%% equation:holstein52
%
%\section{Relation between JTW and Holstein Formalisms}
%	%\\*
%	To conduct a precision search for scalar and tensor couplings, it is necessary to combine the Holstein and JTW models into a single cohesive probability distribution.  
\section{Mathematical Formalism}
	In order to proceed with a measurement, we must find a master equation to describe the probability of beta decay events with any given distribution of energy and momenta among the daughter particles, as a function of the strength of the specific couplings of interest to us.  To do this, two sets of formalisms are combined -- the older formalism from Jackson, Treiman, and Wylde (JTW)~\cite{jtw},~\cite{jtw_coulomb}, which describes the effects of all types of Standard Model and exotic couplings of interest to us here, but which truncates its expression at first order in the (small) parameter of recoil energy, and a newer formalism from Holstein ~\cite{holstein}, which includes terms up to several orders higher in recoil energy, but which does not include any description of the exotic couplings of particular interest to us.  We note that because any exotic couplings present in nature have already been determined to be either small or nonexistant, it is sufficient to describe these parameters with expressions truncated at first order, despite the fact that it is still necessary to describe the larger Standard Model couplings with higher-order terms. 
	
	The procedure for combining the two formalisms is described in detail in Appendix~\ref{appendix_forthepeople}, so we will simply provide the combined master equation here:
	
\bea
\textrm{put  a master equation here.}
\eea
\aside{Do it!  Do the master equation!}



%%%% --- * --- %%%%	
% !TEX root = ../thesis_main.tex


%%%% --- * --- %%%%	
\clearpage
\chapter{The Experimental Setup}
\label{setup_chapter}

%\color{oldcolor}
\section{Overview}

The experimental subject matter of this thesis was conducted at TRIUMF using the apparatus of the TRIUMF Neutral Atom Trap (TRINAT) collaboration.  The TRINAT laboratory offers an experimental set-up which is uniquely suited to precision tests of Standard Model beta decay physics, by virtue of its ability to produce highly localized samples of isotopically pure cold atoms within an open detector geometry.  

The TRINAT lab accepts radioactive ions delivered by the ISAC beamline at TRIUMF.  These ions are collected on the surface of a hot zirconium \aside{Is it definitely zirconium?  I don't remember.} foil where they are electrically neutralized, and subsequently escape from the foil into the first of two experimental chambers (the ``collection chamber").  Within the collection chamber, atoms of one specific isotope -- for the purposes of this thesis,  \isotope[37]{K} -- are continuously collected into a magneto-optical trap (MOT).  Approximately once per second, the atoms in the collection MOT are transferred to a second experimental chamber (the ``detection chamber'') and loaded into a second MOT (see Fig.~\ref{fig:doublemot}).  Because the transfer and trapping mechanisms rely on tuning to specific atomic resonances, this setup allows for the selection of only a single isotope within the detection MOT, and a significantly reduced background relative to the initial beamline output.  The transfer methodology is discussed in some detail within another publication~\cite{swanson}.

\note{Probably describe the laser transfer method slightly.}  

%The TRIUMF Neutral Atom Trap (TRINAT) offers an experimental set-up which is uniquely suited to precision tests of Standard Model beta decay physics.  Radioactive ions are delivered from the ISAC beamline and neutralized before being trapped in the first of two magneto-optical traps (MOTs).  Approximately once per second, atoms from the first MOT are transferred to the second, where their decay products can be observed with significantly less background than would have been possible in the first trap (see Figure~\ref{fig:doublemot}).  The transfer methodology is discussed in some detail in a paper by Swanson et al~\cite{swanson}. \aside{The point is that this eliminates background from the decays of other stuff.  Or the same stuff.  Stuff that's not centered at the trap.}

\begin{figure}[t!h]
	\centering
	\includegraphics[width=.999\linewidth]
	{Figures/doublemot4.pdf}
	\caption{The TRINAT experimental set-up utilizes a two MOT system in order to reduce background in the detection chamber.}	
	\label{fig:doublemot}
\end{figure}

Once the newly transferred atoms have arrived at the second trap, the MOT cycles 500 times between a state where it is `on' and actively confining atoms to a region of approximately 2\,mm$^3$, to a state where it is `off' and instead the atoms are spin-polarized by optical pumping while the atom cloud expands ballistically before being re-trapped.  In order to eliminate systematic effects, the polarization direction is flipped every 16 seconds.  This optical pumping technique and its results are the subject of a recent publication~\cite{ben_OP}.

\note{Below is pretty vague.  I could do better, even for just an overview-summary thing.  Obvs I have to describe it in detail later on *somewhere*, though maybe not in the overview-summary...} 
 
Detectors are positioned about the second MOT for data collection.  The science chamber (shown in Figure \ref{fig:thechamber}) operates at ultra-high vacuum (UHV) and provides the apparatus necessary to intermittently confine atoms within a MOT and then spin-polarize them, and quantify their position, temperature, and initial polarization, and electrostatic hoops to allow for collection and observation of charged recoiling daughter nuclei, as well as further detectors to observe the outgoing betas and reconstruct angular correlations.  
%\comment{(should I elaborate here?) -- "no".} 

\begin{figure}[h!!!tb]
	\centering
%	\hspace*{\fill}%
	\subfloat[A decay event within the TRINAT science chamber.  After a decay, the daughter will be unaffected by forces from the MOT.  Positively charged recoils and negatively charged shake-off electrons are pulled towards detectors in opposite directions.  Although the $\beta^+$ is charged, it is also highly relativistic and escapes the electric field with minimal perturbation.
	%\comment{The pic is still kind-of fuzzy.}
	]
	{\includegraphics[width=.530\linewidth]{Figures/chamber_decayevent3.png}\label{chamber_decayevent} }
	\hspace*{\fill}
%	\hfill
	\hspace*{\fill}
	\subfloat[Inside the TRINAT science chamber.  This photo is taken from the vantage point of one of the microchannel plates, looking into the chamber towards the second microchannel plate.  The current-carrying copper Helmholtz coils and two beta telescopes are visible at the top and bottom.  The metallic piece near the center is one of the electrostatic `hoops' used to generate an electric field within the chamber.  The hoop's central circular hole allows access to the microchannel plate, and the two elongated holes on the sides allow the MOT's trapping lasers to pass unimpeded at an angle of 45 degress `out of the page'.]	
	{\includegraphics[width=.444\linewidth]{Figures/chamber_photo_2.png}}
%	\hspace*{\fill}%
	\caption{The TRINAT detection chamber}	
	\label{fig:thechamber}
\end{figure}
%\clearpage



%\color{black}
%\section{Double MOT System to Supply Atoms}
%%\\*
%Mostly, 
%this requires a diagram.  We take ions supplied by the ISAC beamline, neutralize them and trap them in the first MOT, then periodically transfer them to a second MOT.  Detectors are 
%positioned about the second MOT for data 
%collection.  This double MOT system eliminates a great deal of background.  
%Also, here's some random equation that doesn't really go with the topic of this section:
%\bea
\frac{\partial \rho}{\partial t} &\!\!\!\!\!\! \bigg|_{relax} \!\!\! &= \:\:\: -\frac{1}{2} \left( \hat{\Gamma} \rho + \rho \,\hat{\Gamma} \right) 
\label{eq:relax} \\
\frac{\partial \rho}{\partial t} &\!\!\!\!\!\!  \bigg|_{repop} \!\!\! &= \:\:\: \hat{\Lambda},
\label{eq:repop} 
\eea

%Furthermore, here's a picture that doesn't really go with the topic of this section.
%collection.  This double MOT system 
%\margintodo[color=green]{A thing that's worth noting is that (I think!) recoil-order corrections have been implicitly excluded at some point here.  ...Is this even true??}
%eliminates a great deal of background.  


\section{AC-MOT and Polarization Setup}

%\\*
In order to facilitate a measurement of $A_{\mathrm{\beta}}$, we went to great efforts to polarize the atom cloud, and quantify that polarization.  This resulted in a duty cycle in which the atoms were intermittently trapped in the AC-MOT, then optically pumped to polarize them.  While knowledge of the polarization is less critical in a measurement of $b_{\mathrm{Fierz}}$, we still use only the polarized portion of the duty cycle in order to minimize other systematic errors, such as the scintillator energy calibration and overall trap position.
\note[color=green]{Is that $\uparrow$ even true??  Because I'm really not sure it is.  Via Kofoedhansen, $(E_0 - E_e) = E_\nu$.  So there.}  
Anyway, here's some figures.  Or possibly one figure.  Whatever.  Also, here's a reference to a figure.  See Fig.~\ref*{fig:themot}, or also its subfigures, eg Fig.~\ref{fig:acmot} and Fig.~\ref{fig:mot}.  Maybe I have to subref them?  Like, eg, Subfig.~\subref{fig:acmot} and Subfig.~\subref*{fig:mot}.  What if we try to subref everything?  Consider, eg, Fig.~\subref{fig:themot}.
\margintodo{Does this work?  It really should.}


The Magneto-Optical Trap is a well-known technique from atomic physics, used to confine and cool neutral atoms~\cite{raabprentiss}.  The technique is used predominantly with alkalis due to their simple orbital electron structure, and is quite robust, so is appropriate for use with $^{37}\textrm{K}$.  Once set up, the trapping force is specific to the isotope for which the trap has been tuned, which makes it ideal for use in radioactive decay experiments, since the daughters are unaffected by the trapping forces keeping the parent confined.

There are two primary components necessary for any MOT:  a laser, and a magnetic field.  The laser, which must be circularly polarized in the appropriate directions and tuned slightly to the red of an atomic resonance, is split into three perpendicular retroreflected beams, doppler cooling the atoms and (with the appropriate magnetic field) confining them in all three dimensions (see Figure~\ref{fig:mot}).  The TRINAT science chamber includes 6 `viewports' specifically designed to be used for the trapping laser.

A MOT also requires a quadrupolar magnetic field, which we generate with two current-carrying anti-Helmholtz coils located within the vacuum chamber itself.  The coils themselves are hollow, and are cooled continuously by pumping temperature-controlled water through them.   

One feature which makes our MOT unusual has been developed as a result of our need to rapidly cycle the MOT on and off -- that is, it is an ``AC-MOT''.  Rather than running the trap with one particular magnetic field and one set of laser polarizations to match, we run a sinusoidal AC current in the magnetic field coils, and so the sign and magnitude of the magnetic field alternate smoothly between two extrema, and the trapping laser polarizations are rapidly swapped to remain in sync with the field~\cite{harveymurray}\cite{thesis}.  See Figure~\ref{fig:acmot}.  

\note{Note that because the atoms within a MOT can be treated as following a thermal distribution, some fraction of the fastest atoms continuously escape from the trap's potential well.  Even with the most carefully-tuned apparatus, the AC-MOT cannot quite match a similar standard MOT in terms of retaining atoms.  The TRINAT AC-MOT has a `trapping half-life' of around 6 seconds, and although that may not be particularly impressive by the standards of other MOTs, it is more than adequate for our purposes.  $^{37}\textrm{K}$ itself has a radioactive half-life of only 1.6 seconds 
(cite someone), so our dominant loss mechanism is radioactive decay rather than thermal escape. }



\begin{figure}[ht]
\centering
	\begin{subfigure}[t]{0.242\textwidth}
		\centering
		\includegraphics[width=\textwidth]{mot.png}
	%	\label{fig:mot}
		\caption{Components of a magneto-optical trap, including current-carrying magnetic field coils and counterpropagating circularly polarized laser beams.}
		\label{fig:mot}
	\end{subfigure}
	\hfill
	\begin{subfigure}[t]{0.728\textwidth}
		\centering
		\includegraphics[width=\textwidth]{acmot.png}
		\caption{One cycle of trapping with the AC-MOT, followed by optical pumping to spin-polarize the atoms.  After atoms are transferred into the science chamber, this cycle is repeated 500 times before the next transfer.  The magnetic dipole field is created by running parallel (rather than anti-parallel as is needed for the MOT) currents through the two coils.}
		\label{fig:acmot}
	\end{subfigure}
	\caption{An alternating-current magneto-optical trap with a duty cycle optimized for producing polarized atoms}	
	\label{fig:themot}
\color{black}
\end{figure}


We spin-polarize $^{37}\textrm{K}$ atoms within the trapping region by optical pumping~\cite{ben_OP}.  A circularly polarized laser is tuned to match the relevant atomic resonances, and is directed through the trapping region along the vertical axis in both directions.  When a photon is absorbed by an atom, the atom transitions to an excited state and its total angular momentum (electron spin + orbital + nuclear spin) along the vertical axis is incremented by one unit.  When the atom is de-excited a photon is emitted isotropically, 
%\comment{(is it still isotropic when it's polarized?  I bet it's not.)}
so it follows that if there are available states of higher and lower angular momentum, the \emph{average} change in the angular momentum projection is zero.  If the atom is not yet spin-polarized, it can absorb and re-emit another photon, following a biased random walk towards complete polarization.  

In order to optimally polarize a sample of atoms by this method, it is necessary to have precise control over the magnetic field.  This is because absent other forces, a spin will undergo Larmor precession about the magnetic field lines.  In particular, the magnetic field must be aligned along the polarization axis (otherwise the tendency will be to actually depolarize the atoms), and it must be uniform in magnitude over the region of interest (otherwise its divergencelessness will result in the field also having a non-uniform direction, which results in a spatially-dependent depolarization mechanism).  Note that this type of magnetic field is not compatible with the MOT, which requires a quadrupolar magnetic field \emph{gradient}, and has necessitated our use of the AC-MOT as described in Subsection~\ref{trap}.


%\color{black}
%	\subsection{\textbf{Nuclear Setup}}
\section{Measurement Geometry and Detectors}
	%\\*
	Needs several diagrams.  Back-to-back beta detectors along the polarization axis.  Back-to-back MCPs in an electric field to tag events from the trap, and to measure the trap position and polarization.  Hoops to produce the electric field.  Many laser ports to make the MOT functional, and for optical pumping.  Fancy mirror geometry to combine optical pumping and trapping light along the vertical axis.  Water-cooled (anti-)Helmholz coils within the chamber for the AC-MOT, fast switching to produce an optical pumping field.  
%	\subsection{\textbf{All the Detectors}}

The beta detectors, located above and below the atom cloud along the axis of polarization (see Figure~\ref{chamber_decayevent}), are each the combination of a plastic scintillator and a set of silicon strip detectors.  Using all of the available information, these detectors are able to reconstruct the energy of an incident beta, as well as its hit position, and provide a timestamp for the hit's arrival.  Together the upper and lower beta detectors subtend approximately 1.4\% of the total solid angle as measured with respect to the cloud position. 

It must be noted that the path between the cloud of trapped atoms and either beta detector is blocked by two objects:  a 254$\,\mu$m silicon carbide mirror (necessary for both trapping and optical pumping), and a 229$\,\mu$m beryllium foil (separating the UHV vacuum within the chamber from the outside world).  In order to minimize beta scattering and energy attenuation, these objects have had their materials selected to use the lightest nuclei with the desired material properties, and have been manufactured to be as thin as possible without compromising the experiment.  As the $^{37}\textrm{K} \rightarrow \,^{37}\textrm{\!Ar} + \beta^{+} + \nu_e$ decay proceess releases $Q=5.125$\,MeV of kinetic energy~\cite{Q_value}, the great majority of betas are energetic enough to punch through both obstacles without significant energy loss before being collected by the beta detectors.  

On opposing sides of the chamber, and perpendicular to the axis of polarization, two stacks of $\sim$ 80\,mm diameter microchannel plates (MCPs) have been placed (see Figure~\ref{fig:thechamber}) as detectors, providing a time stamp when a particle is incident on their surfaces.  Behind each stack of MCPs there is a set of delay lines, which provide  position sensitivity for these detectors.   

In order to make best use of these MCPs, we create an electric field in order to draw positively charged particles into one MCP, while drawing negatively charged electrons into the other MCP.  Seven electrostatic hoops have been placed within the chamber (see Figure~\ref{fig:thechamber}), and are connected to a series of high voltage power supplies.  See Sections~\ref{photoions} and~\ref{pos_recoils} for a discussion of what sort of charged particles we expect to observe in these detectors and how they are created.  
  
Scientific data has been collected at field strengths of 395 V/cm, 415 V/cm, and 535 V/cm.  It should be noted that these field strengths are too low to significantly perturb any but the least energetic of the (positively charged) betas from the decay process, and these low energy betas would already have been unable to reach the upper and lower beta detectors due to interactions with materials in the SiC mirror and Be foil vacuum seal.  




%%%% --- * --- %%%%	
% !TEX root = ../thesis_main.tex



%%%% --- * --- %%%%	
\clearpage	
\chapter{Calibrations and Data Selection}
\label{calibrations_chapter}
	
%\section{Polarization}
%	%\\*
%	Polarization measurement was conducted on a different set of data, collected in between the measurements used for $A_{\mathrm{\beta}}$ and $b_{\mathrm{Fierz}}$, and at a higher electric field, because we were unable to run both our MCP detectors simultaneously.  
%	
%\section{Trap Position}
%	%\\*
%	Measured using the same dataset that was used to quantify the polarization.  The trap drifts slightly over the course of our data collection.  Describe the rMCP calibration needed to extract this info.  
%

\section{Cloud Measurements via Photoionization}
\label{cloud}
\label{photoions}
In order to measure properties of the trapped $^{37}\textrm{K}$ cloud, a 10\,kHz pulsed laser at 355\,nm is directed towards the cloud.  These photons have sufficient energy to photoionize neutral $^{37}\textrm{K}$ from its excited atomic state, which is populated by the trapping laser when the MOT is active, releasing 0.77\,eV of kinetic energy, but do not interact with ground state $^{37}\textrm{K}$ atoms.  The laser is of sufficiently low intensity that the great majority of excited state atoms are \emph{not} photoionized, so the technique is only very minimally destructive.  
\note{Probably worth mentioning that we test this stuff offline on stable \isotope[41]{K}. }

Because an electric field has been applied within this region (see Section~\ref{field}) the $^{37}\textrm{K}^+$ ions are immediately pulled into the detector on one side of the chamber, while the freed $e^-$ is pulled towards the detector on the opposite side of the chamber.  Because  $^{37}\textrm{K}^+$ is quite heavy relative to its initial energy, it can be treated as moving in a straight line directly to the detector, where its hit position on the microchannel plate is taken as a 2D projection of its position within the cloud.  Similarly, given a sufficient understanding of the electric field, the time difference between the laser pulse and the microchannel plate hit allows for a calculation of the ion's initial position along the third axis.  

\note{As a check:  the camera measurements for photons from de-excitation.  It's aimed 35 degrees from vertical, with its horizontal axis the same as ..... one of the other axes.  I think it's the TOF axis.  I can check this when my computer comes back.   Also, there's an unknown additional delay between some of our DAQ channels that can't be explained by accounting for cable lengths, so we really like having the check there.}

With this procedure, it is possible to produce a precise map of the cloud's position and size, both of which are necessary for the precision measurements of angular correlation parameters that are of interest to us here.  However, it also allows us to extract a third measurement:  the cloud's polarization.

The key to the polarization measurement is that only atoms in the excited atomic state can be photoionized via the 355 nm laser.  While the MOT runs, atoms are constantly being pushed around and excited by the trapping lasers, so this period of time provides a lot of information for characterizing the trap size and position.  When the MOT is shut off, the atoms quickly return to their ground states and are no longer photoionized until the optical pumping laser is turned on.  As described in Section~\ref{op}, and in greater detail in~\cite{ben_OP}, the optical pumping process involves repeatedly exciting atoms from their ground states until the atoms finally cannot absorb any further angular momentum and remain in their fully-polarized (ground) state until they are perturbed.  Therefore, there is a sharp spike in excited-state atoms (and therefore photoions) when the optical pumping begins, and none once the cloud has been completely polarized.  The number of photoion events that occur once the sample has been maximally polarized, in comparison with the size and shape of the initial spike of photoions, provides a very precise characterization of the cloud's final polarization~\cite{ben_OP}.



\note{Trap position -- Measured using the same dataset that was used to quantify the polarization.  The trap drifts slightly over the course of our data collection.  Describe the rMCP calibration needed to extract this info.}
\note{Polarization measurement was conducted on a different set of data, collected in between the measurements used for $A_{\mathrm{\beta}}$ and $b_{\mathrm{Fierz}}$, and at a higher electric field, because we were unable to run both our MCP detectors simultaneously.  }
%\note{Need to describe how polarization works.  Needs a level diagram.  Needs another(?) level diagram for the photoionization, and maybe a third for the MOT.  Can I combine them all?  idk.}

Anyway, here is a nice table describing the atom cloud, for each of 3 runsets, and I'll immediately reference it right now, as Table~\ref{table:cloudpositions}:


% !TEX root = ../thesis_main.tex
%
%
%
\begin{table}[h!!!!t]
	\begin{center}
	\begin{tabular}{ c  | r || lcl | lcl || lcl | lcl |}
			\multicolumn{1}{c}{ Runsets  }{ } & \multicolumn{1}{  c  }{ } & 
				\multicolumn{3}{  c  }{ \!\!Initial Position\!\! } &  \multicolumn{3}{   c  }{ Final Position } &  \multicolumn{3}{   c  }{ Initial Size } &  \multicolumn{3}{   c  }{ Final Size } \\
			\cline{2-14}
			\multirow{3}{*}{EB $\leftarrow$ RB} %&\multirow{3}{*}{RB}  
								& $x$ & \,\,1.77 & \!\!$\!\! \pm  \!\!$\!\! & 0.03   & \,\,2.06   & \!\!$\!\! \pm  \!\!$\!\! & 0.08    & 0.601 & \!\!$\!\! \pm  \!\!$\!\! & 0.013 & 1.504 & \!\!$\!\! \pm  \!\!$\!\! & 0.047 \\
								& $y$ & -3.51    & \!\!$\!\! \pm  \!\!$\!\! & 0.04   & -3.33     & \!\!$\!\! \pm  \!\!$\!\! & 0.05    & 1.009 & \!\!$\!\! \pm  \!\!$\!\! & 0.013 & 1.551 & \!\!$\!\! \pm  \!\!$\!\! & 0.018 \\
								& $z$ & -0.661  & \!\!$\!\! \pm  \!\!$\!\! & 0.005 & -0.551   & \!\!$\!\! \pm  \!\!$\!\! & 0.021  & 0.891 & \!\!$\!\! \pm  \!\!$\!\! & 0.004 & 1.707 & \!\!$\!\! \pm  \!\!$\!\! & 0.015 \\
			\cline{2-14}
			\multirow{3}{*}{EC $\leftarrow$ RD} %&\multirow{3}{*}{RD}  
								& $x$ & \,\,2.22  & \!\!$\!\! \pm  \!\!$\!\! & 0.05  & \,\,2.33   & \!\!$\!\! \pm  \!\!$\!\! & 0.11    & 1.18   & \!\!$\!\! \pm  \!\!$\!\! & 0.04   & 1.538 & \!\!$\!\! \pm  \!\!$\!\! & 0.087 \\
								& $y$ & -3.68     & \!\!$\!\! \pm  \!\!$\!\! & 0.04  & -3.31      & \!\!$\!\! \pm  \!\!$\!\! & 0.06   & 0.965 & \!\!$\!\! \pm  \!\!$\!\! & 0.012 & 1.460 & \!\!$\!\! \pm  \!\!$\!\! & 0.030 \\
								& $z$ & -0.437   & \!\!$\!\! \pm  \!\!$\!\! & 0.09  & -0.346    & \!\!$\!\! \pm  \!\!$\!\! & 0.037 & 0.927 & \!\!$\!\! \pm  \!\!$\!\! & 0.007 & 1.797 & \!\!$\!\! \pm  \!\!$\!\! & 0.026 \\
			\cline{2-14}
			\multirow{3}{*}{ED $\leftarrow$ RE} %&\multirow{3}{*}{RE}  
								& $x$ & \,\,2.274 & \!\!$\!\! \pm  \!\!$\!\! & 0.012 & \,\,2.46 & \!\!$\!\! \pm  \!\!$\!\! & 0.06   & 0.386 & \!\!$\!\! \pm  \!\!$\!\! & 0.016 & 1.382 & \!\!$\!\! \pm  \!\!$\!\! & 0.046 \\
								& $y$ & -4.54      & \!\!$\!\! \pm  \!\!$\!\! & 0.04   & -4.28    & \!\!$\!\! \pm  \!\!$\!\! & 0.04   & 0.986 & \!\!$\!\! \pm  \!\!$\!\! & 0.08   & 1.502 & \!\!$\!\! \pm  \!\!$\!\! & 0.013 \\
								& $z$ & -0.587    & \!\!$\!\! \pm  \!\!$\!\! & 0.04   & -0.481  & \!\!$\!\! \pm  \!\!$\!\! & 0.018 & 0.969 & \!\!$\!\! \pm  \!\!$\!\! & 0.003 & 1.861 & \!\!$\!\! \pm  \!\!$\!\! & 0.013 \\
			\cline{2-14}
	\end{tabular}
	\end{center}
%	\note{These positions are for the *electron* runsets of those names.  Might want a chart of which rMCP runs are used to measure position for which eMCP runs.  Possibly in that other section.}
	\note{Sig figs here need work.}
	\note{Parameters measured with the recoil runs, and applied on the appropriate electron runs.}
	\caption[Cloud Position and Size]{Cloud Positions and Sizes -- Measured immediately before and immediately following the optical pumping phase of the trapping cycle.  Measurements are evaluated using rMCP runs, and are taken to describe the cloud during associated eMCP runs as well.  All entries are expressed in units of millimetres, and the size parameters describe the gaussian width.}
	\label{table:cloudpositions}
\end{table}




\note{Also, we noticed the trap drifting after one of the runs, because one of the batteries on one of the thingies adjusting the laser frequency (I think) was failing. }
\note[color=jb]{JB:  ``If we rejected the data with the MOT moving (indeed a battery determining the voltage controlled oscillator frequency offset between absorption in stable \isotope[41]{K} cell and the \isotope[37]{K} resonance) then that's all you need to say.''}
\note{describe how you'd turn this into a physical description of the cloud, with like a temperature and a sail velocity and shit.  with equations.}


\section{Plastic Scintillators}
	Energy calibration for the scintillator+PMT setup changed dramatically at one point.    Describe how calibration was done.  
	\note{How the fuck WERE these things done?!}
	\note[color=jb]{JB:  ``You can describe anything you did differently or improved, but you can and should otherwise defer all details of the scintillator calibration and DSSD calibration to Ben's paper and his thesis and Spencer's.  E.g. Section~\ref{section:bb1_systematics} ``statistical agreement between BB1 X and Y detectors' energies only makes a small effect on results" does not need the technical details beyond that statement."
	\label{thesisconventionjb} }
	\note[color=jb]{JB:  ``If you have some way of documenting the coding you used, that would be great."  ... yeah, it would, wouldn't it?}
	
\section{Strip Detectors}
	Also describe how the DSSD calibration was done, even though it wasn't implemented by me. 
	\note{How the fuck WERE these things done?!}

\section{The eMCP}
	I can describe the eMCP calibration here, even though it mostly wasn't implemented by me.  It is tangentially relevant to data selection and background estimation by providing an experimental energy spectrum for shake-off electrons.  It's actually a pretty neat algorithm that I basically wasn't involved with.
	\note[color=jb]{JB:  eMCP.  You need to describe the timing information obtained.  You also need a statement of whether or not you used the position information in your cuts.}

\missingfigure{Needs an SOE timing spectrum.  At least one of them.  Experimental and simulated.  Also, I have to describe how I did the simulating, and how I check that it's OK despite the fact that the simulated spectrum looks nothing like the experimental spectrum.}

\section{The rMCP}
	I did this, and they're absolutely needed to make any sense of the trap position data.  
\missingfigure{Needs two pictures from the rMCP with the grid lines -- before and after corrections are performed.  Just for fun, I could throw in one with the stupid stripes.}
	These calibrations are done during AC-MOT time, and we're actually interested in the rMCP data taken during OP time.  Can I find pictures to estimate the size of the change resulting from the magnetic field?  In any case, the change is pretty small.  

\missingfigure{Position as a function of run number.  I have this somewhere.}















%%%% --- * --- %%%%
% !TEX root = ../thesis_main.tex


%%%% --- * --- %%%%
\clearpage	

\chapter{The Experimental Signature}
\label{analysis_chapter}

\section{TBD}

I really need an excuse to include more pictures of data.  Also, more pictures of simulations

\missingfigure{Show individual beta energy spectra.  ...with a variety of different cuts, perhaps?}

\missingfigure{Show simulated spectra separated by scattering category.}

\missingfigure{Show SimpleMC spectra, show the supersum, show the superratio, show the superratio asymmetry.  Maybe do some simple fits to show how much better the superratio asymmetry is than \emph{not} the superratio asymmetry.  }



\section{The Superratio and Asymmetry}
%\\*
The data can be combined into a superratio asymmetry.  This has the benefit of causing many systematics to cancel themselves out at leading order.  It also will increase the fractional size of the effects we're looking for.  This can be shown by using math.  

\section{Signature of a Fierz Term in This Experiment}
%\\*
Not all systematics effects are eliminated.  We'll want to be careful to propagate through any effects that are relevant.  Using the superratio asymmetry as our physical observable makes this process a bit messier for the things that don't cancel out, but it's all just math.  

\section{Comparative Merits of the Superratio and Supersum for Measurement}
%\\*
Some other groups have performed similar measurements using the supersum as the physical observable.  There are pros and cons to both methods.  I can show, using a back-of-the-envelope calculation, that for this particular dataset, the superratio asymmetry method produces a better result.  





\chapter{Estimating Systematic Effects}
	\section{Low-energy Scintillator Threshold}
	%\\*
	Choice of low-energy scintillator threshold has a large systematic effect...  
	
	\section{BB1 Radius, Energy Threshold, Agreement}
	%\\*
	BB1 radius cut can help to eliminate scattered events.  Energy threshold selection and statistical agreement between BB1 detectors' energies only makes a small effect on results.  

\section{Background Modeling}
	\subsection{Decay from Chamber Surfaces}

\section{Quantifying the Effects Backscatter with Geant4}
	Beta decay (back-)scatter from surfaces within the experimental chamber is a significant systematic, and it must be evaluated, quantified, and corrected for.  This is done via a series of GEANT4 simulations.  While only a small fraction of events are affected, the process results in an energy loss in the beta that can, if not understood, be misinterpreted as the exact signal we're searching for.  It is therefore imperative that this be well understood. 

\section{Lineshape Reconstruction}
	\subsection{Motivation}
	This process is used because the (back-)scatter, which it itself an important systematic, is largely independent of a wide variety of other experimental effects.  These other effects must all be evaluated, but it is computationally prohibitive to re-evaluate the scattering with every other effect under consideration.
	
	\subsection{What is it and how does it work?}
	Mono-energetic beta decay events are generated in GEANT4, which outputs an energy spectrum for unscattered and forward-scattered beta events in the detector.  These spectra are fit to a function to model the scintillator resolution, as well as energy loss in materials that the beta passed through before arriving at the scintillator.  These spectrum fits are performed for a set of beta energies, and parameters are extrapolated to be applied to betas emitted at intermediate energies.  Thus, the whole spectrum can be modeled.  Pictures will make this clearer. 
	
	\subsection{The Math-Specifics}
	I'll write down the specific functions I'm using, and the parameter values I'm using.  (Maybe this should go in an appendix instead?)  I'll describe the adjustments I make to the spectrum so that it can work even for the dataset where the scintillators' resolutions have changed.
	
	\subsection{The Results -- Things That Got Evaluated This Way}
	Trap position, size, sail velocity.  Thicknesses of the SiC mirror, the Be foil, and the DSSD.  Scintillator calibration.  
	
	\subsection{The low-energy tail uncertainty, and what it does}


%%%% --- * --- %%%%	
\clearpage	
%\part{Results}
\chapter{Results}

\section{Measured Limits on $b_{Fierz}$, $C_S$, $C_T$}
	%\\*
	Results go here, with measured limits described and quantified in all formats anyone could ever care about.
	
\section{Discussion of Corrections and Uncertainties}
	...
	
\section{Relation to Other Measurements and New Overall Limits}
	%\\*
	In which I'll show exclusion plots and write down new limits, combining my result with results from the literature.


