% !TEX root = ../thesis_main.tex
% 
% 
% 
% 
%%%% --- * --- %%%%	
\chapter{Simulations}
\label{simulations_chapter}
%I really need an excuse to include more pictures of data.  Also, more pictures of simulations.
\missingfigure{Show simulated spectra separated by scattering category.}
\missingfigure{Show SimpleMC spectra, show the supersum, show the superratio, show the superratio asymmetry.  Maybe do some simple fits to show how much better the superratio asymmetry is than \emph{not} the superratio asymmetry.  }

%\note[color=done]{John says:  ``I doubt I will have further useful comments on the Ch. ((this chapter)) as they are now.'' -- Yeah, but *now* he will.}

The TRINAT collaboration has created a Geant4 (G4) simulation which models the geometry and materials within the experimental chamber, and uses a monte carlo algorithm to describe generalized physical processes such as particle scattering and energy loss, within the geometry specific to the experiment.  This software library has been maintained and updated over several generations of graduate students.  
\note{add comment about how  you need G4 to do scattering/backscattering.  I think I wrote a blurb like this *somewhere* already....}

\section{Considerations for Software Upgrade Implementation}
\label{sec:software_upgrades}
Prior to the simulations required for this particular experiment, two different sets of changes to the G4 code were needed -- the first to enable multithreading, and the second to introduce certain BSM interactions to the decay distribution.  

Enabling multithreading allows for a single instance of the Geant4 simulation to run on several processors at once, effectively speeding up the overall simulation by a factor of the number of processors used.  In the years since the simulation was originally created, the Geant4 collaboration had created libraries intended specifically to support multithread usage, and since the running G4 simulations had historically been very time consuming for the TRINAT collaboration, the decision was made to implement multithreading within our own monte carlo software, on the hopes that this would enable faster progress in analysis. 

Enabling multithreading support turned out to be quite time consuming, and in the end it might have been faster to have spent those months running simulations one processor at a time.  Perhaps the improvement will prove valuable for use in future TRINAT experiments.  

%Throughout the history of this simulation package, 
The TRINAT G4 monte carlo package had never been used to directly model interactions beyond the standard model within the decay physics.  
\aside{because for $\Abeta$ even a BSM interaction will *basically* look like a SM interaction, and I think something somewhere isn't precise enough to distinguish it.}  
It had  previously been set up by the collaboration to use a probability density function (PDF) including most of the terms from Holstein's Eq. (51)~\cite{holstein}, which describes both electron and neutrino momenta from polarized beta decay.  This treatment is quite robust, and includes corrections at recoil-order, as well as certain other corrections of similar size.~\aside[color=bluetodo]{which other corrections?  coulomb and/or radiative corrections, but somehow when I say that, I'm apparently talking about a different thing than everyone else who uses those terms.  also, weak magnetism. also ... ???}

Unfortunately, terms arising from interactions beyond the standard model are not included in Holstein's description of the decay process. \aside{Is this true?  Does it not include *any* BSM interactions?}  To understand the kinematic results of the exotic interactions of interest to us here, we turn to the classic JTW treatment of beta decay~\cite{jtw}~\cite{jtw_coulomb}.  In addition to the (expected) vector and axial interactions, JTW also describes the interaction in terms of (exotic) scalar and tensor interactions, should such be present.  
\note{Furthermore, although it is currently understood that the weak interaction is predominantly or perhaps entirely `left-handed', the JTW treatment leaves certain phase angles unfixed, and is therefore able to accurately describe a decay which is, for example, partially `right-handed' -- however the latter feature is not directly relevant to the project at hand.  [It's several phase angles in JTW, but maybe it's fundamentally only like one angle on some level?  Also I think it's not actually a ``gauge freedom,'' per se.  .... No, I'm pretty sure this `phase angle' description is all wrong.]  Also, consider time reversal!  Anyway, most of this paragraph probably goes better in Section~\ref{sec:math_formalism}.}
Despite JTW's broad ability to describe beta decay under a variety of physical models, this treatment includes only the leading-order terms, and smaller terms, such as recoil-order corrections, are neglected entirely.  

\note[color=org]{Surely I describe what recoil-order corrections even *are* in some appendix somewhere, right?  Or, possibly, in Section~\ref{sec:math_formalism}.  Possibly the paragraphs above need to be moved...}

Because the present project is a precision measurement of the Fierz interference, a term which arises from scalar and tensor couplings, it was imperative to create an event generator for our G4 simulations that could account for these exotic interactions while also including in its PDF the higher-order effects which, in some cases, can mimic the effects of a scalar or tensor current.  

While it might have been possible to directly combine JTW's result with Holstein's Eq. (51), it should be noted that JTW's expression is not compatible in general with the principle of conservation of momentum;  as recoil momentum is neglected entirely, the description is only of two leptons emerging from a nucleus in directions that do not directly oppose one another.  Therefore, the prospect of combining these two slightly incompatible expressions directly might be enough to give one pause.  On an experimental level, the mathematical description of an emerging neutrino is only of interest to us to the extent we can reconstruct it based on detecting both a beta and the recoiling nucleus from a single decay event, and within the present experiment we do not have simultaneous access to both a beta detector and a recoil detector.  

In light of the above considerations, it was decided that an entirely new event generator must be created, based instead on Holstein's Eq.~(52), in which neutrino momentum has been integrated over and is therefore no longer an explicit part of the PDF~\cite{holstein}.  As one might guess, Holstein's Eq.~(52) is greatly simplified in comparison to Holstein's Eq.~(51).  A similar integration over all possible neutrino momenta can also be performed on the JTW PDF, causing several terms to vanish.  The result in both the Holstein and JTW cases is a PDF over only beta energy and direction as measured with respect to nuclear polarization, and the two expressions can be combined in a straightforward manner by comparing similar terms.  

It is this combined Holstein+JTW expression that forms the basis of the new G4 event generator.  It must be noted that although the largest effect from any present scalar or tensor interactions would likely (depending on certain phase angles) be in a non-zero value of $\bFierz$, these interactions can also introduce a perturbation to $\Abeta$ at a higher order.  In order for any precision experimental measurement of $\bFierz$ to be generalized to limits on the parameter space of scalar and tensor currents, it is important to incorporate an accurate representation of the results of such exotic interactions on \emph{all} available observables, and the new G4 event generator does this.  

\note{Things that the G4 simulation did that I kept include:  an accurate representation of the complex details of our experimental geometry.  Also, the noise spectra on the DSSDs.}


\FloatBarrier
\section{The Simple Monte Carlo and Response Function}
\label{sec:responsefunction}
Because of the large number of systematic effects to be examined, and because of the processor time required to run a high statistics Geant4 simulation even after enabling multithreading support, it was desirable to develop a procedure by which it would be possible to evaluate certain systematic effects while avoiding the computationally intensive simulations that might traditionally be used.  To this end, a fast-running Simple Monte Carlo (SMC) was developed together with an empirical ``response function'' similar to the one described by Clifford et al~\cite{clifford} to describe probabilistic beta energy loss before its detection in a scintillator.  In the end, the lineshape description became quite involved, and it is unclear whether, in the end, any time was saved this way. 

The purpose of the SMC was to \emph{quickly} generate initial particle kinematics probabilistically for beta decay events, and it uses the very same event generator based on Holstein's Eq.~(52)~\cite{holstein} that was developed for use with the more sophisticated Geant4 simulations.~\aside{Reference previous section where I discuss this, maybe?}  However, unlike in a G4 simulation, the SMC makes no attempt to track particles through the chamber, and instead simply calculates detector hits based on initial particle momentum.  This procedure obviously neglects scattering effects, which can (in differing regimes) both \emph{increase} and \emph{decrease} the number of beta particles incident on a detector.  Furthermore, this procedure also neglects any energy absorption in materials through which the beta passes before hitting a scintillator -- and the beta \emph{must} pass through several such materials (see Fig.~\ref{chamber_decayevent}).

To make the best use of the SMC for evaluating systematic errors, the energy lost before a beta hits a scintillator must be accounted for somehow in order to ensure all relevant physical effects are propagated through.  In particular, before hitting a scintillator, a beta must pass through a $275\,\mu m$ thick silicon carbide mirror, a $229\,\mu m$ thick beryllium foil, %used to separate the inner portion of the vacuum chamber from the outside world
and finally a set of $300\,\mu m$ thick double-sided silicon strip detectors (DSSDs), before finally having its remaining energy absorbed within a scintillator.  Although the DSSDs are themselves detectors with the ability to record the amount of energy deposited by an incident particle, there are some known problems in achieving a uniform level of precision across the full surface of the DSSDs,\aside{See:  Some other section?  Maybe?} so adding the DSSD energy back to the scintillator energy to produce a better estimate of the original beta energy has the potential to create some problems for the analysis.  Furthermore, given the presence of the mirror, an object with a similar thickness and scattering properties to the DSSDs, re-adding the energy lost in the DSSDs would not eliminate the need to estimate probabilistic energy loss in similar materials.  

In order to create a quantitative description of the effective response function, which varies with initial beta energy, an analytic function of 14 parameters \aside{Is it definitely 14 of them?} has been created to model scintillator output for decays from the central cloud for each of the two polarization states in use.  In other words, although the form of the model is always the same, the 14 individual parameters will take different values for each of the four detector and polarization combinations.  The full response function model is given by the expression,
%for each of four detector and polarization combinations (56 parameters in total), to represent the final beta energy spectra in the scintillators.  
\bea
	R(E_{0}, \textrm{Detector}, \mathrm{Polarization}) &=& p_{\textrm{norm}} \left( f_{\mathrm{moyal}} + f_1 + f_2 + f_3 + f_4 + f_5 \right) + f_{511},
	\nonumber \\
	\label{eq:fullresponsefunction}
\eea
where $p_{\textrm{norm}}$ is one parameter, and the other terms within the expression are themselves functions of multiple parameters and are given by,

% !TEX root = ../thesis_main.tex
%
%
%
%%%% --- * --- %%%%	
%\bea
\begin{flalign}
f_{\mathrm{moyal}} \,\, = \,\, & 
	(1-p_{\mathrm{gfrac}}) \left( 1 + \frac{-p_\alpha - p_\beta }{\left| E_0 \right|} - \frac{p_\Delta}{ p_\gamma \, p_{W} } - p_\gamma \, p_{W} \right)  
	 && \nonumber \\ 
	 & \,\:\;\;\:\, \,\:\;\;\:\, \,\:\;\;\:\, \,\:\;\;\:\, \,\:\;\;\:\, \,\:\;\;\:\,
	 \times \left( 
	\frac{ e^{ \left( \frac{ x-\left( E_0-\frac{1}{2}p_{\mathrm{dE0}} \right) }{ 2 \, p_{\mathrm{lres}}\left|E_0-\frac{1}{2}p_{\mathrm{dE0}}\right| } \right) }  e^{\left(-\frac{1}{2} e^{\left( \frac{x - \left(E_0-\frac{1}{2}p_{\mathrm{dE0}} \right) }{ p_{\mathrm{lres}} \left| E_0-\frac{1}{2}p_{\mathrm{dE0}} \right| } \right)} \right)} }{\sqrt{2\pi \, p_{\mathrm{lres}} \left| E_0-\frac{1}{2}p_{\mathrm{dE0}} \right| } } 
	\right), &&
	\label{eq:fmoyal}
\end{flalign}
%\eea

\note{Obviously, from a physical standpoint, the initial beta energy $E_0$ must be positive, but the response function still includes several expressions of the form, $\left| E_0 \right|$.  This is not done by accident, but rather is an intentional adjustment used to encourage the parameters to behave well within a fit. }
% !TEX root = ../thesis_main.tex
%
%
%
%%%% --- * --- %%%%	
\begin{flalign}
	f_1 \,\, = \,\, & p_{\mathrm{gfrac}}  \left( 1 + \frac{-p_\alpha - p_\beta }{\left|E_{\textrm{in}}\right|} - \frac{p_\Delta}{ p_\gamma \, p_{W} } - p_\gamma \, p_{W} \right) \!
	\left( 
	\frac{ e^{\left( - \frac{ \left(E_{\textrm{out}}-\left(E_{\textrm{in}}+\frac{1}{2}p_{\mathrm{dE0}}\right) \right)^2 }{ 2 \, p_{\textrm{toeres}} \left| E_{\textrm{in}}+\frac{1}{2}p_{\mathrm{dE0}} \right| } \right)} } { \sqrt{2\pi \, p_{\textrm{toeres}} \left|E_{\textrm{in}}+\frac{1}{2} p_{\mathrm{dEin}}\right|} }
	\right), &&
	\label{eq:f1}
\end{flalign}
\begin{flalign}
	f_2 \,\, = \,\, & \frac{p_\alpha}{\left|E_{\textrm{in}}\right|} \left( \frac{1 - \Erf\left[ \frac{(E_{\textrm{out}}-\left|E_{\textrm{in}}\right|)}{\sqrt{2 \, p_{\textrm{toeres}} \left| E_{\textrm{in}} \right| }} \right]}{2 \, \left| E_{\textrm{in}}\right| } \right),
	&&
	\label{eq:f2}
\end{flalign}

\note[color=bluetodo]{In these response functions, $x=$?}
% !TEX root = ../thesis_main.tex
%
%
%
%%%% --- * --- %%%%	
\begin{flalign}
	f_3 \,\, = \,\, & \frac{p_\beta}{\left|E_0\right|}  \left( 	\frac{ e^{ \frac{ p_{\mathrm{k}} *(x-E_0) }{ \left|E_0\right|} } * e^{ \frac{p_{\textrm{toeres}} \, p_{\mathrm{k}}^2 }{2 \left|E_0\right|} } }{ 2 (1-e^{-p_{\mathrm{k}}} ) } \right)
	\left(1 - \Erf\left[ \frac{ (x-E_0+p_{\textrm{toeres}} \, p_{\mathrm{k}} ) } {\sqrt{2 \,p_{\textrm{toeres}} \left|E_0\right| } } \right] \right), &&
	\label{eq:f3}
\end{flalign}
\begin{flalign}
	f_4 \,\, = \,\, & \frac{p_\gamma}{2}   \left( \Erf \left[ \frac{ x-E_0 }{ \sqrt{(2 \, p_{\textrm{toeres}} \left|E_0\right| )} }  \right]  
	- \Erf \left[ \frac{ x-E_0-p_{W} }{ \sqrt{2 \, p_{\textrm{toeres}} \left| E_0+p_{W}\right| } } \right] \right), && 
	\label{eq:f4}
\end{flalign}

\note{To evaluate $\Erf\left[\right]$ for parameter fits, root's built-in function was used.  Root also includes a built-in Landau function, but it makes everything very slow if we use it.}
% !TEX root = ../thesis_main.tex
%
%
%
%%%% --- * --- %%%%	
\bea
	\!\! f_5 \, &\!\!= \,\,& \!\! \frac{p_{\Delta}}{2 \, p_{\gamma} \, p_{W}^3 }  
	\left[  
	(x-E_0) \left( \Erf\left[ \frac{(x-E_0) }{ \sqrt{2 \, p_{\textrm{toeres}} \left|E_0\right| } } \right] 
	- 2 \, \Erf\left[ \frac{ (x-E_0-p_{W}) }{ \sqrt{2 \, p_{\textrm{toeres}} \, \left| E_0+p_{W}\right| }} \right]
	\right. 
	\phantom{\left( \frac{e^{\left(\frac{(p_{W})^2}{ \left| E_0 \right|} \right)}}{\sqrt{2 \left|E_0 \right| }} \right) \!\!\!\!\!\!\!\! \!\!\!\!\!\!\!\! \!\!\!\!\!\!\!\! \!\!\!\! \!\!} %% phantom to get to get bracket size right..
	\right. \nonumber \\ && \left. \left.
	%
	+ \Erf\left[ \frac{(x - E_0 - 2 p_{W}) }{ \sqrt{2 \, p_{\textrm{toeres}} \, \left| E_0 + 2 p_{W}\right| }} \right] \right) 
	+ (2 \, p_{W}) \left( \Erf\left[ \frac{(x-E_0-p_{W})}{\sqrt{2 \, p_{\textrm{toeres}} \left|E_0+p_{W}\right| }}\right] 
	\right. \right. \nonumber \\ && \left. \left.
	%
	- \Erf \left[ \frac{(x-E_0 - 2 p_{W})}{\sqrt{2 \, p_{\textrm{toeres}} \left| E_0+2 p_{W}\right| }} \right] \right) 
	+ \left( 2\, p_{\textrm{toeres}} \left| E_0 \right| \right)  \left( \left( \frac{e^{\left(\frac{- (x-E_0)^2}{(4\,p_{\textrm{toeres}} \left| E_0\right| )} \right)} }{\sqrt{2 \pi \, p_{\textrm{toeres}} \left| E_0 \right| }}\right) 
	\right. \right. \nonumber \\ && \left. \left.
	%
	+ \left( \frac{ - 2 e^{ \left(\frac{ - (x-E_0-p_{W})^2 }{(4 \, p_{\textrm{toeres}} \, \left| E_0+p_{W}\right| )} \right) } }{( \sqrt{2\pi \, p_{\textrm{toeres}}\left| E_0+p_{W}\right| } )} \right) 
	+ \left( \frac{e^{\left(\frac{ - (x-E_0 - 2 p_{W})^2}{(4 \, p_{\textrm{toeres}} \, \left| E_0 + 2 \, p_{W}\right| )} \right)}}{\sqrt{2 \pi \, p_{\textrm{toeres}} \left|E_0 + 2 \,p_{W}\right| }} \right) 
	\right) \right],
%	\!\!\!\!\!\!\!
%	\nonumber \\
	\label{eq:f5}
\eea
 and
% !TEX root = ../thesis_main.tex
%
%
%
%%%% --- * --- %%%%	
\bea
	f_{511} &=& 
	\left| p_{\mathrm{scale}} \right|
	\left[ \left( \frac{195}{17} \sqrt{\frac{2}{\pi}} \, e^{\left(\frac{-(E_{\textrm{in}} - 308)^2}{578}\right) } \right)  
	+ \left(40 + \frac{(E_{\textrm{in}} - 210)^2}{900} \right) \! \left( 1 - \Erf\left[\frac{E_{\textrm{in}} - 334}{30}\right] \right) \!\!\!\! 
	\right. \nonumber \\ && \left.	
	+ \left( \frac{ (E_{\textrm{in}} -505)^2 }{1440}\right) \! \left( 1 - \Erf\left[ \frac{E_{\textrm{in}} - 505}{30}\right] \right) \! \left(1 + \Erf\left[\frac{E_{\textrm{in}}-334}{30}\right] \right)  
	\phantom{e^{\left(\frac{(E_{\textrm{in}})^2}{5} \right)  }} 
	\!\!\!\!\! \!\!\!\!\! \!\!\!\!\! \!\!\!
	\right],
	\label{eq:f511}
\eeawhere a $p$ with any subscript is taken to be a variable parameter that must be evaluated.  The expressions $f_1$, $f_2$, $f_3$, $f_4$, and $f_5$ are motivated by or taken directly from expressions of the same name within Clifford's description, and the individual parameters $p_\alpha$, $p_\beta$, $p_\gamma$, $p_\Delta$, $p_W$, and $p_k$ are closely related to their counterparts of similar name~\cite{clifford}.  The expressions $f_{\mathrm{moyal}}$ and $f_{511}$ represent a departure from the published treatment, however, and arise from physical behaviours within this experiment which are not described within Clifford's treatment.

In particular, $f_{511}$ is a rather inelegant representation of the annihilation radiation compton edge within our geometry.  Although the DSSD provides an effective veto for the overwhelming majority of these events--and indeed within Clifford's treatment this veto is treated as being perfect in its discernment--it is clear both from experimental spectra and the Geant4 simulations intended to represent them that there exist a small number of such events within our scintillator spectra that cannot be vetoed in this manner.  These events must be understood and adequately accounted for.  

It should be noted that no attempt is made to derive the expression for $f_{511}$ from first principles; the expression was chosen only because of its visual similarity to the spectrum's fit residuals before its inclusion.  This expression's contribution to the overall function is negligible at all but the lowest initial beta energies (Eq.~\ref{eq:f511}'s $p_{\mathrm{scale}}$ parameter, showing the absolute normalization of $f_{511}$, is plotted in the top right of Fig.~\ref{fig:lineshapeparams_scale}.), and is always negligible at scintillator energies above $\sim 500\,$keV, as can be seen in Fig.~\ref{fig:lineshape_750}.  We note that within the final analysis, all scintillator spectra will be given a low energy cutoff at 400\,keV, so the only the higher energy tail of $f_{511}$ will make any contribution.  \aside{Also, I think for this particular part of analysis, the cutoff was like 600 keV.} \aside[color=org]{Some of this content needs to go somewhere else.}

The expression $f_{\mathrm{moyal}}$ arises from the beta particles' energy loss within materials (i.e. the mirror, the beryllium foil, and the DSSD itself, as in Fig.~\ref{fig:thechamber}) before its eventual absorption within the scintillator.  Although Clifford's treatment does include a $\Delta E$ detector (our DSSD would be the equivalent), the energy absorbed in this detector is added back in to the total before Clifford's final spectra are modeled.  Although it would be possible to do something similar with our DSSD spectra, \aside[color=org]{We never attempt to do this, because reasons.  Don't I talk about this somewhere?} we would still be left with the problem of accounting for the similarly-shaped energy loss within the mirror and foil. 

The distribution for energy deposition within a thin material by an energetic charged particle, first described by Lev Landau in 1944~\cite{landaudistribution}, is now known as a Landau distribution.  This distribution has a variety of properties that make it challenging to work with -- notably its mean, variance, and all higher moments are undefined, and the distribution itself cannot be written in closed form.  Its primary redeeming mathematical feature, however, is the fact that the convolution of a Landau distribution with another Landau distribution is, itself, a Landau distribution, and this means that we can represent the sum total of energy absorption within three successive thin materials as a single Landau distribution.  

Within the present context, an expression for energy absorption that can be evaluated and re-evaluated quickly by computer with adjusted parameters is needed, as this must be used within a fit function.  To this end, we employ a so-called `Moyal function', which was developed in 1955 to be used as a closed form approximation to the Landau distribution~\cite{moyal}.  Indeed, Eq.~\ref{eq:fmoyal} is little more than a Moyal function.  
\note{...as previously mentioned, re-introducing the energy from the BB1s invites problems with maintaining a uniform energy threshold over the entire detector. }
\note{somewhere I have to talk about the empirical noise spectrum etc. on the BB1s.  Or maybe I've already mentioned it somewhere.}  

The values of these parameters are allowed to vary with initial beta energy, and must be determined empirically by a series of fits to simulated spectra.  
%Since the final response function is expected to be a function of initial beta energy, we must allow for these parameters to also vary as a function of initial beta energy.  
To effect this result, the TRINAT Geant4 simulation is used to generate a series of `mono-energetic' spectra.  That is, for each energy value under consideration (with discrete values selected to span the energy range of betas in our decay), events are generated in which every outgoing beta initially has the same amount of kinetic energy, and the angular distribution of these betas is physically appropriate for the polarization and beta energy under consideration.  These mono-energetic betas are propagated through the experimental geometry via Geant4, and the resulting scintillator spectra are recorded.  Each polarization state must be considered separately, but spectra for both detectors are generated simultaneously, as it is necessary to generate events into a full $4\pi$ steradians in order to fully account for betas scattered into- or away from the detectors.  Cuts identical to those imposed on the experimental data are applied (see Chapter~\ref{dataselection_chapter}).  Several such spectra are shown for the Bottom Detector in the `-' polarization state, with their best fit response functions, components thereof, and residuals of the fit, in Figs.~\ref{fig:lineshape_5000},~\ref{fig:lineshape_3000},~\ref{fig:lineshape_1500},~\ref{fig:lineshape_1000}, and~\ref{fig:lineshape_750}.

\begin{figure}[htb]
	\centering
	\includegraphics[width=.999\linewidth]
	{Figures/MonoFit_5000.png}
	\caption[Fit to Mono-Energetic Spectrum, 5000 keV]{Fit to Mono-Energetic Spectrum, 5000 keV (B-)}	
	\label{fig:lineshape_5000}
\end{figure}
\begin{figure}[h!!tb]
	\centering
	\includegraphics[width=.999\linewidth]
	{Figures/MonoFit_3000.png}
	\caption[Fit to Mono-Energetic Spectrum, 3000 keV]{Fit to Mono-Energetic Spectrum, 3000 keV (B-)}	
	\label{fig:lineshape_3000}
\end{figure}
\begin{figure}[h!!tb]
	\centering
	\includegraphics[width=.999\linewidth]
	{Figures/MonoFit_1500.png}
	\caption[Fit to Mono-Energetic Spectrum, 1500 keV]{Fit to Mono-Energetic Spectrum, 1500 keV (B-)}	
	\label{fig:lineshape_1500}
\end{figure}
\begin{figure}[h!!tb]
	\centering
	\includegraphics[width=.999\linewidth]
	{Figures/MonoFit_1000.png}
	\caption[Fit to Mono-Energetic Spectrum, 1000 keV]{Fit to Mono-Energetic Spectrum, 1000 keV (B-)}	
	\label{fig:lineshape_1000}
\end{figure}
\begin{figure}[h!!tb]
	\centering
	\includegraphics[width=.999\linewidth]
	{Figures/MonoFit_750.png}
	\caption[Fit to Mono-Energetic Spectrum, 750 keV]{Fit to Mono-Energetic Spectrum, 750 keV (B-)}	
	\label{fig:lineshape_750}
\end{figure}
\FloatBarrier  % this will have to go in the end, but it's really hard to edit this thing without it.

The values of the individual parameters contributing to the fit functions in, eg, Figs.~\ref{fig:lineshape_5000},~\ref{fig:lineshape_3000},~\ref{fig:lineshape_1500},~\ref{fig:lineshape_1000}, and~\ref{fig:lineshape_750} are allowed to vary with initial beta energy, and the energy dependence of each parameter must be modeled in order to extrapolate the shape of the response function to intermediate initial beta kinetic energy values that are not explicitly modeled.  For each parameter, the energy dependence is modeled by an analytic function selected to have similar characteristics.  Each of these analytic functions is itself a function of several parameters which can be adjusted to optimize its fit to the true best-fit energy dependence of the parameter it models.  Because some parameters are only weakly independent, it is necessary to perform these fits iteratively on only a single parameter at a time, revisiting earlier parameter fits after fixing other parameters to updated models.  The results of this process are shown in Figs.~\ref{fig:lineshapeparams_part1}, ~\ref{fig:lineshapeparams_part2}, ~\ref{fig:lineshapeparams_part3}, and~\ref{fig:lineshapeparams_part4}.

\begin{figure}[h!!tb]
	\centering
	\includegraphics[width=.999\linewidth]
	{Figures/LineshapeParams_part1.png}
	\note[color=done]{What even \emph{is} the thing plotted below E0 in the `residuals' spot, you ask?  It's `PseudoE', which is describes the difference between the original input energy, $E_0$, and the energy where the output spectrum is maximal. Or something.  To `pretty good' order, it's a straight line.  the `PseudoE' plot shows what's left after you fit it to a straight line.  It's fucking weird that it's negative everywhere.  Like, what?}
	\caption[Lineshape Parameter Fits (Part 1)]{Lineshape Parameter Fits (Part 1)}	
	\label{fig:lineshapeparams_part1}
\end{figure}
\begin{figure}[h!!tb]
	\centering
	\includegraphics[width=.999\linewidth]
	{Figures/LineshapeParams_part2.png}
	\caption[Lineshape Parameter Fits (Part 2)]{Lineshape Parameter Fits (Part 2)}
	\label{fig:lineshapeparams_scale}
	\label{fig:lineshapeparams_part2}
\end{figure}
\begin{figure}[h!!tb]
	\centering
	\includegraphics[width=.999\linewidth]
	{Figures/LineshapeParams_part3.png}
	\caption[Lineshape Parameter Fits (Part 3)]{Lineshape Parameter Fits (Part 3)}	
	\label{fig:lineshapeparams_part3}
\end{figure}
\begin{figure}[h!!tb]
	\centering
	\includegraphics[width=.999\linewidth]
	{Figures/LineshapeParams_part4.png}
	\caption[Lineshape Parameter Fits (Part 4)]{Lineshape Parameter Fits (Part 4)}	
	\label{fig:lineshapeparams_part4}
\end{figure}
%
%\FloatBarrier  % this will have to go in the end, but it's really hard to edit this thing without it.
It is useful to consider how well this empirical response function works to model the spectra.  One can see clearly from Figs.~\ref{fig:lineshape_5000}, ~\ref{fig:lineshape_3000}, ~\ref{fig:lineshape_1500}, ~\ref{fig:lineshape_1000}, and~\ref{fig:lineshape_750} that the fit residuals appear noticeably \emph{worse} at lower initial beta energies.  Fig.~\ref{fig:lineshape_redchi2} shows the reduced $\chi^2$ values arising from comparing mono-energetic G4 spectra to the empirical response functions described above, for all four detector and polarization combinations.

\begin{figure}[h!!tb]
	\centering
	\includegraphics[width=.999\linewidth]
	{Figures/Lineshape_Chi2.png}
	\note{How many DOF for these things?  I should put it on the picture.}
	\caption[Goodness of Fit for Modeled Response Functions]{Goodness of fit for modeled response functions, for all four detector+polarization combinations.  The models are clearly much better behaved at initial energies above $\sim 1200\,$keV.}	
	\label{fig:lineshape_redchi2}
\end{figure}


With the energy dependence for each of the response function's parameters carefully modeled, it becomes possible to make proper use of the full response function.  Given a decay event with a known beta energy from a nucleus with its initial polarization known, we can now predict a probabilistic response from \emph{both} scintillator detectors.  Obviously, for a single decay event, the full spectrum cannot be realized -- however in aggregate the modeled response function agrees well with results from the full Geant4 simulation, particularly at higher beta energies.  

\note{Needs a picture of the *full* beta energy distributions that come out of the lineshape thing.  To compare with (a) data and (b) G4.  Probably a superratio in there somewhere too.}
\note{Um.  Which of the scattering things did I actually put in at the end?  And when did I do it?  Like, how did I account for (back-)scattering?  I tried with/without .... scattering, I think?  and eventually decided not to do it.  for some reason.  I think it breaks normalization in some way that's more subtle than you would think.
\\ ... \\
Do I need the angular distribution in the end?  I think maybe I put in scattering later, and just used a cone for the first round.  I re-did this to do the opposite thing at some point.}

%%%\begin{itemize}
%%%	\item the parameters extracted from the mono-energetic beta spectrum fits, and empirical functions to model how these parameters vary with beta energy, are shown in Figs.~\ref{fig:lineshapeparams_part1}, ~\ref{fig:lineshapeparams_part2}, ~\ref{fig:lineshapeparams_part3}, and~\ref{fig:lineshapeparams_part4}.
%%%	\item How good are these fits, you ask?  See Fig.~\ref{fig:lineshape_redchi2}
%%%\end{itemize}



%\clearpage
\FloatBarrier
\section{Simulating the Background}
\label{sec:tof_bg}
%For this experiment, precise knowledge of the nuclear polarization ....
%The largest source of background events is  ........
One of the largest sources of background events in this experiment is from decaying $\isotope[37]{K}$ atoms that have escaped from the trap and become stuck on the other surfaces within the chamber.  The majority of these events can be eliminated simply by taking a time-of-flight cut on the eMCP relative to a scintillator hit time (as described in Section ~\ref{section:emcp_cuts}).  Unfortunately, this procedure cannot remove the entirety of the background, so what remains must be modeled and understood.  



\begin{figure}[h!tb]
	\centering
	\includegraphics[width=.999\linewidth]
	{Figures/SOE_TOF_Spectra.pdf}
	\note{Rephrase.}
	\caption{SOE TOF, model and data.  In the end, the data was cut to use only events with a TOF between -0.928 ns and +1.416 ns.  Max. possible background is like a factor of two too big.  Similar quality results no matter how you distribute the Levinger spectra between 4S and 3P, however adding the 0 eV electrons makes a big improvement to the agreement. }	
	\label{fig:soetof}
\end{figure}


\note{Reference Section~\ref{section:emcp_cuts}}
Our experiment has some background.  It's annoying and stupid, but as long as we can estimate how much of it there is and what it does, it's basically fine.

\begin{itemize}
		\item TOF cut requires a whole extra model of background in the TOF spectrum..
		\item Suppose background in the TOF spectrum is coming from decays of atoms that have gotten themselves stuck to surfaces within the chamber...
		\item Run G4 to get a beta TOF spectrum (w.r.t. the decay)
		\item Run COMSOL (credit to Alexandre) to track low-energy SOEs through the electric field from wherever they started, into the detectors.  Energy spectra from Levinger.
		\item Combine G4 and COMSOL spectra, event-by-event, while requiring that both the beta detector and the eMCP are hit according to the set of random numbers generated by each monte carlo separately.  Then, the beta and SOE will each have a TOF from decay to detector, and subtracting one from the other gives a timing spectrum that can be observed experimentally.  See Fig.~\ref{fig:soetof}.
		\item Upper limit for the fraction of events generated this way can be estimated by assuming that all losses from the trap not due to radioactive decay emerge isotropically from the trap and then stick to whatever chamber wall is in its path.  This upper limit is too big by a factor of 2.
	\item For each of those 3 simulations, sort the ``good'' data according to emission angle relative to the detector.  Do each detector individually.  For both polarizations.
	\item Assemble the (simulated) superratio asymmetry.  We'll compare it to data, and the $\chi^2$ from that comparison will be our figure of merit.  
	\item We can make a whole 2D parameter space for different values of $\Abeta$ and $\bFierz$, and compare them all (via their superratio asymmetries) to the experimental data.  We get the ``best'' values of $\Abeta$ and $\bFierz$, where $\chi^2$ is minimized.
	\item We can do this whole thing again for simulated data sets with different values of parameters that we vary as systematics.  Note how the best values of $\Abeta$ and $\bFierz$ change when each of the systematics are varied.
\end{itemize}

\begin{figure}[h!tb]
	\centering
	\includegraphics[width=.999\linewidth]
	{Figures/asymmetry_by_tof.pdf}
	\note{I think I want this picture to go in some other section.}
	\caption{The superratio asymmetry, averaged over all scintillator energies between 400-4800 keV, is used to compare the experimental data and simulated TOF model as a proxy for the quality of the model to estimate the background.  All other cuts have been applied.}	
	\label{fig:asymmetry_by_tof}
\end{figure}







%\clearpage
\FloatBarrier
\section{Comparing Simulations to Experimental Data}
%\note{This section shouldn't get a clearpage (or...should it?), but it's just really hard to read my bulletpoint list without it.}
\begin{itemize}
	\item Run 3 sets of G4 simulations with a bunch of statistics (N events, for data with like N/10 events).  Each one has the same nominal value of $\Abeta$, but with 3 different values of the scalar coupling $C_S$:  zero, and +/-(whatever).  Keep $C_T=0$.  Because reasons, we're not really able to distinguish between $C_S$ and $C_T$ in this experiment anyway, so might as well keep the analysis simple.
	\item Just run one set of 0.02*N events for the two percent branch.  We can't neglect it, but it isn't going to change much when we adjust BSM couplings.  Just use the old event generator from Holstein Eq.~(51).
	\item Match cuts in simulated data up to the cuts on experimental data.  Obviously.  DSSD cut, DSSD energy, one hit DSSD, one hit scint.  TOF cut, which requires a whole extra model of background in the TOF spectrum (see Section~\ref{sec:tof_bg}).
	\begin{itemize}
		\item Suppose background in the TOF spectrum is coming from decays of atoms that have gotten themselves stuck to surfaces within the chamber...
		\item Run G4 to get a beta TOF spectrum (w.r.t. the decay)
		\item Run COMSOL (credit to Alexandre) to track low-energy SOEs through the electric field from wherever they started, into the detectors.  Energy spectra from Levinger.
		\item Combine G4 and COMSOL spectra, event-by-event, while requiring that both the beta detector and the eMCP are hit according to the set of random numbers generated by each monte carlo separately.  Then, the beta and SOE will each have a TOF from decay to detector, and subtracting one from the other gives a timing spectrum that can be observed experimentally.  See Fig.~\ref{fig:soetof}.
		\item Upper limit for the fraction of events generated this way can be estimated by assuming that all losses from the trap not due to radioactive decay emerge isotropically from the trap and then stick to whatever chamber wall is in its path.  This upper limit is too big by a factor of 2.
	\end{itemize}
	\item For each of those 3 simulations, sort the ``good'' data according to emission angle relative to the detector.  Do each detector individually.  For both polarizations.
	\item Assemble the (simulated) superratio asymmetry.  We'll compare it to data, and the $\chi^2$ from that comparison will be our figure of merit.  
	\item We can make a whole 2D parameter space for different values of $\Abeta$ and $\bFierz$, and compare them all (via their superratio asymmetries) to the experimental data.  We get the ``best'' values of $\Abeta$ and $\bFierz$, where $\chi^2$ is minimized.
	\item We can do this whole thing again for simulated data sets with different values of parameters that we vary as systematics.  Note how the best values of $\Abeta$ and $\bFierz$ change when each of the systematics are varied.
\end{itemize}

