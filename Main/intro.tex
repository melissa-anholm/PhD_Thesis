% !TEX root = ../thesis_main.tex

%
\chapter{Introduction and Motivation}
\label{IntroductionChapter}
\section{Motivation}
\label{motivation}
Since the magneto-optical trap (MOT) was first described in 1987 by Raab et. al.~\cite{raabprentiss}, it has become a standard technique for confining cold samples of neutral atoms.  These cold trapped atoms may subsequently be used in the measurement of a variety of physical quantities.  

\section{Overview}
\label{overview}
Here's a citation for the 2016 PDG~\cite{PDG2016}.  Another citation of a different thing is here:~\cite{Determination2014}.  So ... now what?

Testing isotopes!  Here's an isotope:  \isotope[56]{Fe}.  And another one, here:  \isotope[37]{K}.  Relatedly, we consider $^{37}$K.


Until recently, one limitation of such samples was the necessity for the presence of a relatively large magnetic field, which is expected to partially destroy atomic polarization, limiting the precision of many types of measurements.  Here we discuss the construction of a newer type of MOT, the AC-MOT, which minimizes residual magnetic fields.  
%
The guys in~\cite{manchester} came up with the idea of the AC-MOT.  They made it work and did some stuff with it.  Good for them.
%
