% !TEX root = ../thesis_main.tex

% PackageColorSetupMacros.tex *probably* requires that we've already loaded ColorNamesMacros.tex, to define some color names that we'll use with packages here.  


% Comments in the Document:
%\newcommand{\comment}[1]{ \!\!{\color{magenta}#1 }\!\!\!\!}
\renewcommand{\comment}[1]{ \!\!{\color{magenta}#1 }\!\!\!\!}
%\newcommand{\co}[1] {{\color{orangered}#1 }}

% Code:
% uses 'listings' package, and it's set up to use colors defined via the 'colors' package (which should be above).
\usepackage{listings} % Blocks of code.
\newcommand{\codestyle}[1]{{\lstinline{#1}}}  % in-body text that looks like code!

% Code Blocks:
% For use with the "listings" package, that makes big blocks of code.  Uses 'color' package settings defined above.
\lstset{ %
	language=Python,                      % the language of the code
	basicstyle=\footnotesize\ttfamily,    % the size of the fonts that are used for the code
	backgroundcolor=\color{white},        % choose the background color; you must add \usepackage{color} or \usepackage{xcolor}
	commentstyle=\color{commentcolor},    % comment style
	numberstyle=\tiny\color{rownumcolor}, % the style that is used for the line-numbers
	stringstyle=\color{stringcolor},      % string literal style
	breakatwhitespace=false,              % sets if automatic breaks should only happen at whitespace
	breaklines=true,                      % sets automatic line breaking
	captionpos=b,                         % sets the caption-position to bottom
	extendedchars=true,                   % lets you use non-ASCII characters; for 8-bits encodings only, does not work with UTF-8
	frame=single,                         % adds a frame around the code
	keywordstyle=\color{keywordcolor},    % keyword style
	%morekeywords={*,...},                % if you want to add more keywords to the set
	%deletekeywords={...},                % if you want to delete keywords from the given language
	numbers=left,                         % where to put the line-numbers; possible values are (none, left, right)
	numbersep=5pt,                        % how far the line-numbers are from the code
	rulecolor=\color{black},              % if not set, the frame-color may be changed on line-breaks within not-black text 	(e.g. comments (green here))
	showspaces=false,                     % show spaces everywhere adding particular underscores; it overrides 	'showstringspaces'
	showstringspaces=false,               % underline spaces within strings only
	showtabs=false,                       % show tabs within strings adding particular underscores
	stepnumber=1,                         % the step between two line-numbers. If it's 1, each line will be numbered
	tabsize=2,                            % sets default tabsize to 2 spaces
	title=\lstname                        % show the filename of files included with \lstinputlisting; also try caption instead of title
}

% Hyperrefs:
% I think this has to be below the list of defined colors??
\usepackage{ifpdf}
\ifpdf
\usepackage{hyperref}
\else
\usepackage[hypertex]{hyperref}
\fi
\hypersetup{  % for hyperrefefs, I guess.
    colorlinks=true,         % false: boxed links;  true: colored links
    linkcolor=black,
    linkbordercolor=cyan,    % color of internal links (change box color with linkbordercolor)
%    citecolor=lime,
    citecolor=blue,
    citebordercolor=blue,    % color of links to bibliography
    filebordercolor=magenta, % color of file links
    urlbordercolor=cyan,     % color of external links
    urlcolor=magenta
}

