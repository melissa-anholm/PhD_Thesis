% !TEX root = ../thesis_main.tex



% MiscMacros.tex *probably* requires that we've already loaded ColorNamesMacros.tex, to define some color names that we'll use with packages here.  


% Comments in the Document:
%\newcommand{\comment}[1]{ \!\!{\color{magenta}#1 }\!\!\!\!}
\renewcommand{\comment}[1]{ \!\,{\color{magenta}#1 }\! }
\newcommand{\greycomment}[1]{ \!\,{\color{lgrey}#1 }\! }

%\renewcommand{\comment}[1]{ \!\!{\color{magenta}#1 }\!\!\!\!}
%\newcommand{\co}[1] {{\color{orangered}#1 }}


%% Some shortcut ToDo Macros:
\newcommand{\todoinline}[2][]{\todo[inline, color=orange, #1]{#2}}  % Can override color.
%\newcommand{\todoinline}[2][]{\todo[inline, color=#1]{#2}}
%
\newcommand{\margintodo}[2][]  % small note in the margin.
	{\todo[caption={#2}, size=\scriptsize, color=orange, #1]{\renewcommand{\baselinestretch}{1.0}\selectfont#2\par}}  % default spacing seems to be like 1.2.  Change to 1.0.  This also allows you to override the color selection.

\newcommand{\note}[2][orange]{\todo[inline, color=#1]{#2}}  % Can still override color.


\newcommand{\bluetodo}[2][]{\todo[inline, color=todoblue, #1]{#2}}  % Can override color.

\newcommand{\aside}[2][orange]  % small note in the margin.
	{\todo[caption={#2}, size=\scriptsize, color=#1]{\renewcommand{\baselinestretch}{1.0}\selectfont#2\par}}  % default spacing seems to be like 1.2.  Change to 1.0.  This also allows you to override the color selection.



