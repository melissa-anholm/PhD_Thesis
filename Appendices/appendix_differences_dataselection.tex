% !TEX root = ../thesis_main.tex



%%%% --- * --- %%%%	
\clearpage	
%\part{Appendices}
\chapter{Notable Differences in Data Selection between this and the Previous Result}
\section{Polarization Cycle Selection}
	Data used for our recent PRL article was slightly less polarized than we thought it was, due to an oversight in the data selection procedure.
	
\section{Leading Edge / Trailing Edge and Walk Correction}
	%\\*
	Using the leading edge rather than the trailing edge to mark the timing of TDC pulses cleans up jitter, eliminates background, and changes the relative delays between different inputs.  It is immediately relevant to the shape of the `walk correction' on scintillator timing pulses, which give a different prediction for beta arrival time as a function of scintillator energy.  %This subsequently affects models for the fraction of background events.  
	
\section{TOF Cut + Background Modelling}
	A SOE-beta time-of-flight cut is necessary to reduce background.  The above mentioned walk correction directly results in an change in which specific events are selected in a given TOF cut.  It further results in an adjustment to the expected fraction of background events in any such cut.
	
\section{BB1 Radius}
	Possibly my default radius cut on the DSSDs is a bit different.  The region of the parameter space that I'm taking for the systematic uncertainty on this is definitely a bit different.  
	
\note{Somebody will surely ask for a justification for why I did this differently, and I don't have one beyond ``this seemed more reasonable to me", which is of course nobody will ever accept as a reason.}