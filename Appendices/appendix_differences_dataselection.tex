% !TEX root = ../thesis_main.tex



%%%% --- * --- %%%%	
\clearpage	
%\part{Appendices}
\chapter{Notable Differences in Data Selection between this and the Previous Result}
%\note[color=done]{Removed some appendices:
%\\
%- Very Obvious Things (on lifetimes and half-lifes) \\
%- On Beta Endpoint Energies ("notationandprelimintegration") \\
%- Wong Nuclear \\ 
%- HowTo Lifetime
%}

\note[color=jb]{JB says Appendix A should all go in the analysis section, and not in an appendix at all.}
\note[color=jb]{JB says:  Appendix A (ie, *this appendix*) is very important, and should at least be a subsection in the Analysis chapter.\\...\\You could condense the Appendix into a set of bullet points at the end of the intro​​ to the Analysis section (which you still need, badly!), and then its content could be interleaved in the Analysis chapter.  E.g. you already have redundancy in the LE and TE discussion vs. the Appendix, and the discussion is more complete in the Analysis chapter, which is good.  }
\note{I really want this appendix to stay here.  I'll make sure to mention everything in the body of the thesis though, since it *is* important.  But at some point, somebody is going to really want to have this info written into a short summary.}


\section{Polarization Cycle Selection}
	Data used for our recent PRL article was slightly less polarized than we thought it was, due to an oversight in the data selection procedure.
	
\section{Leading Edge / Trailing Edge and Walk Correction}
	%\\*
	Using the leading edge rather than the trailing edge to mark the timing of TDC pulses cleans up jitter, eliminates background, and changes the relative delays between different inputs.  It is immediately relevant to the shape of the `walk correction' on scintillator timing pulses, which give a different prediction for beta arrival time as a function of scintillator energy.  %This subsequently affects models for the fraction of background events.  
	
\section{TOF Cut + Background Modelling}
	A SOE-beta time-of-flight cut is necessary to reduce background.  The above mentioned walk correction directly results in an change in which specific events are selected in a given TOF cut.  It further results in an adjustment to the expected fraction of background events in any such cut.
	
\section{BB1 Radius}
	Possibly my default radius cut on the DSSDs is a bit different.  The region of the parameter space that I'm taking for the systematic uncertainty on this is definitely a bit different.  
	
\note{Somebody will surely ask for a justification for why I did this differently, and I don't have one beyond ``this seemed more reasonable to me", which is of course nobody will ever accept as a reason.}

\section{BB1 Energy Threshold}
	I use an overall 50 keV threshold, (taking +/- 10 keV from that as a systematic to be propagated/checked), but I think Ben used 60 keV.  
	
