% !TEX root = ../thesis_main.tex
%
%
%
%
%%%% --- * --- %%%%	
\chapter[SuperRatio]{Derivation of the $\bFierz$ Dependence of the Superratio Asymmetry}
\label{appendix:superratio}
\note[color=jb]{ Appendix KLM you have to pick what you want-- I hope that's Appendix K (that's this one!) -- and remove the rest as you say they're "old". Appendix K could be moved to the end of Experimental Methods because it's absolutely critical and helpful!! but if you want to reference it there and leave it as an Appendix, it's up to you.}


%Consider the following probability distribution for beta decay, which is equivalent to Eq.~(\ref{equation:integrated_jtw}).  %\aside{It's not just equivalent, it's literally identical.  }
Recall the integrated JTW probability distribution for outgoing beta particles from Eq.~(\ref{equation:integrated_jtw}):
\bea
	\textrm{d}^3 \Gamma ( \Ebeta, \mathbf{ \hat{\Omega}}_\beta ) \, \dEe \, \dOmegae
	&=& 
	\frac{2}{(2\pi)^4} \, \FF \, \pe \Ee (E_0 - \Ee)^2 \, \dEe \, \dOmegae \, \xi \nonumber\\ 
	&& \times \left[
		1 + \bFierz \frac{\m c^2}{\Ee} + 
		\A  
		\left(
			\frac{\vecJ}{J} \cdot \frac{\vecpe}{\Ee} 
		\right) 
	\right].
	\label{equation:integrated_jtw_in_superratiosection}
\eea
We note that the only angular dependence remaining in this equation is the dot product between the direction of beta emission and the direction of nuclear spin-polarization.  This allows us to pull out a further factor of $2\pi$ by choosing the axis of polarization as defining our coordinate system, and integrating over the ``$\phi_\beta$'' coordinate.  The result is a bit more friendly to work with:
%Integrating this distribution again over $\phi_\beta$, while reducing the
\bea
	\textrm{d}^2 \Gamma  ( \Ebeta, \theta ) \, \dEe \, \textrm{d} \theta %\, \dEe \, \dOmegae
	&=&
%	\frac{2}{(2\pi)^3} \, \FF \, \pe \Ee (E_0 - \Ee)^2 \, \dEe \, \dOmegae \, \xi \nonumber\\ 
%	&& \times \left[
%		1 + \bFierz \frac{\m c^2}{\Ee} + 
%		\A  
%		\left(
%			\frac{\vecJ}{J} \cdot \frac{\vecpe}{\Ee} 
%		\right) 
%	\right]
%	\\
%	&=&
	W(\Ebeta) \left[ 1 + \bFierz \frac{\m c^2}{\Ebeta} + \Abeta \, \frac{v_\beta }{c} |\vec{P}| \cos\theta  \right] \, \dEe \, \textrm{d} \theta , 
%\label{equation:integrated_jtw}
\eea
where $\theta$ is the angle between the beta emission direction and the polarization direction, and is the only angular dependence that remains.  Here, we have grouped the overall energy dependence into $W(\Ebeta)$, so that
\beq
W(\Ebeta) = \frac{2}{(2\pi)^3} \, \FF \, \pe \Ee (E_0 - \Ee)^2.
\eeq
\note{We could also use this with the Holstein formulation, at least some of it.  The point is, we can put *anything* that only depends on beta energy into $W(\Ebeta)$.  It doesn't matter, because it's already only integrable through numerical methods anyway -- so we can't possibly make it worse.}

In the TRINAT geometry with two polarization states (+/-) and two detectors (T/B) aligned along the axis of polarization, we are able to describe four different count rates, with different combinations of polarization states and detectors.  Thus, neglecting beta scattering effects, we have:
\bea
r_{\mathrm T+}(\Ebeta) &=& \varepsilon_{\mathrm T}(\Ebeta)\, \Omega_{\mathrm T} \, N_+ \left[1 + \bFierz \frac{\m c^2}{\Ebeta}  + \Abeta \, \frac{v}{c} |\vec{P}_+| \langle \cos\theta \rangle_{\mathrm T+} \right] \label{eq:r1} \\
r_{\mathrm B+}(\Ebeta) &=& \varepsilon_{\mathrm B}(\Ebeta)\, \Omega_{\mathrm B} \, N_+ \left[1 + \bFierz \frac{\m c^2}{\Ebeta}  + \Abeta \, \frac{v}{c} |\vec{P}_+| \langle \cos\theta \rangle_{\mathrm B+} \right] \label{eq:r2}\\
r_{\mathrm T-}(\Ebeta) &=& \varepsilon_{\mathrm T}(\Ebeta)\, \Omega_{\mathrm T} \, N_- \left[1 + \bFierz \frac{\m c^2}{\Ebeta}  + \Abeta \, \frac{v}{c} |\vec{P}_-| \langle \cos\theta \rangle_{\mathrm T-} \right] \label{eq:r3}\\
r_{\mathrm B-}(\Ebeta) &=& \varepsilon_{\mathrm B}(\Ebeta)\, \Omega_{\mathrm B} \, N_- \left[1 + \bFierz \frac{\m c^2}{\Ebeta}  + \Abeta \, \frac{v}{c} |\vec{P}_-| \langle \cos\theta \rangle_{\mathrm B-} \right],\label{eq:r4}
\eea
where $\varepsilon_{\mathrm T / \mathrm B}(\Ebeta)$ are the (top/bottom) detector efficiencies, $\Omega_{\mathrm T / \mathrm B}$ are the fractional solid angles for the (top/bottom) detector from the trap position, $N_{+/-}$ are the number of atoms trapped in each (+/-) polarization state, and $|\vec{P}_{+/-}|$ are the magnitudes of the polarization along the detector axis for each polarization state.  $\langle \cos\theta \rangle_{\mathrm T/ \mathrm B, +/-} $ is the average of $\cos\theta$ for \emph{observed} outgoing betas, for each detector and polarization state combination.  This latter term is approximately $\pm 1$ as a result of our detector geometry, but contains important sign information.  For a pointlike trap in the center of the chamber, 103.484 mm from either (DSSSD) detector, each of which is taken to be circular with a radius of 15.5 mm, we find that $\langle | \cos\theta | \rangle_{\mathrm T/ \mathrm B, +/-} \approx 0.994484$, and is the same for all four cases.
\aside{Not quite true.  Some strips are missing.}
% (or for $r=15.0$\,mm, we find that $\langle | \cos\theta | \rangle_{\mathrm T/ \mathrm B, +/-} \approx 0.994829$). 
\aside{This is only true if we neglect (back-)scatter.  This is not actually a good approximation.  But we have pretty good simulations to give us the real numbers, anyway. } Note that a horizontally displaced trap will decrease the magnitude of $\langle | \cos\theta | \rangle $, but as it is an expectation value of an absolute value, all four will remain equal to one another.  In the case of a vertically displaced trap, these four values will no longer all be equal, however it will still be the case that $\langle | \cos\theta | \rangle_{\mathrm T +} = \langle | \cos\theta | \rangle_{\mathrm T -}$, and $\langle | \cos\theta | \rangle_{\mathrm B+} = \langle | \cos\theta | \rangle_{\mathrm B -}$.  \aside{Is that definitely true, or is it only true to lowest order?}

In the case of the present experiment, we note that $|\vec{P}_+| = |\vec{P}_-|$ is correct to a high degree of precision.
