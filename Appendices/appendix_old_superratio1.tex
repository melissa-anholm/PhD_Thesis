% !TEX root = ../thesis_main.tex
%
%
%
%
%%%% --- * --- %%%%	
\chapter[SuperRatio]{Derivation of the $\bFierz$ Dependence of the Superratio Asymmetry}

Consider, from JTW, the probability distribution for outgoing beta particles in terms of only the electron energy and direction w.r.t. parent nuclear polarization (other parameters have been integrated over).  We have
\bea
%P(\Ebeta) &=& W(\Ebeta) \left[ 1 + \bFierz \frac{mc^2}{\Ebeta} + \Abeta \, \frac{v}{c} |\vec{P}| \hat{P} \cdot \hat{\pbeta}  \right].
P(\Ebeta) &=& W(\Ebeta) \left[ 1 + \bFierz \frac{mc^2}{\Ebeta} + \Abeta \, \frac{v}{c} |\vec{P}| \cos\theta  \right].
\eea

In the TRINAT geometry with two polarization states (+/-) and two detectors (T/B), we are able to describe four different count rates, with different combinations of polarization states and detectors.  Thus, we have:
\bea
%\begin{subequations}
%\begin{align}
r_{\mathrm T+}(\Ebeta) &=& \varepsilon_{\mathrm T}(\Ebeta)\, \Omega_T \, N_+ \left[1 + \bFierz \frac{mc^2}{\Ebeta}  + \Abeta \, \frac{v}{c} |\vec{P}_+| \langle \cos\theta \rangle_{\mathrm T+} \right] \label{eq:r1} \\
r_{\mathrm B+}(\Ebeta) &=& \varepsilon_{\mathrm B}(\Ebeta)\, \Omega_B \, N_+ \left[1 + \bFierz \frac{mc^2}{\Ebeta}  + \Abeta \, \frac{v}{c} |\vec{P}_+| \langle \cos\theta \rangle_{\mathrm B+} \right] \label{eq:r2}\\
r_{\mathrm T-}(\Ebeta) &=& \varepsilon_{\mathrm T}(\Ebeta)\, \Omega_T \, N_- \left[1 + \bFierz \frac{mc^2}{\Ebeta}  + \Abeta \, \frac{v}{c} |\vec{P}_-| \langle \cos\theta \rangle_{\mathrm T-} \right] \label{eq:r3}\\
r_{\mathrm B-}(\Ebeta) &=& \varepsilon_{\mathrm B}(\Ebeta)\, \Omega_B \, N_- \left[1 + \bFierz \frac{mc^2}{\Ebeta}  + \Abeta \, \frac{v}{c} |\vec{P}_-| \langle \cos\theta \rangle_{\mathrm B-} \right],\label{eq:r4}
%\end{align}
%\end{subequations}
\eea
where $\varepsilon_{\mathrm T / \mathrm B}(\Ebeta)$ are the (top/bottom) detector efficiencies, $\Omega_{\mathrm T / \mathrm B}$ are the fractional solid angles for the (top/bottom) detector from the trap position, $N_{+/-}$ are the number of atoms trapped in each (+/-) polarization state, and $|\vec{P}_{+/-}|$ are the magnitudes of the polarization along the detector axis for each polarization state.  $\langle \cos\theta \rangle_{\mathrm T/ \mathrm B, +/-} $ is the average of $\cos\theta$ for \emph{observed} outgoing betas, for each detector and polarization state combination.  This latter term is approximately $\pm 1$, and contains important sign information.  For a pointlike trap in the center of the chamber, 103.484 mm from either (DSSSD) detector, each of which is taken to be circular with a radius of 15.5 mm, we find that $\langle | \cos\theta | \rangle_{\mathrm T/ \mathrm B, +/-} \approx 0.994484$, and is the same for all four cases (or for $r=15.0$\,mm, we find that $\langle | \cos\theta | \rangle_{\mathrm T/ \mathrm B, +/-} \approx 0.994829$). Note that a horizontally displaced trap will decrease the magnitude of $\langle | \cos\theta | \rangle $, but all four values will remain equal to one another.  In the case of a vertically displaced trap, these four values will no longer all be equal, however it will still be the case that $\langle | \cos\theta | \rangle_{\mathrm T +} = \langle | \cos\theta | \rangle_{\mathrm T -}$, and $\langle | \cos\theta | \rangle_{\mathrm B+} = \langle | \cos\theta | \rangle_{\mathrm B -}$.

For simplicity, we will henceforth assume that the trap is vertically centered, and take $|\vec{P}_+| = |\vec{P}_-|$.  We also define the following:
\bea
A^\prime &=& A^\prime(\Ebeta) \;\; \equiv \;\; \Abeta \, \frac{v}{c} \, |\vec{P}| \, \langle | \cos\theta | \rangle \\
b^\prime &=& b^\prime(\Ebeta) \;\; \equiv \;\; \bFierz \frac{mc^2}{\Ebeta},
\eea
and choose a coordinate system in which the + polarization state is, in some sense, `pointing up' toward the top detector, such that
\bea
\langle \cos\theta \rangle_{\mathrm T +} &\approx& +1 \\
\langle \cos\theta \rangle_{\mathrm B +} &\approx& -1 \\
\langle \cos\theta \rangle_{\mathrm T -} &\approx& -1 \\
\langle \cos\theta \rangle_{\mathrm B -} &\approx& +1.
\eea
This allows us to rewrite the four count rates in simplified notation, as: 
\bea
r_{\mathrm T+} &=& \varepsilon_{\mathrm T}N_+ \left(1 + b^\prime  + A^\prime \right) \\
r_{\mathrm B+} &=& \varepsilon_{\mathrm B}N_+ \left(1 + b^\prime  - A^\prime \right) \\
r_{\mathrm T-} &=& \varepsilon_{\mathrm T}N_- \left(1 + b^\prime  - A^\prime \right) \\
r_{\mathrm B-} &=& \varepsilon_{\mathrm B}N_- \left(1 + b^\prime  + A^\prime \right).
\eea
We further define the `superratio', $s$, to be:
\bea
s = \frac{ r_{\mathrm T-}\, r_{\mathrm B+} }{ r_{\mathrm T+}\, r_{\mathrm B-} }.
\eea
We are now in a position to define the `superratio asymmetry', $A_{\mathrm{super}}$, as
\bea
A_{\mathrm{super}} &=& A_{\mathrm{super}}(\Ebeta) \;\; \equiv \;\; \frac{1-\sqrt{s}}{1+\sqrt{s}}.
\eea
This is explicitly an experimental quantity that is measured directly by the above combination of count rates.  


Writing the superratio out explicitly in terms of $A^\prime$ and $b^\prime$, factors of $\varepsilon_{\mathrm T / \mathrm B}$ and $N_{+/-}$ cancel out entirely, and we find that
\bea
s &=& \frac{(1+b^\prime - A^\prime)^2}{(1+b^\prime+A^\prime)^2 }.
\eea
From here it is immediately clear that in absence of other corrections (\emph{e.g.} backscattering, unpolarized background, ...), if $b^\prime = 0$ it follows that $A_{\mathrm{super}} = A^\prime$.  In the case where $b^\prime \neq 0$, we find that 
\bea
A_{\mathrm{super}} &=& \frac{A^\prime}{1+b^\prime} \\
&\approx&  A^\prime \, (1 - b^\prime + {b^{\prime}}^2),
\eea
where we have utilized the assumption that $b^\prime \ll 1$.
Thus, 
\bea
A_{\mathrm{super}} &\approx& \Abeta \, \frac{v}{c} \, |\vec{P}| \, \langle | \cos\theta | \rangle - \Abeta \, \frac{v}{c} \, |\vec{P}| \, \langle | \cos\theta | \rangle \left( \bFierz \frac{mc^2}{\Ebeta}\right) + \Abeta \, \frac{v}{c} \, |\vec{P}| \, \langle | \cos\theta | \rangle \left(\bFierz \frac{mc^2}{\Ebeta}\right)^{\!\!2}.
\nonumber \\
\eea
%
%%% -- %%% -- %%% -- %%% -- %%% -- %%% -- %%% -- %%%









