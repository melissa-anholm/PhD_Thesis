% !TEX root = ../thesis_main.tex



%%%% --- * --- %%%%	
\chapter[Very Obvious Things]{Things that Should Be Very \mbox{Obvious}}
\label{fuckingduh}
%This section is not going to make it in to the final version, I hope, even if I really really need more pages.  Really, it's kind-of unfortunate that it gets included it in the drafts, but it's a pretty good place to store relevant information that I keep forgetting.

\note[color=jb]{JB:  ``You don't need anything about the atomic hyperfine structure in this thesis."  \[...\]  Me:  yeah, that's fair.  I probably should cut this whole section actually.}

\section{Lifetimes and Half-Lifes}
Since different people use different notation to describe exponential decay of a physical quantity, it is useful to be able to relate two of the most common methods for describing the decay.  We begin with the rate equation,
\bea
\label{rateequation}
\frac{d N}{d t} &=& -\gamma\: N,
\eea
where it is clear that the ``rate" of decay must be $\gamma\: N$.  If we initially have $N_0$ of the quantity in question, then Eq.~\ref{rateequation} has as its solution
\bea
\label{rateequationsolution}
N(t) &=& N_0 \:e^{-\gamma\: t}.
\eea
Note that the physical interpretation of $\gamma$ is the ``linewidth''.  

We'll wish to convert $\gamma$ into other quantities of interest.  In particular, we can re-write the solution \ref{rateequationsolution} as
\bea
N(t) &=& N_0 \:e^{-t/\tau},
%&=& N_0 \left( \frac{1}{2} \right)^{t/t_{1/2}}.
\eea
where $\tau = 1/\gamma$ is referred to as the ``lifetime''.  Then, we find the half-life $t_{1/2}$ by enforcing the fact that it is the time at which the number of remaining atoms is equal to half of what was originally present.  Therefore, 
\bea
N(t_{1/2}) = N_0 e^{-t_{1/2} / \tau} &=& \frac{1}{2} N_0 \\ 
%\eea
%and
%\bea
e^{-t_{1/2} / \tau} &=&  1/2 \\ 
t_{1/2} / \tau &=& \ln(2).
\eea

Thus, we see that 
\bea
%\tau &=& 1/\gamma \\
t_{1/2} &=& \ln (2) \: \tau, 
\eea
where $\tau$ is the ``lifetime'' of the state, and $t_{1/2}$ is its ``half-life''.
