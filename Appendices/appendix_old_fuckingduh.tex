\chapter[Fucking Duh]{Things that Should Be Very Fucking \mbox{Obvious}}
\label{fuckingduh}
%This section is not going to make it in to the final version, I hope, even if I really really need more pages.  Really, it's kind-of unfortunate that it gets included it in the drafts, but it's a pretty good place to store relevant information that I keep forgetting.

\note[color=jb]{JB:  ``You don't need anything about the atomic hyperfine structure in this thesis."  \[...\]  Me:  yeah, that's fair.  I probably should cut this whole section actually.}

\section{The Center of Gravity}
It's what happens when you set $A = 0$ and $B = 0$.   That is all.

\section{Diagonalizing the Hamiltonian}
Given a Hermitian matrix $\hat{\Omega}$, there exists a unitary matrix $\hat{U}$ such that $\hat{U}^{\dagger}\hat{\Omega}\hat{U}$ is diagonalized.  Solving for this matrix $\hat{U}$ is, in this case, equivalent to solving the eigenvalue problem for $\hat{\Omega}$ \cite{shankar}.  As it turns out, $\hat{U}$ is the matrix of eigenvectors of $\hat{\Omega}$, by which I mean that the eigenvectors are column vectors, and they're all squished together to make $\hat{U}$.  It doesn't matter what order you put them in, but they probably have to be normalized.

Then, if 
\beq
\hat{\Omega}^{'} := \hat{U}^{\dagger}\hat{\Omega}\hat{U},
\eeq	
we find that $\hat{\Omega}^{'}$ is the diagonal matrix with its elements being the eigenvalues.  They're in the same order as the eigenvectors we squished together to make $\hat{U}$ previously.

Also, a unitary operator, $\hat{U}$ is one which satisfies:
\bea
\hat{U}\hat{U}^{\dagger} = \hat{U}^{\dagger}\hat{U}  = \undertilde{I}.
\eea

\section{Rotating Coordinates}
\label{rotating}
If by $\hat{\Omega}$ we really mean the Hamiltonian $\hat{H}$, and by $\hat{U}$ we really mean a coordinate change that takes us to rotating coordinates such that we can easily make a rotating wave approximation, we must take more things into account.  In particular, we find that our new rotating-coordinate Hamiltonian, $\tilde{H}$ is given by:
\beq
\tilde{H} = \hat{U}^{\dagger} \hat{H} \hat{U} - \hbar \hat{A}
\eeq
where
\beq
\hat{U} := e^{-i \hat{A} t}.  
\eeq
This additional term arises from our statement of the Schrodinger equation, 
\beq
\label{schrodingerpartial}
\hat{H} \ket{\psi} = i \hbar \frac{\partial}{\partial t} \ket{\psi}.
\eeq
In particular, note that~\ref{schrodingerpartial} includes only a partial derivative of the wavefunction.  I derive this result explicitly in Chapter~\ref{general_rotating}.  See also Ref.~\cite{budker_opticallypolarized}, pg.~195.
% See, for example, Ref.~\cite{opticallypolarizedatoms}, pg.~195.  

\section{Lifetimes and Half-Lifes}
Since different people use different notation to describe exponential decay of a physical quantity, it is useful to be able to relate two of the most common methods for describing the decay.  We begin with the rate equation,
\bea
\label{rateequation}
\frac{d N}{d t} &=& -\gamma\: N,
\eea
where it is clear that the ``rate" of decay must be $\gamma\: N$.  If we initially have $N_0$ of the quantity in question, then Eq.~\ref{rateequation} has as its solution
\bea
\label{rateequationsolution}
N(t) &=& N_0 \:e^{-\gamma\: t}.
\eea
Note that the physical interpretation of $\gamma$ is the ``linewidth''.  

We'll wish to convert $\gamma$ into other quantities of interest.  In particular, we can re-write the solution \ref{rateequationsolution} as
\bea
N(t) &=& N_0 \:e^{-t/\tau},
%&=& N_0 \left( \frac{1}{2} \right)^{t/t_{1/2}}.
\eea
where $\tau = 1/\gamma$ is referred to as the ``lifetime''.  Then, we find the half-life $t_{1/2}$ by enforcing the fact that it is the time at which the number of remaining atoms is equal to half of what was originally present.  Therefore, 
\bea
N(t_{1/2}) = N_0 e^{-t_{1/2} / \tau} &=& \frac{1}{2} N_0 \\ 
%\eea
%and
%\bea
e^{-t_{1/2} / \tau} &=&  1/2 \\ 
t_{1/2} / \tau &=& \ln(2).
\eea

Thus, we see that 
\bea
%\tau &=& 1/\gamma \\
t_{1/2} &=& \ln (2) \: \tau, 
\eea
where $\tau$ is the ``lifetime'' of the state, and $t_{1/2}$ is its ``half-life''.

%\section{Lifetimes and Half-Lifes}
%Since different people use different notation to describe exponential decay of a physical quantity, it is useful to be able to relate two of the most common methods for describing the decay.  We begin with the rate equation,
%\bea
%\label{rateequation}
%\frac{d N}{d t} &=& -\gamma\: N,
%\eea
%where it is clear that the ``rate" of decay must be $\gamma\: N$.  If we initially have $N_0$ of the quantity in question, then Eq.~\ref{rateequation} has as its solution
%\bea
%\label{rateequationsolution}
%N(t) &=& N_0 \:e^{-\gamma\: t}.
%\eea
%Note that the physical interpretation of $\gamma$ is the ``linewidth''.  
%
%We'll wish to convert $\gamma$ into other quantities of interest.  In particular, we can re-write the solution \ref{rateequationsolution} as
%\bea
%N(t) &=& N_0 \:e^{-t/\tau} \\
%&=& N_0 \left( \frac{1}{2} \right)^{t/t_{1/2}}.
%\eea
%Thus, we see that 
%\bea
%\tau &=& 1/\gamma \\
%t_{1/2} &=& \ln (2) \: \tau, 
%\eea
%where $\tau$ is the ``lifetime'' of the state, and $t_{1/2}$ is its ``half-life''.

\section{Reduced Matrix Elements}
% via http://quantummechanics.ucsd.edu/ph130a/130_notes/node426.html , which is nice but also I should find some other source if I ever use this for anything ever.
% Actually, see Shankar pg. 420.  But I still don't know what it *means*.
% Similarly for Sakurai pg. 239.
The Wigner-Eckart Theorem says, for vector operator $V^q$,
\beq
\langle \alpha ' j' m' | V^q | \alpha j m \rangle = \langle j' m' | j1 m q\rangle \langle \alpha ' j' \| V \|\alpha j \rangle .
\eeq
The point being that $\langle \alpha ' j' \| V \|\alpha j \rangle$ is the same for all $m$ and $q$.


\section{Doppler Cooling Limit}
Here it is!
\beq
k T_{\textrm{D}} = \frac{1}{2} \hbar \Gamma
\eeq

%\section{Order of Magnitude Estimates for Magnetic Perturbations}
%We'll optically pump our atoms with a laser tuned close to some resonant frequency.  Also, there might be transverse magnetic fields.  The Hamiltonian term for an atom in an electric field is:
%\beq
%\label{stark}
%\hat{H}_{\vec{E}} = -\vec{d} \cdot \vec{E}, 
%\eeq
%where
%\beq
%\vec{d} = \alpha \vec{E}, 
%\eeq
%and $\alpha$ is the ``atomic polarizability", which is different for different types of atoms.  We like to refer to this part of the Hamiltonian as the Stark Effect, and in fact, this is only the lowest-order (dipole) term of the perturbation.  For Hydrogen, $\alpha_{H} = 0.67 \times 10^{-24} \mathrm{cm}^3$, though I don't know where I got that number from originally, and should probably check on that.
%
%The term in the Hamiltonian for the interaction between the atom and the external magnetic field is given by:
%\beq
%\label{zeeman}
%\hat{H}_{\vec{B}} = - \vec{\mu} \cdot \vec{B}.
%\eeq
%We call that one the Zeeman Effect, and again this is only the lowest-order (dipole) term.
%
%As it turns out, those Hamiltonians are really made for constant, non-varying fields.  We probably have to do more things to find the Hamiltonians for the AC-Stark and AC-Zeeman effects.
%
%\emph{(Cite someone for this stuff.  Ref.~\cite{budker_opticallypolarized} pg.~63 is good for the DC Stark Effect.)}