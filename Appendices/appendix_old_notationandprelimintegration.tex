% !TEX root = ../thesis_main.tex




\chapter[Notation]{Proposed Notation}
\label{notation}

\section{Things.}
The jtw expression, after integrating over $d\Omega_\nu$ and making some substitutions (and more controversially, adjustments so that neutrinos aren't massless, and recoil energy isn't zero), turns out like this:
\bea
& \!\!\!\!  \omega \mathrm{d(stuff})  \!\!\!\!\!\!\!\! & 
\nonumber\\
&= \!\!\!\!\!\!\!\! & 2\left( \frac{1}{2\pi} \right)^4 p_\beta E_\beta \left( |\vec{p}_\beta + \vec{p}_r | \right)(Q-E_\beta -E_r) dE_\beta d\Omega_\beta \xi \nonumber \\
&& \times \left[ \phantom{\frac{\left< \vec{J} \right>}{J}   \!\!\!\!\!\!\!\!   \!\!\!\!} 
 1 + a_{\beta\nu} \left( \frac{-|\vec{p}_\beta|^2 - \vec{p}_\beta \cdot \vec{p}_r}{E_\beta(Q-E_\beta -E_r)}\right) + b_{\mathrm{Fierz}} \left( \frac{m_e}{E_\beta} \right) \right. \nonumber \\
&& \left. + c_{\mathrm{align}} \left( \frac{J(J+1) - 3 \left<(\vec{J} \cdot \hat{j})^2 \right>}{ J(2J+1) } \right) \!\!
\left( \frac{- \frac{1}{3}(|\vec{p}_\beta|^2 + \vec{p}_\beta \cdot \vec{p}_r )+ (\vec{p}_\beta \cdot \hat{j})^2 + (\vec{p}_\beta \cdot \hat{j})(\vec{p}_r \cdot \hat{j}) }{E_\beta (Q-E_\beta-E_r)}\right) \right.\nonumber \\
&& \left. +\frac{\left< \vec{J} \right>}{J} \cdot \left[ A_\beta \frac{\vec{p}_\beta}{E_\beta} + B_\nu \left(\frac{-\vec{p}_\beta - \vec{p}_r}{Q-E_\beta-E_r}\right) + D \left( \frac{ (\vec{p}_r \times \vec{p}_\beta) }{E_\beta(Q-E_\beta-E_r)} \right) 
\right] \right].
\eea
Assumptions/substitutions that I made in order to arrive at that expression are:
\bea
&& Q = E_\beta + E_\nu + E_r \\
&& \vec{p}_\beta + \vec{p}_\nu + \vec{p}_r = 0 \\
&& (E_0 - E_e)^2 \rightarrow p_\nu E_\nu = \left|\vec{p}_\beta + \vec{p}_r \right| (Q-E_\beta-E_r) \\
&& \int d\Omega_\nu = 4\pi .
\eea
\section{Things.2}
Actually though, I'm'a back up.  Let's talk about dot products in spherical coordinates.  Of course, in Cartesian coordinates,
\bea
\vec{p}_\beta = p_\beta
\begin{bmatrix}
\sin\theta_\beta \cos\phi_\beta \\
\sin\theta_\beta \sin\phi_\beta \\
\cos\theta_\beta \\
\end{bmatrix} ;
& \vec{p}_r = p_r 
\begin{bmatrix}
\sin\theta_r \cos\phi_r \\
\sin\theta_r \sin\phi_r \\
\cos\theta_r \\
\end{bmatrix} ;
&\vec{p}_\nu= p_\nu
\begin{bmatrix}
\sin\theta_\nu \cos\phi_\nu \\
\sin\theta_\nu \sin\phi_\nu \\
\cos\theta_\nu \\
\end{bmatrix}
\eea
and
\bea
\vec{p}_\beta + \vec{p}_r + \vec{p}_\nu = 0.
\eea
We'd like to find an expression for $\vec{p}_\beta \cdot \vec{p}_\nu$, and we'd probably like to get rid of all dependence on the direction of $\nu$, which will make the integral over $d\Omega_\nu$ easier to deal with.  
\bea
\vec{p}_\beta \cdot \vec{p}_\nu 
&=& - \vec{p}_\beta \cdot (\vec{p}_\beta + \vec{p}_r ) \\
& = & -p_\beta^2 - \vec{p}_\beta \cdot \vec{p}_r \\ 
& = & -p_\beta^2 - p_\beta p_r  
\begin{bmatrix}
\sin\theta_\beta \cos\phi_\beta \\
\sin\theta_\beta \sin\phi_\beta \\
\cos\theta_\beta \\
\end{bmatrix}
\cdot
\begin{bmatrix}
\sin\theta_r \cos\phi_r \\
\sin\theta_r \sin\phi_r \\
\cos\theta_r \\
\end{bmatrix} \\
& = & -p_\beta^2 - p_\beta p_r  
\left[ \sin\theta_\beta \cos\phi_\beta \sin\theta_r \cos\phi_r
+ \sin\theta_\beta \sin\phi_\beta \sin\theta_r \sin\phi_r  \right.\nonumber\\
&& \left.
+ \cos\theta_\beta \cos\theta_r 
\right] \\
&=&  -p_\beta^2 - p_\beta p_r  
\left[ \cos\theta_\beta \cos\theta_r + \sin\theta_\beta \sin\theta_r \cos(\phi_\beta - \phi_r) \right],
\eea
where that last step happens via the use of some nice trig identities, and Mathematica confirms it. Equivalently, though:
\bea
\vec{p}_\beta \cdot \vec{p}_\nu 
&=& 
p_\beta p_\nu 
\left[ \cos\theta_\beta \cos\theta_\nu + \sin\theta_\beta \sin\theta_\nu \cos(\phi_\beta - \phi_\nu) \right].
\eea
That version is actually probably better, since recoil momentum isn't even a thing, for the purposes of integration.

Consider, now, the integral:
\bea \!\!\!\! \!\!\!\! \!\!\!\! \!\!\!\!
\int \frac{\vec{p}_\beta \cdot \vec{p}_\nu}{E_\beta E_\nu} d\Omega_\nu 
&=& \int_0^{2\pi} \!\!\! \int_0^{\pi} \frac{\vec{p}_\beta \cdot \vec{p}_\nu}{E_\beta E_\nu} \sin\theta_\nu d\theta_\nu d\phi_\nu \\ 
&=& \frac{p_\beta p_\nu }{E_\beta E_\nu}  \int_0^{2\pi} \!\!\! \int_0^{\pi} 
\left[ 
\cos\theta_\beta \cos\theta_\nu + \sin\theta_\beta \sin\theta_\nu \cos(\phi_\beta - \phi_\nu)
\right] 
\sin\theta_\nu d\theta_\nu d\phi_\nu \\
&=&  \frac{p_\beta p_\nu }{E_\beta E_\nu} \left[
2\pi  \cos\theta_\beta \int_0^{\pi} \cos\theta_\nu \sin\theta_\nu d\theta_\nu \right.\nonumber\\
&&\left.+ \sin\theta_\beta \int_0^{2\pi} \!\!\! \cos(\phi_\beta - \phi_\nu)
\int_0^{\pi}  \sin^2\theta_\nu  d\theta_\nu d\phi_\nu
\right] \\
&=& \frac{p_\beta p_\nu }{E_\beta E_\nu} \frac{\pi}{2} \sin\theta_\beta \int_0^{2\pi} \!\!\! \cos(\phi_\beta - \phi_\nu) d\phi_\nu \\
&=& 0.
\eea
That obviously didn't work.  Probably $p_\nu = p_\nu(\hat{p}_\nu)$.  How do I even *deal* with that?!?

--

Actually, according to insight that Alexandre thought was very obvious, because it was, this notation only happened in the first place because the lab frame matters.  it's measured w.r.t. polarization or alignment or something.  The integration over the two leptons is really an integration over the lab frame + one lepton.  

\section{Things.3}

Suppose I integrate jtw over everything but electron energy.  What's left?
\bea
\omega \mathrm{d(stuff}) &=& \left( \frac{1}{2\pi} \right)^5 \! (4\pi)^2 \,p_\beta E_\beta E_\nu^2 \, dE_\beta \, \xi  
 \left[1 + b_{\mathrm{Fierz}} \left( \frac{m_e}{E_\beta} \right) \right].
\eea
...Or, in units that don't suck, with terminology that's useful to us (and assuming massless neutrinos, which is fine, 'cause the uncertainty in $Q$ is like ~200 eV or something (check value) ), 
\bea
\!\!\!\! \!\!\!\! \!\!\!\! \!\!\!\!
\omega \mathrm{d(stuff}) &=& 4 \left( \frac{1}{2\pi} \right)^3 \!
\frac{(E_\beta^2 - m^2c^4)^{1/2} }{c}  \, E_\beta 
(Q - E_\beta - E_{r\textrm{kin}})^2 
\, dE_\beta \, \xi  
\left[1 + b_{\mathrm{Fierz}} \left( \frac{m_ec^2}{E_\beta} \right) \right]
\eea
Integrate over the beta energy (**NOT** between $E_\beta = m_e c^2$ and $E_\beta = m_e c^2 + Q$ ; fix this in the code!) and we find:
\bea
\omega &=& 4 \left( \frac{1}{2\pi} \right)^3 \! (...)
\eea 
This is *probably* interpreted as the overall decay rate, somehow.

\section{Beta End-point Energy}
Note that nuclear physics notational convention, which is fucking retarded, apparently defines kinetic energy $T$ s.t. 
\beq
E = T + m c^2,
\eeq
but of course, it's still the case that
\beq
E = (p^2 c^2 + m^2 c^4)^{1/2}.
\eeq
But since $Q=T_{final} - T_{initial}$, we're sort-of stuck, if we want to describe the beta end-point energy in terms of $Q$, we find that
\bea
&& E_\textrm{end} = Q + m_e c^2 \\
&& Q = (p_\textrm{max}^2 c^2 + m_e^2 c^4)^{1/2} - m_e c^2
\eea

... actually, pretty sure that's all wrong.  ... ... Actually-actually, it's fine.  See \\ ``jtw\_integration\_scratch3.nb'' for more-other details.  That version, I believe, finally got all the thingies right.

\section{Beta End-point Energy.2}
Firstly, some Q-values for $^{37}$K (via bnl).
\bea
Q_{\textrm{EC}} &=& 6.147\,45 (23) \, \textrm{MeV} \\
Q_{\beta+}          &=& 5.125\, 45 (23) \, \textrm{MeV}
\eea
It would be better if I knew the branching ratio for EC/$\beta+$ though.

Anyway, in the jtw notation, 
\bea
E_0 := Q + m_e c^2.
\eea
\todo[inline]{Is that $\uparrow$ even true??  Because I'm really not sure it is.  Via Kofoedhansen, $(E_0 - E_e) = E_\nu$.  So there.
}
What does that mean in terms of the individual particles' energies?  Well, if that's the beta end-point energy, it just means that that's the total (kinetic + rest) energy for the beta.  So, we can distribute $Q$ of kinetic energy around to the other particles.  
\bea
Q &=& T_\beta + T_\nu + T_r 
\label{Q_def}
\\
&=& (E_\beta - m_\beta c^2) + (E_\nu - m_\nu c^2) + (E_r - M_r c^2) \\
&\approx& E_\beta - m_\beta c^2 + E_\nu + T_r \\
E_0 &\approx& E_\beta + E_\nu + T_r 
\eea
\todo{A thing that's worth noting is that (I think!) recoil-order corrections have been implicitly excluded at some point here.  ...Is this even true??}
\bea
T_\beta &=& \left( p_\beta^2 c^2 + m_e^2 c^4 \right)^{1/2} - m_e c^2 \\
T_\nu &=& \left( p_\nu^2 c^2 + m_\nu^2 c^4 \right)^{1/2} - m_\nu c^2 \nonumber \\
&\approx& p_\nu c \\
T_r &=& \left( p_r^2 c^2 + m_r^2 c^4 \right)^{1/2} - m_r c^2 \nonumber\\
&\approx& \frac{p_r^2}{2 m_r}
\eea

%
%\section{Things.4}
%From jtw, the unpolarized P.D.F. is like this:
%\bea
%\omega d(\textrm{stuff}) &=& \left(\frac{1}{2\pi}\right)^5 p_e E_e (E_0 - E_e)^2 d E_e d\Omega_e d\Omega_\nu \xi 
%\left(1+ a\frac{\vec{p}_e \cdot \vec{p}_\nu }{E_e E_\nu} + b \frac{m}{E_e} \right)
%\eea
%But recall (recall!!) that the lab frame coordinates are implicit in at least one of the lepton coordinates.  Probably both.  Then, 
%\bea
%d\Omega_e     &=& \sin(\theta_e - \theta_L)d(\theta_e - \theta_L) d(\phi_e - \phi_L)     \\
%d\Omega_\nu &=& \sin(\theta_\nu - \theta_L)d(\theta_\nu - \theta_L) d(\phi_\nu - \phi_L),
%\eea
%and of course, 
%\bea
%\vec{p}_e + \vec{p}_\nu + \vec{p}_r = \vec{0},
%\eea
%so we can always switch everything from neutrino- into recoil coordinates.  If we were to do that, we'd find that
%\bea
%\vec{p}_e \cdot \vec{p}_\nu &=& \vec{p}_e \cdot (-\vec{p}_e - \vec{p}_r) \\ 
%&=& -|\vec{p}_e|^2 - \vec{p}_e\cdot \vec{p}_r.
%\eea
%Right about now, we really wish we knew how to convert everything between spherical and cartesian coordinates.  To that end, we'll write out the explicit conversion convention to be used:
%\bea
%p_x &=& p \sin\theta \cos\phi \\
%p_y &=& p \sin\theta \sin\phi \\
%p_z &=& p \cos\theta .
%\eea
%\bea
%p         &=& \left( p_x^2 + p_y^2 + p_z^2 \right)^{1/2} \\
%\theta &=& \arctan\left( \frac{ \left(p_x^2 + p_y^2 \right)^{1/2} }{p_z} \right) \\
%\phi    &=& \arctan\left( \frac{p_y}{p_x} \right)
%\eea
%
%Putting this trivial bullshit stuff together, we find:
%\bea
%\omega d(\textrm{stuff})|_{\nu} &=& \left(\frac{1}{2\pi}\right)^5 \xi p_e E_e (E_0 - E_e)^2 d E_e \nonumber\\
%&& \times \sin(\theta_e - \theta_L) d(\theta_e - \theta_L) d(\phi_e - \phi_L)  \phantom{^2} 
%\sin(\theta_\nu - \theta_L) d(\theta_\nu - \theta_L) d(\phi_\nu - \phi_L) \nonumber\\
%&& \times \left[1+ b \frac{m}{E_e} + a\frac{1}{E_e E_\nu}\left(|\vec{p}_e||\vec{p}_\nu | \right) \right.  \nonumber\\
%&& \times \left( \phantom{\frac{1}{1}\!\!\!\!\! }  \sin(\theta_e - \theta_L)\sin(\theta_\nu-\theta_L)
%\left( \cos(\phi_e-\phi_L)\cos(\phi_\nu-\phi_L) + \sin(\phi_e-\phi_L)\sin(\phi_\nu-\phi_L) \phantom{1^2\!\!\!\!\! } \right) \right. \nonumber\\
%&& + \left.\left. \cos(\theta_e-\theta_L)\cos(\theta_\nu- \theta_L) \phantom{\frac{1}{1}\!\!\!\!\!} \right) \right] .
%\label{neutrino_omega}
%\eea
%Or equivalently,
%\bea
%\omega d(\textrm{stuff})|_{r} &=& \left(\frac{1}{2\pi}\right)^5 \xi p_e E_e (E_0 - E_e)^2 d E_e \nonumber\\
%&& \times \sin(\theta_e - \theta_L) d(\theta_e - \theta_L) d(\phi_e - \phi_L)  \phantom{^2} 
%\sin(\theta_\nu - \theta_L) d(\theta_\nu - \theta_L) d(\phi_\nu - \phi_L) \nonumber\\
%&& \times \left[1+ b \frac{m}{E_e} + a\frac{1}{E_e E_\nu}\left( -|\vec{p}_e|^2 - |\vec{p}_e||\vec{p}_r| \right.\right. 
%\nonumber\\
%&& \times \left( \phantom{\frac{1}{1}\!\!\!\!\! }  \sin(\theta_e - \theta_L)\sin(\theta_r-\theta_L)
%( \cos(\phi_e-\phi_L)\cos(\phi_r-\phi_L) + \sin(\phi_e-\phi_L)\sin(\phi_r-\phi_L) ) \right. 
%\nonumber\\
%&& + \left.\left. \cos(\theta_e-\theta_L)\cos(\theta_r - \theta_L) \phantom{\frac{1}{1}\!\!\!\!\!} \right) \left.\phantom{1^2 \!\!\!\! \!
%\! }\right)  \right].
%\label{recoil_omega}
%\eea
%
%And I now would very much like to integrate over $d\Omega_\nu$. \textit{Very much.}  SO MUCH.
%So I'm going to play a fucktarded integration trick, in which I hold the neutrino direction fixed, and then integrate over the lab frame.  The lab frame stuff will be automagically eliminated from the integral (and infinitesimals) over beta-direction, and everything will be all happy.  So...
%
%\todo[inline]{Start of nu-basis integration stuff.}
%%%
%\bea
%\int \int \omega d(\textrm{stuff}) &:=& \omega^\prime d(\textrm{stuff}) \nonumber\\
%&=& \left(\frac{1}{2\pi}\right)^5 \xi p_e E_e (E_0 - E_e)^2 d E_e \, d\theta_e d\phi_e \nonumber\\
%&& \times \int_0^\pi d \theta_L \int_0^{2\pi} d \phi_L \,\,
%\sin(\theta_e - \theta_L) \sin(\theta_\nu - \theta_L)   \phantom{^2} 
%%\nonumber\\
%%&& \times 
%\left[1+ b \frac{m}{E_e} + a\frac{|\vec{p}_e ||\vec{p}_\nu | }{E_e E_\nu} \right.  \nonumber\\
%&& \times \left( \phantom{\frac{1}{1}\!\!\!\!\! }  \sin(\theta_e - \theta_L)\sin(\theta_\nu-\theta_L)
%\left( \cos(\phi_e-\phi_L)\cos(\phi_\nu-\phi_L) + \sin(\phi_e-\phi_L)\sin(\phi_\nu-\phi_L) \phantom{1^2\!\!\!\!\! } \right) \right. \nonumber\\
%&& + \left.\left. \cos(\theta_e-\theta_L)\cos(\theta_\nu- \theta_L) \phantom{\frac{1}{1}\!\!\!\!\!} \right) \right] 
%\label{nu_integral}
%\eea
%
%Here, it may be useful (or not...) to recall (from a trig identity) that:
%\bea
%\cos \left(\phi_e - \phi_\nu \right) 
%&=& 
%\Big( \cos(\phi_e-\phi_L)\cos(\phi_\nu-\phi_L) + \sin(\phi_e-\phi_L)\sin(\phi_\nu-\phi_L) \Big) \\ 
%&=&
% \cos\phi_e\cos\phi_\nu + \sin\phi_e\sin\phi_\nu,
%\eea
%and further (from conservation of momentum), that:
%\bea
%\cos \theta_\nu &=&
%- \frac{1}{p_\nu} \big(  p_e \cos\theta_e + p_r \cos \theta_r  \big) 
%\label{costhetanu}
%\\
%%\eea
%%\bea
%\!\!\!\! \!\!\!\! \!\!\!\!  \!\!\!\! \!\!\!\! 
%\cos(\phi_\nu - \phi_L) 
%&=&
%\frac{-1}{p_\nu \sin(\theta_\nu - \theta_L) } 
%\Big( p_\beta \sin(\theta_\beta - \theta_L) \cos(\phi_\beta - \phi_L) + 
%p_r \sin(\theta_r - \theta_L) \cos(\phi_r - \phi_L)
%\Big)
%\\
%\!\!\!\! \!\!\!\! \!\!\!\!  \!\!\!\! \!\!\!\! 
%\sin(\phi_\nu - \phi_L) 
%&=&
%\frac{-1}{p_\nu \sin(\theta_\nu - \theta_L) } 
%\Big( p_\beta \sin(\theta_\beta - \theta_L) \sin(\phi_\beta - \phi_L) + 
%p_r \sin(\theta_r - \theta_L) \sin(\phi_r - \phi_L)
%\Big),
%\eea
%and if $Q$ is as defined in Eq.~(\ref{Q_def}), then we also have:
%\bea
%p_\nu \, c &\approx& \left(Q - \frac{p_r^2}{2M_r} - E_\beta + m_e c^2 \right),
%\label{pnu}
%\eea
%and as always, 
%\bea
%p_\beta \, c &=& \left( E_\beta^2 - m_e^2 c^4 \right)^{1/2} .
%\eea
%
%It's straightforward to extract $\sin \theta_\nu$ too, and the sign ambiguity is OK, because $\theta$ is only defined over $0$-$\pi$, where `$\sin(\cos^{-1}(...) )$' comes out correctly.  The goal, of course, is to get rid of all references to $\nu$ in Eq~(\ref{nu_integral}).
%
%At any rate, the term that looks like $\cos \left(\phi_e - \phi_\nu \right)$ becomes:
%\bea
%\!\!\!\! \!\!\!\! \!\!\!\!
%\cos \left(\phi_e - \phi_\nu \right) 
%&=&
% \Big( \cos(\phi_e-\phi_L)\cos(\phi_\nu-\phi_L) + \sin(\phi_e-\phi_L)\sin(\phi_\nu-\phi_L) \Big) 
%\\ 
%&=& 
%\lbracket{26} \!\!
%\cos(\phi_e-\phi_L) \!
%	\left(
%		\frac{-1}{p_\nu \sin(\theta_\nu - \theta_L) } 
%	\right) \!
%	\Big( 
%		p_e \sin(\theta_e - \theta_L) \cos(\phi_e - \phi_L) + 
%		p_r \sin(\theta_r - \theta_L) \cos(\phi_r - \phi_L) 
%	\Big)
%	\nonumber \\
%	&& + \:
%	\sin(\phi_e-\phi_L) \!
%	\left(
%		\frac{-1}{p_\nu \sin(\theta_\nu - \theta_L) } 
%	\right) \!
%	\Big( 
%		p_e \sin(\theta_e - \theta_L) \sin(\phi_e - \phi_L) + 
%		p_r \sin(\theta_r - \theta_L) \sin(\phi_r - \phi_L)
%	\Big)	
%\!\! \rbracket{26}
%\nonumber \\
%\eea
%%\\
%%% 
%\bea
%\cos \left(\phi_e - \phi_\nu \right) 
%&=&
%\left(
%	\frac{-1}{p_\nu \sin(\theta_\nu - \theta_L) } 
%\right) \!\!
%\nonumber\\
%&& \times
%\lbracket{26} \!\!
%	\Big( 
%		p_e \sin(\theta_e - \theta_L) \cos^2(\phi_e - \phi_L) % \cos(\phi_e-\phi_L) 
%		+ p_r \sin(\theta_r - \theta_L) \cos(\phi_r - \phi_L) \cos(\phi_e-\phi_L)
%	\Big)
%	\nonumber \\
%	&& + \:
%	\Big( 
%		p_e \sin(\theta_e - \theta_L) \sin^2(\phi_e - \phi_L)%\sin(\phi_e-\phi_L) 
%		+ p_r \sin(\theta_r - \theta_L) \sin(\phi_r - \phi_L)\sin(\phi_e-\phi_L)
%	\Big)	
%\!\! \rbracket{26}
%\nonumber \\
%\\
%&=&
%\left(
%	\frac{-1}{p_\nu \sin(\theta_\nu - \theta_L) } 
%\right) \!\!
%\lbracket{26} \!\!
%	p_e \sin(\theta_e - \theta_L) 
%	\left(
%		\cos^2(\phi_e - \phi_L) 
%		+ \sin^2(\phi_e - \phi_L)
%	\right)
%	\nonumber\\
%	&&
%	+ \:
%	p_r \sin(\theta_r - \theta_L) 
%	\big(
%		\cos(\phi_r - \phi_L) \cos(\phi_e-\phi_L)
%		+ 
%		 \sin(\phi_r - \phi_L)\sin(\phi_e-\phi_L)
%	\big)
%\!\! \rbracket{26}
%\nonumber \\
%\\
%&=&
%\left(
%	\frac{-1}{p_\nu \sin(\theta_\nu - \theta_L) } 
%\right) \!\!
%\lbracket{26} \!\!
%	p_e \sin(\theta_e - \theta_L) 
%	+ 
%	p_r \sin(\theta_r - \theta_L) 
%		\cos(\phi_r - \phi_e)
%\!\! \rbracket{26}
%\eea
%...and that's really not so bad at all.  
%
%Back to the master equation (\ref{nu_integral}) now.  It turns out, 
%\bea
%%\!\!\!\! \!\!\!\! \!\!\!\! 
%\omega^\prime d(\textrm{stuff}) 
%&=& 
%\left(\frac{1}{2\pi}\right)^5 \xi p_e E_e (E_0 - E_e)^2 d E_e \, d\theta_e d\phi_e 
%\nonumber\\
%&& \times \int_0^\pi d \theta_L \int_0^{2\pi} d \phi_L \,\,
%\sin(\theta_e - \theta_L) \sin(\theta_\nu - \theta_L)  
%\!
%\lbracket{22} \!
%    1+ b \frac{m}{E_e} 
%    \nonumber\\
%    &&
%    + a\frac{|\vec{p}_e ||\vec{p}_\nu | }{E_e E_\nu} 
%    \lparen{18} \!
%	    \cos(\theta_e-\theta_L)\cos(\theta_\nu- \theta_L) %\phantom{\frac{1}{1}\!\!\!\!\!} 
%	%    \nonumber \\
%	%    && +
%	    +
%	    \sin(\theta_e - \theta_L)\sin(\theta_\nu-\theta_L)
%	    \nonumber\\
%	    && \times
%	    %
%	    \left(
%		\frac{-1}{p_\nu \sin(\theta_\nu - \theta_L) } 
%	    \right) \!\!
%	    \lbracket{26} \!\!
%		p_e \sin(\theta_e - \theta_L) 
%		+ 
%		p_r \sin(\theta_r - \theta_L) 
%		\cos(\phi_r - \phi_e)
%	    \!\! \rbracket{26}
%	   %
%    \!
%    \rparen{18}
%\! 
%\rbracket{22} 
%\\ %%%    %%% 
%\omega^\prime d(\textrm{stuff}) 
%&=& 
%\left(\frac{1}{2\pi}\right)^5 \xi p_e E_e (E_0 - E_e)^2 d E_e \, d\theta_e d\phi_e 
%\nonumber\\
%&& \times \int_0^\pi d \theta_L \int_0^{2\pi} d \phi_L \,\,
%\sin(\theta_e - \theta_L) \sin(\theta_\nu - \theta_L)  
%\!
%\lbracket{22} \!
%    1+ b \frac{m}{E_e} 
%    \nonumber\\
%    &&
%    + \:
%    a\frac{p_e \, p_\nu}{E_e E_\nu} 
%    \lparen{18} \!
%	    \cos(\theta_e-\theta_L)\cos(\theta_\nu- \theta_L) 
%    \rparen{18}
%    \nonumber\\
%    &&
%    -
%    a\frac{ p_e }{E_e E_\nu} 
%    \lparen{18} \!
%	    \sin(\theta_e - \theta_L)
%	    \! \lbracket{16} \!\!
%		p_e \sin(\theta_e - \theta_L) 
%		+ 
%		p_r \sin(\theta_r - \theta_L) 
%		\cos(\phi_r - \phi_e)
%	    \!\! \rbracket{16}
%    \!
%    \rparen{18}
%\! 
%\rbracket{22} 
%\eea
%
%...Aaaaaand, we're almost done.  I think.  Let's do the $\theta_L$ integral now.
%\bea
%\omega^\prime d(\textrm{stuff}) 
%%&=& 
%%\left(\frac{1}{2\pi}\right)^5 \xi p_e E_e (E_0 - E_e)^2 d E_e \, d\theta_e d\phi_e 
%%\nonumber\\
%%&& \times 
%%\int_0^{2\pi} d \phi_L \,\,
%%\lbracket{22} \!
%%\frac{\pi}{2} \cos\left(\theta_e - \theta_\nu \right)
%%    \left( 1+ b \frac{m}{E_e} \right)
%%    + a\frac{p_e \, p_\nu}{E_e E_\nu} 
%%    \lparen{18} \!
%%	    \frac{\pi}{8} 
%%	    \cos \Big( 
%%	    	2(\theta_e - \theta_\nu) 
%%	    \Big) \!\!
%%    \rparen{18}
%%    \nonumber\\
%%    &&
%%    -
%%    a\frac{ p_e }{E_e E_\nu} 
%%    \int_0^\pi d \theta_L 
%%    \lparen{18} \!
%%	    p_e
%%	    \sin^3(\theta_e - \theta_L) \sin(\theta_\nu - \theta_L)
%%	    + 
%%	    p_r  \sin^2(\theta_e - \theta_L) \sin(\theta_r - \theta_L) \cos(\phi_r - \phi_e)
%%    \!
%%    \rparen{18}
%%\! 
%%\rbracket{22} 
%%\nonumber\\
%%\\
%%%%  %%%
%&=& 
%\left(\frac{1}{2\pi}\right)^5 \xi p_e E_e (E_0 - E_e)^2 d E_e \, d\theta_e d\phi_e 
%\nonumber\\
%&& \times 
%\int_0^{2\pi} d \phi_L \,\,
%\lbracket{22} \!
%\frac{\pi}{2} \cos\left(\theta_e - \theta_\nu \right)
%    \left( 1+ b \frac{m}{E_e} \right)
%    + a\frac{p_e \, p_\nu}{E_e E_\nu} 
%    \lparen{18} \!
%	    \frac{\pi}{8} 
%	    \cos \Big( 
%	    	2(\theta_e - \theta_\nu) 
%	    \Big) \!\!
%    \rparen{18}
%    \nonumber\\
%    &&
%    -
%    a\frac{ p_e }{E_e E_\nu} 
%    \lparen{18} \!
%	    \frac{3\pi}{8} p_e \cos(\theta_e - \theta_\nu)
%	    + \frac{\pi}{8}
%	    p_r \cos(\phi_r - \phi_e)
%	    \Big( \!
%	    	\cos \big(
%			2\theta_e - \theta_\nu - \theta_r
%		\big)
%		+ 
%		2 \cos \big(
%			\theta_\nu - \theta_r
%		\big)
%	    \!
%	    \Big)
%    \!
%    \rparen{18}
%\! 
%\rbracket{22} 
%\nonumber\\
%\eea
%That's not really good enough though.  We have to get rid of all those pesky $\theta_\nu$'s, $p_\nu$'s, and $E_\nu$'s.  Do we do it before or after the next integral?  In principle, it doesn't matter.  In practice, one way is probably messier than the other.  
%
%Whatever, let's simplify some shit.
%\bea
%\omega^\prime d(\textrm{stuff}) 
%&=& 
%\left(\frac{1}{2\pi}\right)^4 \frac{\pi}{8} \:
%\xi p_e E_e 
%%(E_0 - E_e)^2 
%E_\nu^2 \,
%d E_e \, d\theta_e d\phi_e 
%\nonumber\\
%&& \times 
%\lbracket{22} \!
%4
%\cos\left(\theta_e - \theta_\nu \right)
%    \left( 1+ b \frac{m}{E_e} \right)
%    + a \frac{p_e }{E_e} 
%    \cos \Big( 
%    	2 \big(\theta_e - \theta_\nu \big) \!
%    \Big)
%    -
%    3 \,
%    a\frac{ p_e^2 }{E_e E_\nu} 
%	    \cos \big(\theta_e - \theta_\nu \big)
%    \nonumber\\
%    &&
%    - \:
%    a\frac{ p_e \, p_r}{E_e E_\nu} 
%    \Big( \!
%	    \cos \big(
%		2\theta_e - \theta_\nu - \theta_r
%	\big)
%	+ 
%	2 \cos \big(
%		\theta_\nu - \theta_r
%	\big) \!
%    \Big)
%    \cos \big(\phi_r - \phi_e \big)
%\! 
%\rbracket{22} 
%% \nonumber\\
%\eea
%
%So we're down to just $E_\nu$'s and $\theta_\nu$'s.  It's OK though -- there are expressions for those!  To reiterate Eqs.~(\ref{pnu}, \ref{costhetanu}), 
%\bea
%p_\nu \, c &\approx& \left(Q - \frac{p_r^2}{2M_r} - E_\beta + m_e c^2 \right)
%\\
%%\eea
%%\bea
%\cos \theta_\nu &=&
%- \frac{1}{p_\nu} \big(  p_e \cos\theta_e + p_r \cos \theta_r  \big) .
%\eea
%%%
%%%
%...
%
%
%\todo[inline]{End of nu-basis integration stuff.}
%%%%%%    %%%%%    %%%%%    %%%%%    %%%%%
%
%But actually, we've still got to integrate over beta direction before this'll do anything useful for us.  I think.  Therefore, define
%\bea
%\omega^{\prime \prime} d(\textrm{...}) 
%&:=& 
%\int_0^{2\pi} \!\! d\phi_e 
%\int_0^\pi \!\! d \theta_e 
%\:
%\frac{  \omega^{\prime}  d(\textrm{stuff}) }{ d\phi_e d \theta_e},
%\eea
%which is totally legit, since we've already converted $d \Omega_e $ into things with a factor of $\sin \theta_e$.  On that note, 
%%%
%\bea
%\omega^{\prime \prime} d(\textrm{...}) 
%&=& 
%\left(\frac{1}{2\pi}\right)^4 \frac{\pi}{8} \: \xi 
%\int_0^{2\pi} \!\! d\phi_e 
%\int_0^\pi \!\! d \theta_e 
%\lparen{24}
%p_e E_e 
%%(E_0 - E_e)^2 
%E_\nu^2 \,
%d E_e \, 
%%d\theta_e d\phi_e 
%\nonumber\\
%&& \times 
%\lbracket{22} \!
%4
%\cos\left(\theta_e - \theta_\nu \right)
%    \left( 1+ b \frac{m}{E_e} \right)
%    + a \frac{p_e }{E_e} 
%    \cos \Big( 
%    	2 \big(\theta_e - \theta_\nu \big) \!
%    \Big)
%    -
%    3 \,
%    a\frac{ p_e^2 }{E_e E_\nu} 
%	    \cos \big(\theta_e - \theta_\nu \big)
%    \nonumber\\
%    &&
%    - \:
%    a\frac{ p_e \, p_r}{E_e E_\nu} 
%    \Big( \!
%	    \cos \big(
%		2\theta_e - \theta_\nu - \theta_r
%	\big)
%	+ 
%	2 \cos \big(
%		\theta_\nu - \theta_r
%	\big) \!
%    \Big)
%    \cos \big(\phi_r - \phi_e \big)
%\! 
%\rbracket{22} 
%\rparen{24},
%\eea
%and in a step that I'm thoroughly suspicious of, the last term that looks like $\cos(\phi_r - \phi_e)$ integrates to zero.  Hey, it \emph{might} be true.  Anyway, that means:
%
%\bea
%\omega^{\prime \prime} d(\textrm{...}) 
%&=& 
%\left(\frac{1}{2\pi}\right)^3 \frac{\pi}{8} \: \xi 
%\int_0^\pi \!\! d \theta_e 
%\lparen{24}
%p_e E_e 
%E_\nu^2 \,
%d E_e \, 
%\nonumber\\
%&&\times
%\lbracket{22} \!
%\cos\left(\theta_e - \theta_\nu \right)
%    \lbracket{16}
%    4 \left( 1+ b \frac{m}{E_e} \right)
%    -
%    3 \,
%    a\frac{ p_e^2 }{E_e E_\nu} 
%    \rbracket{16}
%    \nonumber\\
%    && 
%    + a \frac{p_e }{E_e} 
%    \cos \Big( 
%    	2 \big(\theta_e - \theta_\nu \big) \!
%    \Big)
%\rbracket{22} 
%\rparen{24},
%\eea
%and sadly, I'm doomed to put in the $\theta_\nu$'s (and $p_\nu$'s ???) in before doing the $\theta_e$ integral, because those things are functions of $\theta_e$. Sooo...
%%%    %%    %% 
%\def \thetanu {\theta_\nu }
%%\def \thetanu { \cos^{-1} \Big( \frac{-1}{p_\nu} \big(  p_e \cos\theta_e + p_r \cos \theta_r  \big) \Big) }
%%\def \costhetanu {\frac{-1}{p_\nu} \big(  p_e \cos\theta_e + p_r \cos \theta_r  \big)}
%\def \costemtn {\cos \theta_e \cos\thetanu + \sin\theta_e \sin\theta_\nu}
%\def \sintemtn {\sin \theta_e \cos\thetanu - \cos\theta_e \sin\theta_\nu}
%%%    %%    %% 
%\bea
%\omega^{\prime \prime} d(\textrm{...}) 
%&=& 
%\left(\frac{1}{2\pi}\right)^3 \frac{\pi}{8} \: \xi 
%\int_0^\pi \!\! d \theta_e 
%\lparen{24}
%p_e E_e 
%E_\nu^2 \,
%d E_e \, 
%\nonumber\\
%&&\times
%\lbracket{22} \!
%	\left( \costemtn \right)
%    \lbracket{16}
%    4 \left( 1+ b \frac{m}{E_e} \right)
%    -
%    3 \,
%    a\frac{ p_e^2 }{E_e E_\nu} 
%    \rbracket{16}
%    \nonumber\\
%    && 
%    + a \frac{p_e }{E_e} 
%        \Bigg(
%    \left( \costemtn \right)^2
%    - 
%    \left( \sintemtn \right)^2
%    \Bigg)
%\rbracket{22} 
%\rparen{24},
%\eea
%and therefore, 
%%
%%%    %%    %% 
%\def \thetanu {\theta_\nu }
%\def \costhetanu {\frac{-1}{p_\nu} \big(  p_e \cos\theta_e + p_r \cos \theta_r  \big)}
%\def \sinthetanu {\sin \! \left[ \cos^{-1}\! \left(\frac{-1}{p_\nu} \big(  p_e \cos\theta_e + p_r \cos \theta_r  \big) \right) \right] }
%%\def \sinthetanu { \sqrt{1-\left(\frac{-1}{p_\nu} \big(  p_e \cos\theta_e + p_r \cos \theta_r  \big) \right)^2}}
%\def \costemtn {\cos \theta_e \left(\costhetanu \right) + \sin\theta_e \sinthetanu}
%\def \sintemtn {\sin \theta_e \left(\costhetanu \right) - \cos\theta_e \sinthetanu}
%%%    %%    %% 
%\bea
%\!\!\!\!\!\!
%\omega^{\prime \prime} d(\textrm{...}) 
%&=& 
%\left(\frac{1}{2\pi}\right)^3 \frac{\pi}{8} \: \xi 
%\int_0^\pi \!\! d \theta_e 
%\lparen{22}
%	p_e E_e 
%	E_\nu^2 \,
%	d E_e \, 
%	\lbracket{28} \!
%		\Big( 
%			\cos \theta_e \left(\costhetanu\right) 
%			\nonumber\\
%			&&
%			+ \sin\theta_e \sinthetanu
%		\Big)
%	    \lbracket{16}
%	    4 \left( 
%			1+ b \frac{m}{E_e} 
%		\right)
%	    -
%	    3 \, a\frac{ p_e^2 }{E_e E_\nu} 
%	    \rbracket{16}
%			\nonumber\\
%			&& 
%			+ a \frac{p_e }{E_e} 
%			\Bigg(
%				\left( 
%					\cos \theta_e \left(\costhetanu\right) + \sin\theta_e \sinthetanu
%				\right)^2
%				\nonumber\\
%				&&
%				- 
%				\left( 
%					\sintemtn 
%				\right)^2
%			\Bigg)
%		\rbracket{28} 
%	\rparen{22},
%	\nonumber\\
%\eea
%...
%
%I might try to simplify that and make it look pretty.  Perhaps this is horribly misguided.  I don't know.  
%%%
%\def \sinthetanu {\sin \! \left[ \cos^{-1}\!\! \left(  \frac{- p_e}{p_\nu} \cos\theta_e +\frac{ -p_r}{p_\nu} \cos \theta_r \right)\! \right] }
%%%
%\bea
%\omega^{\prime \prime} d(\textrm{...}) 
%&=& 
%\left(\frac{1}{2\pi}\right)^3 \frac{\pi}{8} \: \xi 
%\int_0^\pi \!\! d \theta_e 
%\lparen{22}
%	p_e E_e 
%	E_\nu^2 \,
%	d E_e 
%	\lbracket{28} \!
%	    \frac{-1}{p_\nu} 
%	    \lparen{13} \!
%		    4 \left( \!
%				1+ b \frac{m}{E_e} 
%			\! \right)
%		    - 3 \, a\frac{ p_e^2 }{E_e E_\nu} 
%	    \rparen{13}			
%		\cos \theta_e 
%		\Big(
%			p_e \cos\theta_e + p_r \cos \theta_r  
%		\Big) 
%		\nonumber\\
%		&&
%		+ 
%	    \lparen{13} \!
%		    4 \left( \!
%				1+ b \frac{m}{E_e} 
%			\! \right)
%		    - 3 \, a\frac{ p_e^2 }{E_e E_\nu} 
%	    \rparen{13}
%			\sin\theta_e 
%			\, 
%			\sinthetanu
%			\nonumber\\
%			&& 
%			+ a \frac{p_e }{E_e} 
%			\lparen{36} \!\!
%				\left( 
%					\sin\theta_e 
%					\, 
%					\sinthetanu
%					+
%					\frac{-1}{p_\nu} 
%					\cos \theta_e 
%					\Big(
%						p_e \cos\theta_e + p_r \cos \theta_r  
%					\Big) 
%				\right)^2
%				\nonumber\\
%				&&
%				+
%				\left( 
%					\frac{+1}{p_\nu} 
%					\sin \theta_e 
%					\Big(
%							p_e \cos\theta_e + p_r \cos \theta_r  
%					\Big) 
%					+ \cos\theta_e \,
%					\sinthetanu
%				\right)^2
%			\!\! \rparen{36}
%		\rbracket{28} 
%	\rparen{22},
%	\nonumber\\
%\eea
%
%Anyway.  We'll want to list off some nice integrals here.  See integration\_scratch.nb.
%%%% first, some definitions, for ease of latex-ing:
%%%
%\def \intsinthetaesinthetanu {\frac{p_\nu }{2p_e}
%\left( 
%	\sin^{-1}\!\left[\frac{p_e}{p_\nu}-\frac{p_r }{p_\nu}\cos\theta_r\right] 
%	+ 
%	\sin^{-1}\!\left[\frac{p_e}{p_\nu}+\frac{p_r }{p_\nu}\cos\theta_r\right]
%\right)
%\nonumber\\
%&&
%+
%\frac{1}{2p_e p_\nu}
%\left(
%	(p_e+ p_r\cos\theta_r) \sqrt{p_\nu^2- \left( p_e+p_r \cos\theta_r\right)^2}
%	+
%	(p_e-p_r\cos\theta_r) \sqrt{p_\nu^2- \left(p_e - p_r\cos\theta_r\right)^2}
%\right)
%}
%%%
%\def \sintesintenpcostecostensq {
%-\frac{1}{24 p_e^2} 
%\Bigg( \!
%	-\frac{6 p_e^4 \pi }{p_{\nu}^2}
%	-12 p_e^2 \pi 
%	+ 6p_{\nu}^2 \sin^{-1}\! \left[\frac{p_e}{p_{\nu}}-\frac{p_r \cos\theta_r}{p_{\nu}}\right]
%	+ 6 p_{\nu}^2\sin^{-1}\!\left[\frac{p_e}{p_{\nu}}+\frac{p_r \cos\theta_r}{p_{\nu}}\right]
%	\!
%\Bigg)
%	\nonumber\\
%&&	
%-\frac{1}{24 p_e^2}
%\bigg(
%	\frac{1}{p_{\nu}^2} 
%\bigg) \!
%\Bigg( \!\!
%	\left(
%		\sqrt{
%			p_{\nu}^2-(p_e + p_r \cos\theta_r)^2
%		} 
%		+ 
%		\sqrt{
%			p_{\nu}^2-(p_e - p_r \cos\theta_r)^2
%		} \,
%	\right)
%	\left(
%		%\frac{2 p_e}{1}
%		2 p_e
%	\right)
%	\left(
%		6 p_e^2 + 2p_r^2 \cos^2\theta_r - 3p_{\nu}^2 
%	\right) 
%	\nonumber\\
%&&	+ 
%	\left( 
%		\sqrt{ 
%			p_{\nu}^2-(p_e + p_r \cos\theta_r)^2
%		}
%		- 
%		\sqrt{
%			p_{\nu}^2-(p_e - p_r \cos\theta_r)^2
%		} \,
%	\right)
%	\left(
%		%\frac{2p_r}{1} 
%		2p_r 
%		\cos\theta_r
%	\right)
%	\left(
%		10p_e^2 -2p_r^2\cos^2\theta_r + 5p_{\nu}^2
%	\right)
%	\!\!
%\Bigg)
%}
%%
%\def \outfour {
%\frac{1}{24 \text{pe}^2}\left(-\frac{6 \text{pe}^4 \pi}{\text{pn}^2}+12 \text{pe}^2 \pi +6\text{pn}^2 \text{ArcSin}\left[\frac{\text{pe}}{\text{pn}}-\frac{\text{pr} \text{Cos}[\text{tr}]}{\text{pn}}\right]+6 \text{pn}^2\text{ArcSin}\left[\frac{\text{pe}}{\text{pn}}+\frac{\text{pr} \text{Cos}[\text{tr}]}{\text{pn}}\right]\right)
%\nonumber\\
%&&
%+\frac{1}{24 \text{pe}^2}\frac{1}{\text{pn}^2}
%\Bigg(
%\!\!
%\left(\sqrt{\text{pn}^2 -(\text{pe}+\text{pr} \text{Cos}[\text{tr}])^2} +\sqrt{\text{pn}^2 -(\text{pe}-\text{pr} \text{Cos}[\text{tr}])^2}\right)(2\text{pe})\left(6 \text{pe}^2-3\text{pn}^2+2\text{pr}^2 \text{Cos}[\text{tr}]^2\right)
%\nonumber\\
%&&
%+
%\left(
%\sqrt{\text{pn}^2 -(\text{pe}+\text{pr} \text{Cos}[\text{tr}])^2} -\sqrt{\text{pn}^2 -(\text{pe}-\text{pr} \text{Cos}[\text{tr}])^2}\right)(2\text{pr} \text{Cos}[\text{tr}]) \left(10 \text{pe}^2 +5 \text{pn}^2-2\text{pr}^2 \text{Cos}[\text{tr}]^2
%\right)
%\!\!
%\Bigg)
%}
%%% End of latex-y definitions.  That was fun.
%\bea
%\int_0^\pi \!\! d \theta_e \,
%\cos\theta_e \Big( p_e \cos\theta_e + p_r \cos \theta_r \Big) 
%&=&
%\frac{\pi}{2} p_e 
%\eea
%%
%\bea
%&&
%\int_0^\pi \!\! d \theta_e \,
%\sin\theta_e 
%\sin \!
%\left( \!
%	\cos^{-1} \!
%	\Big( 
%		 \frac{- p_e}{p_\nu} \cos\theta_e +\frac{ -p_r}{p_\nu} \cos \theta_r 
%	\Big) \!
%\right)
%\nonumber\\
%&&
%=
%\intsinthetaesinthetanu
%%\frac{p_\nu }{2p_e}\left( \sin^{-1}\!\left[\frac{p_e}{p_\nu}-\frac{p_r }{p_\nu}\cos\theta_r\right] 
%%+ \sin^{-1}\!\left[\frac{p_e}{p_\nu}+\frac{p_r }{p_\nu}\cos\theta_r\right]\right)
%%\nonumber\\
%%+\frac{1 }{2p_e p_\nu}
%%\left((p_e+ p_r\cos\theta_r) \sqrt{p_\nu^2- \left( p_e+p_r \cos\theta_r\right)^2}
%%+
%%(p_e-p_r\cos\theta_r) \sqrt{p_\nu^2- \left(p_e - p_r\cos\theta_r\right)^2}\right)
%\nonumber\\
%\eea
%%
%\bea
%&&\int_0^\pi \!\! d \theta_e \,
%%\sin\theta_e 
%\lparen{26}
%	\!
%	\sin\theta_e 
%	\sin \!
%	\left( \!
%		\cos^{-1} \!
%		\Big( 
%			 \frac{- p_e}{p_\nu} \cos\theta_e +\frac{ -p_r}{p_\nu} \cos \theta_r 
%		\Big) \!
%	\right)
%	+ 
%	\frac{-1}{p_\nu}
%	\cos\theta_e 
%	\Big( 
%		p_e \cos\theta_e + p_r \cos \theta_r  
%	\Big)
%	\!
%\rparen{26}^2
%\nonumber\\
%&&=
%\sintesintenpcostecostensq
%%
%\nonumber\\
%\eea
%\bea
%&&
%\int_0^\pi \!\! d \theta_e \,
%\bigg( 
%	\frac{+1}{p_\nu} 
%	\sin \theta_e 
%	\Big(
%			p_e \cos\theta_e + p_r \cos \theta_r  
%	\Big) 
%	+ \cos\theta_e \,
%	\sinthetanu
%\bigg)^2
%\nonumber\\
%&&
%=
%\outfour
%\nonumber\\
%\eea
%%% End of math-y definitions to be used for future reference.
%...Or whatever.  The point is, those last two equations add up to $\pi$.  Exciting!!!
%
%
%We can ``simplify'' that 
%%shitty 
%integral thing quite a bit now.
%
%%%
%\bea
%\omega^{\prime \prime} d(\textrm{...}) 
%&=& 
%\left(\frac{1}{2\pi}\right)^3 \frac{\pi}{8} \: \xi 
%	p_e E_e 
%	E_\nu^2 \,
%	d E_e 
%\lbracket{30}\!\!
%	\left(
%		\frac{\pi}{2} p_e
%	\right)
%	\left( \frac{-1}{p_\nu} \right)
%	\lparen{13} \!
%		4 \left( \!
%			1+ b \frac{m}{E_e} 
%		\! \right)
%	    - 3 \, a\frac{ p_e^2 }{E_e E_\nu} 
%	\rparen{13}
%	+ 
%	a \frac{p_e }{E_e} 
%	\left( \pi \right)			
%\nonumber\\
%&&
%+ 
%\lparen{13} 
%	\! 4 
%    \left( \!
%		1+ b \frac{m}{E_e} 
%	\! \right)
%    - 3 \, a\frac{ p_e^2 }{E_e E_\nu} 
%\rparen{13}
%\lbracket{20}
%	\intsinthetaesinthetanu
%\rbracket{20}
%\rbracket{30},
%\nonumber\\
%\eea
%%%    %%
%\bea
%\omega^{\prime \prime} d(\textrm{...}) 
%&=& 
%\left(\frac{1}{2\pi}\right)^3 \frac{\pi}{8} \: \xi 
%p_e E_e 
%E_\nu^2 \,
%d E_e 
%\lbracket{30}\!\!
%%	\left(
%%		\frac{ - \pi \,p_e }{2 p_\nu}
%%	\right)
%%	\lparen{13} \!
%%		4 \left( \!
%%			1+ b \frac{m}{E_e} 
%%		\! \right)
%%	    - 3 \, a\frac{ p_e^2 }{E_e E_\nu} 
%%	\rparen{13}
%	+ 
%	a \frac{p_e }{E_e} 
%	\left( \pi \right)			
%\nonumber\\
%&&
%+ 
%\lparen{13} 
%	\! 4 
%    \left( \!
%		1+ b \frac{m}{E_e} 
%	\! \right)
%    - 3 \, a\frac{ p_e^2 }{E_e E_\nu} 
%\rparen{13}
%\lbracket{20}
%	\left(
%		\frac{ - \pi \,p_e }{2 p_\nu}
%	\right)
%	+
%	\intsinthetaesinthetanu
%\rbracket{20}
%\rbracket{30},
%\nonumber\\
%\eea
%
%
%




