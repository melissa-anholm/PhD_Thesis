% !TEX root = ../thesis_main.tex
%
%
%
%%%% --- * --- %%%%	
\chapter[Derivation of the Beta Decay Probability Density Function]{Derivation of the Probability Density Function}
\label{appendix_forthepeople}
%\note{More coulomb corrections in JTW notation:~\cite{EbelFeldman1957}.  Also, an update to Falkowski's review article\cite{Falkowski2021}, which currently still liives on the arxiv, is here:~\cite{Falkowski2022}. }
%\note{Start with Lee-Yang Hamiltonian, or at least point to it in the first chapter (it's Eq.~\ref{eq:lee_yang_hamiltonian}), because that's where all the BSM stuff lives.}
%\note{Define all the letters.}
%\note[color=done]{John says to keep this appendix, because it's great now.}
%\note[color=jb]{JB:  Appendix B is in good shape. "someone other than me" please change to "the collaboration."}
In order to obtain a \acf{PDF} to describe the beta decay process which includes both a representation of the exotic physics that is the subject of this search, as well as the small higher-order corrections which are known to be present, it is necessary to combine features from two disparate formalisms.  The \acf{JTW} description incorporates many parameters representing various flavours of exotic physics, but incorporates only the leading order terms in the resulting expression.  By contrast, the Holstein formulation makes no reference to exotic physics, but does a much more thorough job of describing the beta decay physics that we collectively do expect to see, such as higher-order terms and small corrections to the form of the spectra.

% has included many smaller corrections 

Because the \ac{JTW} formulation has neglected these higher-order terms, this model is much less messy to work with than Holstein's -- an assertion that may come as a surprise to the reader given the content within Sec.~\ref{sec:jtw_formalism}.
% -- therefore \ac{JTW}'s work will be considered first.  
We begin by recalling the Fermi contact interaction description of beta decay from Eq.~(\ref{eq:fermitransition}).  Recalling the definition of an interaction Hamiltonian, we can write:
\bea
\mathcal{M}_{fi} \;\;=\;\; G_F \int \bar{\psi}_f \, \mathcal{\hat{O}} \, \psi_i \,\textrm{d}V .
\;\;=\;\;
\int \! \mathcal{H}_{\mathrm{int}} \, \textrm{d}V
\label{eq:transitionmatrixhamiltonian}
\eea 

This approximation is adequate for the purpose of characterizing \ac{BSM} interactions because any exotic physics that we might search for has already been constrained, through decades of clever experiments, to be a comparatively small part of the overall behaviour.  These small adjustments to already small terms may be safely neglected.  This model leads directly to the \ac{JTW} result described in Sec.~\ref{sec:jtw_formalism}, however the Holstein results of Sec.~\ref{sec:holstein_formalism} arise from a more modern description of the beta decay process.  

%The neglected small adjustments would have been made to already small terms.  


We will use the Lee-Hang interaction Hamiltonian here, which incorporates a linear combination of all possible operators that obey Lorentz invariance at the nucleon level.  In particular\cite{LeeYang}:
% !TEX root = ../thesis_main.tex
%
%
%%%% --- * --- %%%%	
\bea
{\mathcal H} &=& (\bar{\psi}_p \psi_n)( C_S \,\bar{\psi}_e \psi_\nu + C_S^\prime \, \bar{\psi}_e \gamma_5 \psi_\nu )
%
\nonumber \\ &&
+ \: (\bar{\psi}_p \gamma_\mu \psi_n)( C_V \,\bar{\psi}_e \gamma_\mu \psi_\nu + C_V^\prime \, \bar{\psi}_e \gamma_\mu \gamma_5 \psi_\nu )
%
\nonumber \\ &&
+ \: \frac{1}{2} (\bar{\psi}_p \sigma_{\lambda \mu} \psi_n)( C_T \,\bar{\psi}_e \sigma_{\lambda \mu} \psi_\nu + C_T^\prime \, \bar{\psi}_e \sigma_{\lambda \mu} \gamma_5 \psi_\nu ) 
%
\nonumber \\ &&
+ \: (\bar{\psi}_p \gamma_\mu \gamma_5 \psi_n)( C_A \,\bar{\psi}_e \gamma_\mu \gamma_5 \psi_\nu + C_A^\prime \, \bar{\psi}_e \gamma_\mu \psi_\nu )
%
\nonumber \\ &&
+ \: (\bar{\psi}_p \gamma_5 \psi_n)( C_P \,\bar{\psi}_e \gamma_5 \psi_\nu + C_P^\prime \, \bar{\psi}_e \psi_\nu ) 
%
+ \textrm{H.C.},
\label{eq:lee_yang_hamiltonian} 
\eea
%%%\unskip
where $C_X$ and $C_X^{\prime}$ (with $X=\{V,A,S,T,P\}$) are complex coupling constants for vector, axial, scalar, tensor, and pseudoscalar interactions, and $\psi_Y$ (with $Y=\{p,n,e,\nu\}$) are the wavefunctions for the interaction's proton, neutron, electron, and neutrino.  Operators $\gamma_5$ and $\gamma_\mu$ are Dirac gamma matrices, and $\mbox{$\sigma_{\lambda\mu} = -\frac{i}{2}(\gamma_\lambda \gamma_\mu - \gamma_\mu\gamma_\lambda )$}$.  As usual, ``H.C.'' represents the Hermitian conjugate of the previous terms within the Hamiltonian.

With these expressions in place, it is possible to obtain a complete solution for the differential decay rate of Eq.~\ref{eq:fermidifferentialdecayrate} in terms of physical observables.

\note{Fermi formalism really only *has* the leading order things.  we've already gotten rid of some.}
\note{Blurb continuation:  JTW is for the BSM shit, because they created it before they really knew which form the operators take.~\cite{jtw}\cite{jtw_coulomb}  Holstein is for the small contributions to physics that we actually know to exist.~\cite{holstein}\cite{holstein_errata}}

\section[JTW Formalism]{JTW Formalism}
\label{sec:jtw_formalism}
Rather than going through the full calculation to find the differential decay rate, I will instead simply give the result, taken from \ac{JTW} and others\cite{LeeYang}\cite{jtw}\cite{jtw_coulomb}\cite{EbelFeldman1957}.  To leading order, we find a five-dimensional \ac{PDF} for beta decay kinematics:  
%\cite{LeeYang}\cite{jtw}\cite{jtw_coulomb}\cite{EbelFeldman1957}:
%
%
%
%%\note{They start with the Lee-Yang interaction Hamiltonian in Eq.~(\ref{eq:lee_yang_hamiltonian}), with the pseudoscalar terms dropped for convenience, as they're suppressed anyway.}
%
%%We begin a statement of the five-dimensional \ac{PDF} for beta decay kinematics, taken from \acf{JTW}, which arises directly from the Lee-Yang interaction Hamiltonian of Eq.~(\ref{eq:lee_yang_hamiltonian})~\cite{LeeYang}\cite{jtw}\cite{jtw_coulomb}\cite{EbelFeldman1957}:
%
%%, taken from \acf{JTW} for beta decay kinematics~\cite{LeeYang}\cite{jtw}\cite{jtw_coulomb}:
%%Here's a master equation from JTW to describe beta decay kinematics~\cite{jtw}\cite{jtw_coulomb}:
%%%%\note{Rephrase some of this stuff...}
%%\note[done]{Item 12. from Georg:
%%\\
%%I have no idea what the title of Appendix B means. It is for sure original and even funny though ... Also, the actual appendix title in the thesis is not the same as that listed in the TOC.
%%}
%%\label{note:gs_12}
% !TEX root = ../thesis_main.tex



% "A PDF for the People"
\bea
\omega(\cdots) \!\!\!\! \!\!\!\! \!\!\!\! \!\!\!\! && \,\,\,\, \,\,\,\, \mathrm{d} \E \, \dOmegae \, \dOmeganu 
\,\, = \,\, \frac{\FF}{(2\pi)^5} \, \pe \Ee (E_0 - \Ee)^2 \dEe \, \dOmegae \, \dOmeganu \, \nonumber\\ 
&&	\times \,\, \xi \left[
	1 + \a \frac{\vecpe\cdot\vecpnu}{\Ee\Enu} + \bFierz \frac{\m c^2}{\Ee} 
%	&& 
    + \,\,  \calign \,\, \Talign(\vecJ) 
	\left(
		\frac{\vecpe \cdot \vecpnu}{3\Ee\Enu}
		- \frac{ (\vecpe\cdot \hatj) (\vecpnu\cdot\hatj) }{\Ee\Enu}
	\right)
	\!
	%\left(
	%	\TalignExpand
	%\right)
\right. \nonumber\\ 
&&	\left. + 
	 \frac{\vecJ}{J} \cdot
	\left(
		\A \frac{\vecpe}{\Ee} 
		+ \B \frac{\vecpnu}{\Enu} 
		+ \D \frac{\vecpe \times \vecpnu}{\Ee\Enu} 
	\right)
\right]
\label{equation:jtw_master}
\eea

where, for convenience, we have defined a nuclear alignment term,
\bea
\Talign(\vecJ) &\equiv& \TalignExpand.
\label{eq:Talign}
\eea
\note{Missing factors of c ??}
%%%\aside{We have already specialized to $\beta^+$ decay.  .... Wait, we have?  Where?  ... There.  Fixed. It's not like that anymore.}

In Eqs.~(\ref{equation:jtw_master}) and~(\ref{eq:Talign}), $\Ebeta$, $\vecpbeta$, and $\pbeta$ are the outgoing $\beta$ particle's (total) energy, momentum 3-vector, and momentum scalar, while $\Enu$, $\vecpnu$, and $\pnu$ are the equivalent quantities for the outgoing (anti-)neutrino, and $E_0$ is the maximum possible $\beta$ energy associated with the transition.\aside{Holstein gives a formula for it.  Maybe I should write it out...}   $\vecJ$ is the nuclear angular momentum vector of the parent, and $J$ is its projection onto the axis of quantization. $\hatj$ is a unit vector in the direction of $\vecJ$ (note that in general, $\hatj \neq \frac{\vecJ}{J}$).  As usual, $\me$ is the mass of the electron, and $c$ is the speed of light.  The infinitesimal surface element $\dOmegabeta$ ($\dOmeganu$) represents the direction of $\beta$ (neutrino) emission.  The function $\FFpm$ is known as a Fermi function for outgoing electrons (top) and positrons (bottom), with $Z^\prime$ the proton number of the daughter nucleus, and is evaluated as in e.g. Refs.~\cite{wilkinson2}\cite{wilkinson3}\cite{wilkinson4}. 

%\aside{Possibly it should be $Z^\prime$ and for the daughter nucleus...}  The Fermi function accounts for the electric force between the nucleus and the charged outgoing beta particle, and its integral over $\Ebeta$ has no closed form expression.

%%%indicates that the direction of beta emission must be considered
%%%Eq.~\ref{equation:jtw_master} must be 
%%%$\beta$ emission must be e 
%%%the probability distribution must be considered to be a function of the direction of $\beta$ emission.

For the reader's convenience, the kinematic factors unique to a particular transition are written out here in terms of their couplings.~\aside{...which are the same as in the Lee-Yang Hamiltonian!}
% !TEX root = ../thesis_main.tex
%
%
%
%%%% --- * --- %%%%	
\begin{eqnarray}
    \xi &=& 
    	|M_F|^2    \left( |C_S|^2 + |C_V|^2 + |C_S^\prime|^2 + |C_V^\prime|^2 \right) 
		\nonumber \\ && + \;\; 
		|M_{GT}|^2 \left( |C_T|^2 + |C_A|^2 + |C_T^\prime|^2 + |C_A^\prime|^2 \right)
	\label{eq:jtw_xi} \\
    \bFierz \, \xi &=& \pm \: 2\gamma \, \textrm{Re}\!\left[ |M_F|^2 \left( C_S C_V^* + C_S^\prime C_V^{\prime *} \right) + |M_{GT}|^2 \left( C_T C_A^* + C_T^{\prime} C_A^{\prime *} \right) \right] 
    \label{eq:jtw_bxi} \\
    \Abeta \, \xi &=& |M_{GT}|^2 \lambda_{J^\prime J} \left[ \pm 2 \textrm{Re}\!\left[ C_T C_T^{\prime *} - C_A C_A^{\prime *} \right] + 2 \frac{\alpha Z \me }{\pbeta} \,\textrm{Im}\!\left[ C_T C_A^{\prime *} + C_T^\prime C_A^* \right] \right] 
		\nonumber \\ && + \;\; 
		\delta_{J^\prime J} \, |M_F| |M_{GT}| \left( \frac{J}{J+1} \right)^{\!\!\! 1/2} \left[ \phantom{\frac{1}{1}\!\!\!} 2 \,\textrm{Re} \! \left[ C_S C_T^{\prime *} +  C_S^\prime C_T^* - C_V C_A^{\prime *} - C_V^\prime C_A^* \right] 
		\right.
		\nonumber \\ && \pm \;\;
		\left.
		2 \frac{\alpha Z \me }{\pbeta} \,\textrm{Im}\!\left[ C_S C_A^{\prime *} + C_S^\prime C_A^* - C_V C_T^{\prime *} -C_V^\prime C_T^* \right] \right]
	\label{eq:jtw_Abetaxi}
\end{eqnarray}
%
%

In Eqs. \ref{eq:jtw_xi} - \ref{eq:jtw_DTRxi}, we have used $\gamma := \left( 1-\alpha^2 Z^{\prime 2} \right)^{1/2}$, and as usual, $\alpha$ is the fine structure constant.  Here, $M_F$ and $M_{GT}$ are the Fermi and Gamow-Teller matrix elements, and are unique to the transition under consideration, 
while the $C_X$ and $C_X^\prime$ (for $X = \{ V,A,S,T\} $) are as in Eq.~(\ref{eq:lee_yang_hamiltonian}).
%are, as in the Lee-Yang Hamiltonian in Eq.~(\ref{eq:lee_yang_hamiltonian}), the complex coupling constants describing the global strength of vector, axial, scalar, and tensor couplings within charged weak interactions.  
%
The possible pseudoscalar couplings ($C_P$ and $C_P^\prime$) have been dropped here because they are relativisitically suppressed.  
\note[jbn]{(Any quark-lepton pseudoscalar couplings have usually been ignored in beta decay, because they are suppressed by (beta momentum)/(nucleon mass).  Note that more recently it's been pointed out that C\_P is naturally quite large in the nucleon (M. Gonzalez-Alonso and J. Martin Camalich Phys Rev Lett 112 042501 (2014)) and allows for significant constraints from allowed beta decay.)}
%~\aside{Somewhere I have to talk about what the primes and the imaginary parts mean.  But maybe later.}  
We have also made use of the following shorthand definitions of $\lambda_{J^\prime J}$, $\Lambda_{J^\prime J}$, and $\delta_{J^\prime J}$ for transitions with parent and daughter nuclear angular momenta given by $J$ and $J^\prime$ respectively:
\bea
\lambda_{J^\prime J} \;\; = \;\; 
	\begin{cases}
		1 				& J^\prime = J - 1 \\
		\frac{1}{J+1} 	& J^\prime = J \\
		\frac{-J}{J+1} 	& J^\prime = J + 1 
	\end{cases}
\eea
\bea
\Lambda_{J^\prime J} \;\; = \;\; 
	\begin{cases}
		1 							& J^\prime = J - 1 \\
		\frac{-(2J-1)}{J+1} 		& J^\prime = J \\
		\frac{J(2J-1)}{(J+1)(2J-3)} & J^\prime = J + 1 
	\end{cases}
\eea
\bea
\delta_{J^\prime J} \;\; = \;\; 
	\begin{cases}
		1 	& J^\prime = J \\
		0	& J^\prime \neq J 
	\end{cases}
	\label{eq:kronecker}
\eea

%%\bea
%%
%%\eea
%%\bea
%%
%%\eea


%%%\note{Define all the things *here*.
%%%\\...\\
%%%%$\vecpbeta$, $\pbeta$, $\vecpnu$, $\pnu$, $\Ebeta$, $\Enu$, $E_0$, 
%%%%$F_{\mp}(Z, \Ebeta)$, 
%%%%$\Omegahatbeta$, $\Omegahatnu$, $\vecJ$, $J$, $\jhat$, 
%%%%\\...\\
%%%$\xi$, $\bFierz \xi$, $\Abeta \xi$, $\calign \xi$, $\Bnu \xi$, $\DTR \xi$.
%%%}


Note that Eq.~\ref{equation:jtw_master} depends on neutrino momentum, which we cannot observe directly;  to make use of the neutrino momentum, it must first be reconstructed from the momenta of the outgoing beta and daughter nucleus.
%Furthermore, we cannot use this expression to reconstruct neutrino momenta in our decay events either, because it would be necessary to account for the momentum of the recoiling daughter nucleus, treating the decay as a three-body problem.  
From an experimental standpoint, within the present experiment we it is not possible to reconstruct the the neutrino momenta with the available data, because we failed to measure the momenta of the daughters in conjunction with the tagged beta decay events with which we are primarily concerned in this thesis, so there is insufficient kinematic information available.   

From a theoretical standpoint, JTW has intentionally neglected recoil-order terms -- meaning that the daughter nucleus is treated, for the purpose of kinetic energy calculations, as being infinitely massive, and as such it must have no change in kinetic energy from the decay--however the approximation still allows for it to undergo a change in momentum.  One result of this approximation is that the neutrino energy, $\Enu$, is not a free variable within Eq.~\ref{equation:jtw_master}, since the total amount of energy released is fixed for a given transition.  The inherent inconsistencies of this approximation make it a tricky starting point for a description of neutrino and recoil kinematics.

% re-formulate Eq.~(\ref{equation:jtw_master}) in terms of the momentum of the daughter instead of the momentum of the neutrino.  

It is fortunately possible to simplify Eq.~(\ref{equation:jtw_master}) by integrating over all possible neutrino directions %(but not the neutrino energy), 
such that the resulting distribution no longer depends on parameters that are not observed.  
%The neutrino energy itself is not a free variable in this equation, because the energy release in the decay is fixed, and given the approximation that none of that energy is allocated to the recoiling daughter, it is very straightforward to calculate the neutrino energy for a decay event in which the beta energy is known.
%
%We haven't integrated out the neutrino momentum.  Neutrino energy itself is a redundant parameter, I think, because we are already using an endpoint energy and a beta energy, and we are not taking recoil-order effects into account.  Also, we treat neutrinos as massless here, which is a perfectly reasonable approximation for our purposes.  For ``convenience'', let's define a nuclear alignment term, $\Talign$, so that:
We take this as an opportunity to specialize the equations for $\beta^+$ decay transitions with $J = J^\prime = \frac{3}{2}$. The result 
%of performing this integration over neutrino direction 
is:
% !TEX root = ../thesis_main.tex
%
%
%
% The JTW Proto-Master
\bea
	\textrm{d}^3 \Gamma \dEe \, \dOmegae
	&=& 
	\frac{2}{(2\pi)^4} \, \FF \, \pe \Ee (E_0 - \Ee)^2 \, \dEe \, \dOmegae \, \xi \nonumber\\ 
	&& \times \left[
		1 + \bFierz \frac{\m c^2}{\Ee} + 
		\A  
		\left(
			\frac{\vecJ}{J} \cdot \frac{\vecpe}{\Ee} 
		\right) 
	\right],
\label{equation:integrated_jtw}
\eea
%
\unskip with the remaining parameters,
% !TEX root = ../thesis_main.tex
%
%
%
%%%% --- * --- %%%%	
\begin{eqnarray}
    \xi &=& 
    	|M_F|^2    \left( |C_S|^2 + |C_V|^2 + |C_S^\prime|^2 + |C_V^\prime|^2 \right) 
		\nonumber \\ && + \;\,
		|M_{GT}|^2 \left( |C_T|^2 + |C_A|^2 + |C_T^\prime|^2 + |C_A^\prime|^2 \right)
		\label{eq:jtw_xi_integrated} \\
    \bFierz \, \xi &=& 
    	- \: 2 \gamma 
	%	\left( 1-\alpha^2 Z^{\prime 2} \right)^{1/2} 
		\textrm{Re}\!\left[ |M_F|^2 \left( C_S C_V^* + C_S^\prime C_V^{\prime *} \right) 
    	+ |M_{GT}|^2 \left( C_T C_A^* + C_T^{\prime} C_A^{\prime *} \right) \right] 
		\nonumber \\
    	\label{eq:jtw_bxi_integrated} \\
%%%    \Abeta \, \xi &=& 
%%%    	|M_{GT}|^2 \frac{1}{J+1}
%%%		% \lambda_{J^\prime J} 
%%%		\left[ - %\pm 
%%%		2 \textrm{Re}\!\left[ C_T C_T^{\prime *} - C_A C_A^{\prime *} \right] 
%%%		+ 2 \frac{\alpha Z \me c^2 }{\pbeta c} \,\textrm{Im}\!\left[ C_T C_A^{\prime *} + C_T^\prime C_A^* \right] \right] 
%%%		\nonumber \\ && + \;\, 
%%%		%\delta_{J^\prime J} \, 
%%%		M_F\,M_{GT} \left( \frac{J}{J+1} \right)^{\!\!\! 1/2} \left[ \phantom{\frac{1}{1}\!\!\!} 2 \,\textrm{Re} \! \left[ C_S C_T^{\prime *} +  C_S^\prime C_T^* - C_V C_A^{\prime *} - C_V^\prime C_A^* \right] 
%%%		\right.
%%%		\nonumber \\ && - % \pm %\;\;
%%%		\left.
%%%		2 \frac{\alpha Z \me c^2}{\pbeta c} \,\textrm{Im}\!\left[ C_S C_A^{\prime *} + C_S^\prime C_A^* - C_V C_T^{\prime *} -C_V^\prime C_T^* \right] \right]
%%%	\\
    \Abeta \, \xi &=& 
    	\frac{4}{5} \, |M_{GT}|^2 \left[ \textrm{Re}\!\left[ C_A C_A^{\prime *} - C_T C_T^{\prime *} \right] + \frac{\alpha Z \me c^2}{\pbeta c} \,\textrm{Im}\!\left[ C_T C_A^{\prime *} + C_T^\prime C_A^* \right] \right] 
		\nonumber \\ && + \;\, 
		2 %\sqrt{ 3/5 } 
		\left( \frac{3}{5} \right)^{\!\! 1/2} \!\!
		M_F\,M_{GT}  \left[ \phantom{\frac{1}{1}\!\!\!\!} \textrm{Re} \! \left[ C_S C_T^{\prime *} +  C_S^\prime C_T^* - C_V C_A^{\prime *} - C_V^\prime C_A^* \right] 
		\right.
		\nonumber \\ && - \;
		\left.
		\frac{\alpha Z \me c^2}{\pbeta c} \,\textrm{Im}\!\left[ C_S C_A^{\prime *} + C_S^\prime C_A^* - C_V C_T^{\prime *} -C_V^\prime C_T^* \right] \right].
	\label{eq:jtw_Abetaxi_integrated}
\end{eqnarray}
%
\aside{I'm missing factors of $c$ in this.  Probably put them in.}
%
\unskip \!\!/This is a great simplification on Eqs.~(\ref{equation:jtw_master})-(\ref{eq:kronecker}).  
%\aside{}
%We still must write the remaining parameters in terms of the relevant nuclear matrix elements and fundamental coupling constants.  These coupling constants are, in general, complex-valued, and JTW does not choose a phase angle for us.  We write them out in 
%Eqs.~(\ref{eq:jtw_xi}-\ref{eq:jtw_Abetaxi}).
Note that JTW presents slightly different expressions for one component of $\Abeta$ within~\cite{jtw} and~\cite{jtw_coulomb}, and the latter convention is what has been adopted here.  
%Here, we adopt the convention from the latter publication.  

We further require that both $M_F$ and $M_{GT}$ must be real, 
%which can be done without loss of generality,~\aside{I don't think this is true...} 
however we do not require that they be positive, which would make the two conventions equivalent.  
%We do not require that either $M_F$ or $M_{GT}$ be positive (which would make the two conventions equivalent), but we can require that both must be \emph{real} without loss of generality.

There are a number of degrees of freedom left in the expressions that remain, and it is not immediately obvious how certain choices about these parameters might affect the physical results, or which approximations or assumptions ought to be made in order to arrive at a theory that matches the observational reality.  This is in part a result of the fact that the theory behind \ac{JTW} was developed before we had a detailed experimental understanding of much of the behaviour of beta decay, so all mathematically consistent behaviours are treated with roughly the same amount of consideration within the model.  

%The qualitative meaning behind some such parameter choices is described in what follows.  
Perhaps the most notable improvement to our understanding of Eqs.~(\ref{equation:integrated_jtw})-(\ref{eq:jtw_Abetaxi_integrated}) is that the weak interaction arises predominantly from vector ($C_V$, $C_V^\prime$) and axial-vector ($C_A$, $C_A^\prime$) couplings, and the scalar and tensor couplings, if present at all, are comparatively small.  
% and must be relegated to the realm of precision measurements to search for exotic physics. 
In particular, 
\bea
\frac{ |C_S|^2 + |C_S^\prime|^2 }{ |C_V|^2 + |C_V^\prime|^2 }        &\ll& 1 
\\
\frac{ |C_T|^2 + |C_T^\prime|^2 }{ |C_A|^2 + |C_A^\prime|^2 }        &\ll& 1.
\eea
\aside{Ugh.  Is this even the right thing to compare?}
%($C_S$, $C_S^\prime$) and tensor ($C_T$, $C_T^\prime$) couplings 


Imaginary components of $C_X$ and $C_X^\prime$ are associated with breaking of time-reversal symmetry for the transition.  Since this effect has never been observed, it must be comparatively small if it is present, and so it follows that
%To the best of our collective understanding, 
\bea
\left| \frac{\Im \,[C_X]}{\Re \,[C_X]} \, \right|               &\ll& 1 
\\
\left| \frac{\Im \,[C_X^\prime]}{\Re \,[C_X^\prime]} \, \right| &\ll& 1, 
\eea
which holds \emph{at least} for the cases of $X=\{V,A\}$, as there is comparatively little experimental information on how scalar and tensor interactions might behave with regard to time reversal, for the obvious reason that they have never been observed.  \aside{This isn't quite right, there *are* limits....}
%\note{I think I made an even number of sign errors...}
In order to obtain the correct, physically observed value for $\Abeta$, we require that the 
$M_{F}\,M_{GT}$ term in Eq.~(\ref{eq:jtw_Abetaxi}) have an overall positive value.  Because we know that the scalar and tensor couplings must be small, and any imaginary contributions to the term must be small, we conclude that
\bea
	M_{F}\,M_{GT} \left( C_V C_A^{\prime *} + C_V^\prime C_A^* \right) < 0.
\eea

\aside{JTWcoulomb to find sources that thought that S,T.}
%\note{In order to make JTW give us the correct, physically observed value for $\Abeta$, we need ... something.  JTW writes an expression for $\Abeta$ with slightly different sign convention in~\cite{jtw} and~\cite{jtw_coulomb}.  There is some subtlety involved in getting the correct signs on things, but the easiest way to figure it out is to make sure the calculation of the value of $\Abeta$ matches reality.  To this end, ... either for both $M_{GT}$ and $M_{F}$ to be positive, or we use the notation for $\Abeta$ from JTW's earlier paper rather than the one with the coulomb corrections.  Somehow, we run into problems getting JTW to agree with Holstein if we require $M_{GT}$ and $M_{F}$ to have the same sign, so we're going to kludge together two different expressions for $\Abeta$ from both of JTW's papers.  Or ... wait ... maybe the opposite thing is what I should be doing?  In handwritten notes, I explicitly use the *later* JTW sign convention.}

%\note{In order to make JTW give us the correct, physically observed value for $\Abeta$, we need either for both $M_{GT}$ and $M_{F}$ to be positive, or we use the notation for $\Abeta$ from JTW's earlier paper rather than the one with the coulomb corrections.  Somehow, we run into problems getting JTW to agree with Holstein if we require $M_{GT}$ and $M_{F}$ to have the same sign, so we're going to kludge together two different expressions for $\Abeta$ from both of JTW's papers.  Or ... wait ... maybe the opposite thing is what I should be doing?  In handwritten notes, I explicitly use the *later* JTW sign convention.}

\aside{Also, $\xi = G_v^2 \, \cos\theta_C \, f_1(E).$}

\note{Handedness considerations?}



%%%% --- * --- %%%%	
%%%% --- * --- %%%%	
%%%% --- * --- %%%%	
\section[Holstein Formalism]{Holstein Formalism} 
\label{sec:holstein_formalism}
\note{Probably the phrasing here is way too casual.  Fix it.}
Holstein~\cite{holstein}~\cite{holstein_errata} generously provides explicit equations to match both Eq.~(\ref{equation:jtw_master}) (i.e. Holstein's Eq.~(51), where neutrino direction is a parameter of the probability distribution) and Eq.~(\ref{equation:integrated_jtw}) (Holstein's Eq.~(52), where neutrino direction has already been integrated over).  

%Holstein~\cite{holstein} generously provides explicit equations to match both Eq.~\ref{equation:jtw_master} (where neutrino direction is a parameter of the probability distribution), and Eq.~\ref{equation:integrated_jtw} (where the neutrino direction has already been integrated over).

%This one is harder.  But here, we've already integrated over neutrino momentum at least.  That's something.  

Here's Holstein's Eq.~(52):
% !TEX root = ../thesis_main.tex



% "A PDF for the People"
\bea
\mathrm{d}^3 \Gamma &=& 2  G_v^2 \cos^2\theta_c \frac{\FF}{(2\pi)^4} \, \pe \Ee (E_0 - \Ee)^2 \dEe \, \dOmegae 
\nonumber\\
&& \times
\left\{
	F_0(\E) 
	+ \Lambda_1 F_1(\E) \hatn \cdot \frac{\vecpe}{\Ee}
	+ \Lambda_2 F_2(\E) \left[ \left( \nhat \cdot \frac{\vecpe}{\Ee} \right)^2 - \frac{1}{3}\frac{\pe^2}{\Ee^2} \right]
	\right. \nonumber\\ && \left.
	+ \Lambda_3 F_3(\E) 
		\left[ 
			\left( \hatn \cdot \frac{\vecpe}{\Ee} \right)^3
			- \frac{3}{5}\frac{\pe^2}{\Ee^2}\hatn \cdot \frac{\vecpe}{\Ee}
		\right]
\right\}
\label{equation:holstein52}
\eea
\unskip  % equation:holstein52
where $\Ebeta$, $\vecpbeta$, $\pbeta$, $E_0$, $\dOmegabeta$, and $\FFpm$ are as in Eq.~(\ref{equation:jtw_master}), 
%\note{... where $\Ebeta$, $\vecpbeta$, $\pbeta$, $E_0$, $\dOmegabeta$, and $\FFpm$ are as in Eq.\ref{equation:jtw_master}.
%}
and the $\Lambda_i$ are given by Holstein's Eq.~(48), where, \emph{within this context}, $M$ is the nuclear spin projection along the axis of quantization:
\bea
    \Lambda_1   &:=& \LambdaOne   
    \label{eq:lambda1} \\
    \Lambda_2   &:=& \LambdaTwo 
    \label{eq:lambda2} \\
    \Lambda_3   &:=& \LambdaThree .
    \label{eq:lambda3}
\eea
and we immediately see a relation between several terms in \ac{JTW}'s and Holstein's descrioptions:
\bea
\textrm{Holstein's \,} \mathbf{\hat{n}} &=& \textrm{JTW's \,} \mathbf{\hat{j}}
\label{eq:nequalsj} \\
\Lambda_1 \hatj &=& \LambdaOne \hatj \;\; = \;\; \frac{\vecJ}{J}  \\
\Lambda_2 &=& \Talign \frac{(2J-1)}{(J+1)}.
\eea
Note that $\Lambda_3$ is a quadrupole term, and \ac{JTW} has no equivalent.
\note{Make this shit a table.}
\note{Also define:  $G_v$, $\theta_c$.}
\note{Note:  It's not the case that $ | \vecJ | == J $.  It's actually super fucking infuriating notation. }

%%%%
%%%%It can be immediately seen that Holstein's $\Lambda_1$ is closely related to JTW's $\frac{\vecJ}{J}$, and 
%%%%%a bit later after John points it out to us, we see that 
%%%%Holstein's $\Lambda_2$ is closely related to JTW's $\Talign$.  JTW has no equivalent to $\Lambda_3$.  We find:
%%%%\bea
%%%%\Lambda_1 \hatj &=& \LambdaOne \hatj \;\; = \;\; \frac{\vecJ}{J}  \\
%%%%%\Lambda_2 \frac{(J+1)}{(2J-1)} &=& \Talign
%%%%\Lambda_2 &=& \Talign \frac{(2J-1)}{(J+1)}.
%%%%\eea
%%%%We proceed by defining some notational conventions.  Firstly, 
%%%%\bea
%%%%\textrm{Holstein's \,} \mathbf{\hat{n}} &=& \textrm{JTW's \,} \mathbf{\hat{j}},
%%%%\label{eq:nequalsj}
%%%%\eea




\note{}
The careful reader will eventually note that despite the deceivingly similar notation, Holstein's spectral functions $F_i(\Ee)$ are not the same as the $F_i(\Ee, u, v, s)$ in any limit.
%, despite the notational similarities.  
Among other rules, Holstein's spectral functions obey these:
\bea
	F_i(\Ee) &\neq& F_i(\Ee, u, v, s)    \\
	F_i(\Ee) &=&    H_i(\Ee, u, v, 0)    \\
	f_i(\Ee) &=&    F_i(\Ee, u, v, 0).
\eea
For the $F_i(\Ee)$ functions of interest to us here, we find the following relationships:
%though expressions for $F_i(\Ee)$ can still be obtained through a chain of substitutions: 
\begin{align}
F_0(\Ee) & = H_0(\Ee, J, J^\prime, 0) = F_1(\Ee, J, J^\prime, 0) 
	\!\!\! 
	& = &\; f_1(\Ee) 
	\nonumber \\
F_1(\Ee) & = H_1(\Ee, J, J^\prime, 0) = \textstyle F_4(\Ee, J, J^\prime, 0) \,+\! \frac{1}{3}F_7(\Ee, J, J^\prime, 0) 
	\!\!\! 
	& = &\; \textstyle f_4(\Ee) \,+\! \frac{1}{3}f_7(\Ee) 
	\nonumber \\
F_2(\Ee) & = H_2(\Ee, J, J^\prime, 0) = \textstyle F_{10}(\Ee, J, J^\prime, 0) \!+\! \frac{1}{2}F_{13}(\Ee, J, J^\prime, 0) 
	\!\!\! 
	& = &\; \textstyle f_{10}(\Ee) \!+\! \frac{1}{3}f_{13}(\Ee) 
	\nonumber \\
F_3(\Ee) & = H_3(\Ee, J, J^\prime, 0) = F_{18}(\Ee, J, J^\prime, 0) 
	\!\!\!
	& = &\; f_{18}(\Ee) .
\label{eq:holstein_FHFf}
\end{align}
\unskip  % eq:holstein_FHFf
\aside{Check: what's the deal with $f_{13}$'s scaling in $F_2$? }

Note that the $f_i(\Ee)$ in Eq.~\ref{eq:holstein_FHFf} are the same spectral functions used to describe a polarized decay spectrum when the neutrino (ie, the recoil) is also observed -- though of course such a spectrum must have other terms as well.  For the spectrum of interest to us here, in which the neutrino direction has already been integrated over, we can simply look up the $H_i(\Ee, J, J^\prime, 0) = H_i(E, u, v, s\!=\!0)$ spectral functions, and leave it at that.  We find:
%Here, we've taken $u=J$ and $v=J^\prime$ to be the initial and final angular momenta respectively, because apparently I'm having a hard time keeping my notation straight.
% !TEX root = ../thesis_main.tex
%
%
%
%%%% --- * --- %%%%	
\begin{multline}
F_0(\Ee) = 
\left| a_1 \right|^2 
+ 2 \Re\left[ a_1^* a_2 \right] \frac{1}{3 M^2} 
\left[  
	\m^2 + 4 \Ee E_0 + 2 \frac{\me^2}{\Ee}E_0 - 4\Ee^2
\right]
\\
+ \left| c_1 \right|^2
+ 2 \Re\left[ c_1^* c_2\right] \frac{1}{9 M^2} 
\left[
	11 \me^2 + 20 \Ee E_0 
	- 2\frac{\me^2}{\Ee}E_0
	- 20\Ee^2
\right]
- 2 \frac{E_0}{3M} \Re\left[ c_1^*(c_1 + d \pm b)\right]
\\
+ \frac{2\Ee}{3M} 
\left( 
	3 \left| a_1 \right|^2 + \Re \left[ c_1^*(5c_1 \pm 2 b) \right]
\right)
- \frac{\me^2}{3 M \Ee} 
\Re \left[ 
	-3 a_1^*e + c_1^*\left(2c_1 + d \pm 2b - h\frac{E_0 - \Ee}{2M} \right)
\right]
\end{multline}
%\unskip
% !TEX root = ../thesis_main.tex
%
%
%
%%%% --- * --- %%%%	
\begin{align}
F_1(\Ee) = & \:\:
\deltauv \left( \frac{u}{\!u+1\!} \right)^{\!\!1/2} \!\!
\left\{
	2 \Re\left[ 
		a_1^*\left(\! c_1 - \frac{E_0}{3M}(c_1 + d \pm b) + \frac{\Ee}{3M}( 7 c_1 \pm b + d )\! \right)
	\right]
	\right. \nonumber \\ & \left. 
	+
	2 \Re\left[
		a_1^* c_2 + c_1^* a_2
	\right] \!
	\left(
		\frac{4 \E(E_0 - \E) + 3 \me^2}{3M^2}
	\right) \!
\right\}
\nonumber \\ &
\mp \frac{ (-1)^s \gammauv}{u+1} 
\Re \left\{ \!
	c_1^* \! \left(
		c_1 + 2 c_2 \left(\frac{8\Ee(E_0-\Ee)+3\me^2}{3M^2}\right)
		- \frac{2 E_0}{3 M} (c_1 + d \pm b) 
		\right.\right. \nonumber \\ & \left.\left.
		+ \frac{\Ee}{3M} (11 c_1 - d \pm 5 b)
	\right)
\right\}
%\nonumber \\ &
+ 
\frac{\lambdauv}{u+1}
\Re \left\{ \!
	c_1^* \! \left[
		- f \left(\frac{5\Ee}{M}\right)
		\right. \right. \nonumber \\ & \left. \left.
		+ g \left( \frac{3}{2} \right)^{\!\!1/2} \!
		\left(
			\frac{E_0^2 - 11 E_0 \Ee + 6 \me^2 + 4\Ee^2 }{6M^2}
		\right) 
		%\right.\right. \nonumber \\ & \left.\left.
		\pm 3 j_2 
		\left(
			\frac{8 \Ee^2 - 5E_0 \Ee - 3 \me^2}{6M^2}
		\right)
	\right] \!
\right\}
\end{align}
%\unskip
% !TEX root = ../thesis_main.tex
%
%
%
%%%% --- * --- %%%%	
\begin{flalign}
F_2(\Ee) = &
\thetauv \frac{\Ee}{2M} 
\Re\left[
	c_1^*\left(
		c_1 + c_2 \frac{8(E_0-\Ee)}{3M}
		-d \pm b
	\right)
\right]
&& \nonumber \\ &
- \deltauv \frac{\Ee}{M} 
\left[ \frac{u(u+1)}{(2u-1)(2u+3)} \right]^{1/2} \!
\Re \left\{ \!
	a_1^*\left( 
		\left( \! \frac{3}{2} \right)^{\!\!1/2}\!\! f
		+ g \frac{\Ee+2E_0}{4M} 
		\right.\right.
		&& \nonumber \\ &
		\left.\left.
		\pm \left( \frac{3}{2} \right)^{\!\!1/2}\!\! j_2 \frac{E_0-\E}{2M}
	\right) \!
\right\}
+ (-1)^s \, \kappauv \frac{\E}{2M}
\Re \left[
	c_1^* \! \left(
		\pm \, 3 f 
		\pm \left( \frac{3}{2} \right)^{\!\!1/2}\!\! g \frac{E_0-\Ee}{M}
		\right.\right.
		&& \nonumber \\ &
		\left.\left.
		+ 3 j_2 \frac{E_0-2\Ee}{2M}
	\right)
\right]
+ \epsilonuv \Re\left[ c_1^* j_3 \right]
\left( 
	\frac{21 \E^2}{8 M^2}
\right) &&
\end{flalign}
%\unskip
% !TEX root = ../thesis_main.tex
%
%
%
%%%% --- * --- %%%%	
\begin{flalign}
F_3(\Ee) = &
- \deltauv \, (3 u^2 + 3 u -1)
\left[
	\frac{u}{(u-1)(u+1)(u+2)(2u-1)(2u+3)}
\right]^{1/2}\!
&& \nonumber \\ &
\times 
\Re \left[
	a_1^* j_3
\right]
\left( 
	\frac{\Ee^2 \sqrt{15} }{4M^2}
\right)
+ 
\frac{\rhouv}{u+1} 
\Re \left[
	c_1^*(g\sqrt{3} + j_2\sqrt{2})
	\left(
		\frac{5 \Ee^2}{4 M^2}
	\right)
\right]
&& \nonumber \\ &
\pm
\frac{(-1)^s \sigmauv }{u+1} 
\Re \left[ c_1^*j_3 \right] 
\left( 
	\frac{5\Ee^2}{2 M^2}
\right) &&
\label{equation:holstein_F3}
\end{flalign}
%\unskip
%and we might really appreciate if these things could be simplified a bit.  
where most of the terms in Eqs.~\ref{equation:holstein_F0}-\ref{equation:holstein_F3} have yet to be defined.

The terms $a_1, a_2, b, c_1, c_2, d, e, f, g, h, j_2, j_3$ are described as being structure functions.  Holstein gives some predictions for their form, assuming the impulse approximation holds, in his Eq.~(67).
%\aside{There was something wrong with this assumption.  Something circular.  I forget.  Blah.}  
For the most part, the forms of these structure functions are beyond the scope of this thesis, so I will not re-state them here, however
%\aside{Or will I?}  
it should be noted that the numerical values used for these parameters were taken from a private communication from Ian Towner to the TRINAT collaboration\cite{itownerCalcs}.
%\comment{It should be noted that the numerical values used for these parameters came from a private communication from Ian Towner to ... someone other than me.}  

An exception is made for parameters $a_i$ and $c_i$, as these are closely related to the Fermi- and Gamow-Teller couplings for the transition, and must be compared to the equivalent expressions within JTW's formalism.  
%therefore there are equivalent expressions available in JTW's formalism which must be compared.  
%it is important to note the expressions for $a_i$ and $c_i$, because these will directly come into play when we try to reconcile Holstein's expression with JTW's.  
In fact, $a_1$ and $a_2$ ($c_1$ and $c_2$) are terms within a series expansion for the vector (axial) couplings, including \ac{ROC}, with recoil energy $q$ and average nuclear mass (of the parent and daughter) $M$, such that:
\bea
a(q^2) &=& a_1 + \left(\! \frac{q^2}{M^2} \! \right) a_2 + \cdots \label{equation:series_expand_a} \\
c(q^2) &=& c_1 + \left(\! \frac{q^2}{M^2} \! \right) c_2 + \cdots \label{equation:series_expand_c}
\eea
Using the impulse approximation, Holstein finds:
\bea
a(q^2) &\approx& \frac{g_V(q^2)}{(1 + \frac{\Delta}{2M})} \left[ M_F    + \frac{1}{6}(q^2 - \Delta^2) M_{r^2} + \frac{1}{3} \Delta M_{\mathbf{r} \cdot \mathbf{p} } \right] 
\label{equation:full_a}
\\ 
c(q^2) &\approx& \frac{g_A(q^2)}{(1 + \frac{\Delta}{2M})} \left[ M_{GT} + \frac{1}{6}(q^2 - \Delta^2) M_{\sigma r^2} + \frac{1}{6 \sqrt{10} }(2\Delta^2 + q^2) M_{1y} 
\right. \nonumber \\ && \left.
+ A \frac{\Delta}{2 M} M_{\sigma L} + \frac{1}{2} \Delta M_{\sigma r p} \right],
\label{equation:full_c}
\eea
where $\Delta$ is the difference between the masses of the parent and daughter nuclei, $M_F$ and $M_{GT}$ are the familiar Fermi and Gamow-Teller matrix elements specific to the transition, and $g_V(q^2)$ and $g_A(q^2)$ are the universally applicable vector and axial couplings (which vary according to the energy scale involved).  The terms $M_{r^2}$, $M_{\mathbf{r} \cdot \mathbf{p} }$, $M_{\sigma r^2}$, $M_{1y}$, $M_{\sigma L}$, and $M_{\sigma r p}$ are matrix elements relating to the nuclear structure of the parent and daughter isotopes.  For the sake of simplicity, we note that for the transition of primary concern to us here, $^{37}\textrm{K} \rightarrow \,^{37}\textrm{\!Ar} + \beta^{+} + \nu_e$, we find that \mbox{$M_{\mathbf{r} \cdot \mathbf{p} } = M_{\sigma L} = M_{\sigma r p} = 0$}, so those terms can safely be dropped\cite{itownerCalcs}.  Recalling that the energy dependence in $g_V(q^2)$ and $g_A(q^2)$ only becomes relevant at much higher energy scales, we will take the approximation that they are to be treated as constant.
%are nearly constant at the energy scales of concern to 
It immediately follows that:
\bea
a_1 &\approx& g_V \left( 1 - \frac{\Delta}{2 M} \right) \left[ M_F - \frac{1}{6} \Delta^2 M_{r^2} \right] \label{eq:a1} \\
a_2 &\approx& \frac{1}{6} M^2 \, g_V \left( 1 - \frac{\Delta}{2 M} \right) M_{r^2} \label{eq:a2} \\
c_1 &\approx& g_A \left( 1 - \frac{\Delta}{2 M} \right) \left[ M_{GT} + \frac{1}{6} \Delta^2 \left( \frac{2}{\sqrt{10}} M_{1y} - M_{\sigma r^2} \right) \right] \label{eq:c1} \\
c_2 &\approx& \frac{1}{6} M^2 \, g_A \left( 1 - \frac{\Delta}{2 M} \right) \left[ \frac{1}{\sqrt{10}} M_{1y} + M_{\sigma r^2} \right] \label{eq:c2}
\eea

%%%\note{The fuck is $A$?}
%%%$M_{r^2} =$ something, \\
%%%$M_{\mathbf{r} \cdot \mathbf{p} } = 0$ \\
%%%$M_{\sigma r^2} =$ something \\
%%%$M_{1y} =$ something \\
%%%$M_{\sigma L} = 0$ \\
%%%$M_{\sigma r p} =0$ \\


\note{}
%%%%Therefore, 
%%%%\bea
%%%%a(q^2) &\approx& \frac{g_V(q^2)}{(1 + \frac{\Delta}{2M})} \left[ M_F    + \frac{1}{6}(q^2 - \Delta^2) M_{r^2} + \frac{1}{3} \Delta M_{\mathbf{r} \cdot \mathbf{p} } \right] 
%%%%\label{equation:full_a}
%%%%\\ 
%%%%c(q^2) &\approx& \frac{g_A(q^2)}{(1 + \frac{\Delta}{2M})} \left[ M_{GT} + \frac{1}{6}(q^2 - \Delta^2) M_{\sigma r^2} + \frac{1}{6 \sqrt{10} }(2\Delta^2 + q^2) M_{1y} 
%%%%\right. \nonumber \\ && \left.
%%%%+ A \frac{\Delta}{2 M} M_{\sigma L} + \frac{1}{2} \Delta M_{\sigma r p} \right]
%%%%\label{equation:full_c}
%%%%\eea
%%%%\note{Somewhere I have to define $q^2$ and $\Delta$ are.}
%%%%...where the $M_{xxx}$'s are certain nuclear matrix elements.  \aside{Should I just list the values of things that I inherited from Ian Towner's personal communication that one time?}  However, Eqs.~(\ref{equation:holstein_F0}-\ref{equation:holstein_F3}) are not written in terms of $a(q^2)$ and $c(q^2)$, but rather in terms of $a_1$, $a_2$, $c_1$, and $c_2$.  In fact, Holstein is implicitly using series expansions to remove the dependence on recoil momentum, so that
%%%%\bea
%%%%a(q^2) &=& a_1 + \left(\! \frac{q^2}{M^2} \! \right) a_2 + \cdots \label{equation:series_expand_a} \\
%%%%c(q^2) &=& c_1 + \left(\! \frac{q^2}{M^2} \! \right) c_2 + \cdots \label{equation:series_expand_c}
%%%%\eea

%Also, Holstein proceeds to split up $a(q^2)$ and $c(q^2)$ into their first two Taylor series terms for an expansion of .... $q/M$, maybe?  Or possibly $q^2 / M^2$?  Anyway, that's $a_1$ and $a_2$, and $c_1$ and $c_2$, in Holstein notation.  He doesn't do that with any of the other structure functions. 
%\bluetodo{Seriously, I need to check what the taylor series is even expanding in.}


%\note{In fact, we might want to add the other different-er terms in to this thing now, before we get ahead of ourselves.}
Next, Holstein goes and tweaks those $F_i(\Ebeta)$ terms that we've already written out, by adding in an adjustment for Coulomb corrections.  Those corrections have this form:
\beq
	F_i(\Ebeta) \rightarrow \tilde{F}_i(\Ebeta) := \FF \left[ F_i(\Ebeta) + \Delta F_i(\Ebeta) \right]
\eeq

To obtain expressions for the $\Delta F_i(\Ebeta)$, Holstein invokes some Feynman diagrams and provides expressions for several integrals, all of which are both complex and complicated.  The modified spectral functions are provided in terms of functions of these integrals.  Since nobody wants to have to evaluate those integrals, Holstein makes a further approximation by taking only the first term in an expansion of the $\Delta F_i(\Ebeta)$ in terms of $Z\alpha$, where $Z\alpha \ll 1$.  Then, the resulting expressions for $\Delta F_i(\Ebeta)$ can be written in terms of much more straightforward integrals over form factors for electric change and weak charge.  

If we make the further assumption that these form factors are identical, and that both types of charge are spread over a ball of uniform density with radius $R$,\aside{and also, I think something like that the weak charge is the same distribution as the electric charge} then we find:
\bea
	X = Y = \frac{9\pi R}{140}
\eea
in the Eqs.~(\ref{eq:holstein_DeltaF1_Euvs} - \ref{eq:holstein_DeltaF7_Euvs}) that follow.

Because Holstein doesn't actually write 
%this stuff out 
these expressions
in terms of $F_i(\Ebeta)$, but rather in terms of $F_i(\Ebeta, u,v,s)$, this correction presents yet another opportunity for the reader to interpret his notation incorrectly.  We note that one must remember to make use of the relations in Eq.~(\ref{eq:holstein_FHFf}).  Furthermore, Holstein notes that some of the terms $F_i(\Ebeta, u,v,s)$ are suppressed already, and he does not consider those terms further.  We will take this approximation to be adequate for our purposes here.

%\note{So clearly I'm going to need terms for $\Delta F_1(\Ebeta, u, v, s)$, $\Delta F_4(\Ebeta, u, v, s)$, $\Delta F_7(\Ebeta, u, v, s)$, $\Delta F_{10}(\Ebeta, u, v, s)$, $\Delta F_{13}(\Ebeta, u, v, s)$, and $\Delta F_{18}(\Ebeta, u, v, s)$. We really only have expressions for some of them in Holstein's Eq.~(C4).  In particular, we've got $\Delta F_1(\Ebeta, u, v, s)$, $\Delta F_4(\Ebeta, u, v, s)$ and $\Delta F_7(\Ebeta, u, v, s)$, but we're missing $\Delta F_{10}(\Ebeta, u, v, s)$, $\Delta F_{13}(\Ebeta, u, v, s)$, and $\Delta F_{18}(\Ebeta, u, v, s)$.  That's annoying.  Holstein gives as an excuse for that (have to check to make sure it works and that it's actually an excuse is for this) that the recoil terms $b$, $d$, and $f$ are already suppressed in their contribution to beta decay spectra. }

\note{ What is less clear, given the context in the paper, is whether or not when Holstein writes out his simplified expressions for $\Delta F_{x}(\Ebeta, u, v, s)$ he actually means $ \FF \Delta F_{i}(\Ebeta, u, v, s)$.  These terms are pretty small, so it probably doesn't *really* matter, but it would still be really nice to *know*, damn it.}

So, we'll write out the functions for these corrections.  
% !TEX root = ../thesis_main.tex
%
%
%
%%%% --- * --- %%%%	
\begin{flalign}
\Delta F_1(\Ee, u, v, s) = &
\mp \left( \frac{8 \alpha Z}{3\pi } \right) \left\{ |a|^2 \left[ 4 \Ee (X+Y) + E_0 X + \textstyle{\frac{\m c^2}{\Ee }}(X+2Y) \right] 
\right. &&\nonumber \\ & \left.
+ |c|^2\left[ \Ee ( \textstyle{\frac{16}{3}} X + 4 Y)  - \textstyle{\frac{1}{3}}E_0 X + \textstyle{\frac{\m c^2}{\Ee }} (X+2Y) \right]
\right\} &&
\label{eq:holstein_DeltaF1_Euvs}
\end{flalign}
% 
% 
% 
% \textstyle{\frac{16}{3}}\unskip
% !TEX root = ../thesis_main.tex
%
%
%
%%%% --- * --- %%%%	
\begin{flalign}
\Delta F_4(\Ee, u, v, s) = &
\mp \left( \frac{8 \alpha Z}{3\pi } \right) \Ee \, (5X + 4Y) \left[ \deltauv \left( \frac{u}{u+1} \right)^{1/2} 2\Re \,[a^* c] 
\right. && \nonumber \\ & \left.
\mp (-1)^s \left( \frac{\gammauv}{u+1} \right)^{\phantom{1/2}}\!\!\!\!\! |c|^2 \right] 
&&
\end{flalign}
% 
% 
% 
% \textstyle{\frac{16}{3}}\unskip
% !TEX root = ../thesis_main.tex
%
%
%
%%%% --- * --- %%%%	
\begin{align}
\Delta F_7(\Ee, u, v, s) = &
\mp \left( \frac{8 \alpha Z}{3\pi } \right) \left[ \deltauv \left( \frac{u}{u+1} \right)^{1/2} 2\Re \,[a^* c] 
\right. \nonumber \\ & \left.
\mp (-1)^s \left( \frac{\gammauv}{u+1} \right)^{\phantom{1/2}}\!\!\!\!\! |c|^2 \right] (E_0 - \Ee) X
\label{eq:holstein_DeltaF7_Euvs}
\end{align}
% 
% 
% 
% \textstyle{\frac{16}{3}}\unskip

We note that the above corrections have been written in terms of $a=a(q^2)$ and $c=c(q^2)$, and we must use Eqs.~(\ref{equation:series_expand_a}, \ref{equation:series_expand_c}) to put the results in terms of $a_1$,  $a_2$, $c_1$, and $c_2$ so that they can be correctly combined with Eqs.~(\ref{equation:holstein_F0}-\ref{equation:holstein_F3}).

If we evaluate Holstein's Eqs.~(B8) for $\beta^+$ decay modes (i.e., the \emph{lower} sign when the option arises), taking
%, which I will absolutely not type out here, 
%for the case where 
$u=v=J=J^\prime=3/2$ and $s=0$, we find the following values:
% for $\beta^+$ decay modes (the \emph{lower} sign when the option arises):
\begin{align}
\deltauv     &= 1 
& \thetauv   &= 1 
& \rhouv     &= \frac{-41}{40}
	\nonumber\\
\gammauv     &= 1 
& \kappauv   &=\frac{1}{2\sqrt{2}} % \;\; ! \!=\;\; \frac{3}{\sqrt{2}}
& \sigmauv   &= \frac{-41}{4\sqrt{35}}
	\nonumber\\
\lambdauv    &= \frac{-\sqrt{2} }{5} % 2\sqrt{3} 
& \epsilonuv &= \frac{-1}{2\sqrt{5}}
& \phiuv     &= 0 % \frac{1}{32} \left(\frac{3}{5}\right)^{\!1/2}.
	\nonumber\\
\label{eq:holstein_greekletterfunctions}
%
\end{align}
\unskip  % eq:holstein_greekletterfunctions
\note{Also, pretty sure one of those never gets used.  Which one was it?  idk.}
%%
%%Furthermore, in our calculations here, we will be considering only the $\beta^+$ decay modes, and therefore we take the \emph{lower} sign when the option arises.  We also will use $s=0$, so that $(-1)^s = +1$.


%%% -- %%%
%- -- --- -- - 
%Let's define some of that notation!

%%%%We proceed by defining some notational conventions.  Firstly, 
%%%%\bea
%%%%\textrm{Holstein's \,} \hat{n} &=& \textrm{JTW's \,} \mathbf{j},
%%%%\label{eq:nequalsj}
%%%%\eea
%%%%and the 
%%%%$\Lambda_i$ are given by Holstein's Eq.~(48):
%%%%\bea
%%%%    \Lambda_1   &:=& \LambdaOne   
%%%%    \label{eq:lambda1} \\
%%%%    \Lambda_2   &:=& \LambdaTwo 
%%%%    \label{eq:lambda2} \\
%%%%    \Lambda_3   &:=& \LambdaThree .
%%%%    \label{eq:lambda3}
%%%%\eea
%%%%
%%%%\aside{Note:  It's not the case that $ | \vecJ | == J $.  It's actually super fucking infuriating notation. }
%%%%%\note[color=tag]{Clean up phrasing at the end of the PDF for the People.}
%%%%It can be immediately seen that Holstein's $\Lambda_1$ is closely related to JTW's $\frac{\vecJ}{J}$, and 
%%%%%a bit later after John points it out to us, we see that 
%%%%Holstein's $\Lambda_2$ is closely related to JTW's $\Talign$.  JTW has no equivalent to $\Lambda_3$.  We find:
%%%%\bea
%%%%\Lambda_1 \hatj &=& \LambdaOne \hatj \;\; = \;\; \frac{\vecJ}{J}  \\
%%%%%\Lambda_2 \frac{(J+1)}{(2J-1)} &=& \Talign
%%%%\Lambda_2 &=& \Talign \frac{(2J-1)}{(J+1)}.
%%%%\eea


%Now we'll have to deal with expanding the $F_i(\Ee)$.  %Note that these are very different from the $F_i(\Ee, u, v, s)$, and also different from the $f_i(\Ee)$.  
%Holstein makes a goddamn mess of this, so here we go!  From Holstein's Eq.~(B10):
%\bea
%F_i(\Ee) &=& H_i(\Ee, J, J^\prime, 0).
%\eea
%%and from Holstein's Eq.~(B9), we see that 
%%\bea
%%f_i(\Ee) &=& F_i(\Ee, J, J^\prime, 0)
%%\eea
%From Holstein's many Eqs.~(B7), we see that the $H_i(\Ee, u, v, s)$ can be written in terms of the functions $F_i(\Ee, u, v, s)$, which we carefully note \emph{are not the same} as the functions $F_i(\Ee)$.  We further see, from Holstein's Eq.~(B9) that a further set of functions, $f_i(\Ee)$ are defined in terms of the $F_i(\Ee, u, v, s)$.  In particular, Holstein's Eq.~(B9) states that
%\bea
%f_i(\Ee) &=& F_i(\Ee, J, J^\prime, 0).
%\eea
%Then, if we combine (some parts of) Holstein's Eqs.~(B7) with (B9) and (B10):
%\begin{align}
F_0(\Ee) & = H_0(\Ee, J, J^\prime, 0) = F_1(\Ee, J, J^\prime, 0) 
	\!\!\! 
	& = &\; f_1(\Ee) 
	\nonumber \\
F_1(\Ee) & = H_1(\Ee, J, J^\prime, 0) = \textstyle F_4(\Ee, J, J^\prime, 0) \,+\! \frac{1}{3}F_7(\Ee, J, J^\prime, 0) 
	\!\!\! 
	& = &\; \textstyle f_4(\Ee) \,+\! \frac{1}{3}f_7(\Ee) 
	\nonumber \\
F_2(\Ee) & = H_2(\Ee, J, J^\prime, 0) = \textstyle F_{10}(\Ee, J, J^\prime, 0) \!+\! \frac{1}{2}F_{13}(\Ee, J, J^\prime, 0) 
	\!\!\! 
	& = &\; \textstyle f_{10}(\Ee) \!+\! \frac{1}{3}f_{13}(\Ee) 
	\nonumber \\
F_3(\Ee) & = H_3(\Ee, J, J^\prime, 0) = F_{18}(\Ee, J, J^\prime, 0) 
	\!\!\!
	& = &\; f_{18}(\Ee) .
\label{eq:holstein_FHFf}
\end{align}
\unskip
%So that's fun.  Note that the $f_i(\Ee)$ are what goes into the polarized decay spectrum when the neutrino (ie, the recoil) is also observed.  It's a more complicated spectrum that way.  For this spectrum in which the neutrino has already been integrated over, we can just look up the $H_i(\Ee, J, J^\prime, 0) = H_i(\Ee, u, v, s)$ spectral functions, and leave it at that.
%So let's do this thing!
%% !TEX root = ../thesis_main.tex
%
%
%
%%%% --- * --- %%%%	
\begin{multline}
F_0(\Ee) = 
\left| a_1 \right|^2 
+ 2 \Re\left[ a_1^* a_2 \right] \frac{1}{3 M^2} 
\left[  
	\m^2 + 4 \Ee E_0 + 2 \frac{\me^2}{\Ee}E_0 - 4\Ee^2
\right]
\\
+ \left| c_1 \right|^2
+ 2 \Re\left[ c_1^* c_2\right] \frac{1}{9 M^2} 
\left[
	11 \me^2 + 20 \Ee E_0 
	- 2\frac{\me^2}{\Ee}E_0
	- 20\Ee^2
\right]
- 2 \frac{E_0}{3M} \Re\left[ c_1^*(c_1 + d \pm b)\right]
\\
+ \frac{2\Ee}{3M} 
\left( 
	3 \left| a_1 \right|^2 + \Re \left[ c_1^*(5c_1 \pm 2 b) \right]
\right)
- \frac{\me^2}{3 M \Ee} 
\Re \left[ 
	-3 a_1^*e + c_1^*\left(2c_1 + d \pm 2b - h\frac{E_0 - \Ee}{2M} \right)
\right]
\end{multline}
%\unskip
%% !TEX root = ../thesis_main.tex
%
%
%
%%%% --- * --- %%%%	
\begin{align}
F_1(\Ee) = & \:\:
\deltauv \left( \frac{u}{\!u+1\!} \right)^{\!\!1/2} \!\!
\left\{
	2 \Re\left[ 
		a_1^*\left(\! c_1 - \frac{E_0}{3M}(c_1 + d \pm b) + \frac{\Ee}{3M}( 7 c_1 \pm b + d )\! \right)
	\right]
	\right. \nonumber \\ & \left. 
	+
	2 \Re\left[
		a_1^* c_2 + c_1^* a_2
	\right] \!
	\left(
		\frac{4 \E(E_0 - \E) + 3 \me^2}{3M^2}
	\right) \!
\right\}
\nonumber \\ &
\mp \frac{ (-1)^s \gammauv}{u+1} 
\Re \left\{ \!
	c_1^* \! \left(
		c_1 + 2 c_2 \left(\frac{8\Ee(E_0-\Ee)+3\me^2}{3M^2}\right)
		- \frac{2 E_0}{3 M} (c_1 + d \pm b) 
		\right.\right. \nonumber \\ & \left.\left.
		+ \frac{\Ee}{3M} (11 c_1 - d \pm 5 b)
	\right)
\right\}
%\nonumber \\ &
+ 
\frac{\lambdauv}{u+1}
\Re \left\{ \!
	c_1^* \! \left[
		- f \left(\frac{5\Ee}{M}\right)
		\right. \right. \nonumber \\ & \left. \left.
		+ g \left( \frac{3}{2} \right)^{\!\!1/2} \!
		\left(
			\frac{E_0^2 - 11 E_0 \Ee + 6 \me^2 + 4\Ee^2 }{6M^2}
		\right) 
		%\right.\right. \nonumber \\ & \left.\left.
		\pm 3 j_2 
		\left(
			\frac{8 \Ee^2 - 5E_0 \Ee - 3 \me^2}{6M^2}
		\right)
	\right] \!
\right\}
\end{align}
%\unskip
%% !TEX root = ../thesis_main.tex
%
%
%
%%%% --- * --- %%%%	
\begin{flalign}
F_2(\Ee) = &
\thetauv \frac{\Ee}{2M} 
\Re\left[
	c_1^*\left(
		c_1 + c_2 \frac{8(E_0-\Ee)}{3M}
		-d \pm b
	\right)
\right]
&& \nonumber \\ &
- \deltauv \frac{\Ee}{M} 
\left[ \frac{u(u+1)}{(2u-1)(2u+3)} \right]^{1/2} \!
\Re \left\{ \!
	a_1^*\left( 
		\left( \! \frac{3}{2} \right)^{\!\!1/2}\!\! f
		+ g \frac{\Ee+2E_0}{4M} 
		\right.\right.
		&& \nonumber \\ &
		\left.\left.
		\pm \left( \frac{3}{2} \right)^{\!\!1/2}\!\! j_2 \frac{E_0-\E}{2M}
	\right) \!
\right\}
+ (-1)^s \, \kappauv \frac{\E}{2M}
\Re \left[
	c_1^* \! \left(
		\pm \, 3 f 
		\pm \left( \frac{3}{2} \right)^{\!\!1/2}\!\! g \frac{E_0-\Ee}{M}
		\right.\right.
		&& \nonumber \\ &
		\left.\left.
		+ 3 j_2 \frac{E_0-2\Ee}{2M}
	\right)
\right]
+ \epsilonuv \Re\left[ c_1^* j_3 \right]
\left( 
	\frac{21 \E^2}{8 M^2}
\right) &&
\end{flalign}
%\unskip
%% !TEX root = ../thesis_main.tex
%
%
%
%%%% --- * --- %%%%	
\begin{flalign}
F_3(\Ee) = &
- \deltauv \, (3 u^2 + 3 u -1)
\left[
	\frac{u}{(u-1)(u+1)(u+2)(2u-1)(2u+3)}
\right]^{1/2}\!
&& \nonumber \\ &
\times 
\Re \left[
	a_1^* j_3
\right]
\left( 
	\frac{\Ee^2 \sqrt{15} }{4M^2}
\right)
+ 
\frac{\rhouv}{u+1} 
\Re \left[
	c_1^*(g\sqrt{3} + j_2\sqrt{2})
	\left(
		\frac{5 \Ee^2}{4 M^2}
	\right)
\right]
&& \nonumber \\ &
\pm
\frac{(-1)^s \sigmauv }{u+1} 
\Re \left[ c_1^*j_3 \right] 
\left( 
	\frac{5\Ee^2}{2 M^2}
\right) &&
\label{equation:holstein_F3}
\end{flalign}
%\unskip
%...Phew!  I typed all of that out just so that I can have a record of what's going on here, but actually, the very first thing I want to do is make some simplifications here.  


% %%%%%%%%%%%%%%%%%%%%%%%%%%%%%% 
% \subsubsection[Algebra]{Algebra}
% Oookay.  Here's some stuff I'll want to keep track of now, but will also want to not have cluttering my documents later.
% \begin{align}
% \deltauv \left( \frac{u}{u+1} \right)^{\!\!1/2} \;\;&=\;\; \left( \frac{3}{5} \right)^{1/2}
% \\
% \mp \frac{ (-1)^s \gammauv}{u+1} \;\;&=\;\; \frac{2}{5}
% \\
% \frac{\lambdauv}{u+1} \;\;&=\;\; \frac{4 \sqrt{3}}{5}
% \\
% - \deltauv \left[ \frac{u(u+1)}{(2u-1)(2u+3)} \right]^{1/2}  \;\;&=\;\;  \frac{-\sqrt{5}}{4}
% \\
% %(-1)^s \kappauv \;\;&=\;\; \frac{3}{\sqrt{2}}  % I math-ed this wrong before.  I think.
% (-1)^s \kappauv \;\;&=\;\; \frac{1}{2\sqrt{2}}
% \\
% \frac{21}{8} \epsilonuv \;\;&=\;\; \frac{-21}{16 \sqrt{35}} %\;\;=\;\;
% \\
% \!\!\!\! \!\!\!
% - \deltauv (3 u^2 + 3 u -1)
% \left[
% 	\frac{u}{(u-1)(u+1)(u+2)(2u-1)(2u+3)}
% \right]^{1/2}\!
% \left(
% 	\frac{\sqrt{15}}{4}
% \right)
% \;\;&=\;\; \frac{-41 \sqrt{3}}{16 \sqrt{7}}
% \\
% \frac{\rhouv}{u+1}\left( \frac{5}{4} \right) \;\;&=\;\; \frac{41}{80}
% \\
% \pm
% \frac{(-1)^s \sigmauv }{u+1} \left( \frac{5}{2} \right) \;\;&=\;\; \frac{-41}{4\sqrt{35}}
% \end{align}
% %%%%%%%%%%%% Probably check the math on all of these over again.
% Note:  I calculated $\kappa_{u,v}$ wrong before.  Probably.  Unclear about the order of operations in really old papers.
%
% \subsubsection[More Algebra]{More Algebra}
% So here are some simplifications, using those expressions we just calculated in the stupid section above.
% %
% \begin{multline}
% F_0(\Ee) =
%   \left| a_1 \right|^2
% + \left( \!
% 	\frac{2\Ee}{M}
% \! \right)
% \left| a_1 \right|^2
% + \left| c_1 \right|^2
% + \left( \!
% 	\frac{- 2 E_0 + 10\Ee - 2\me^2/\Ee}{3M}
% \! \right)
% \left| c_1 \right|^2
% \\
% %
% + \frac{\me^2}{3 M \Ee}
% \left(
% 	3 Re\left[a_1^*e\right] - Re\left[c_1^*d\right] + 2 Re\left[c_1^*b\right]
% \right)
% + \frac{-4\Ee}{3M} Re \left[c_1^* b \right]
% + \frac{- 2 E_0}{3M} \left( Re\left[c_1^* d\right] - Re\left[c_1^* b \right] \vphantom{2_2^2} \right)
% \\
% + \frac{2}{3 M^2}
% \left(
% 	\m^2 + 4 \Ee E_0 + 2 \me^2\frac{E_0}{\Ee} - 4\Ee^2
% \right)
% Re\left[ a_1^* a_2 \right]
% \\
% + \frac{2}{9 M^2}
% \left(
% 	11 \me^2 + 20 \Ee E_0
% 	- 2\me^2\frac{E_0}{\Ee}
% 	- 20\Ee^2
% \right)
% Re\left[ c_1^* c_2\right]
% +
% \frac{\me^2}{6 M^2} \left(\frac{E_0 - \Ee}{\Ee} \right) Re\left[c_1^*h\right]
% %%%%%%% simplified enough for now.  continue it later.
% \end{multline}
% So that one's looking pretty good.  I can probably simplify it further later.
% %
% \begin{multline}
% F_1(\Ee) =
% \frac{2\sqrt{3}}{\sqrt{5}} Re \left[ a_1^*c_1 \right]
% + \frac{2}{5} \left| c_1 \right|^2
% %\\
% +
% \frac{2}{\sqrt{15}} \left( \frac{7\Ee -E_0}{M} \right)
% 	Re \left[ a_1^*c_1 \right]
% +
% \frac{2}{15}
% \left( \frac{11 \E- 2 E_0}{M} \right)
% 	\left| c_1 \right|^2
% \\
% +
% \frac{2}{\sqrt{15}} \left( \frac{E_0 - \Ee}{M} \right)
% 	Re \left[ a_1^*b \right]
% +
% \frac{2}{\sqrt{15}} \left( \frac{\Ee - E_0}{M} \right)
% 	Re \left[ a_1^*d \right]
% \\
% +
% \frac{2}{15}
% \left( \frac{2 E_0 - 5 \E}{M} \right)
% 	Re \left[c_1^*b \right]
% +
% \frac{2}{15}
% \left( \frac{-2 E_0 - \Ee }{M} \right)
% 	Re \left[c_1^*d \right]
% \\
% -
% \frac{20\sqrt{3}}{5} \left(\frac{\Ee}{M}\right)
% 	Re \left[ c_1^* f \right]
% \\
% +
% \frac{2}{\sqrt{15}} \left( \frac{4 \E(E_0 - \E) + 3 \me}{M^2} \right) \!
% 	Re\left[ a_1^* c_2 + c_1^* a_2 \right]
% +
% \frac{4}{15} \left(\frac{8\Ee(E_0-\Ee)+3\me}{M^2}\right)
% 	Re \left[ c_1^*c_2 \right]
% \\
% +
% \left(\frac{12}{30 \sqrt{2} }\right) \left( \frac{E_0^2 - 11 E_0 \Ee + 6 \me^2 + 4\Ee^2 }{M^2} \right)
% 	Re \left[ c_1^* g \right]
% +
% \left( \frac{-12\sqrt{3}}{30}\right) \left( \frac{8 \Ee^2 - 5E_0 \Ee - 3 \me^2}{M^2} \right)
% 	Re \left[c_1^* j_2 \right]
% \end{multline}
% $F_1(\Ee)$ is definitely looking better than it was before.  Onwards!
% %
% \begin{multline}
% F_2(\Ee) =
% \left( \frac{\Ee}{M} \right)
% \left[
% 	\frac{1}{2} \left| c_1 \right|^2
% 	- \frac{1}{2} Re\left[ c_1^*d \right]
% 	- \frac{1}{2} Re\left[ c_1^* b \right]
% 	+
% 	\frac{-\sqrt{15}}{4\sqrt{2}}
% 	Re \left[ a_1^* f \right]
% 	+
% 	\frac{-3}{4\sqrt{2}}
% 		Re \left[ c_1^* f \right]
% \right]
% \\
% +
% \frac{1}{M^2}
% \left\{
% 	\frac{4}{3} \left[ \Ee(E_0-\Ee) \right] Re\left[ c_1^*c_2 \right]
% 	+
% 	\frac{-\sqrt{5}}{16} \left[ \Ee(\Ee+2E_0) \right] Re \left[ a_1^* g \right]
% 	+
% 	\frac{\sqrt{15}}{8\sqrt{2}} \left[ \Ee(E_0-\E) \right] Re \left[ a_1^* j_2 \right]
% \right.
% \\
% \left.
% 	+
% 	\frac{-\sqrt{3}}{8} \left[ \Ee(E_0-\Ee) \right] Re \left[ c_1^* g \right]
% 	+
% 	\frac{3}{8\sqrt{2}}\left[ \Ee(E_0-2\Ee) \right] Re \left[ c_1^* j_2 \right]
% 	+
% 	\frac{-21}{16\sqrt{35}} \left( \E^2 \right) Re \left[ c_1^* j_3 \right]
% \right\}
% \end{multline}
% %
% And that's good enough for now for $F_2(\Ee)$, even though it's not really very good.
% \bea
% F_3(\Ee) &=&
% \left( \frac{\Ee^2}{M^2} \right)
% \left[
% \frac{-41\sqrt{3}}{16\sqrt{7}}
% 	Re \left[ a_1^* j_3 \right]
% +
% \frac{41\sqrt{3}}{80}
% Re\left[
% 	c_1^* g
% \right]
% 	+
% \frac{41\sqrt{2}}{80}
% Re\left[
% 	c_1^* j_2
% \right]
% + \frac{-41}{4\sqrt{35}}
% 	Re \left[ c_1^*j_3 \right]
% \right]
% \eea
% To be honest, $F_3(\Ee)$ came out a lot cleaner than I was expecting.  That's pretty nice.
%
% %
% % \begin{multline}
% % F_2(\Ee) =
% % \frac{-9}{\sqrt{2}} Re \left[c_1^* f \right]
% % +
% % \frac{\Ee}{2M}
% % Re\left[
% % 	c_1^*c_1
% % \right]
% % +
% % \frac{8\Ee(E_0-\Ee)}{6M^2} Re\left[ c_1^*c_2 \right]
% % +
% % \frac{-\Ee}{2M}
% % Re\left[
% % 	c_1^*d + c_1^*b
% % \right]
% % \\
% % + \frac{-\sqrt{15}}{4 \sqrt{2}}
% % \left( \frac{\Ee}{M} \right)
% % 	Re \left[ a_1^*f \right]
% % +
% % \frac{-\sqrt{5}}{16}
% % \left( \frac{\Ee(\Ee+2E_0)}{M^2} \right)
% % 	Re \left[ a_1^*g \right]
% % +
% % \frac{\sqrt{15}}{8\sqrt{2}}
% % \left( \frac{\Ee(E_0-\E)}{M^2} \right)
% % 	Re \left[ a_1^*j_2 \right]
% % \\
% % + \frac{-3 \sqrt{3}}{2} \left(\frac{E_0-\Ee}{M} \right)
% % 	Re \left[ c_1^* g \right]
% % + \frac{9}{\sqrt{2}} \left( \frac{E_0-2\Ee}{2M} \right)
% % 	Re \left[ c_1^* j_2 \right]
% % +
% % \frac{-21}{16\sqrt{35}}
% % \left(
% % 	\frac{\E^2}{M^2}
% % \right)
% % Re\left[ c_1^* j_3 \right]
% % \end{multline}
% % %
% % I'll leave $F_2(\Ee)$ for now.  Onwards!
%
%
%
% Let's see what they are in the limit of infinite recoil mass!  (I think that corresponds to $M$ in the equations, but I should really double check that...)
% \bea
% F_0(\Ee) |_{M\rightarrow\infty} &=& \left| a_1 \right|^2 + \left| c_1 \right|^2 \\
% F_1(\Ee) |_{M\rightarrow\infty} &=& \frac{2\sqrt{3}}{\sqrt{5}} Re \left[ a_1^*c_1 \right]+ \frac{2}{5} \left| c_1 \right|^2 \\
% F_2(\Ee) |_{M\rightarrow\infty} &=& 0 \\ %\frac{-9}{\sqrt{2}} Re \left[c_1^* f \right] \\
% F_3(\Ee) |_{M\rightarrow\infty} &=& 0
% \eea
%
% Anyway, let's expand our Eq.~(\ref{eq:holsteinpdf}) in terms of $\Lambda_i$s and things.  Using Eqs.~(\ref{eq:nequalsj}, \ref{eq:lambda1}, \ref{eq:lambda2}, \ref{eq:lambda3}) and changing the labelling on our quantization axis to $\hatj$:
% \bea
% \d^3 \Gamma &=& 2  G_v^2 \cos^2\theta_c \frac{\FF}{(2\pi)^4} \, \pe \Ee (E_0 - \Ee)^2 \dEe \, \dOmegae
% \nonumber\\
% && \times
% \left\{
% 	F_0(\E)
% 	+
% 	\left(\! \LambdaOne \!\right)
% 	\left[ \hatj \cdot \hatp \right] \left(\! \frac{\pe}{\Ee} \!\right)
% 	F_1(\E)
% 	+
% 	\left(\! \LambdaTwo \!\right)
% 	\left[ \left( \jhat \cdot \hatp \right)^2 - \frac{1}{3} \right] \left(\! \frac{ \pe^2 }{\Ee^2} \! \right)%^2
% 	F_2(\E)
% 	\right. \nonumber\\ && \left.
% 	+
% 	\left(\! \LambdaThree \!\right)
% 	\left[
% 		\left( \hatj \cdot \hatp \right)^{3}
% 		-
% 		\frac{3}{5} \left( \hatj \cdot \hatp \right)
% 	\right]
% 	\left(\! \frac{\pe^3}{\Ee^3} \! \right)
% 	F_3(\E)
% \right\}
% \eea
%
%


%%%% --- * --- %%%%	
%\section{Qualitative Parameter Descriptions and Convention Compatibility}
\section{Combining Formalisms}
\label{sec:combo_formalism}
To combine the two formalisms, we begin by comparing individual terms within \ac{JTW}'s integrated \ac{PDF} (Eq.~\ref{equation:integrated_jtw}) and Holstein's comparable \ac{PDF} (Eq.~\ref{equation:holstein52}).  We find:
\bea
\xi &\approx& G_v^2 \cos^2\theta_c \,\, F_0(\Ebeta)  \label{eq:jtw_holstein_xi}
\\
\Abeta \xi &\approx& G_v^2 \cos^2\theta_c \,\, F_1(\Ebeta), \label{eq:jtw_holstein_Abetaxi}
\eea
where the equality is exact within certain limits as described below.

With these relationships established, we can proceed to compare individual terms within each of the above expressions.  Recalling that \ac{JTW} retains fewer expansion terms than Holstein, and also neglects the smaller nuclear structure functions entirely, it is clear that many terms from Holstein simply have no equivalent within \ac{JTW}.  In fact, of the Holstein structure functions $a_1, a_2, b, c_1, c_2, d, e, f, g, h, j_2$, and $j_3$, only $a_1$ and $c_1$ are represented within \ac{JTW}.  Holstein has made it clear that $a_1$ and $c_1$ are related, respectively, to the Fermi (vector) and Gamow-Teller (axial) couplings.  However, \ac{JTW} uses more parameters to describe each of these:  the vector coupling is parameterized by $M_F$, $C_V$, and $C_V^\prime$, and the axial coupling by $M_{GT}$, $C_A$, and $C_A^\prime$.

To properly compare the Holstein and JTW formalisms, Eqs.~(\ref{eq:jtw_holstein_xi}-\ref{eq:jtw_holstein_Abetaxi}) must be evaluated with $C_S \!=\! C_S^\prime \!=\! C_T \!=\! C_T^\prime = 0$, and $a_2 \!=\! b \!=\! c_2 \!=\! d \!=\! e \!=\! f \!=\! g \!=\! h \!=\! j_2 \!=\! j_3 \!=\! 0$, and $M = \infty$.  We find, from Eq.~(\ref{eq:jtw_holstein_xi}):
\bea
| a_1 |^2 \,\,\, G_v^2 \cos^2\theta_c &=& | M_F |^2 \, (|C_V|^2 + |C_V^\prime|^2 )  
\label{eq:holstein_a1_from_xi} 
\\
| c_1 |^2 \,\,\, G_v^2 \cos^2\theta_c &=& |M_{GT}|^2 (|C_A|^2 + |C_A^\prime|^2 ) 
\label{eq:holstein_c1_from_xi}, 
\eea
and treating Eq.~(\ref{eq:jtw_holstein_Abetaxi}) in a similar manner, 
\bea
| c_1 |^2 \,\,\, G_v^2 \cos^2\theta_c &=& 2 \, |M_{GT}|^2 \Re \left[ C_A C_A^{\prime *} \right]
\label{eq:holstein_a1_from_Abeta} 
\\
\Re \left[ a_1^* c_1 \right] \,\,\, G_v^2 \cos^2\theta_c &=& - M_F \, M_{GT} \, \Re \left[ C_V C_A^{\prime *} + C_V^\prime C_A^* \right]
\label{eq:holstein_a1c1}
\eea

At this point, given that we are working with complex coupling constants, it becomes clear that there may not be a uniquely defined relationship between the two formalisms' coupling constants.  Therefore, we will proceed by simply picking a convention and checking that it is self-consistent and produces the physical behaviour we expect.  

Because we expect Holstein to use only the physically observed left-handed couplings, and because we are not presently \emph{searching} for right-handed couplings, we will enforce left-handedness within \ac{JTW}'s description as well.  In particular, for left-handed couplings, we have~\cite{Falkowski2021}:
\bea
C_V &=& C_V^\prime 
\label{eq:cVcVprime} \\
C_A &=& C_A^\prime.
\label{eq:cAcAprime}
\eea
\note{Aaaaand here's where I start possibly getting signs of things wrong.}
One possible convention for the relationship between the two formalisms which is consistent with the constraints described above is:
\bea
a_1 &=& \frac{M_F}     {G_v \cos\theta_c} \, \frac{1}{\sqrt{2}} \, (C_V + C_V^\prime )  
\label{eq:a1_def}
\\
c_1 &=& \frac{M_{GT}}{G_v \cos\theta_c} \, \frac{-1}{\sqrt{2}} \, (C_A + C_A^\prime ). 
\label{eq:c1_def}
\eea
We note that in the above expressions, the terms ${G_v \cos\theta_c}$, $(C_V + C_V^\prime )$, and $(C_A + C_A^\prime )$ are universally applicable, while the portion of the couplings dependent on the structure of the nucleus resides entirely within the nuclear matrix elements $M_F$ and $M_{GT}$.
%\note{}

For convenience, we define the following left-handed couplings for vectors, axial vectors, scalars, and tensors:
\bea
g_V &=& \frac{1}{\sqrt{2}}  \, (C_V + C_V^\prime )  \;\; = \;\; +\, 1.0
\label{eq:gV_def}
\\
g_A &=& \frac{-1}{\sqrt{2}} \, (C_A + C_A^\prime )  \;\; \approx \;\; +\, 0.91210
\label{eq:gA_def}
\\
g_S &=& \frac{1}{\sqrt{2}} \, (C_S + C_S^\prime ) 
\label{eq:gS_def}
\\
g_T &=& \frac{-1}{\sqrt{2}} \, (C_T + C_T^\prime), 
\label{eq:gT_def}
\eea
where, 
\aside{In my notes, $g_T$ has a minus sign in front of it.  Is there really any reason to keep that?  I guess probably not...\\
Also in my notes, $g_A$ and $c_1$ have a minus sign in front of them too.  I think it's superfluous.  Recall that $M_{GT}$ is negative.   } 
in addition to the requirements of Eqs.~(\ref{eq:cVcVprime})-(\ref{eq:cAcAprime}) we will henceforth consider only the left-handed scalar and tensor couplings, so that 
\bea
C_S &=& C_S^\prime 
\label{eq:cScSprime} \\
C_T &=& C_T^\prime.
\label{eq:cTcTprime}
\eea
Furthermore, we will require that all $C_X$ and $C_X^\prime$ are \emph{real}.  By enforcing this requirement, we lose the ability to describe a violation of time reversal symmetry, an exotic behaviour that is not the focus of the present search.  
\note{Also, maybe I should at some point mention that my $g_X$ aren't the same thing that Holstein uses with that notation?  Ugh.}
In order to produce the correct physical behaviour in the decay of $^{37}$K, we also require, for the Gamow-Teller matrix elements in Holstein and \ac{JTW}:
\bea
M_{GT, \textrm{\,Holstein}} &=& \pm M_{GT, \textrm{\,JTW}} \;\; \approx \;\; -\, 0.62376
\label{eq:jtw_holstein_sign}
\eea
%
\note[bluetodo, nolist]{Pretty sure I resolved it.  I think.  Now, with this fix (implemented in Eq.~(\ref{eq:jtw_Abetaxi_integrated_LH}), so don't fucking change it now), we have:
\bea
M_{GT, \textrm{\,Holstein}} &=& + M_{GT, \textrm{\,JTW}} \;\; \approx \;\; -\, 0.62376
\eea
Now, Eqs.~(\ref{eq:jtw_holstein_sign},\ref{eq:jtw_Abetaxi_integrated_LH}, \ref{eq:definerho},  and \ref{eq:Awithrho}) are all consistent.  Yay!
\\
Eq.~(\ref{bFierzwithrho}) is also consistent.  
}
%
The sign ambiguity can be attributed to a sign ambiguity in the original \ac{JTW} publications.  The result is that the literature can't seem to agree on a single sign convention, or even whether the convention ought to be the same in the cases of $\beta^-$/$\beta^+$ decay.  See, for example, Refs~\cite{dan_thesis}\cite{GCBall2000}\cite{Raman1978}.  The important point is that, for whichever convention is in use, the equations result in broadly correct physical behaviour.  In the case of $^{37}$K, we can use the fact that the overall sign of $\Abeta$ \emph{must} be negative within our mathematical expressions, because it is experimentally measured to be negative.  

We are now in a position to write out Eqs.~(\ref{eq:jtw_xi_integrated}), (\ref{eq:jtw_bxi_integrated}), and (\ref{eq:jtw_Abetaxi_integrated}) in terms of only real, left-handed couplings.  Doing so, we note that if the couplings are required to be real, then $\bFierz$ is \emph{only} sensitive to left-handed scalar or tensor couplings, and within $\Abeta$, the requirement is only that the scalar and tensor couplings must have \emph{the same} handedness.

The resulting expressions are:
% !TEX root = ../thesis_main.tex
%
%
%
%%%% --- * --- %%%%	
\begin{eqnarray}
\xi &=& 
	|M_F|^2    \left( |g_S|^2 + |g_V|^2 \right) \,+\, 
	|M_{GT}|^2 \left( |g_T|^2 + |g_A|^2 \right)
	\label{eq:jtw_xi_integrated_LH} 
	\\
\bFierz \, \xi &=& 
	- \: 2 \gamma \,
	\left( |M_F|^2 \, g_V \,g_S  \;+\; |M_{GT}|^2 \, g_A \, g_T \right) 
	\label{eq:jtw_bxi_integrated_LH} 
	\\
\Abeta \, \xi &=& 
	\frac{2}{5} \, |M_{GT}|^2 \left( g_A^2 - g_T^2 \right) + 
	2 \left( \frac{3}{5} \right)^{\!\! 1/2} \!
	M_F\,M_{GT, \textrm{\,JTW}} \left(\, g_S \, g_T - g_V \, g_A \phantom{g_A^2 \!\!\!\! \!\! } \right) 
	\;\;\;\;
	\label{eq:jtw_Abetaxi_integrated_LH}
\end{eqnarray}
%
%  eq:jtw_xi_integrated_LH
%  eq:jtw_bxi_integrated_LH
%  eq:jtw_Abetaxi_integrated_LH

%\note[bluetodo]{Pretty sure I fucked up the sign of $M_{GT}$, again.  
%\\...\\
%if $C_A$ is negative and $g_A$ is positive, like I think is the case in the simulation, then Eq.~(\ref{eq:jtw_Abetaxi_integrated_LH}) gets the wrong sign with the JTW $M_{GT}$.
%\\...\\
%Eq.~(\ref{eq:jtw_Abetaxi_integrated_LH}) I think doesn't follow the sign convention of jtw\_coulomb.   ....fixed.
%}

We will take this opportunity to define and utilize some standard nuclear physics notation:
\bea
\rho &:=& \frac{C_A M_{GT}}{C_V M_F} \;\; = \;\; \frac{- g_A M_{GT}}{g_V M_F}
\label{eq:definerho}
\eea
Since each specific transition may, as a result of the nuclear structure relationships involved, take different values for the matrix elements $M_F$ and $M_{GT}$, a particular transition can often be described by the ratio, $\rho$, of the Gamow-Teller and Fermi couplings specific to it.  This notation offers a cleaner way to characterize standard model predictions of the observables in Eqs.~(\ref{eq:jtw_xi})-(\ref{eq:jtw_DTRxi}), but can become somewhat inelegant when used within a description of \ac{BSM} physics.  
%In particular,
Taking \emph{only} the leading order (linear) terms in exotic couplings, we find:
\bea
\xi &=& g_V^2 | M_F |^2 \left( 1 + \rho^2 \right)
\label{eq:xiwithrho} 
\\
\bFierz &=& \frac{-2\gamma}{1 + \rho^2} \left( \frac{g_S}{g_V} + \rho^2 \frac{g_T}{g_A} \right) 
\label{bFierzwithrho}
\\
\Abeta &=& \frac{\frac{2}{5} \rho^2 - 2 \rho \sqrt{\frac{3}{5}}  }{1 + \rho^2}, 
\label{eq:Awithrho}
\eea
where, for $^{37}$K, Eq.~(\ref{eq:definerho}) must produce a positive value for $\rho$ in order to be compatible with both Eq.~(\ref{eq:Awithrho}) and the physically observed results.









