% !TEX root = ../thesis_main.tex



%%%% --- * --- %%%%	
\chapter[A PDF]{A PDF For The People}
\label{appendix_forthepeople}

\section[JTW]{JTW}
Here's a master equation from JTW to describe beta decay kinematics~\cite{jtw},~\cite{jtw_coulomb}:

% !TEX root = ../thesis_main.tex



% "A PDF for the People"
\bea
\omega(\cdots) \!\!\!\! \!\!\!\! \!\!\!\! \!\!\!\! && \,\,\,\, \,\,\,\, \mathrm{d} \E \, \dOmegae \, \dOmeganu 
\,\, = \,\, \frac{\FF}{(2\pi)^5} \, \pe \Ee (E_0 - \Ee)^2 \dEe \, \dOmegae \, \dOmeganu \, \nonumber\\ 
&&	\times \,\, \xi \left[
	1 + \a \frac{\vecpe\cdot\vecpnu}{\Ee\Enu} + \bFierz \frac{\m c^2}{\Ee} 
%	&& 
    + \,\,  \calign \,\, \Talign(\vecJ) 
	\left(
		\frac{\vecpe \cdot \vecpnu}{3\Ee\Enu}
		- \frac{ (\vecpe\cdot \hatj) (\vecpnu\cdot\hatj) }{\Ee\Enu}
	\right)
	\!
	%\left(
	%	\TalignExpand
	%\right)
\right. \nonumber\\ 
&&	\left. + 
	 \frac{\vecJ}{J} \cdot
	\left(
		\A \frac{\vecpe}{\Ee} 
		+ \B \frac{\vecpnu}{\Enu} 
		+ \D \frac{\vecpe \times \vecpnu}{\Ee\Enu} 
	\right)
\right]
\label{equation:jtw_master}
\eea

where, for convenience, we have defined a nuclear alignment term,
\bea
\Talign(\vecJ) &\equiv& \TalignExpand.
\eea
\aside{We have already specialized to $\beta^+$ decay.}

Note that this master equation depends on neutrino momentum, which we cannot observe directly.  Furthermore, we cannot reconstruct neutrino momenta in our decay events either, because it would be necessary to account for the momentum of the recoiling daughter nucleus, treating the decay as a three-body problem.  From an experimental standpoint, we failed to measure the momenta of the daughters in conjunction with the ``tagged'' beta decay events with which we are primarily concerned in this thesis.  From a theoretical standpoint, JTW has intentionally neglected recoil-order terms -- meaning that the daughter nucleus is treated, for the purpose of kinetic energy calculations, as being infinitely massive, and as such it must have no change in kinetic energy from the decay.  This approximation makes it a bit tricky to correctly re-formulate Eq.~(\ref{equation:jtw_master}) in terms of the momentum of the daughter instead of the momentum of the neutrino.  

Fortunately, it is possible to simplify Eq.~(\ref{equation:jtw_master}) by integrating over all possible neutrino directions, such that the result no longer depends on parameters that we do not observe.  The neutrino energy itself is not a free variable in this equation, because the energy release in the decay is fixed, and given the approximation that none of that energy is allocated to the recoiling daughter, it is very straightforward to calculate the neutrino energy for a decay event in which the beta energy is known.

%We haven't integrated out the neutrino momentum.  Neutrino energy itself is a redundant parameter, I think, because we are already using an endpoint energy and a beta energy, and we are not taking recoil-order effects into account.  Also, we treat neutrinos as massless here, which is a perfectly reasonable approximation for our purposes.  For ``convenience'', let's define a nuclear alignment term, $\Talign$, so that:


The result of performing this integration over neutrino direction is:
% !TEX root = ../thesis_main.tex
%
%
%
% The JTW Proto-Master
\bea
	\textrm{d}^3 \Gamma \dEe \, \dOmegae
	&=& 
	\frac{2}{(2\pi)^4} \, \FF \, \pe \Ee (E_0 - \Ee)^2 \, \dEe \, \dOmegae \, \xi \nonumber\\ 
	&& \times \left[
		1 + \bFierz \frac{\m c^2}{\Ee} + 
		\A  
		\left(
			\frac{\vecJ}{J} \cdot \frac{\vecpe}{\Ee} 
		\right) 
	\right],
\label{equation:integrated_jtw}
\eea
%\unskip which is a great simplification on Eq.~(\ref{equation:jtw_master}).  We still must write the remaining parameters in terms of the relevant nuclear matrix elements and fundamental coupling constants.  These coupling constants are, in general, complex-valued, and JTW does not choose a phase angle for us.  We write them out in 
Eqs.~(\ref{eq:jtw_xi}-\ref{eq:jtw_Abetaxi}).
% !TEX root = ../thesis_main.tex
%
%
%
%%%% --- * --- %%%%	
\begin{eqnarray}
    \xi &=& 
    	|M_F|^2    \left( |C_S|^2 + |C_V|^2 + |C_S^\prime|^2 + |C_V^\prime|^2 \right) 
		\nonumber \\ && + \;\; 
		|M_{GT}|^2 \left( |C_T|^2 + |C_A|^2 + |C_T^\prime|^2 + |C_A^\prime|^2 \right)
	\label{eq:jtw_xi} \\
    \bFierz \, \xi &=& \pm \: 2\gamma \, \textrm{Re}\!\left[ |M_F|^2 \left( C_S C_V^* + C_S^\prime C_V^{\prime *} \right) + |M_{GT}|^2 \left( C_T C_A^* + C_T^{\prime} C_A^{\prime *} \right) \right] 
    \label{eq:jtw_bxi} \\
    \Abeta \, \xi &=& |M_{GT}|^2 \lambda_{J^\prime J} \left[ \pm 2 \textrm{Re}\!\left[ C_T C_T^{\prime *} - C_A C_A^{\prime *} \right] + 2 \frac{\alpha Z \me }{\pbeta} \,\textrm{Im}\!\left[ C_T C_A^{\prime *} + C_T^\prime C_A^* \right] \right] 
		\nonumber \\ && + \;\; 
		\delta_{J^\prime J} \, |M_F| |M_{GT}| \left( \frac{J}{J+1} \right)^{\!\!\! 1/2} \left[ \phantom{\frac{1}{1}\!\!\!} 2 \,\textrm{Re} \! \left[ C_S C_T^{\prime *} +  C_S^\prime C_T^* - C_V C_A^{\prime *} - C_V^\prime C_A^* \right] 
		\right.
		\nonumber \\ && \pm \;\;
		\left.
		2 \frac{\alpha Z \me }{\pbeta} \,\textrm{Im}\!\left[ C_S C_A^{\prime *} + C_S^\prime C_A^* - C_V C_T^{\prime *} -C_V^\prime C_T^* \right] \right]
	\label{eq:jtw_Abetaxi}
\end{eqnarray}
%
%
Note that JTW presents slightly different expressions for the sign convention in components of $\Abeta$ within~\cite{jtw} and~\cite{jtw_coulomb}.  Here, we adopt the convention from the latter publication.  Furthermore, we do not require that either $M_F$ or $M_{GT}$ be positive (which would allow us to safely drop their absolute value indicators and make the conventions of these two papers equivalent).  In order to obtain the correct, physically observed value for $\Abeta$, we require that the 
$M_{F}\,M_{GT}$ term in Eq.~(\ref{eq:jtw_Abetaxi}) have an overall positive value.  Because we know that the scalar and tensor couplings must be small, and any imaginary contributions to the term must be small, we conclude that
\bea
	M_{F}\,M_{GT} \left( C_V C_A^{\prime *} + C_V^\prime C_A^* \right) < 0.
\eea

%\note{In order to make JTW give us the correct, physically observed value for $\Abeta$, we need ... something.  JTW writes an expression for $\Abeta$ with slightly different sign convention in~\cite{jtw} and~\cite{jtw_coulomb}.  There is some subtlety involved in getting the correct signs on things, but the easiest way to figure it out is to make sure the calculation of the value of $\Abeta$ matches reality.  To this end, ... either for both $M_{GT}$ and $M_{F}$ to be positive, or we use the notation for $\Abeta$ from JTW's earlier paper rather than the one with the coulomb corrections.  Somehow, we run into problems getting JTW to agree with Holstein if we require $M_{GT}$ and $M_{F}$ to have the same sign, so we're going to kludge together two different expressions for $\Abeta$ from both of JTW's papers.  Or ... wait ... maybe the opposite thing is what I should be doing?  In handwritten notes, I explicitly use the *later* JTW sign convention.}

%\note{In order to make JTW give us the correct, physically observed value for $\Abeta$, we need either for both $M_{GT}$ and $M_{F}$ to be positive, or we use the notation for $\Abeta$ from JTW's earlier paper rather than the one with the coulomb corrections.  Somehow, we run into problems getting JTW to agree with Holstein if we require $M_{GT}$ and $M_{F}$ to have the same sign, so we're going to kludge together two different expressions for $\Abeta$ from both of JTW's papers.  Or ... wait ... maybe the opposite thing is what I should be doing?  In handwritten notes, I explicitly use the *later* JTW sign convention.}

\aside{Also, $\xi = G_v^2 \, \cos\theta_C \, f_1(E).$}


\section[Holstein]{Holstein} 
Holstein~\cite{holstein}~\cite{holstein_errata} generously provides explicit equations to match both Eq.~(\ref{equation:jtw_master}) (i.e. Holstein's Eq.~(51), where neutrino direction is a parameter of the probability distribution) and Eq.~(\ref{equation:integrated_jtw}) (Holstein's Eq.~(52), where neutrino direction has already been integrated over).  

%Holstein~\cite{holstein} generously provides explicit equations to match both Eq.~\ref{equation:jtw_master} (where neutrino direction is a parameter of the probability distribution), and Eq.~\ref{equation:integrated_jtw} (where the neutrino direction has already been integrated over).

%This one is harder.  But here, we've already integrated over neutrino momentum at least.  That's something.  

Here's Holstein's Eq.~(52):
% !TEX root = ../thesis_main.tex



% "A PDF for the People"
\bea
\mathrm{d}^3 \Gamma &=& 2  G_v^2 \cos^2\theta_c \frac{\FF}{(2\pi)^4} \, \pe \Ee (E_0 - \Ee)^2 \dEe \, \dOmegae 
\nonumber\\
&& \times
\left\{
	F_0(\E) 
	+ \Lambda_1 F_1(\E) \hatn \cdot \frac{\vecpe}{\Ee}
	+ \Lambda_2 F_2(\E) \left[ \left( \nhat \cdot \frac{\vecpe}{\Ee} \right)^2 - \frac{1}{3}\frac{\pe^2}{\Ee^2} \right]
	\right. \nonumber\\ && \left.
	+ \Lambda_3 F_3(\E) 
		\left[ 
			\left( \hatn \cdot \frac{\vecpe}{\Ee} \right)^3
			- \frac{3}{5}\frac{\pe^2}{\Ee^2}\hatn \cdot \frac{\vecpe}{\Ee}
		\right]
\right\}
\label{equation:holstein52}
\eea
\unskip

A careful reader will eventually note that Holstein's spectral functions $F_i(\Ee)$ are not the same as the $F_i(\Ee, u, v, s)$ in any limit, despite the notational similarities.  Among other rules, Holstein's spectral functions obey these:
\bea
	F_i(\Ee) &\neq& F_i(\Ee, u, v, s)    \\
	F_i(\Ee) &=&    H_i(\Ee, u, v, 0)    \\
	f_i(\Ee) &=&    F_i(\Ee, u, v, 0).
\eea
For the $F_i(\Ee)$ functions of interest to us here, we find the following relationships:
%though expressions for $F_i(\Ee)$ can still be obtained through a chain of substitutions: 
\begin{align}
F_0(\Ee) & = H_0(\Ee, J, J^\prime, 0) = F_1(\Ee, J, J^\prime, 0) 
	\!\!\! 
	& = &\; f_1(\Ee) 
	\nonumber \\
F_1(\Ee) & = H_1(\Ee, J, J^\prime, 0) = \textstyle F_4(\Ee, J, J^\prime, 0) \,+\! \frac{1}{3}F_7(\Ee, J, J^\prime, 0) 
	\!\!\! 
	& = &\; \textstyle f_4(\Ee) \,+\! \frac{1}{3}f_7(\Ee) 
	\nonumber \\
F_2(\Ee) & = H_2(\Ee, J, J^\prime, 0) = \textstyle F_{10}(\Ee, J, J^\prime, 0) \!+\! \frac{1}{2}F_{13}(\Ee, J, J^\prime, 0) 
	\!\!\! 
	& = &\; \textstyle f_{10}(\Ee) \!+\! \frac{1}{3}f_{13}(\Ee) 
	\nonumber \\
F_3(\Ee) & = H_3(\Ee, J, J^\prime, 0) = F_{18}(\Ee, J, J^\prime, 0) 
	\!\!\!
	& = &\; f_{18}(\Ee) .
\label{eq:holstein_FHFf}
\end{align}
\unskip  % eq:holstein_FHFf

Note that the $f_i(\Ee)$ in Eq.~\ref{eq:holstein_FHFf} are the same spectral functions used to describe a polarized decay spectrum when the neutrino (ie, the recoil) is also observed -- though of course such a spectrum must have other terms as well.  For the spectrum of interest to us here, in which the neutrino direction has already been integrated over, we can simply look up the $H_i(\Ee, J, J^\prime, 0) = H_i(E, u, v, s\!=\!0)$ spectral functions, and leave it at that.  We find:
%Here, we've taken $u=J$ and $v=J^\prime$ to be the initial and final angular momenta respectively, because apparently I'm having a hard time keeping my notation straight.
% !TEX root = ../thesis_main.tex
%
%
%
%%%% --- * --- %%%%	
\begin{multline}
F_0(\Ee) = 
\left| a_1 \right|^2 
+ 2 \Re\left[ a_1^* a_2 \right] \frac{1}{3 M^2} 
\left[  
	\m^2 + 4 \Ee E_0 + 2 \frac{\me^2}{\Ee}E_0 - 4\Ee^2
\right]
\\
+ \left| c_1 \right|^2
+ 2 \Re\left[ c_1^* c_2\right] \frac{1}{9 M^2} 
\left[
	11 \me^2 + 20 \Ee E_0 
	- 2\frac{\me^2}{\Ee}E_0
	- 20\Ee^2
\right]
- 2 \frac{E_0}{3M} \Re\left[ c_1^*(c_1 + d \pm b)\right]
\\
+ \frac{2\Ee}{3M} 
\left( 
	3 \left| a_1 \right|^2 + \Re \left[ c_1^*(5c_1 \pm 2 b) \right]
\right)
- \frac{\me^2}{3 M \Ee} 
\Re \left[ 
	-3 a_1^*e + c_1^*\left(2c_1 + d \pm 2b - h\frac{E_0 - \Ee}{2M} \right)
\right]
\end{multline}
%\unskip
% !TEX root = ../thesis_main.tex
%
%
%
%%%% --- * --- %%%%	
\begin{align}
F_1(\Ee) = & \:\:
\deltauv \left( \frac{u}{\!u+1\!} \right)^{\!\!1/2} \!\!
\left\{
	2 \Re\left[ 
		a_1^*\left(\! c_1 - \frac{E_0}{3M}(c_1 + d \pm b) + \frac{\Ee}{3M}( 7 c_1 \pm b + d )\! \right)
	\right]
	\right. \nonumber \\ & \left. 
	+
	2 \Re\left[
		a_1^* c_2 + c_1^* a_2
	\right] \!
	\left(
		\frac{4 \E(E_0 - \E) + 3 \me^2}{3M^2}
	\right) \!
\right\}
\nonumber \\ &
\mp \frac{ (-1)^s \gammauv}{u+1} 
\Re \left\{ \!
	c_1^* \! \left(
		c_1 + 2 c_2 \left(\frac{8\Ee(E_0-\Ee)+3\me^2}{3M^2}\right)
		- \frac{2 E_0}{3 M} (c_1 + d \pm b) 
		\right.\right. \nonumber \\ & \left.\left.
		+ \frac{\Ee}{3M} (11 c_1 - d \pm 5 b)
	\right)
\right\}
%\nonumber \\ &
+ 
\frac{\lambdauv}{u+1}
\Re \left\{ \!
	c_1^* \! \left[
		- f \left(\frac{5\Ee}{M}\right)
		\right. \right. \nonumber \\ & \left. \left.
		+ g \left( \frac{3}{2} \right)^{\!\!1/2} \!
		\left(
			\frac{E_0^2 - 11 E_0 \Ee + 6 \me^2 + 4\Ee^2 }{6M^2}
		\right) 
		%\right.\right. \nonumber \\ & \left.\left.
		\pm 3 j_2 
		\left(
			\frac{8 \Ee^2 - 5E_0 \Ee - 3 \me^2}{6M^2}
		\right)
	\right] \!
\right\}
\end{align}
%\unskip
% !TEX root = ../thesis_main.tex
%
%
%
%%%% --- * --- %%%%	
\begin{flalign}
F_2(\Ee) = &
\thetauv \frac{\Ee}{2M} 
\Re\left[
	c_1^*\left(
		c_1 + c_2 \frac{8(E_0-\Ee)}{3M}
		-d \pm b
	\right)
\right]
&& \nonumber \\ &
- \deltauv \frac{\Ee}{M} 
\left[ \frac{u(u+1)}{(2u-1)(2u+3)} \right]^{1/2} \!
\Re \left\{ \!
	a_1^*\left( 
		\left( \! \frac{3}{2} \right)^{\!\!1/2}\!\! f
		+ g \frac{\Ee+2E_0}{4M} 
		\right.\right.
		&& \nonumber \\ &
		\left.\left.
		\pm \left( \frac{3}{2} \right)^{\!\!1/2}\!\! j_2 \frac{E_0-\E}{2M}
	\right) \!
\right\}
+ (-1)^s \, \kappauv \frac{\E}{2M}
\Re \left[
	c_1^* \! \left(
		\pm \, 3 f 
		\pm \left( \frac{3}{2} \right)^{\!\!1/2}\!\! g \frac{E_0-\Ee}{M}
		\right.\right.
		&& \nonumber \\ &
		\left.\left.
		+ 3 j_2 \frac{E_0-2\Ee}{2M}
	\right)
\right]
+ \epsilonuv \Re\left[ c_1^* j_3 \right]
\left( 
	\frac{21 \E^2}{8 M^2}
\right) &&
\end{flalign}
%\unskip
% !TEX root = ../thesis_main.tex
%
%
%
%%%% --- * --- %%%%	
\begin{flalign}
F_3(\Ee) = &
- \deltauv \, (3 u^2 + 3 u -1)
\left[
	\frac{u}{(u-1)(u+1)(u+2)(2u-1)(2u+3)}
\right]^{1/2}\!
&& \nonumber \\ &
\times 
\Re \left[
	a_1^* j_3
\right]
\left( 
	\frac{\Ee^2 \sqrt{15} }{4M^2}
\right)
+ 
\frac{\rhouv}{u+1} 
\Re \left[
	c_1^*(g\sqrt{3} + j_2\sqrt{2})
	\left(
		\frac{5 \Ee^2}{4 M^2}
	\right)
\right]
&& \nonumber \\ &
\pm
\frac{(-1)^s \sigmauv }{u+1} 
\Re \left[ c_1^*j_3 \right] 
\left( 
	\frac{5\Ee^2}{2 M^2}
\right) &&
\label{equation:holstein_F3}
\end{flalign}
%\unskip
and we might really appreciate if these things could be simplified a bit.  

The terms $a_1, a_2, b, c_1, c_2, d, e, f, g, h, j_2, j_3$ are ``structure functions''.  Holstein gives some predictions for their form, assuming the impulse approximation holds, in his Eq.~(67). \aside{There was something wrong with this assumption.  Something circular.  I forget.  Blah.}  For the most part, the values and form of these structure functions are beyond the scope of this thesis, so I will not re-write them all here. \aside{Or will I?}  \comment{It should be noted that the numerical values used for these parameters came from a private communication from Ian Towner to ... someone other than me.}  However, it is important to note the expressions for $a_i$ and $c_i$, because these will directly come into play when we try to reconcile Holstein's expression with JTW's.  Therefore, 
\bea
a(q^2) &\approx& \frac{g_V(q^2)}{(1 + \frac{\Delta}{2M})} \left[ M_F    + \frac{1}{6}(q^2 - \Delta^2) M_{r^2} + \frac{1}{3} \Delta M_{\mathbf{r} \cdot \mathbf{p} } \right] 
\label{equation:full_a}
\\ 
c(q^2) &\approx& \frac{g_A(q^2)}{(1 + \frac{\Delta}{2M})} \left[ M_{GT} + \frac{1}{6}(q^2 - \Delta^2) M_{\sigma r^2} + \frac{1}{6 \sqrt{10} }(2\Delta^2 + q^2) M_{1y} 
\right. \nonumber \\ && \left.
+ A \frac{\Delta}{2 M} M_{\sigma L} + \frac{1}{2} \Delta M_{\sigma r p} \right]
\label{equation:full_c}
\eea
\note{Somewhere I have to define $q^2$ and $\Delta$ are.}
...where the $M_{xxx}$'s are certain nuclear matrix elements.  \aside{Somewhere, just list out the goddamn values of things that I inherited from Ian Towner's personal communication that one time, over multiple generations of grad students.}  However, Eqs.~(\ref{equation:holstein_F0}-\ref{equation:holstein_F3}) are not written in terms of $a(q^2)$ and $c(q^2)$, but rather in terms of $a_1$, $a_2$, $c_1$, and $c_2$.  In fact, Holstein is implicitly using series expansions to remove the dependence on recoil momentum, so that
\bea
a(q^2) &=& a_1 + \left(\! \frac{q^2}{M^2} \! \right) a_2 + \cdots \label{equation:series_expand_a} \\
c(q^2) &=& c_1 + \left(\! \frac{q^2}{M^2} \! \right) c_2 + \cdots \label{equation:series_expand_c}
\eea

%Also, Holstein proceeds to split up $a(q^2)$ and $c(q^2)$ into their first two Taylor series terms for an expansion of .... $q/M$, maybe?  Or possibly $q^2 / M^2$?  Anyway, that's $a_1$ and $a_2$, and $c_1$ and $c_2$, in Holstein notation.  He doesn't do that with any of the other structure functions. 
%\bluetodo{Seriously, I need to check what the taylor series is even expanding in.}


%\note{In fact, we might want to add the other different-er terms in to this thing now, before we get ahead of ourselves.}
Next, Holstein goes and tweaks those $F_i(\Ebeta)$ terms that we've already written out, by adding in an adjustment for Coulomb corrections.  Those corrections have this form:
\beq
	F_i(\Ebeta) \rightarrow \tilde{F}_i(\Ebeta) := \FF \left[ F_i(\Ebeta) + \Delta F_i(\Ebeta) \right]
\eeq

To obtain expressions for the $\Delta F_i(\Ebeta)$, Holstein invokes some Feynman diagrams and provides expressions for several integrals, all of which are both complex and complicated.  The modified spectral functions are provided in terms of functions of these integrals.  Since nobody wants to have to evaluate those integrals, Holstein makes a further approximation by taking only the first term in an expansion of the $\Delta F_i(\Ebeta)$ in terms of $Z\alpha$, where $Z\alpha \ll 1$.  Then, the resulting expressions for $\Delta F_i(\Ebeta)$ can be written in terms of much more straightforward integrals over form factors for electric change and weak charge.  

If we make the further assumption that these form factors are identical, and that both types of charge are spread over a ball of uniform density with radius $R$,\aside{and also, I think something like that the weak charge is the same distribution as the electric charge} then we find:
\bea
	X = Y = \frac{9\pi R}{140}
\eea
in the Eqs.~(\ref{eq:holstein_DeltaF1_Euvs} - \ref{eq:holstein_DeltaF7_Euvs}) that follow.

Because Holstein doesn't actually write this stuff out in terms of $F_i(\Ebeta)$, but rather in terms of $F_i(\Ebeta, u,v,s)$, this correction presents yet another opportunity for the reader to interpret his notation incorrectly.  We note that one must remember to make use of the relations in Eq.~(\ref{eq:holstein_FHFf}).  Furthermore, Holstein notes that some of the terms $F_i(\Ebeta, u,v,s)$ are suppressed already, and he does not consider those terms further.  We will take this approximation to be adequate for our purposes here.

%\note{So clearly I'm going to need terms for $\Delta F_1(\Ebeta, u, v, s)$, $\Delta F_4(\Ebeta, u, v, s)$, $\Delta F_7(\Ebeta, u, v, s)$, $\Delta F_{10}(\Ebeta, u, v, s)$, $\Delta F_{13}(\Ebeta, u, v, s)$, and $\Delta F_{18}(\Ebeta, u, v, s)$. We really only have expressions for some of them in Holstein's Eq.~(C4).  In particular, we've got $\Delta F_1(\Ebeta, u, v, s)$, $\Delta F_4(\Ebeta, u, v, s)$ and $\Delta F_7(\Ebeta, u, v, s)$, but we're missing $\Delta F_{10}(\Ebeta, u, v, s)$, $\Delta F_{13}(\Ebeta, u, v, s)$, and $\Delta F_{18}(\Ebeta, u, v, s)$.  That's annoying.  Holstein gives as an excuse for that (have to check to make sure it works and that it's actually an excuse is for this) that the recoil terms $b$, $d$, and $f$ are already suppressed in their contribution to beta decay spectra. }

\note{ What is less clear, given the context in the paper, is whether or not when Holstein writes out his simplified expressions for $\Delta F_{x}(\Ebeta, u, v, s)$ he actually means $ \FF \Delta F_{i}(\Ebeta, u, v, s)$.  These terms are pretty small, so it probably doesn't *really* matter, but it would still be really nice to *know*, damn it.}

So, we'll write out the functions for these corrections.  
% !TEX root = ../thesis_main.tex
%
%
%
%%%% --- * --- %%%%	
\begin{flalign}
\Delta F_1(\Ee, u, v, s) = &
\mp \left( \frac{8 \alpha Z}{3\pi } \right) \left\{ |a|^2 \left[ 4 \Ee (X+Y) + E_0 X + \textstyle{\frac{\m c^2}{\Ee }}(X+2Y) \right] 
\right. &&\nonumber \\ & \left.
+ |c|^2\left[ \Ee ( \textstyle{\frac{16}{3}} X + 4 Y)  - \textstyle{\frac{1}{3}}E_0 X + \textstyle{\frac{\m c^2}{\Ee }} (X+2Y) \right]
\right\} &&
\label{eq:holstein_DeltaF1_Euvs}
\end{flalign}
% 
% 
% 
% \textstyle{\frac{16}{3}}\unskip
% !TEX root = ../thesis_main.tex
%
%
%
%%%% --- * --- %%%%	
\begin{flalign}
\Delta F_4(\Ee, u, v, s) = &
\mp \left( \frac{8 \alpha Z}{3\pi } \right) \Ee \, (5X + 4Y) \left[ \deltauv \left( \frac{u}{u+1} \right)^{1/2} 2\Re \,[a^* c] 
\right. && \nonumber \\ & \left.
\mp (-1)^s \left( \frac{\gammauv}{u+1} \right)^{\phantom{1/2}}\!\!\!\!\! |c|^2 \right] 
&&
\end{flalign}
% 
% 
% 
% \textstyle{\frac{16}{3}}\unskip
% !TEX root = ../thesis_main.tex
%
%
%
%%%% --- * --- %%%%	
\begin{align}
\Delta F_7(\Ee, u, v, s) = &
\mp \left( \frac{8 \alpha Z}{3\pi } \right) \left[ \deltauv \left( \frac{u}{u+1} \right)^{1/2} 2\Re \,[a^* c] 
\right. \nonumber \\ & \left.
\mp (-1)^s \left( \frac{\gammauv}{u+1} \right)^{\phantom{1/2}}\!\!\!\!\! |c|^2 \right] (E_0 - \Ee) X
\label{eq:holstein_DeltaF7_Euvs}
\end{align}
% 
% 
% 
% \textstyle{\frac{16}{3}}\unskip

We note that the above corrections have been written in terms of $a(q^2)$ and $c(q^2)$, and we must use Eqs.~(\ref{equation:series_expand_a}, \ref{equation:series_expand_c}) to put the results in terms of $a_1$,  $a_2$, $c_1$, and $c_2$ so that they can be correctly combined with Eqs.~(\ref{equation:holstein_F0}-\ref{equation:holstein_F3}).

If we evaluate Holstein's Eqs.~(B8), which I will absolutely not type out here, for the case $u=v=J=J^\prime=3/2$, we find the following values:
\begin{align}
\deltauv     &= 1 
& \thetauv   &= 1 
& \rhouv     &= \frac{-41}{40}
	\nonumber\\
\gammauv     &= 1 
& \kappauv   &=\frac{1}{2\sqrt{2}} % \;\; ! \!=\;\; \frac{3}{\sqrt{2}}
& \sigmauv   &= \frac{-41}{4\sqrt{35}}
	\nonumber\\
\lambdauv    &= \frac{-\sqrt{2} }{5} % 2\sqrt{3} 
& \epsilonuv &= \frac{-1}{2\sqrt{5}}
& \phiuv     &= 0 % \frac{1}{32} \left(\frac{3}{5}\right)^{\!1/2}.
	\nonumber\\
\label{eq:holstein_greekletterfunctions}
%
\end{align}
\unskip  % eq:holstein_greekletterfunctions
Furthermore, in our calculations here, we will be considering only the $\beta^+$ decay modes, and therefore we take the \emph{lower} sign when the option arises.  We also will use $s=0$, so that $(-1)^s = +1$.
\note{Also, pretty sure one of those never gets used.  Which one was it?  idk.}


%%% -- %%%
- -- --- -- - 

Let's define some of that notation!
Firstly, 
\bea
\textrm{Holstein's \,} \hat{n} &=& \textrm{JTW's \,} \mathbf{j},
\label{eq:nequalsj}
\eea
and the 
$\Lambda_i$ are given by Holstein's Eq.~(48):
\bea
    \Lambda_1   &:=& \LambdaOne   
    \label{eq:lambda1} \\
    \Lambda_2   &:=& \LambdaTwo 
    \label{eq:lambda2} \\
    \Lambda_3   &:=& \LambdaThree .
    \label{eq:lambda3}
\eea

\aside{Note:  It's not the case that $ | \vecJ | == J $.  It's actually super fucking infuriating notation. }

We immediately see that Holstein's $\Lambda_1$ is closely related to JTW's $\frac{\vecJ}{J}$, and a bit later after John points it out to us, we see that Holstein's $\Lambda_2$ is closely related to JTW's $\Talign$.  JTW doesn't have any equivalent to $\Lambda_3$.  In particular, we find:
\bea
\Lambda_1 \hatj &=& \LambdaOne \hatj \;\; = \;\; \frac{\vecJ}{J}  \\
%\Lambda_2 \frac{(J+1)}{(2J-1)} &=& \Talign
\Lambda_2 &=& \Talign \frac{(2J-1)}{(J+1)}.
\eea
%Now we'll have to deal with expanding the $F_i(\Ee)$.  %Note that these are very different from the $F_i(\Ee, u, v, s)$, and also different from the $f_i(\Ee)$.  
%Holstein makes a goddamn mess of this, so here we go!  From Holstein's Eq.~(B10):
%\bea
%F_i(\Ee) &=& H_i(\Ee, J, J^\prime, 0).
%\eea
%%and from Holstein's Eq.~(B9), we see that 
%%\bea
%%f_i(\Ee) &=& F_i(\Ee, J, J^\prime, 0)
%%\eea
%From Holstein's many Eqs.~(B7), we see that the $H_i(\Ee, u, v, s)$ can be written in terms of the functions $F_i(\Ee, u, v, s)$, which we carefully note \emph{are not the same} as the functions $F_i(\Ee)$.  We further see, from Holstein's Eq.~(B9) that a further set of functions, $f_i(\Ee)$ are defined in terms of the $F_i(\Ee, u, v, s)$.  In particular, Holstein's Eq.~(B9) states that
%\bea
%f_i(\Ee) &=& F_i(\Ee, J, J^\prime, 0).
%\eea
%Then, if we combine (some parts of) Holstein's Eqs.~(B7) with (B9) and (B10):
%\begin{align}
F_0(\Ee) & = H_0(\Ee, J, J^\prime, 0) = F_1(\Ee, J, J^\prime, 0) 
	\!\!\! 
	& = &\; f_1(\Ee) 
	\nonumber \\
F_1(\Ee) & = H_1(\Ee, J, J^\prime, 0) = \textstyle F_4(\Ee, J, J^\prime, 0) \,+\! \frac{1}{3}F_7(\Ee, J, J^\prime, 0) 
	\!\!\! 
	& = &\; \textstyle f_4(\Ee) \,+\! \frac{1}{3}f_7(\Ee) 
	\nonumber \\
F_2(\Ee) & = H_2(\Ee, J, J^\prime, 0) = \textstyle F_{10}(\Ee, J, J^\prime, 0) \!+\! \frac{1}{2}F_{13}(\Ee, J, J^\prime, 0) 
	\!\!\! 
	& = &\; \textstyle f_{10}(\Ee) \!+\! \frac{1}{3}f_{13}(\Ee) 
	\nonumber \\
F_3(\Ee) & = H_3(\Ee, J, J^\prime, 0) = F_{18}(\Ee, J, J^\prime, 0) 
	\!\!\!
	& = &\; f_{18}(\Ee) .
\label{eq:holstein_FHFf}
\end{align}
\unskip
%So that's fun.  Note that the $f_i(\Ee)$ are what goes into the polarized decay spectrum when the neutrino (ie, the recoil) is also observed.  It's a more complicated spectrum that way.  For this spectrum in which the neutrino has already been integrated over, we can just look up the $H_i(\Ee, J, J^\prime, 0) = H_i(\Ee, u, v, s)$ spectral functions, and leave it at that.
%So let's do this thing!
%% !TEX root = ../thesis_main.tex
%
%
%
%%%% --- * --- %%%%	
\begin{multline}
F_0(\Ee) = 
\left| a_1 \right|^2 
+ 2 \Re\left[ a_1^* a_2 \right] \frac{1}{3 M^2} 
\left[  
	\m^2 + 4 \Ee E_0 + 2 \frac{\me^2}{\Ee}E_0 - 4\Ee^2
\right]
\\
+ \left| c_1 \right|^2
+ 2 \Re\left[ c_1^* c_2\right] \frac{1}{9 M^2} 
\left[
	11 \me^2 + 20 \Ee E_0 
	- 2\frac{\me^2}{\Ee}E_0
	- 20\Ee^2
\right]
- 2 \frac{E_0}{3M} \Re\left[ c_1^*(c_1 + d \pm b)\right]
\\
+ \frac{2\Ee}{3M} 
\left( 
	3 \left| a_1 \right|^2 + \Re \left[ c_1^*(5c_1 \pm 2 b) \right]
\right)
- \frac{\me^2}{3 M \Ee} 
\Re \left[ 
	-3 a_1^*e + c_1^*\left(2c_1 + d \pm 2b - h\frac{E_0 - \Ee}{2M} \right)
\right]
\end{multline}
%\unskip
%% !TEX root = ../thesis_main.tex
%
%
%
%%%% --- * --- %%%%	
\begin{align}
F_1(\Ee) = & \:\:
\deltauv \left( \frac{u}{\!u+1\!} \right)^{\!\!1/2} \!\!
\left\{
	2 \Re\left[ 
		a_1^*\left(\! c_1 - \frac{E_0}{3M}(c_1 + d \pm b) + \frac{\Ee}{3M}( 7 c_1 \pm b + d )\! \right)
	\right]
	\right. \nonumber \\ & \left. 
	+
	2 \Re\left[
		a_1^* c_2 + c_1^* a_2
	\right] \!
	\left(
		\frac{4 \E(E_0 - \E) + 3 \me^2}{3M^2}
	\right) \!
\right\}
\nonumber \\ &
\mp \frac{ (-1)^s \gammauv}{u+1} 
\Re \left\{ \!
	c_1^* \! \left(
		c_1 + 2 c_2 \left(\frac{8\Ee(E_0-\Ee)+3\me^2}{3M^2}\right)
		- \frac{2 E_0}{3 M} (c_1 + d \pm b) 
		\right.\right. \nonumber \\ & \left.\left.
		+ \frac{\Ee}{3M} (11 c_1 - d \pm 5 b)
	\right)
\right\}
%\nonumber \\ &
+ 
\frac{\lambdauv}{u+1}
\Re \left\{ \!
	c_1^* \! \left[
		- f \left(\frac{5\Ee}{M}\right)
		\right. \right. \nonumber \\ & \left. \left.
		+ g \left( \frac{3}{2} \right)^{\!\!1/2} \!
		\left(
			\frac{E_0^2 - 11 E_0 \Ee + 6 \me^2 + 4\Ee^2 }{6M^2}
		\right) 
		%\right.\right. \nonumber \\ & \left.\left.
		\pm 3 j_2 
		\left(
			\frac{8 \Ee^2 - 5E_0 \Ee - 3 \me^2}{6M^2}
		\right)
	\right] \!
\right\}
\end{align}
%\unskip
%% !TEX root = ../thesis_main.tex
%
%
%
%%%% --- * --- %%%%	
\begin{flalign}
F_2(\Ee) = &
\thetauv \frac{\Ee}{2M} 
\Re\left[
	c_1^*\left(
		c_1 + c_2 \frac{8(E_0-\Ee)}{3M}
		-d \pm b
	\right)
\right]
&& \nonumber \\ &
- \deltauv \frac{\Ee}{M} 
\left[ \frac{u(u+1)}{(2u-1)(2u+3)} \right]^{1/2} \!
\Re \left\{ \!
	a_1^*\left( 
		\left( \! \frac{3}{2} \right)^{\!\!1/2}\!\! f
		+ g \frac{\Ee+2E_0}{4M} 
		\right.\right.
		&& \nonumber \\ &
		\left.\left.
		\pm \left( \frac{3}{2} \right)^{\!\!1/2}\!\! j_2 \frac{E_0-\E}{2M}
	\right) \!
\right\}
+ (-1)^s \, \kappauv \frac{\E}{2M}
\Re \left[
	c_1^* \! \left(
		\pm \, 3 f 
		\pm \left( \frac{3}{2} \right)^{\!\!1/2}\!\! g \frac{E_0-\Ee}{M}
		\right.\right.
		&& \nonumber \\ &
		\left.\left.
		+ 3 j_2 \frac{E_0-2\Ee}{2M}
	\right)
\right]
+ \epsilonuv \Re\left[ c_1^* j_3 \right]
\left( 
	\frac{21 \E^2}{8 M^2}
\right) &&
\end{flalign}
%\unskip
%% !TEX root = ../thesis_main.tex
%
%
%
%%%% --- * --- %%%%	
\begin{flalign}
F_3(\Ee) = &
- \deltauv \, (3 u^2 + 3 u -1)
\left[
	\frac{u}{(u-1)(u+1)(u+2)(2u-1)(2u+3)}
\right]^{1/2}\!
&& \nonumber \\ &
\times 
\Re \left[
	a_1^* j_3
\right]
\left( 
	\frac{\Ee^2 \sqrt{15} }{4M^2}
\right)
+ 
\frac{\rhouv}{u+1} 
\Re \left[
	c_1^*(g\sqrt{3} + j_2\sqrt{2})
	\left(
		\frac{5 \Ee^2}{4 M^2}
	\right)
\right]
&& \nonumber \\ &
\pm
\frac{(-1)^s \sigmauv }{u+1} 
\Re \left[ c_1^*j_3 \right] 
\left( 
	\frac{5\Ee^2}{2 M^2}
\right) &&
\label{equation:holstein_F3}
\end{flalign}
%\unskip
%...Phew!  I typed all of that out just so that I can have a record of what's going on here, but actually, the very first thing I want to do is make some simplifications here.  


% %%%%%%%%%%%%%%%%%%%%%%%%%%%%%% 
% \subsubsection[Algebra]{Algebra}
% Oookay.  Here's some stuff I'll want to keep track of now, but will also want to not have cluttering my documents later.
% \begin{align}
% \deltauv \left( \frac{u}{u+1} \right)^{\!\!1/2} \;\;&=\;\; \left( \frac{3}{5} \right)^{1/2}
% \\
% \mp \frac{ (-1)^s \gammauv}{u+1} \;\;&=\;\; \frac{2}{5}
% \\
% \frac{\lambdauv}{u+1} \;\;&=\;\; \frac{4 \sqrt{3}}{5}
% \\
% - \deltauv \left[ \frac{u(u+1)}{(2u-1)(2u+3)} \right]^{1/2}  \;\;&=\;\;  \frac{-\sqrt{5}}{4}
% \\
% %(-1)^s \kappauv \;\;&=\;\; \frac{3}{\sqrt{2}}  % I math-ed this wrong before.  I think.
% (-1)^s \kappauv \;\;&=\;\; \frac{1}{2\sqrt{2}}
% \\
% \frac{21}{8} \epsilonuv \;\;&=\;\; \frac{-21}{16 \sqrt{35}} %\;\;=\;\;
% \\
% \!\!\!\! \!\!\!
% - \deltauv (3 u^2 + 3 u -1)
% \left[
% 	\frac{u}{(u-1)(u+1)(u+2)(2u-1)(2u+3)}
% \right]^{1/2}\!
% \left(
% 	\frac{\sqrt{15}}{4}
% \right)
% \;\;&=\;\; \frac{-41 \sqrt{3}}{16 \sqrt{7}}
% \\
% \frac{\rhouv}{u+1}\left( \frac{5}{4} \right) \;\;&=\;\; \frac{41}{80}
% \\
% \pm
% \frac{(-1)^s \sigmauv }{u+1} \left( \frac{5}{2} \right) \;\;&=\;\; \frac{-41}{4\sqrt{35}}
% \end{align}
% %%%%%%%%%%%% Probably check the math on all of these over again.
% Note:  I calculated $\kappa_{u,v}$ wrong before.  Probably.  Unclear about the order of operations in really old papers.
%
% \subsubsection[More Algebra]{More Algebra}
% So here are some simplifications, using those expressions we just calculated in the stupid section above.
% %
% \begin{multline}
% F_0(\Ee) =
%   \left| a_1 \right|^2
% + \left( \!
% 	\frac{2\Ee}{M}
% \! \right)
% \left| a_1 \right|^2
% + \left| c_1 \right|^2
% + \left( \!
% 	\frac{- 2 E_0 + 10\Ee - 2\me^2/\Ee}{3M}
% \! \right)
% \left| c_1 \right|^2
% \\
% %
% + \frac{\me^2}{3 M \Ee}
% \left(
% 	3 Re\left[a_1^*e\right] - Re\left[c_1^*d\right] + 2 Re\left[c_1^*b\right]
% \right)
% + \frac{-4\Ee}{3M} Re \left[c_1^* b \right]
% + \frac{- 2 E_0}{3M} \left( Re\left[c_1^* d\right] - Re\left[c_1^* b \right] \vphantom{2_2^2} \right)
% \\
% + \frac{2}{3 M^2}
% \left(
% 	\m^2 + 4 \Ee E_0 + 2 \me^2\frac{E_0}{\Ee} - 4\Ee^2
% \right)
% Re\left[ a_1^* a_2 \right]
% \\
% + \frac{2}{9 M^2}
% \left(
% 	11 \me^2 + 20 \Ee E_0
% 	- 2\me^2\frac{E_0}{\Ee}
% 	- 20\Ee^2
% \right)
% Re\left[ c_1^* c_2\right]
% +
% \frac{\me^2}{6 M^2} \left(\frac{E_0 - \Ee}{\Ee} \right) Re\left[c_1^*h\right]
% %%%%%%% simplified enough for now.  continue it later.
% \end{multline}
% So that one's looking pretty good.  I can probably simplify it further later.
% %
% \begin{multline}
% F_1(\Ee) =
% \frac{2\sqrt{3}}{\sqrt{5}} Re \left[ a_1^*c_1 \right]
% + \frac{2}{5} \left| c_1 \right|^2
% %\\
% +
% \frac{2}{\sqrt{15}} \left( \frac{7\Ee -E_0}{M} \right)
% 	Re \left[ a_1^*c_1 \right]
% +
% \frac{2}{15}
% \left( \frac{11 \E- 2 E_0}{M} \right)
% 	\left| c_1 \right|^2
% \\
% +
% \frac{2}{\sqrt{15}} \left( \frac{E_0 - \Ee}{M} \right)
% 	Re \left[ a_1^*b \right]
% +
% \frac{2}{\sqrt{15}} \left( \frac{\Ee - E_0}{M} \right)
% 	Re \left[ a_1^*d \right]
% \\
% +
% \frac{2}{15}
% \left( \frac{2 E_0 - 5 \E}{M} \right)
% 	Re \left[c_1^*b \right]
% +
% \frac{2}{15}
% \left( \frac{-2 E_0 - \Ee }{M} \right)
% 	Re \left[c_1^*d \right]
% \\
% -
% \frac{20\sqrt{3}}{5} \left(\frac{\Ee}{M}\right)
% 	Re \left[ c_1^* f \right]
% \\
% +
% \frac{2}{\sqrt{15}} \left( \frac{4 \E(E_0 - \E) + 3 \me}{M^2} \right) \!
% 	Re\left[ a_1^* c_2 + c_1^* a_2 \right]
% +
% \frac{4}{15} \left(\frac{8\Ee(E_0-\Ee)+3\me}{M^2}\right)
% 	Re \left[ c_1^*c_2 \right]
% \\
% +
% \left(\frac{12}{30 \sqrt{2} }\right) \left( \frac{E_0^2 - 11 E_0 \Ee + 6 \me^2 + 4\Ee^2 }{M^2} \right)
% 	Re \left[ c_1^* g \right]
% +
% \left( \frac{-12\sqrt{3}}{30}\right) \left( \frac{8 \Ee^2 - 5E_0 \Ee - 3 \me^2}{M^2} \right)
% 	Re \left[c_1^* j_2 \right]
% \end{multline}
% $F_1(\Ee)$ is definitely looking better than it was before.  Onwards!
% %
% \begin{multline}
% F_2(\Ee) =
% \left( \frac{\Ee}{M} \right)
% \left[
% 	\frac{1}{2} \left| c_1 \right|^2
% 	- \frac{1}{2} Re\left[ c_1^*d \right]
% 	- \frac{1}{2} Re\left[ c_1^* b \right]
% 	+
% 	\frac{-\sqrt{15}}{4\sqrt{2}}
% 	Re \left[ a_1^* f \right]
% 	+
% 	\frac{-3}{4\sqrt{2}}
% 		Re \left[ c_1^* f \right]
% \right]
% \\
% +
% \frac{1}{M^2}
% \left\{
% 	\frac{4}{3} \left[ \Ee(E_0-\Ee) \right] Re\left[ c_1^*c_2 \right]
% 	+
% 	\frac{-\sqrt{5}}{16} \left[ \Ee(\Ee+2E_0) \right] Re \left[ a_1^* g \right]
% 	+
% 	\frac{\sqrt{15}}{8\sqrt{2}} \left[ \Ee(E_0-\E) \right] Re \left[ a_1^* j_2 \right]
% \right.
% \\
% \left.
% 	+
% 	\frac{-\sqrt{3}}{8} \left[ \Ee(E_0-\Ee) \right] Re \left[ c_1^* g \right]
% 	+
% 	\frac{3}{8\sqrt{2}}\left[ \Ee(E_0-2\Ee) \right] Re \left[ c_1^* j_2 \right]
% 	+
% 	\frac{-21}{16\sqrt{35}} \left( \E^2 \right) Re \left[ c_1^* j_3 \right]
% \right\}
% \end{multline}
% %
% And that's good enough for now for $F_2(\Ee)$, even though it's not really very good.
% \bea
% F_3(\Ee) &=&
% \left( \frac{\Ee^2}{M^2} \right)
% \left[
% \frac{-41\sqrt{3}}{16\sqrt{7}}
% 	Re \left[ a_1^* j_3 \right]
% +
% \frac{41\sqrt{3}}{80}
% Re\left[
% 	c_1^* g
% \right]
% 	+
% \frac{41\sqrt{2}}{80}
% Re\left[
% 	c_1^* j_2
% \right]
% + \frac{-41}{4\sqrt{35}}
% 	Re \left[ c_1^*j_3 \right]
% \right]
% \eea
% To be honest, $F_3(\Ee)$ came out a lot cleaner than I was expecting.  That's pretty nice.
%
% %
% % \begin{multline}
% % F_2(\Ee) =
% % \frac{-9}{\sqrt{2}} Re \left[c_1^* f \right]
% % +
% % \frac{\Ee}{2M}
% % Re\left[
% % 	c_1^*c_1
% % \right]
% % +
% % \frac{8\Ee(E_0-\Ee)}{6M^2} Re\left[ c_1^*c_2 \right]
% % +
% % \frac{-\Ee}{2M}
% % Re\left[
% % 	c_1^*d + c_1^*b
% % \right]
% % \\
% % + \frac{-\sqrt{15}}{4 \sqrt{2}}
% % \left( \frac{\Ee}{M} \right)
% % 	Re \left[ a_1^*f \right]
% % +
% % \frac{-\sqrt{5}}{16}
% % \left( \frac{\Ee(\Ee+2E_0)}{M^2} \right)
% % 	Re \left[ a_1^*g \right]
% % +
% % \frac{\sqrt{15}}{8\sqrt{2}}
% % \left( \frac{\Ee(E_0-\E)}{M^2} \right)
% % 	Re \left[ a_1^*j_2 \right]
% % \\
% % + \frac{-3 \sqrt{3}}{2} \left(\frac{E_0-\Ee}{M} \right)
% % 	Re \left[ c_1^* g \right]
% % + \frac{9}{\sqrt{2}} \left( \frac{E_0-2\Ee}{2M} \right)
% % 	Re \left[ c_1^* j_2 \right]
% % +
% % \frac{-21}{16\sqrt{35}}
% % \left(
% % 	\frac{\E^2}{M^2}
% % \right)
% % Re\left[ c_1^* j_3 \right]
% % \end{multline}
% % %
% % I'll leave $F_2(\Ee)$ for now.  Onwards!
%
%
%
% Let's see what they are in the limit of infinite recoil mass!  (I think that corresponds to $M$ in the equations, but I should really double check that...)
% \bea
% F_0(\Ee) |_{M\rightarrow\infty} &=& \left| a_1 \right|^2 + \left| c_1 \right|^2 \\
% F_1(\Ee) |_{M\rightarrow\infty} &=& \frac{2\sqrt{3}}{\sqrt{5}} Re \left[ a_1^*c_1 \right]+ \frac{2}{5} \left| c_1 \right|^2 \\
% F_2(\Ee) |_{M\rightarrow\infty} &=& 0 \\ %\frac{-9}{\sqrt{2}} Re \left[c_1^* f \right] \\
% F_3(\Ee) |_{M\rightarrow\infty} &=& 0
% \eea
%
% Anyway, let's expand our Eq.~(\ref{eq:holsteinpdf}) in terms of $\Lambda_i$s and things.  Using Eqs.~(\ref{eq:nequalsj}, \ref{eq:lambda1}, \ref{eq:lambda2}, \ref{eq:lambda3}) and changing the labelling on our quantization axis to $\hatj$:
% \bea
% \d^3 \Gamma &=& 2  G_v^2 \cos^2\theta_c \frac{\FF}{(2\pi)^4} \, \pe \Ee (E_0 - \Ee)^2 \dEe \, \dOmegae
% \nonumber\\
% && \times
% \left\{
% 	F_0(\E)
% 	+
% 	\left(\! \LambdaOne \!\right)
% 	\left[ \hatj \cdot \hatp \right] \left(\! \frac{\pe}{\Ee} \!\right)
% 	F_1(\E)
% 	+
% 	\left(\! \LambdaTwo \!\right)
% 	\left[ \left( \jhat \cdot \hatp \right)^2 - \frac{1}{3} \right] \left(\! \frac{ \pe^2 }{\Ee^2} \! \right)%^2
% 	F_2(\E)
% 	\right. \nonumber\\ && \left.
% 	+
% 	\left(\! \LambdaThree \!\right)
% 	\left[
% 		\left( \hatj \cdot \hatp \right)^{3}
% 		-
% 		\frac{3}{5} \left( \hatj \cdot \hatp \right)
% 	\right]
% 	\left(\! \frac{\pe^3}{\Ee^3} \! \right)
% 	F_3(\E)
% \right\}
% \eea
%
%









