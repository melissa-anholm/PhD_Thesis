% !TEX root = ../thesis_main.tex



%\chapter[Nuclear]{Misc. Nuclear Physics}
%\label{nuclear}
%\section{Things.}

\chapter[Misc. Nuclear Physics]{The Parity Operator:  Vectors and Axial Vectors and Pseudoscalars, oh my!}
\note[color=jb]{JB says:  Appendix G -> internal (needed for understanding for later ST analysis).}

%\section{The Parity Operator:  Vectors and Axial Vectors and Pseudoscalars, oh my!}
via Samuel Wong, 1990.  pg 212.

\section{Scalars}
A scalar does not change sign under the parity operation, because why even would it?

\section{Vectors}
Vectors, or ``polar vectors'' (V) are exactly what I think they are.  Position $\vec{r}$ and (linear) momentum $\vec{p}$ are examples.  The thing about these vectors is that they change sign (or, really, direction) under a parity transformation.  

\section{Axial Vectors}
Axial vectors (A) \emph{don't} change sign under a parity transformation.  An example is angular momentum, $\vec{L} = \vec{r} \times \vec{p}$.  This is where the "mirror" thing breaks down.  A mirror only is only a one-dimensional parity operator, so if you think of a thing with angular momentum and its reflection in a mirror, you imagine the mirror image with reversed angular momentum too.  But in reality, if you change the sign of \emph{all components} of $\vec{r}$ and simultaneously all components of $\vec{p}$, it's clear that the quantity $(\vec{r} \times \vec{p})$ remains unchanged.  Pauli spin matrices, $\vec{\mathbf{\sigma}}$, are axial vectors.

\section{Pseudoscalars}
Pseudoscalars, (P).  You'd think they were scalars, because they're just a number, but have to remember that you got them by taking the scalar product of a (polar) vector and an axial vector.  If you apply the parity operator to the quantity $P = (\vec{V} \cdot \vec{A})$, $\vec{V}$ changes sign (for all components) while $\vec{A}$ does not.  So the resultant ``scalar'' $P$ has to change sign too.  That's how you know that $P$ is really a pseudoscalar.

\section{Tensors}
...Yeah, Wong doesn't really get into that.  At least not here.

\section{Comments on Parity Conservation}
An interaction made from a mixture of scalars and pseudoscalars, or a mixture of vectors and axial vectors, does \emph{not} conserve parity.  

For the strong and electromagnetic forces, parity \emph{is} strictly conserved.  Not so for the weak force!

\section{Q-Values}
The $Q$-value for a particular interaction is defined as the difference between the kinetic energies of the final and initial systems.
\bea
Q&=&T_f - T_i
\eea
In particular, for $\beta^+$ decay, we're including .... which things? ... as part of the mass of the various systems?  


\section{Helicity}
Helicity, $h$, is a pseudoscalar.
\bea
h &=& \frac{\vec{\sigma} \cdot \vec{p}}{|\vec{p}|}
\eea

%\section{Allowed Decays}
%via Krane:
%
%The beta and the neutrino can't carry away any \emph{orbital} angular momentum, because we pretend like everything starts at the origin and moves directly outwards.  You \emph{can} change the nuclear angular momentum in an allowed decay though, because your outgoing leptons have spin.  So the nuclear angular momentum changes by either 0 or 1 in an allowed decay.
%
%via severjins\_beck\_cuncic\_2006.pdf:
%
%Selection rules for an allowed transition are:
%\bea
%\Delta I = I_f - I_i = \{0, \pm 1\} \\ 
%\hat{\Pi}_i \, \hat{\Pi}_f = +1
%\eea
%Then, you divide the allowed transitions into singlet (anti-parallel lepton spins, $S=0$ -- a Fermi transition) and triplet states (parallel lepton spins, $S=1$ -- a Gamow-Teller transition).
%
%\section{Fermi Decays}
%The spins of the two leptons are antiparallel, so there's no change in nuclear angular momentum (assuming it's  an Allowed transition).  They're ``vector'' interactions.
%
%\section{Gamow-Teller Decays}
%The leptons have parallel spins, so (in the Allowed approximation) they change the projection of the nuclear angular momentum ($M_I$) by 1 unit.  The transition may or may not simultaneously change I by one unit.  These are ``axial vector'' interactions.  Note that $I=0 \rightarrow I=0$ transitions are not permitted as Gamow-Teller decays.

\section{Conserved Vector Current Hypothesis}
The CVC hypothesis asserts that the Fermi interactions of nucleons within a nucleus are \emph{not} changed by all the surrounding mesons.  (What?)
\note[color=jb]{JB:  You're describing a consequence of CVC, and are not stating the actual hypothesis.}
\note[color=jb]{JB:  I think you don't have time to explain the CBC hypothesis.  You'll just have to assume it.  I personally found that the technical derivation was the only way to see what was going on.  }

\note[color=jb]{JB:  A good sketch of CVC is done in Commins' notes and his book with Bucksbaum~\cite{ComminsBucksbaum}.  Ian towner did it in his notes.  I reproduce this in my own course notes for Phys 505 (p.21-28 of the thing John attached to that email).  First you show in Dirac notation that the E\&M current for pointlike particles is conserved--this just needs one use of the Dirac Equation and conservation of electric charge.  Then look what happens for finite nucleons.  Then construct the analogous weak interaction current for the nucleon, including three possible current terms that transform like Lorentz vectors, and note the consequences if that current is still conserved.}

\note{Figure out how to cite somebody's unpublished/semi-unpublished notes.}
\note[color=jb]{JB:  Jelley~\cite{Jelley} describes the qualitative background on his text page 110 (attached).  Feynman and Gell-Mann proposed that the isovector weak vector current and the isovector E\&M vector current are members of an isotriplet of currents all of which are conserved.  [This idea eventually went straight into the SM electroweak interaction, with the addition of nonabelian and massive operators for the weak part, though figuring out how to make a consistent theory with the massive bosons took the combination of Yang-Mills gauge theories and then Weinberg and Salam's approaches.] }

\note[color=jb]{JB:  What you say is one consequence, that meson exchange currents don't change the vector part of the weak interaction-- this is consistent with conservation of electric charge and its analog in the weak interaction.  This particular consequence is a very powerful tool in electromagnetic effects in nuclei, e.g. if you take the isovector combination of magnetic moments (37Ar - 37K e.g.) you get something without meson exchange corrections and therefore precisely sensitive to other higher-order physics effects.  Arima and Towner separately studied this for significant parts of their careers.  The axial vector interaction is changed by meson exchange currents.  The recent calculation of Gysbeg et al. accounting for most of the Gamow-Teller strength is including meson exchange currents in what they call 2-body currents, natural higher-order corrections in their chiral EFT expansion of the strong interaction between nucleons when you consider electroweak interactions.}

\note[color=jb]{JB:  But there are lots of other things that also don't change the Fermi interaction... so picking out meson exchange currents for discussion is maybe not fully motivated.  Lots of people state one consequence or another of CVC to motivate what they are doing, without explaining.}

\[\]
\note[color=jb]{Paraphrased JB:  Don't try to derive CVC in this thesis.  Just cite the hypothesis and say that our Abeta stuff provides a test of it.}
\note[color=jb]{JB:  Something else beyond the thesis scope, sorry:  a nonzero $\bFierz$ does not necessarily break CVC, as the vector part of the SM weak interaction could still be conserved whether or not there are other quark-lepton currents with different Lorentz structure.  You could make a nonzero $\bFierz$ from a 2nd-class scalar in the nucleon-electron weak current which would break CVC, but that could not be distinguished from a quark-electron extra scalar current.  }






