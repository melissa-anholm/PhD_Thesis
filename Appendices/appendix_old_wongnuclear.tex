%\chapter[Nuclear]{Misc. Nuclear Physics}
%\label{nuclear}
%\section{Things.}

\chapter[Misc. Nuclear Physics]{The Parity Operator:  Vectors and Axial Vectors and Pseudoscalars, oh my!}

%\section{The Parity Operator:  Vectors and Axial Vectors and Pseudoscalars, oh my!}
via Samuel Wong, 1990.  pg 212.

\section{Scalars}
A scalar does not change sign under the parity operation, because why even would it?

\section{Vectors}
Vectors, or ``polar vectors'' (V) are exactly what I think they are.  Position $\vec{r}$ and (linear) momentum $\vec{p}$ are examples.  The thing about these vectors is that they change sign (or, really, direction) under a parity transformation.  

\section{Axial Vectors}
Axial vectors (A) \emph{don't} change sign under a parity transformation.  An example is angular momentum, $\vec{L} = \vec{r} \times \vec{p}$.  This is where the "mirror" thing breaks down.  A mirror only is only a one-dimensional parity operator, so if you think of a thing with angular momentum and its reflection in a mirror, you imagine the mirror image with reversed angular momentum too.  But in reality, if you change the sign of \emph{all components} of $\vec{r}$ and simultaneously all components of $\vec{p}$, it's clear that the quantity $(\vec{r} \times \vec{p})$ remains unchanged.  Pauli spin matrices, $\vec{\mathbf{\sigma}}$, are axial vectors.

\section{Pseudoscalars}
Pseudoscalars, (P).  You'd think they were scalars, because they're just a number, but have to remember that you got them by taking the scalar product of a (polar) vector and an axial vector.  If you apply the parity operator to the quantity $P = (\vec{V} \cdot \vec{A})$, $\vec{V}$ changes sign (for all components) while $\vec{A}$ does not.  So the resultant ``scalar'' $P$ has to change sign too.  That's how you know that $P$ is really a pseudoscalar.

\section{Tensors}
...Yeah, Wong doesn't really get into that.  At least not here.

\section{Comments on Parity Conservation}
An interaction made from a mixture of scalars and pseudoscalars, or a mixture of vectors and axial vectors, does \emph{not} conserve parity.  

For the strong and electromagnetic forces, parity \emph{is} strictly conserved.  Not so for the weak force!

\section{Q-Values}
The $Q$-value for a particular interaction is defined as the difference between the kinetic energies of the final and initial systems.
\bea
Q&=&T_f - T_i
\eea
In particular, for $\beta^+$ decay, we're including .... which things? ... as part of the mass of the various systems?  


\section{Helicity}
Helicity, $h$, is a pseudoscalar.
\bea
h &=& \frac{\vec{\sigma} \cdot \vec{p}}{|\vec{p}|}
\eea

%\section{Allowed Decays}
%via Krane:
%
%The beta and the neutrino can't carry away any \emph{orbital} angular momentum, because we pretend like everything starts at the origin and moves directly outwards.  You \emph{can} change the nuclear angular momentum in an allowed decay though, because your outgoing leptons have spin.  So the nuclear angular momentum changes by either 0 or 1 in an allowed decay.
%
%via severjins\_beck\_cuncic\_2006.pdf:
%
%Selection rules for an allowed transition are:
%\bea
%\Delta I = I_f - I_i = \{0, \pm 1\} \\ 
%\hat{\Pi}_i \, \hat{\Pi}_f = +1
%\eea
%Then, you divide the allowed transitions into singlet (anti-parallel lepton spins, $S=0$ -- a Fermi transition) and triplet states (parallel lepton spins, $S=1$ -- a Gamow-Teller transition).
%
%\section{Fermi Decays}
%The spins of the two leptons are antiparallel, so there's no change in nuclear angular momentum (assuming it's  an Allowed transition).  They're ``vector'' interactions.
%
%\section{Gamow-Teller Decays}
%The leptons have parallel spins, so (in the Allowed approximation) they change the projection of the nuclear angular momentum ($M_I$) by 1 unit.  The transition may or may not simultaneously change I by one unit.  These are ``axial vector'' interactions.  Note that $I=0 \rightarrow I=0$ transitions are not permitted as Gamow-Teller decays.

\section{Conserved Vector Current Hypothesis}
The CVC hypothesis asserts that the Fermi interactions of nucleons within a nucleus are \emph{not} changed by all the surrounding mesons.  (What?)



