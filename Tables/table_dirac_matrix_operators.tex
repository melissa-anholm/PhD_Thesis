% !TEX root = ../thesis_main.tex
%
%
%
%%%% --- * --- %%%%	
%\renewcommand{\arraystretch}{1.6}
\begin{table}[h!!!!t]
	\begin{center}
	\begin{tabular}{ l  l  l}
		\multicolumn{3}{l}{ \textbf{Lorentz Invariant Operators}}
		\\  %\hline
		\multicolumn{1}{l}{Name} 		& \multicolumn{1}{l}{ Form}   								& \multicolumn{1}{l}{Parity}   	
		\\  \hline
		%%% % %%%
		Scalar 			   				& $1$														& $+$									
		\\
		Pseudoscalar \;\;\;\;\;\;\;		& $\gamma_5$									 			& $-$				
		\\
		Vector							& $\gamma_\mu$												& $-$				
		\\
		Axial-vector					& $\gamma_\mu \gamma_5$										& $+$
		\\
		Tensor							& $\gamma_\mu \gamma_\nu - \gamma_\nu \gamma_\mu$ \;\;		& N/A									
		\\  \hline
		%%% % %%%
	\end{tabular}
	\end{center}
%	\caption[Dirac Matrix Operators]{Dirac Matrix Operators.  I need to reference this table somewhere.  Also, these are dirac matrices in 4D.  More D, different Dirac matrices.}
	\note{Ugh, I've largely just avoided talking about parity....}
	\caption[Lorentz Invariant Operators]{A complete list of operators that obey Lorentz invariance, defined in terms of Dirac $\gamma$-matrices~\cite{ben_thesis,dan_thesis}.  It can be shown that the operators listed here span the entire space, meaning that any other Lorentz invariant operator can be expressed as a sum of the above. }
%%%%%	as a result of certain symmetries in the Dirac matrices that all other Lorentz invariant operators can be reduced to these.}
	\label{table:dirac_matrix_operators}
\end{table}
%\renewcommand{\arraystretch}{1}