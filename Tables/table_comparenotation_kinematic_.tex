% !TEX root = ../thesis_main.tex
%
%
%
%%%% --- * --- %%%%	
\renewcommand{\arraystretch}{1.6}
\begin{table}[h!!!!t]
	\begin{center}
	\begin{tabular}{ | l | l | p{3.35in} | }
		\multicolumn{1}{c}{Holstein} 				& \multicolumn{1}{c}{JTW} 					& \multicolumn{1}{c}{Comments}
		\\  \hline
		$\hatn$ 									& $\hatj$									& Nuclear polarization unit vector.  Also the axis of quantization.  %In Holstein, this is actually just the axis of quantization, and the mathematical framework is provided such that it need not be aligned with the polarization.  However, JTW makes the simplification that the axis of quantization and the axis of polarization are the same, so we will do the likewise here.  \comment{(is this even true?)}
		\\  \hline
		$\hatk$ 									& $\frac{\vecpnu}{E_\nu}$					& Neutrino emission direction unit vector (3D).  Neutrinos are always treated as massless.
		\\  \hline
		$\bm{\vec{p}} $								& $\bm{\vec{p}_e}$							& 3D Beta momentum.  We use $\bm{\vec{p}_\beta}$ in this document.
		\\  \hline
		$E$											& $E_e$										& Beta energy.  We use $\Ebeta$ here.
		\\  \hline
%		$q$											& $E_\nu$									& Neutrino energy.  Because JTW does not include ROC, their treatment uses $E_\nu$ only to aid clarity, rather than as an independent variable.  Holstein claims that $q$ is a four-vector, but that cannot be strictly true since $q^2 \neq - m_\nu c^2$.
%		\\  \hline
		$p$											& ...										& Beta momentum 4-vector.
		\\  \hline
		$k$											& ...										& Neutrino 4-vector.
		\\  \hline
		$q$											& No equivalent								& Recoil momentum 4-vector.
		\\  \hline
	\end{tabular}
	\end{center}
	\caption[Notation Guide]{A comparison of some kinematic terms in JTW~\cite{jtw}~\cite{jtw_coulomb} and Holstein~\cite{holstein}.}
	\label{table:compare_notation_kinematic}
\end{table}
\renewcommand{\arraystretch}{1}
%
%