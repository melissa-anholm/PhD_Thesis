% !TEX root = ../thesis_main.tex
%
%
%
%%%% --- * --- %%%%	
\renewcommand{\arraystretch}{1.6}
\begin{table}[h!!!!t]
	\begin{center}
	\begin{tabular}{ | c | c | c | p{3.35in} | }
		\multicolumn{1}{c}{Holstein} 				& \multicolumn{1}{c}{JTW} 								& \multicolumn{1}{c}{Thesis} 	& \multicolumn{1}{c}{Comments}
		\\  \hline
		$k$											& 														& 								& Neutrino momentum 4-vector% (its magnitude is 0)
		\\  \hline
		$ $											& $E_\nu$												& 								& Neutrino energy
		\\  \hline
		$\hat{k}$ 									& $\displaystyle \frac{\mathbf{p_{\bm{\nu}}}}{E_\nu} $	&  %$\displaystyle \frac{\vecpnu}{E_\nu}$		
																																			& 3D Neutrino emission direction unit vector.  Neutrinos are always treated as massless.
		\\  \hline
		$p$											& 														& 								& Beta momentum 4-vector, or sometimes the magnitude of the beta momentum 3-vector.  Never the magnitude of the 4-vector.
		\\  \hline
		$E$											& $E_e$													& $\Ebeta$						& Beta energy
		\\  \hline
		$ \mathbf{p} $								& $\mathbf{p_e}$										& $\bm{\vec{p}_\beta}$			& Beta momentum 3-vector
		\\  \hline
		$q$											&  														&								& Recoil momentum 4-vector, or sometimes a magnitude.
		\\  \hline
	\end{tabular}
	\end{center}
	\caption[Notation Guide]{A comparison of some kinematic terms in JTW~\cite{jtw}~\cite{jtw_coulomb} and Holstein~\cite{holstein}.  Yes, the bolding/italicization carries meaning.  }
	\label{table:compare_notation_kinematic}
\end{table}
\renewcommand{\arraystretch}{1}
%
%